\documentclass{article}
\usepackage[spanish]{babel}
\usepackage{amsmath, amssymb, amsthm}

% Configuración de teoremas
\newtheorem{theorem}{Teorema}
\newtheorem{lemma}{Lema}
\newtheorem{proposition}{Proposición}
\newtheorem{corollary}{Corolario}

% Comandos útiles
\newcommand{\norm}[1]{\left\lVert #1 \right\rVert}
\newcommand{\abs}[1]{\left\lvert #1 \right\rvert}
\newcommand{\e}{e^{i n x}}
\newcommand{\Lone}{L^1(0,2\pi)}

\begin{document}

\title{Una función en $L^1([0,2\pi])$ tal que su serie de Fourier diverge en todas partes.}
\author{Andrés David Cadena Simons}
\date{\today}
\maketitle

\section{Preliminares}\label{prop:3.17}
  \begin{proposition}
    Existe una sucesión de polinomios trigonométricos \( (\varphi_n)_{n=2}^{+\infty} \) tal que para todo \( n \in \{2,3,\dots\} \), cumple las siguientes propiedades:
    
    \begin{enumerate}
        \item[A)] \( \varphi_n \geq 0 \), para todo \( x \in [0,2\pi] \).
        \item[B)] \( \int_{[0,2\pi]} \varphi_n d\lambda = \pi \), donde \( \lambda \) es la medida de Lebesgue en la recta.
        \item[C)] Sea \( v_n \in \mathbb{Z}^+ \) el orden del polinomio trigonométrico \( \varphi_n \). Entonces, es posible determinar números \( Q_n \in \mathbb{R} \), \( \lambda_n \in \mathbb{Z}^+ \) y un conjunto \( F_n \subset [0,2\pi] \) tales que:
        
        \begin{enumerate}
            \item[C.i)] \( \lim\limits_{n \to +\infty} Q_n = +\infty \),
            \item[C.ii)] \( F_2 \subset \cdots \subset \bigcup\limits_{p=2}^{+\infty} F_p = [0,2\pi] \),
            \item[C.iii)] \( \lim\limits_{n \to +\infty} \lambda_n = +\infty \) y
            \item[C.iv)] para todo \( x \in F_n \), existe un \( k := k_x \in \mathbb{Z}^+ \) tal que \( \lambda_n \leq k < v_n \) y
            \[
            S_k(\varphi_n; x) > Q_n,
            \]
        \end{enumerate}
    \end{enumerate}
    
    donde \( S_k(\varphi_n, \cdot) \) es la \( k \)-ésima suma parcial de la serie de Fourier de la función \( \varphi_n \).
\end{proposition}


\section{Teorema}
\begin{theorem}{Teorema de Kolmogórov}
  Existe una función $\Phi\in L^{1}([0,2\pi])$, tal que si serie de Fourier diverge en todas partes.
\end{theorem}
\begin{proof} 
  Consideremos una sucesión estrictamente decreciente de números enteros $(n_{\alpha})_{\alpha\in\mathbb{N}}$, tal que $n_0\geq 2$ y que para todo $\alpha\in \mathbb{Z}^{+}$ se cumplen las siguientes propiedades:
  \begin{itemize}
    \item $\lambda_{n_{\alpha}}>v_{n_{\alpha-1}}$.
    \item $Q_{n_{\alpha}}>4Q_{n_{\alpha-1}}$.
    \item $\sqrt{Q_{n_{\alpha}}}>v_{n_{\alpha-1}}$.
  \end{itemize}
  donde para todo $s\in\{2,3,\cdots\}, \lambda_s,v_{s}$ y $Q_{s}$ son los índices de la proposición \ref{prop:3.17}
\end{proof}

\section{Bibliografía}
\bibliographystyle{plain}
\bibliography{references}

\end{document}

