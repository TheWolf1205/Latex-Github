
\documentclass[12pt]{article}
\usepackage[utf8]{inputenc}
\usepackage{amsmath, amssymb}
\usepackage{geometry}
\usepackage{graphicx}
\usepackage{enumitem}

\geometry{margin=1in}

\title{Desarrollo de Integral usando Método de Residuos}
\author{}
\date{}

\begin{document}

\maketitle

Queremos resolver la integral:
\begin{equation*}
I = \int_{0}^{2\pi} \frac{dx}{(a - b\cos(x))^2}
\end{equation*}
donde \( a > b > 0 \).

Haciendo el cambio de variable \( z = e^{ix} \), tenemos:
\begin{equation*}
\cos(x) = \frac{z^2 + 1}{2z}
\end{equation*}
y \( dx = \frac{dz}{iz} \), lo que convierte la integral en un contorno en el plano complejo:
\begin{equation*}
I = \oint_{|z|=1} \frac{\frac{1}{iz} dz}{\left(a - b \frac{z^2 + 1}{2z}\right)^2}
\end{equation*}

Simplificando:
\begin{equation*}
I = \frac{2}{i} \oint_{|z|=1} \frac{dz}{\left( 2az - b(z^2 + 1) \right)^2}
\end{equation*}
\begin{equation*}
I = \frac{2}{i} \oint_{|z|=1} \frac{dz}{\left( -bz^2 + 2az - b \right)^2}
\end{equation*}

Los polos se encuentran al resolver:
\begin{equation*}
-bz^2 + 2az - b = 0
\end{equation*}
\begin{equation*}
z = \frac{2a \pm \sqrt{4a^2 - 4b^2}}{2b} = \frac{a \pm \sqrt{a^2 - b^2}}{b}
\end{equation*}

Definiendo \( k = \frac{a - \sqrt{a^2 - b^2}}{b} \), tenemos un polo dentro del contorno unitario, ya que \( k < 1 \) cuando \( a > b \).

El residuo en \( z = k \) se calcula como:
\begin{equation*}
\text{Res} \left[ \frac{2}{i} \frac{1}{\left( -b(z-k)(z-k') \right)^2}, z = k \right]
\end{equation*}
Realizando las operaciones, obtenemos:
\begin{equation*}
I = \frac{2\pi}{a^2 - b^2}
\end{equation*}

\end{document}
