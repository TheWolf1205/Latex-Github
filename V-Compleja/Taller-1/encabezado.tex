\documentclass{article}
\usepackage{tikz}
\usetikzlibrary{positioning, automata}
\usepackage{fancyhdr}
\usepackage{extramarks}
\usepackage[plain]{algorithm}
\usepackage{algpseudocode}
\usepackage[utf8]{inputenc} 
\usepackage[T1]{fontenc}    
\usepackage{hyperref}       
\usepackage{url}        
\usepackage{booktabs}
\usepackage{pdfpages}
\usepackage{amsfonts}   
\usepackage{nicefrac}       
\usepackage{microtype}      
\usepackage{amsmath}
\usepackage{graphicx}
\usepackage{float}
\usepackage{caption}
\usepackage{ragged2e}
\usepackage{amssymb}
\usepackage{mathtools}
\usepackage{xcolor}
\usepackage[spanish]{babel} % ¡Solo una vez!
\usepackage{array}
\usepackage{multirow}
\usepackage{multicol}
\usepackage{manfnt}
\usepackage{layout}
\usepackage{manfnt}
\usepackage{phaistos}
\usepackage{polynom}
\usepackage[most]{tcolorbox}
\RequirePackage{algorithm}
\RequirePackage{algpseudocode}


\setcounter{page}{0}
%
% Basic Document Settings
%

\topmargin=-0.45in
\evensidemargin=0in
\oddsidemargin=0in
\textwidth=6.5in
\textheight=9.0in
\headsep=0.25in

\linespread{1.1}
%|#*******************************************************************#|
\pagestyle{fancy}
\lhead{Taller}
\chead{\hmwkClass\ : \hmwkTitle}
\rhead{\firstxmark}
\lfoot{\lastxmark}
\cfoot{\thepage}

\renewcommand\headrulewidth{0.4pt}
\renewcommand\footrulewidth{0.4pt}

\setlength\parindent{0pt}

%
% Create Problem Sections
%

\newcommand{\enterProblemHeader}[1]{
    \nobreak\extramarks{}{Problema \arabic{#1} continúa en la siguiente página\ldots}\nobreak{}
    \nobreak\extramarks{Problema \arabic{#1} (continuación}{Problema \arabic{#1} continúa en la siguiente página\ldots}\nobreak{}
}

\newcommand{\exitProblemHeader}[1]{
    \nobreak\extramarks{Problema \arabic{#1} (continuación)}{Problema \arabic{#1} continúa en la siguiente página\ldots}\nobreak{}
    \stepcounter{#1}
    \nobreak\extramarks{Problema \arabic{#1}}{}\nobreak{}
}



\setcounter{secnumdepth}{0}
\newcounter{partCounter}
\newcounter{homeworkProblemCounter}
\setcounter{homeworkProblemCounter}{1}
\nobreak\extramarks{Problema \arabic{homeworkProblemCounter}}{}\nobreak
%margen de una pagina
\newenvironment{changemargin}[2]{%
\begin{list}{}{%
\setlength{\topsep}{0pt}%
\setlength{\leftmargin}{#1}%
\setlength{\rightmargin}{#2}%
\setlength{\listparindent}{\parindent}%
\setlength{\itemindent}{\parindent}%
\setlength{\parsep}{\parskip}%
}%
\item[]}{\end{list}}

%
% Homework Problem Environment
%
% This environment takes an optional argument. When given, it will adjust the
% problem counter. This is useful for when the problems given for your
% assignment aren't sequential. See the last 3 problems of this template for an
% example.
%
\newenvironment{homeworkProblem}[1][-1]{
    \ifnum#1>0
        \setcounter{homeworkProblemCounter}{#1}
    \fi
    \section{Problema \arabic{homeworkProblemCounter}:}
    \setcounter{partCounter}{1}
    \enterProblemHeader{homeworkProblemCounter}
}{
    \exitProblemHeader{homeworkProblemCounter}
}


%|#*******************************************************************#|

\newcommand{\hmwkTitle}{Taller 1}
\newcommand{\hmwkDueDate}{\today}
\newcommand{\hmwkClass}{Variable Compleja}
\newcommand{\hmwkUniversity}{Universidad Nacional de Colombia}
\newcommand{\hmwkAuthorName}{Andrés David Cadena Simons \and acadenas@unal.edu.co}
\newcommand{\hmwkInstructor}{Cesar Augusto Gómez Sierra}

%
% Title Page
%

\title{
    \vspace{2in}
    \textmd{\textbf{\hmwkClass:\ \hmwkTitle}}\\
    \normalsize\vspace{0.1in}\small{Nose de cuando del 2024}\\
    \vspace{0.1in}\large{\textit{\hmwkUniversity}}\\
    \vspace{1.5in} \textrm{\hmwkInstructor}
    \vspace{1.5in}
}

\author{\hmwkAuthorName}
\date{}

\renewcommand{\part}[1]{\textbf{\large Part \Alph{partCounter}}\stepcounter{partCounter}\\}
%
%   Nuevos tipos de columna
%
\newcolumntype{C}{>{$}c<{$}}
\newcolumntype{L}{>{$}l<{$}}
\newcolumntype{R}{>{$}r<{$}}
% ---------------------------
%|  Various Helper Commands  |
% ---------------------------

% Useful for algorithms
\newcommand{\alg}[1]{\textsc{\bfseries \footnotesize #1}}

% -For derivatives-
\newcommand{\deriv}[2]{\frac{\mathrm{d #1 }}{\mathrm{d} #2} }

%For derivatives of degree >1
\newcommand{\mderiv}[3]{\frac{\mathrm{d^{#3} #1 }}{\mathrm{d} #2} }

% -For partial derivatives-
\newcommand{\pderiv}[2]{\frac{\partial}{\partial #1} (#2)}

% -Integral dx-
\newcommand{\dx}{\hspace{3pt}\mathrm{d}}

% Alias for the Solution section header
% \newcommand{\solution}{\textbf{\\\\\large Solución:\\ \hspace*{5pt}}}

%parentesis automatico
\newcommand{\parauto}[1]{\ensuremath{\left( #1 \right)}}

% Probability commands: Expectation, Variance, Covariance, Bias
\newcommand{\E}{\mathrm{E}}
\newcommand{\Var}{\mathrm{Var}}
\newcommand{\Cov}{\mathrm{Cov}}
\newcommand{\Bias}{\mathrm{Bias}}


%TcolorBox

% Definir colores y la tcolorbox de la solución
\definecolor{myDColor}{HTML}{101010} 

\definecolor{myLColor}{RGB}{153,204,255} 

\definecolor{LinkColor}{HTML}{9669d9} 


\newtcolorbox{solution}[1][]{%
    enhanced,
    skin first=enhanced,
    skin middle=enhanced,
    skin last=enhanced,
    before upper={\parindent15pt},
    breakable,
    boxrule = 0pt,
    frame hidden,
    borderline west = {4pt}{0pt}{myDColor},
    colback = myLColor!5,
    coltitle = myLColor!5,
    sharp corners,
    rounded corners = southeast,
    arc is angular,
    arc = 3mm,
    attach boxed title to top left,
    boxed title style = {%
        enhanced,
        colback = myDColor,
        colframe = myDColor,
        top = 0pt,
        bottom = 0pt,
        sharp corners,
        rounded corners = northeast,
        arc is angular,
        arc = 2mm,
        rightrule = 0pt,
        bottomrule = 0pt,
        toprule = 0pt,
    },
    title = {\bfseries\large Solución:}, 
    overlay unbroken={%
        \node[anchor=west, color=black!70] at (title.east) {#1};
        \path[fill = tcbcolback!80!black] 
            ([yshift = 3mm]interior.south east) -- ++(-0.4,-0.1) -- ++(0.1,-0.2);
    },
    overlay first = {%
        \node[anchor=west, color=black!70] at (title.east) {#1};
        \path[fill = tcbcolback!80!black] 
            ([yshift = 3mm]interior.south east) -- ++(-0.4,-0.1) -- ++(0.1,-0.2);
    },
    overlay middle={%
        \path[fill = tcbcolback!80!black] 
            ([yshift = -3mm]interior.north east) -- ++(-0.4,0.1) -- ++(0.1,0.2);
        \path[fill = tcbcolback!80!black] 
            ([yshift = 3mm]interior.south east) -- ++(-0.4,-0.1) -- ++(0.1,-0.2);
    },
    overlay last={%
        \path[fill = tcbcolback!80!black] 
            ([yshift = -3mm]interior.north east) -- ++(-0.4,0.1) -- ++(0.1,0.2);
        \path[fill = tcbcolback!80!black] 
            ([yshift = 3mm]interior.south east) -- ++(-0.4,-0.1) -- ++(0.1,-0.2);
    },
    extras middle and last = { rounded corners = northeast }
}

% Ambientes de teorema, nota, etc.
\newtheorem{theorem}{Teorema}[section] % Vinculado a la sección
\newtheorem{proposition}{Proposición}[section] % Vinculado a la sección
\newtheorem{note}{Nota}[section] % Vinculado a la sección
\newtheorem{example}{Ejemplo}[section] % Vinculado a la sección
\newtheorem{lemma}{Lema}[section] % Vinculado a la sección
\newtheorem{axiom}{Axioma}[section] % Vinculado a la sección
\newtheorem{definition}{Definición}[section] % Vinculado a la sección




%Cuadrito de Demostración
\newcommand{\heart}{\begin{tikzpicture}[scale=0.001cm,rotate=180]
\fill[black] (0,0) 
        .. controls (0,-0.5) and (0.3,-1.8) .. (2,-2)
        .. controls (4.2,-2) and (5.5,0) .. (5.5,3)
        .. controls (5.5,5.5) and (3.5,7.5) .. (0,10)
        .. controls (-3.5,7.5) and (-5.5,5.5) .. (-5.5,3)
        .. controls (-5.5,0) and (-4.2,-2) .. (-2,-2)
        .. controls (-0.3,-1.8) and (0,-0.5) .. (0,0);
\end{tikzpicture}}

\newcommand{\demostrado}[0]{ \begin{flushright} $\heart{}$ \end{flushright}}
