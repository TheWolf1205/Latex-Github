\begin{homeworkProblem}
  Calcular el argumento principal, y escribir en forma polar $z=\frac{(1-i\sqrt{3})^{7}}{(\sqrt{3}-i)^{3}}$.
  \begin{solution}
    Calculemos la forma polar de $z_1=1-i\sqrt{3}$:\\
    Note que:
    \begin{align*}
      r_1&=\sqrt{(1)^2+(-\sqrt{3})^2},\\
      &=\sqrt{1+3},\\
      &=2,
    \end{align*}
    por otro lado,
    \begin{align*}
      \theta_1&=arctan\left( \frac{-\sqrt{3}}{1} \right),\\
      &=-\frac{\pi}{3}.
    \end{align*}
    Razonando de forma similar para $z_2=\sqrt{3}-i$:
    \begin{align*}
      r_2&=\sqrt{(\sqrt{3})^2+(-1)^2},\\
      &=\sqrt{3+1},\\
      &=2,
    \end{align*}
    por otro lado,
    \begin{align*}
      \theta_2&=arctan\left( \frac{-1}{\sqrt{3}} \right),\\
      &=-\frac{\pi}{6}.
    \end{align*}
    Por lo que como $z=\frac{z_1^7}{z_2^3}$ podemos afirmar que:
    \begin{align*}
      z&=\frac{z_1^7}{z_2^3},\\
      &=\frac{(2e^{-i\frac{\pi}{3}})^{7}}{(2e^{-i\frac{\pi}{6}})^{3}},\\
      &=2^{4}e^{i\left( \frac{3\pi}{6}-\frac{7\pi}{3} \right)},\\
      &=16e^{i\frac{-11\pi}{6}},\\
      &=16e^{i\frac{\pi}{6}},
    \end{align*}
    por lo que se concluye que la forma polar de $z$ es $16e^{i\frac{\pi}{6}}$ y el argumento principal de $z$ es $Arg(z)=\frac{\pi}{6}$. 
  \end{solution}
\end{homeworkProblem}
