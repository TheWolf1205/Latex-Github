\begin{homeworkProblem}
  Sea $f(z)=\sum_{n=0}^{\infty}a_nz^n$, con radio de convergencia $r$. Probar:
  \begin{itemize}
    \item $g(z)=\sum_{n=1}^{\infty}na_nz^{n-1}$ tiene el mismo radio de convergencia de $f$.
      \begin{solution}
        Sabemos que:
        \begin{align*}
          \lim_{n \to \infty}|a_n|^{\frac{1}{n}}=\frac{1}{r}
        \end{align*}
        luego:
        \begin{align*}
          \lim_{n \to \infty}|na_n|^{\frac{1}{n}}&=\lim_{n \to \infty}|n|^{\frac{1}{n}}|a_n|^{\frac{1}{n}}\\
          &=(1)\left( \frac{1}{r} \right)\\
          &=\frac{1}{r}
        \end{align*}
        de lo que se puede concluir que el radio de convergencia de $g$ es $r$.
        \demostrado
      \end{solution}
    \item $f$ es holomorfa en $D(0,r)$ y $f'(z)=g(z)$.
      \begin{solution}
        Sea $|z|<r$ y $\delta>0$ tal que $|z|+\delta < r$. Sea $h\in\mathbb{C}$ tal que $|h|<\delta$, luego:
        \begin{align*}
          f(z+h)&=\sum_{n=0}^{\infty}a_n(z+h)^{n}\\
          &=\sum_{n=0}^{\infty}a_n(z^n+nz^{n-1}h+h^2p_n(z,h))
        \end{align*}
        donde $p_n(z,h)$ es un polinomio con coeficientes en los enteros.\\
        Note que:
        \begin{align*}
          \left| p_n(z,h) \right|&\leq\left| \sum_{k=2}^{n} \binom{n}{k}\delta^{k-2}z^{n-k}\right|\\
          &\leq \sum_{k=2}^{n}\binom{n}{k}\delta^{k-2}|z|^{n-k}\\
          &\leq p_n(|z|,\delta)
        \end{align*}
        Ahora:
        \begin{align*}
          f(z+h)-f(z)-\sum_{n=1}^{\infty}na_nz^{n-1}h=h^2\sum_{n=2}^{\infty}a_np_n(z,h)
        \end{align*}
        lo que implica:
        \begin{align*}
          \frac{f(z+h)-f(z)}{h}-\sum_{n=1}^{\infty}na_nz^{n-1}&=h\sum_{n=2}^{\infty}a_np_n(z,h)
        \end{align*}
        luego:
        \begin{align*}
          \lim_{h \to 0}\left| h\sum_{n=2}^{\infty}a_np_n(z,h) \right|&\leq \lim_{h \to 0}|h|\left| \sum_{n=2}^{\infty}a_np_n(z,h) \right|\\
          &\leq \lim_{h \to 0}|h|\sum_{n=2}^{\infty}|a_n|p_n(|z|,\delta)\\
          &\leq 0
        \end{align*}
        luego $f$ es diferenciable y su derivada es $g$.
        \demostrado
      \end{solution}
    \item Pruebe $a_n=\frac{f^{(n)}(0)}{n!}$.
      \begin{solution}
        Note que:
        \begin{align*}
          f^{(n)}(z)&=\sum_{k=n}^{\infty}\left(\prod_{i=k-n+1}^{k}i\right)a_k(z)^{k-n}\\
          &=n!a_n+\sum_{k=n+1}^{\infty}\left(\prod_{i=k-n+1}^{k}i\right)a_k(z)^{k-n}
        \end{align*}
        luego $f^{(n)}(0)=n!a_n$, lo que implica $a_n=\frac{f^{(n)}(0)}{n!}$.
        \demostrado
      \end{solution}
    \item Sea $h(z)=\sum_{n=0}^{\infty}\frac{a_n}{n+1}z^{n+1}$. Pruebe que tiene radio de convergencia $r$. (Note que $h'(z)=f(z)$, $h$ se le llama primitiva de $f$).
    \begin{solution}
      Note que:
      \begin{align*}
        \lim_{n \to \infty}\left| \frac{a_{n-1}}{n} \right|^{\frac{1}{n}}&\leq \lim_{n \to \infty}\frac{|a_{n-1}|^{\frac{1}{n}}}{|n|^{\frac{1}{n}}}\\
        &\leq \frac{\lim_{n \to \infty}|a_{n-1}|^{\frac{1}{n}}}{\lim_{n \to \infty}|n|^{\frac{1}{n}}}\\
        &\leq \frac{\frac{1}{r}}{1}\\
        &\leq \frac{1}{r}
      \end{align*}
      luego $h(z)$ tiene radio de convergencia $r$.
      \demostrado
    \end{solution}
  \end{itemize}
\end{homeworkProblem}
