\begin{homeworkProblem}
  Sea $f$ analítica sobre un abierto $U$, sea $z_0\in U$ un máximo para $Re(f)$  (parte real de $f$) ($Re(f(z_0)) > Re(f(z)$), para todo $z \in U$). Probar que $f$ es localmente constante en $z_0$.
  \begin{solution}
    Note que $e^{f(z)}$ es analítica en $U$, además $|e^{f(z)}|=e^{Re(f(z))}$, luego la existencia de un máximo para $Re(f)$ implica la existencia de un máximo para $e^{Re(f)}=|e^{f(z)}|$, que por el teorema anterior implicaría que si el máximo es $z_0$, entonces $e^{f(z)}$ es localmente constante en $z_0$ y por ende $f$ es localmente constante en $z_0$.
    \demostrado
  \end{solution}
\end{homeworkProblem}
\begin{homeworkProblem}
  Sea $f$ analítica sobre un abierto $U$, sea $z_0\in U$ un máximo para $Im(f)$  (parte imaginaria de $f$) ($Im(f(z_0)) > Im(f(z)$), para todo $z \in U$). Probar que $f$ es localmente constante en $z_0$.
  \begin{solution}
    Note que $e^{-if(z)}$ es analítica en $U$, además $|e^{-if(z)}|=e^{Im(f(z))}$, luego la existencia de un máximo para $Im(f)$ implica la existencia de un máximo para $e^{Im(f)}=|e^{-if(z)}|$, que por el teorema anterior implicaría que si el máximo es $z_0$, entonces $e^{-if(z)}$ es localmente constante en $z_0$ y por ende $f$ es localmente constante en $z_0$.
    \demostrado
  \end{solution}
\end{homeworkProblem}
