\begin{homeworkProblem}
  Considere $E=c_0$ donde
  \begin{align*}
    c_0=\{u=\{u_n\}_{n\geq 1}: \text{ tales que }u_{n}\in\mathbb{R},n\geq 1,\lim_{n \to \infty}u_{n}=0\}.
  \end{align*}
  Es decir, $c_0$ es el conjunto de las secuencias reales que tienden a $0$. Dotamos a este espacio con la norma $\norm{u}_{l^{\infty}}=\displaystyle \sup_{n\in\mathbb{Z}^{+}}|u_{n}|$. Considere el funcional $f:E\to \mathbb{R}$ dado por
  \begin{align*}
    f(u)=\sum_{n=1}^{\infty}\frac{1}{2^{n}}u_{n}.
  \end{align*}
  \begin{enumerate}[(i)]
    \item Muestre que $f\in E^{*}$ y calcule $\norm{f}_{E^{*}}$.
    \item ¿Es posible encontrar $u\in E$ tal que $\norm{u}=1$ y $f(u)=\norm{f}_{E^{*}}$? 
  \end{enumerate}
  \begin{solution}
    \begin{enumerate}[(i)]
      \item Veamos que $f\in E^{*}$ es decir, que $f$ es una transformación lineal de $E$ en $\mathbb{R}$.\\
        Antes de empezar, note que como la serie $\sum_{n=1}^{\infty}\frac{1}{2^{n}}$ es absolutamente convergente y $u_n$ es una sucesión que tiende a $0$, entonces la serie determinada por $f$ es absolutamente convergente y por ende permite reordenamientos, siendo así, continuemos.\\
        Dadas $u,v\in E$ y $c\in \mathbb{R}$ note que
        \begin{align*}
          f(cu+v)&=\sum_{n=1}^{\infty}\frac{1}{2^{n}}\left( cu_n+v_n \right),\\
          &= c\sum_{n=1}^{\infty}\frac{1}{2^{n}}u_{n}+\sum_{n=1}^{\infty}\frac{1}{2^n}v_n,\\
          &=cf(u)+f(v).
        \end{align*}
        lo que nos permite concluir que $f$ es una transformación lineal en $E^*$.\\
        Ahora para calcular $\norm{f}_{E^{*}}$ nos será de gran utilidad pensar en los $u\in E$ tales que $\norm{u}_{E}=1$, ya que si bien le estamos pidiendo converger a $0$, no estamos exigiendo que sea de alguna forma especifica si no en el infinito, por lo que sabremos que el supremo que estamos buscando se encontrará justamente en la sucesión constante $1$ (esto ya que podemos pedirle a la sucesión ser $1$ hasta un punto arbitrario y luego si decaimiento a $0$), es por esto que 
        \begin{align*}
          \norm{f}_{E^{*}}&=\sup_{\substack{u\in E\\ \norm{u}_{E}=1}}f(u),\\
          &=\sup_{\substack{u\in E\\\norm{u}_{E}=1}}\sum_{n=1}^{\infty}\frac{1}{2^n}u_n,\\
          &=\sum_{n=1}^{\infty}\frac{1}{2^n},\\
          &=\sum_{n=0}^{\infty}\frac{1}{2^{n}} - 1,\\
          &=\frac{1}{1-\frac{1}{2}}-1,\\
          &=2-1,\\
          &=1.
        \end{align*}
        lo que concluye el numeral.
      \item \textbf{No}, note que esto es claro por el comentario que realizamos para calcular la norma de $f$, puesto que de lo contrario ese supremo realmente sería un máximo, es por esto que si existiera $u$ de norma $1$ con tendencia a $0$ tal que $\norm{f}_{E^*}$ fuera exactamente $f(u)$, se cumpliría que
        \begin{align*}
          \norm{f}_{E^{*}}&=f(u),\\
          &=\sum_{n=1}^{\infty}\frac{1}{2^n}u_n,\\
          &<\sum_{n=1}^{\infty}\frac{1}{2^n},\\
          &<\sum_{n=0}^{\infty}\frac{1}{2^{n}} - 1,\\
          &<\frac{1}{1-\frac{1}{2}}-1,\\
          &<2-1,\\
          &<1.
        \end{align*}
      entonces $f(u)=\norm{f}_{E^{*}}<1$, lo que nos lleva a una contradicción de la forma $1<1$, puesto que ya verificamos que $\norm{f}_{E^{*}}=1$.   
    \end{enumerate}
  \end{solution}
\end{homeworkProblem}
