\begin{homeworkProblem}
  Demuestre que si $T\in L(E,F)$\footnote{Recuerde que $L(E,F)$ denota el conjunto de operadores lineales de $E$ en $F$. Dado $T\in L(E,F)$ definimos la norma de $T$ como $\displaystyle \norm{T} = \sup_{\substack{x \in E \\ \norm{x}_E \leq 1}} \norm{Tx}_F$.}, entonces
  \begin{enumerate}[(i)]
    \item $\norm{Tx}_{F}\leq \norm{T}\norm{x}_{E}$, para todo $x\in E$.
    \item $\displaystyle \norm{T} = \sup_{\substack{x \in E \\ x\neq 0}} \frac{\norm{Tx}_{f}}{\norm{x}_{E}}$.
    \item $\displaystyle \norm{T} = \sup_{\substack{x \in E \\ \norm{x}_E = 1}} \norm{Tx}_F$.
    \item $\norm{T}=\inf\{M>0:\norm{Tx}_{F}\leq M\norm{x}_{E}, \text{ para todo }x\in E\}$.
  \end{enumerate}
  \begin{solution}
    \begin{enumerate}[(i)]
      \item Note que como $T$ es un operador lineal, podemos obviar el caso en el que $x=0$, pues $\norm{Tx}_{F}=0\leq 0=\norm{T}\norm{x}_{E}$.\\
        Ahora, con el fin de simplificar la idea, si tomamos $x\neq 0$, entonces podemos reescribir $y=\frac{x}{\norm{x}_{E}}$.\\
        Siendo así, note que dado $y$ por propiedades del supremo se satisface que
        \begin{align*}
          \norm{Ty}_{F}&\leq \sup_{\substack{y\in E\\ \norm{y}_{E}\leq 1}}\norm{Ty}_{F} &&\text{Reescribiendo la norma y multiplicando a la derecha por $\norm{y}_{E}=1$,}\\
          \norm{Ty}_{F}&\leq \norm{T}\norm{y}_{E} &&\text{Reescribiendo $y=\frac{x}{\norm{x}_{E}}$ y usando la sublinealidad de la norma y el operador,}\\
          \frac{1}{\norm{x}_{E}}\norm{Tx}_{E}&\leq \frac{1}{\norm{x}_{E}}\norm{T}\norm{x}_{E},
        \end{align*}
        Lo que implica que $\norm{Tx}_{F}\leq \norm{T}\norm{x}_{E}$, luego como se toma $y$ arbitrario se extiende el resultado a todo $x\in E$ y por ende se concluye el resultado.
      \item Note que por la sublinealidad de la norma y el operador podemos asegurar que
        \begin{align*}
          \sup_{\substack{x\in E\\x\neq 0}}\frac{\norm{Tx}_{F}}{\norm{x}_{E}}&=\sup_{\substack{x\in E\\x\neq 0}}\left\|T\frac{x}{\norm{x}_{E}}\right\|_{F},\\
          &=\sup_{\substack{x\in E\\x\neq 0}}\norm{Ty}_{F} &&\text{como $y$ es unitario y distinto de $0$},\\
          &\leq \sup_{\substack{x\in E\\\norm{x}\leq 1}}\norm{Tx}_{F},\\
          &\leq \norm{T}.
        \end{align*}
        Por otro lado veamos que si asumimos que $\norm{x}\leq 1$, entonces
        \begin{align*}
          \norm{T}&= \sup_{\substack{x\in E\\\norm{x}_{E}\leq 1}}\norm{Tx}_{F},\\
          &\leq \sup_{\substack{x\in E\\ \norm{x}_{E}\leq 1}}\frac{\norm{Tx}_{F}}{\norm{x}_{E}},\\
          &\leq \sup_{\substack{x\in E\\x\neq 0}}\frac{\norm{Tx}_{F}}{\norm{x}_{E}}.
        \end{align*}
        Ya que $\{x\in E:\norm{x}_{E}\leq 1\}\subset E$ y omitimos el caso en el que $x=0$ ya que $Tx=0$ y por ende no es el supremo del conjunto a menos de que $T$ sea el operador nulo.
      \item Note que si usamos la sublinealidad del operador y de la norma podemos ver que
        \begin{align*}
          \sup_{\substack{x\in E\\x\neq 0}}\frac{\norm{Tx}_{F}}{\norm{x}_{E}}=\sup_{\substack{x\in E\\\norm{x}=1}}\norm{Tx}_{F}
        \end{align*}
        luego usando (II) podemos afirmar que
        \begin{align*}
          \norm{T}=\sup_{\substack{x\in E\\\norm{x}=1}}\norm{Tx}_{F}.
        \end{align*}
      \item Note que el conjunto de los $M$ que satisfacen la condición del conjunto no son afectados cuando se divide por la norma de $x$ en ambos lados de la desigualdad, es decir, podemos suponer que los $x$ dados en la condición del conjunto son unitarios. Luego la condición se transforma en ver el menor de los $M>0$ que satisface $\norm{Tx}_{F}\leq M$ para todo $x\in E$ que satisface $\norm{x}_{E}=1$, luego por (III) podemos afirmar que este $M$ es precisamente $\norm{T}$. 
    \end{enumerate}
  \end{solution}
\end{homeworkProblem}
