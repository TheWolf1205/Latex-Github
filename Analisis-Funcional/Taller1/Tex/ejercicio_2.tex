\begin{homeworkProblem}
  Sea $(E,\norm{\cdot}_{E})$, $(F,\norm{\cdot}_{F})$ espacios vectoriales normados. Considere $T:E\to F$ una transformación lineal. Muestre que las siguientes afirmaciones son equivalentes:
  \begin{enumerate}[(i)]
    \item $T$ es continua.
    \item $T$ es continua en cero.
    \item $T$ es acotada. Es decir, existe $M>0$ tal que para todo $x\in E$,
      \begin{align*}
        \norm{Tx}_{F}\leq M\norm{x}_{E}.
      \end{align*}
    \item Si $\overline{B(0,1)}=\{x\in E:\norm{x}\leq 1\}$, entonces la imagen directa $T\left( \overline{B(0,1)} \right)$ es un conjunto acotado de $F$.
  \end{enumerate}
  \begin{solution}
    \begin{itemize}
      \item (I)$\to$(II).\\
        Este sentido es trivial.
      \item (II)$\to$(III).\\
        Note que si tomamos $\epsilon=1$ tenemos que existe un $\delta>0$ tal que si $\norm{x}_{E}<\delta$, entonces $\norm{Tx}_{F}<1$.\\
        Ahora, dado $x\in E$ arbitrario suponga $y=\frac{\delta}{2\norm{x}}x$, note que $\norm{y}<\delta$ y por ende $\norm{Ty}_{F}<1$, luego se sigue que
        \begin{align*}
          \frac{\delta}{2\norm{x}_{E}}\norm{Tx}_{F}<1,
        \end{align*}
        lo que implica que
        \begin{align*}
          \norm{Tx}_{F}<M\norm{x}_{E}
        \end{align*}
        En dónde $M$ es una constante tal que $M\leq\frac{2}{\delta}$, lo que concluye en que el operador $T$ es acotado.
      \item (III)$\to$(IV).\\
        Note que como $T$ es un operador acotado, significa que para todo $x\in \overline{B(0,1)}$ se cumple que $\norm{Tx}_{F}\leq M$, luego podemos asegurar que $T\left( \overline{B(0,1)} \right)\subseteq B_{F}(0,M)$, lo que concluye el resultado esperado.
      \item (IV)$\to$(I).\\
        Note que (IV) nos dice que la imagen directa $T\left( \overline{B(0,1)} \right)\subseteq B_{F}(0,M)$, en particular $T\left( B(0,1) \right)\subseteq B_{F}(0,M)$, es decir, que para todo $x\in B(0,1)$ se satisface que $\norm{Tx}_{F}\leq M$, ahora, veamos que dados $u,v\in E$ y $\epsilon>0$ existe $\delta=\frac{\epsilon}{M}>0$ tal que si:
        \begin{align*}
          \norm{u-v}_{E}<\delta
        \end{align*}
        entonces
        \begin{align*}
          \norm{\frac{1}{\delta}u-v}_{E}<1
        \end{align*}
        luego
        \begin{align*}
          \norm{T\left( \frac{1}{\delta}u-v \right)}_{F}<M
        \end{align*}
        lo que implica que
        \begin{align*}
          \norm{Tu-Tv}_{F}<M\delta=\epsilon.
        \end{align*}
        Lo que nos permite concluir que $T$ es un operador continuo. 
    \end{itemize}
  \end{solution}
\end{homeworkProblem}
