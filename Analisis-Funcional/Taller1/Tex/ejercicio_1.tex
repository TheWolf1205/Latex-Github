\begin{homeworkProblem}
  Sea $(E,\norm{\cdot})$ un espacio vectorial normado. Defina
  \begin{align*}
    \mathcal{K}=\{x\in E:\norm{x}=1\}.
  \end{align*}
  Demuestre que $E$ es de Banach si y solamente si $\mathcal{K}$ es completo. 
  \begin{solution}
    Supongamos que $E$ es de Banach y veamos que $\mathcal{K}$ es completo.\\
    Razonemos por contradicción.\\
    Suponga $\{x_n\}\subset \mathcal{K}$ sucesión de Cauchy que converge a $x\notin \mathcal{K}$ cuando $n\to\infty$, es decir, $\norm{x}\neq 1$.\\
    Primero, note que como $\{x_n\}$ es una sucesión de Cauchy, dado $\epsilon>0$ existe $N>0$ tal que si $m,n>N$, entonces 
    \begin{align*}
      \norm{x_m-x_n}<\epsilon.
    \end{align*}
    Ahora note que si $m\to\infty$ se satisface que 
    \begin{align*}
      \left| \norm{x}-1 \right|&\leq\left| \norm{x}-\norm{x_n} \right|,\\
      &\leq\norm{x-x_n},\\
      &< \epsilon.
    \end{align*}
    Luego como $\epsilon$ es arbitrario sabemos que $\norm{x}=1$, contradicción, pues desde un principio se asumió que $\norm{x}\neq 1$, luego podemos concluir que $\matcal{K}$ es completo. 
    Por otro lado, supongamos que $\mathcal{K}$ es completo y veamos que esto implica que $E$ es de Banach.\\
    Primero, recuerde que $0\in E$, por lo que si tomamos $\{x_k\}\subset E$ sucesión de Cauchy obviaremos el caso en el que esta converge a $0$, ya que si esta converge a $0$ estaría convergiendo en el espacio.\\ 
    Suponga $\{x_n\}\subset E$ sucesión de Cauchy, entonces se tiene que dado $\epsilon>0$ existe $N>0$ tal que si $n,m>N$ entonces
    \begin{align*}
      \norm{x_n-x_m}<\epsilon.
    \end{align*}
    Note que $\{\norm{x_n}\}\subset \mathbb{R}$ es una sucesión de Cauchy, ya que
    \begin{align*}
      \left| \norm{x_n}-\norm{x_m} \right|&\leq \norm{x_n-x_m},\\
      &\leq \epsilon.
    \end{align*}
    Por lo que como $\mathbb{R}$ es completo, entonces sabemos que $\norm{x_n}\to l\in\mathbb{R}\setminus\{0\}$ cuando $n\to \infty$.\\
    Ahora suponga $\{y_n\}=\left\{\frac{x_n}{\norm{x_n}}\right\}\subset\mathcal{K}$ y note que como $\mathcal{K}$ es completo, entonces existe $y\in \mathcal{K}$ tal que $y_n\to y\in \mathcal{K}$ cuando $n\to\infty$, luego tenemos que dado $\epsilon>0$ existe $N>0$ tal que si $n>N$, entonces
    \begin{align*}
      \norm{y-y_n}<\epsilon,
    \end{align*}
    es decir
    \begin{align*}
      \norm{y-\frac{x_n}{\norm{x_n}}}&<\epsilon &&\text{Multiplicando por $\norm{x_n}$},\\
      \norm{\norm{x_n}y-x_n}&<\norm{x_n}\epsilon,
    \end{align*}
    Ahora si tomamos $n\to\infty$
    \begin{align*}
      \norm{ly-x}<l\epsilon,
    \end{align*}
    Luego como $\epsilon$ es arbitrario, entonces sabemos que $x_n\to x=ly$ cuando $n\to\infty$, de lo que se puede concluir que $E$ es un espacio de Banach. 
  \end{solution}
\end{homeworkProblem}
