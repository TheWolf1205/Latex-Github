\begin{homeworkProblem}
  Sean $(E,\norm{\cdot}_{E})$, $(F,\norm{\cdot}_{F})$ espacios vectoriales normales. Suponga que $F$ es un espacio de Banach. Muestre que $L(E,F)$ es un espacio de Banach con la norma usual de $L(E,F)$. En particular, $E^{*}=L(E,\mathbb{R}),E^{**}=L(E^{*},\mathbb{R})$ son espacios de Banach. 
  \begin{solution}
    Dado $T\in L(E,F)$ definimos la norma de $L(E,F)$ como
    \begin{align*}
      \norm{T}&=\sup_{\substack{x\in E\\\norm{x}_{E}=1}}\norm{Tx}_{F}.
    \end{align*}
    Suponga $\{T_n\}\subset L(E,F)$ sucesión de Cauchy y veamos que esta converge a $T\in L(E,F)$.\\
    Note que como $\{T_n\}$ es sucesión de Cauchy, entonces se cumple que dado $\epsilon>0$ existe $N>0$ tal que si $n,m>N$, entonces
    \begin{align*}
      \norm{T_n-T_m}<\epsilon
    \end{align*}
    Pero note que dado $x\in E$ (distinto del nulo) podemos tomar $\epsilon$ de la forma $\frac{\epsilon}{\norm{x}_{E}}>0$ que nos permite afirmar que 
    \begin{align*}
      \norm{T_nx-T_mx}_{F}&\leq \norm{T_n-T_m}\norm{x}_{E},\\
      &<\frac{\epsilon}{\norm{x}_{E}}\norm{x}_{E},\\
      &<\epsilon.
    \end{align*}
    Luego $\{T_nx\}\subset F$ es una sucesión de Cauchy, luego como $F$ es Banach, podemos afirmar que $T_{n}x\to g_x\in F$ cuando $n\to \infty$.\\
    Siendo así, dado $x$ podemos definir un $g_x$ de la forma anterior, por lo que vamos a definir $T:E\to F$ como $Tx=g_x$, luego podemos afirmar que $T_n\to T$ puntualmente cuando $n\to \infty$.\\
    Ahora veamos que la convergencia realmente es uniforme, es decir que $\norm{T-T_n}\to 0$ cuando $n\to \infty$.\\
    Para eso, tomamos un $\epsilon > 0$, y como $\{T_n\}$ es Cauchy en $L(E,F)$, existe $N$ tal que para todo $n, m > N$,
    $$\|T_n - T_m\| < \frac{\epsilon}{2}.$$
    En particular, fijando $m$ y tomando el límite cuando $n \to \infty$, usando la convergencia puntual $T_n \to T$, podemos aplicar lo siguiente:\\
    Sea $x \in E$ con $\|x\|_E \leq 1$, entonces
    $$\|T_n x - T x\|_F = \lim_{m \to \infty} \|T_n x - T_m x\|_F \leq \limsup_{m \to \infty} \|T_n - T_m\| \cdot \|x\|_E < \frac{\epsilon}{2}.$$
    Tomando el supremo sobre todas las $x$ con $\|x\|_E \leq 1$, se obtiene:
    $$\|T_n - T\| = \sup_{\|x\|_E \leq 1} \|T_n x - T x\|_F < \frac{\epsilon}{2},$$
    para todo $n$ suficientemente grande. Por tanto,
    $$\|T_n - T\| \to 0,$$
    lo cual muestra que la convergencia es uniforme.\\
    Ahora, veamos que $T\in L(E,F)$.\\
    Sea $\alpha$ un escalar y $x,y\in E$, entonces 
    \begin{align*}
      T(\alpha x+y)&=\lim_{n \to \infty}T_{n}(\alpha x+y),\\
      &=\lim_{n \to \infty}\alpha T_{n}(x)+T_{n}(y),\\
      &=\alpha \lim_{n \to \infty}T_{n}(x)+\lim_{n \to \infty}T_{n}(y),\\
      &=\alpha Tx + Ty.
    \end{align*}
    Luego $T\in L(E,F)$ lo que concluye el resultado esperado. 
  \end{solution}
\end{homeworkProblem}
