\thispagestyle{empty}

\begin{minipage}{0.3\textwidth}
  \includegraphics[scale=0.35]{logounal.png}
\end{minipage}%
\hfill
\begin{minipage}{0.65\textwidth}
  \begin{center}
    \scshape
    \Large \textsc{Universidad Nacional de Colombia} \\
    \textcolor{white}{\tiny.} \Large \textsc{Departamento de Matemáticas} \\
    \textcolor{white}{\tiny.} \large \textsc{Análisis Funcional} \\
    \textcolor{white}{\tiny.} \large \textsf{Taller 1: Espacios vectoriales normados} \normalsize (I-2025)
  \end{center}
\end{minipage}

\vspace{0.3cm}
\normalfont

\textbf{Profesor:} Oscar Guillermo Riaño Castañeda\\
\textbf{Integrantes:} Andrés David Cadena Simons \hspace{2.8cm}  Jairo Sebastián Niño Castro\hspace{2.8cm}
Iván Felipe Salamanca Medina\\
\hspace*{2.1cm}\hspace{2.25cm}\textbf{Fecha:} 03 de Junio del 2025\\
\vspace{0.25cm}\\


\textbf{Ejercicio 1}.

\begin{enumerate}
    \item[(I)] Sea $\mathbb{R}$ con la $\sigma$-álgebra de Borel.
    \begin{enumerate}
        \item[(a)] Dado $x_0\in \mathbb{R}$, considere $\delta_{x_0}$ la medida de Dirac centrada en $x_0$ dada por: $\delta_{x_0}(A)=1$ si $x_0\in A$ y $\delta_{x_0}(A)=0$ si $x_0\notin A$ para cada $A\in \mathcal{B}(\mathbb{R})$. Muestre que $x_0$ es una medida.
        \item[(b)] Sea $f:\mathbb{R}\to \mathbb{R}$ una función medible. Muestre que 
        \begin{align*}
            \int_{\mathbb{R}}f(x)\delta_{x_0}=f(x_0).
        \end{align*}
        \item[(c)] De un ejemplo de una función que sea integrable con la medida $\delta_{x_0}$ para algún $x_0\in \mathbb{R}$ pero que no sea integrable con la medida de Lebesgue.
    \end{enumerate}
    \item[(II)] Sea $\mathbb{N}=\{1,2,3,...\}$ con la $\sigma$-álgebra $\mathcal{P}(\mathbb{N})$.
    \begin{enumerate}
        \item[(a)] Considere la medida contadora $\mu$ dada por $\mu(A):=\text{cardinal}(A)$ si $A$ es finito y $\mu(A)=\infty$ si $A$ es infinito, para cada $A \in \mathcal{P}(\mathbb{N})$. Muestre que $\mu$ es una medida.
        \item[(b)] Dada $f:\mathbb{N}\to \mathbb{R}$ una función medible, es decir, $f$ es una sucesión $f=\{a_j\}_{j\in \mathbb{N}}$ para algunos $a_j\in \mathbb{R}$. Muestre que si $f$ es integrable (es decir, $\displaystyle \int_{\mathbb{N}}|f|\, d\mu<\infty$), entonces
        \begin{align*}
            \int_{\mathbb{N}}f\, dx=\sum_{j=1}^\infty a_j.
        \end{align*}
    \end{enumerate}
\end{enumerate}

\begin{proof}
  \begin{enumerate}
    \item[(a)] Veamos que $\delta_{x_0}$ es una medida.\\
      Note que $\delta_{x_0}(\emptyset)=0$ ya que $x_0\notin \emptyset$ por definición del conjunto $\emptyset$.\\
      Por otro lado, veamos que si tomamos una unión numerable de conjuntos disjuntos y le calculamos su medida, esto va a ser igual que la suma de la medida de cada conjunto, es decir, dados $\{A_{n}\}_{n\in\mathbb{N}}$ conjuntos disjuntos se satisface que
      \begin{align*}
        \delta_{x_0}\left( \bigcup_{n\in\mathbb{N}}A_{n} \right)&=\sum_{n\in\mathbb{N}}\delta_{x_0}\left( A_{n} \right), &&\text{para toda familia numerable de conjuntos disjuntos.}
      \end{align*}
      Esto ya que si $\delta_{x_0}\left( \bigcup_{n\in\mathbb{N}}(A_{n}) \right)=0$, significa que $x_0\notin \bigcup_{n\in\mathbb{N}}(A_{n})$, en particular, $x_0\notin A_{n}$ para todo $n\in\mathbb{N}$, luego es válido afirmar que
      \begin{align*}
        \delta_{x_0}\left( \bigcup_{n\in\mathbb{N}}(A_{n}) \right)&=0,\\
        &=\sum_{n\in\mathbb{N}}\delta_{x_0}(A_{n}).
      \end{align*}
      Por otra parte, si $\delta_{x_0}\left( \bigcup_{n\in\mathbb{N}}(A_{n}) \right)=1$, entonces $x_0\in \bigcup_{n\in\mathbb{N}}(A_{n})$, pero como la familia de conjuntos $\{A_{n}\}$ es disjunta $2$ a $2$, entonces podemos afirmar que existe un único $N\in\mathbb{N}$ tal que $x_0\in A_N$ y $x_0\notin n$ para todo $n\neq N$, es decir que $\delta_{x_0}(A_{N})=1$ y $\delta_{x_0}(A_{n})=0$ para todo $n\neq N$, luego
      \begin{align*}
        \delta_{x_0}\left( \bigcup_{n\in\mathbb{N}}(A_{n}) \right)&=1,\\
        &=\delta_{x_0}(A_{N}),\\
        &=\sum_{n\in N}\delta_{x_0}(A_{n}).
      \end{align*}
      lo que nos permite afirmar que $\delta_{x_0}:\mathbb{R}\to[0,+\infty]$ es una medida.
    \item[(b)] Veamos que dada $f:\mathbb{R}\to\mathbb{R}$ función medible se cumple que
      \begin{align*}
        \int_{\mathbb{R}} f(x) d \delta_{x_0}=f\left(x_0\right).
      \end{align*}
      Usaremos funciones simples para demostrar el resultado para funciones simples positivas, luego usaremos la densidad de las funciones simples positivas en las funciones medibles no negativas para extender este resultado a las funciones medibles no negativas y por último esto nos servirá para concluir el resultado a cualquier función medible.\\
      Sea $f(x) = \sum_{i=1}^n a_i \chi_{A_i}(x)$, donde $a_i>0$ para todo $i\in\mathbb{N}$, además 
      \begin{align*}
        \chi_{A_i}(x) = \begin{cases} 
          1, & \text{si } x \in A_i,\\
          0, & \text{si } x \notin A_i
        \end{cases}
      \end{align*}
      para todo $i = 1,\cdots,n$. Además, podemos suponer que los conjuntos sobre los cuales se definen las funciones características son disjuntos $2$ a $2$, es decir, $A_i \cap A_j = \emptyset$ para todo $i \neq j$.\\
      Note que,
      \begin{align*}
        \int_{\mathbb{R}} f(x) \, d\delta_{x_0} &= \int_{\mathbb{R}} \left( \sum_{i=1}^n a_i \chi_{A_i}(x) \right) d\delta_{x_0},\\ 
        &= \sum_{i=1}^n a_i \delta_{x_0}(A_i),
      \end{align*}
      no obstante, como en un principio asumimos que la familia numerable de conjuntos $\{A_i\}$ son disyuntos, entonces $x_0$ solo puede pertenecer a uno de los $A_i$, sin pérdida de generalidad suponga que $x_0\in A_{I}$ (caso contrario $x\notin \bigcup_{i=1}^{n}(A_i)$ y por ende $f(x_0)=0$ lo que concluye el resultado), entonces
      \begin{align*}
        \int_{\mathbb{R}} f(x) \, d\delta_{x_0} &= \sum_{i=1}^n a_i \delta_{x_0}(A_i),\\
        &= a_I,\\
        &= f(x_0).
      \end{align*}
      Lo que nos permite afirmar el resultado para funciones simples positivas, ahora veamos que esto se repite para funciones simples no negativas.\\
      Tomemos $f:\mathbb{R}\to[0, \infty]$ una función medible no negativa.\\ Entonces, como las funciones simples no negativas se pueden aproximar por funciones simples positivas, sabemos que existe una sucesión monótona de funciones $\{f_n\}_{n\in\mathbb{N}}$ simples positivas tales que
      \begin{align*}
        0\leq f_{1}(x)\leq\cdots\leq f_n(x)\leq f_{n+1}(x)\leq \cdots \leq f(x) \text{ y }  \lim_{n \to \infty} f_n(x) = f(x),  
      \end{align*}
      para todo $x \in \mathbb{R}$. Luego como cada $f_{n}$ es simple positiva, es medible y por ende usando el teorema de la convergencia monótona podemos afirmar que
      \begin{align*}
        \int_{\mathbb{R}} f(x) \, d\delta_{x_0} &= \lim_{n \to \infty} \int_{\mathbb{R}} f_n(x) \, d\delta_{x_0},\\
        &= \lim_{n \to \infty} f_n(x_0),\\
        &= f(x_0),       
      \end{align*}
      por lo que podemos afirmar el resultado para funciones medibles no negativas.\\
      Ahora usemos que las funciones medibles se pueden reescribir como suma de funciones medibles no negativas, sea $f: \mathbb{R} \to \mathbb{R}$ una función medible, vamos a particionar a $f$ como su parte no negativa ($f^{+}$) y su parte negativa $(f^{-})$ de forma que $f = f^+ - f^-$. Luego,
      \begin{align*}
        \int_{\mathbb{R}} f(x) \, d\delta_{x_0} &= \int_{\mathbb{R}} f^+(x) \, d\delta_{x_0} + \int_{\mathbb{R}} -f^-(x) \, d\delta_{x_0}.       
      \end{align*}
      Pero como el resultado vale para funciones medibles no negativas se puede afirmar que
      \begin{align*}
        \int_{\mathbb{R}} f(x) \, d\delta_{x_0} &= f^+(x_0) + -f^-(x_0),\\
        &= f(x_0).       
      \end{align*}
      Lo que concluye el teorema.
     \item[(c)] Sea $f(x)=|x|$ y $x_0=0$.\\
      Sabemos que
      \begin{align*}
        \int_{\mathbb{R}}|x|\, dx=\infty.
      \end{align*}
      pero note que
      \begin{align*}
        \int_{\mathbb{R}}|x|\, d\delta_{0}=|0|=0.
      \end{align*}
      por lo cual podemos afirmar que $|x|$ no es integrable respecto a Lebesgue, pero si respecto a la medida de Dirac centrada en $0$. 
  \end{enumerate} 
\end{proof}


%%%%%%%%%%%%%%%%%%%%%%%%%%%%%%%%%%%%%%%%%%%%%%%%%%%%%%%%%%%%


\textbf{Ejercicio 3.} Sea $\Omega\subseteq \mathbb{R}^n$ abierto. Sea $1\leq p\leq \infty$. Entonces $L^p(\Omega)$ es un espacio de Banach.

\begin{proof}
    Vamos a seguir la prueba como en el libro de Haim Brezis. Veamos dos casos
    \begin{itemize}
        \item \textbf{Caso 1:} Si $p=\infty$, sea $(f_j)$ una sucesión de Cauchy en $L^\infty(\Omega)$. Por definición, para todo $\epsilon>0$ existe $N\in \mathbb{Z}^+$ y un conjunto $E_\epsilon$ con $\mu(E_\epsilon)=0$ ($\mu$ es la medida de Lebesgue), tal que si $j,m\geq N$, entonces 
        \begin{align*}
            \norm{f_j-f_m}_\infty=\sup_{x\in \Omega\setminus E_\epsilon}|f_j(x)-f_m(x)|<\epsilon.
        \end{align*}
        En particular, para todo $k \in \mathbb{Z}^+$ existe $N_k$ y $E_k$ con $\mu(E_k)=0$ tal que si $j,m\geq N_k$, entonces
        \begin{align*}
            \norm{f_j-f_m}_\infty=\sup_{x \in\Omega\setminus E_k}|f_j(x)-f_m(x)|<\dfrac{1}{k}.
        \end{align*}
        Sea $\displaystyle E=\bigcup_{k=1}^\infty E_k$, por tanto
        \begin{align*}
            \mu(E)\leq \sum_{k=1}^\infty\mu(E_k)=0,
        \end{align*}
        por tanto $\mu(E)=0$ y, además, la sucesión $(f_j(x))$ es una sucesión de Cauchy en $\mathbb{R}$ para todo $x\in \Omega\setminus E$, de manera que $f_j(x)\to f(x)$, con $f(x)\in \mathbb{R}$ para todo $x \in \Omega\setminus E$. Sea $x \in \Omega\setminus E$ cualquiera y $j\geq N_{2k}$. Sea $m\geq j$, suficientemente grande para que $|f(x)-f_m(x)|<\dfrac{1}{2k}$, entonces
        \begin{align*}
            |f(x)-f_j(x)|\leq |f(x)-f_m(x)|+|f_m(x)-f_j(x)|<\dfrac{1}{2k}+\dfrac{1}{2k}=\dfrac{1}{k},
        \end{align*}
        como $x$ es arbitrario, concluimos que $|f(x)-f_j(x)|<\dfrac{1}{k}$, para todo $x \in \Omega\setminus E$, y como $\mu(E)=0$, concluimos que $\norm{f-f_j}_\infty<\dfrac{1}{k}$ para $j\geq N_k$, es decir, $f_j\to f$ en $L^\infty(\Omega)$.

        \item \textbf{Caso 2:} Si $1\leq p<\infty$, sea $(f_j)$ una sucesión de Cauchy en $L^p(\Omega)$. Para todo $\epsilon>0$ existe $N_\epsilon>0$ tal que si $j,m\geq N_\epsilon$, entonces $\norm{f_j-f_m}_p<\epsilon$. Para $\epsilon=\dfrac{1}{2}$, existe $j_1\in \mathbb{Z}^+$ tal que si $m\geq j_1$, entonces
        \begin{align*}
            \norm{f_{j_1}-f_m}_p<\dfrac{1}{2}.
        \end{align*}
        Ahora, para $\epsilon=\dfrac{1}{4}$, existe $j_2\in \mathbb{Z}^+$, con $j_2>j_1$ tal que si $m\geq j_2$, entonces
        \begin{align*}
            \norm{f_{j_2}-f_m}_p\leq \dfrac{1}{4},
        \end{align*}
        en particular, $\norm{f_{j_1}-f_{j_2}}_p<\dfrac{1}{2}$. Análogamente, para $\epsilon=\dfrac{1}{8}$, existe $j_3\in \mathbb{Z}^+$ con $j_3>j_2$ tal que si $m\geq j_3$, entonces
        \begin{align*}
            \norm{f_{j_3}-f_m}_p<\dfrac{1}{8},
        \end{align*}
        en particular $\norm{f_{j_2}-f_{j_3}}<\dfrac{1}{4}$. Realizando un argumento inductivo, encontramos que para todo $k\in \mathbb{Z}^+$, existe una sucesión creciente de enteros positivos $(j_k)$, tal que la subsucesión $(f_{j_k})$ cumple que
        \begin{align*}
            \norm{f_{j_k}-f_{j_{k+1}}}_p<\dfrac{1}{2^k}.
        \end{align*}
        Veamos que la subsucesión $(f_{j_k})$ converge en $L^p(\Omega)$. Dado $m \in \mathbb{Z}^+$ definimos
        \begin{align*}
            g_m(x)=\sum_{k=1}^m |f_{j_{k+1}}(x)-f_{j_k}(x)|,
        \end{align*}
        entonces
        \begin{align*}
            \norm{g_m}_p=\norm{\sum_{k=1}^m |f_{j_{k+1}}-f_{j_k}|}_p\leq \sum_{k=1}^m\norm{f_{j_{k+1}}-f_{j_k}}_p\leq \sum_{k=1}^m\dfrac{1}{2^k}\leq \sum_{k=1}^\infty\dfrac{1}{2^k}=1,
        \end{align*}
        además, si $m_1\leq m_2$, entonces
        \begin{align*}
            g_{m_1}(x)=\sum_{k=1}^{m_1} |f_{j_{k+1}}(x)-f_{j_k}(x)|=\sum_{k=1}^{m_2} |f_{j_{k+1}}(x)-f_{j_k}(x)|=g_{m_2}(x),
        \end{align*}
        es decir, $g_{m_1}$ es una sucesión monótona creciente. Así, por el Teorema de la Convergencia Monótona, tenemos que 
        \begin{align*}
            1=1^p\geq \lim_{m\to \infty}\norm{g_m}_p^p=\lim_{m\to \infty}\int_{\Omega}|g_m(x)|^p\, dx=\int_{\Omega}\lim_{m\to \infty}|g_m(x)|^p\, dx,
        \end{align*}
        por tanto, $\displaystyle \lim_{m\to \infty}|g_m(x)|<\infty$ para casi todo $x \in \Omega$, es decir, $\displaystyle \lim_{m\to \infty}|g_m(x)|<\infty$ para todo $x\in \Omega\setminus E$, donde $\mu(E)=0$, definimos
        \begin{align*}
            g(x)=\begin{cases}
                \displaystyle \lim_{m\to \infty}g_m(x), \hspace{3mm} &\text{ si } x\in \Omega\setminus E\\
                0, &\text{ en otro caso }.
            \end{cases}
        \end{align*}
        Por lo hecho anteriormente, $g\in L^p(\Omega)$ y $\norm{g}_p\leq 1$, además, para $x \in \Omega\setminus E$, podemos escribir
        \begin{align*}
            g(x)=\sum_{k=1}^\infty|f_{j_{k+1}}(x)-f_{j_k}(x)|.
        \end{align*}
        
        Sean $m,k\in \mathbb{Z}^+$ con $m\geq k\geq2$, entonces existe $l\in \mathbb{N}$ tal que $m=k+l$ y para $x \in \Omega\setminus E$
        \begin{align*}
            |f_{j_m}(x)-f_{j_k}(x)|&=|(f_{j_m}(x)-f_{j_{k+l-1}}(x))+(f_{j_{k+l-1}}(x)-f_{j_{k+l-2}}(x))+\dotsb+(f_{j_{k+1}}(x)-f_{j_{k}}(x))|\\
            &\leq\sum_{i=1}^l|f_{j_{k+i}}(x)-f_{j_{k+i-1}}(x)|\\
            &=\sum_{i=k}^{m}|f_{j_{i+1}}(x)-f_{j_i}(x)|\\
            &\leq \sum_{i=k}^\infty |f_{j_{i+1}}(x)-f_{j_i}(x)|\\
            &=\sum_{i=1}^\infty |f_{j_{i+1}}(x)-f_{j_i}(x)|-\sum_{i=1}^{k-1} |f_{j_{i+1}}(x)-f_{j_i}(x)|\\
            &=g(x)-g_{k-1}(x), 
        \end{align*}
        por tanto, 
        \begin{align*}
            |f_{j_m}(x)-f_{j_k}(x)|\leq g(x)-g_{k-1}(x)\xrightarrow{k\to \infty}0,
        \end{align*}
        es decir, la sucesión $(f_{j_k}(x))$ es de Cauchy en $\mathbb{R}$ para casi todo $x \in \Omega$, más precisamente, para $x \in \Omega \setminus E$. Definimos entonces
        \begin{align*}
            f(x)=\begin{cases}
                \displaystyle\lim_{k\to \infty}f_{j_k}(x), \hspace{3mm} &\text{ si } x\in \Omega\setminus E\\
                0, &\text{ en otro caso.}
            \end{cases}
        \end{align*}
        Sea $x \in \Omega\setminus E$. Vimos que, dados $m\geq k\geq 2$, se tiene que
        \begin{align*}
            |f_{j_m}(x)-f_{j_k}(x)|\leq g(x)-g_{k-1}(x)\leq g(x),
        \end{align*}
        como esto se tiene para todo $m\geq k\geq 2$, por la definición de $f$ $k\geq 2$ y $x \in \Omega\setminus E$
        \begin{align}\label{eq: dominada}
            |f(x)-f_{j_k}(x)|\leq g(x),
        \end{align}
        en particular, como $|f(x)|-|f_{j_k}(x)|\leq |f(x)-f_{j_k}(x)|$, tenemos que
        \begin{align*}
            |f(x)|\leq g(x)+|f_{j_k}(x)|,
        \end{align*}
        de manera que $\norm{f}_p\leq \norm{g}_p+\norm{f_{j_k}}_p<\infty$, dado que $g \in L^p(\Omega)$ y $f_{j_k}\in L^p(\Omega)$. Finalmente por la desigualdad (\ref{eq: dominada}) y el Teorema de la Convergencia Dominada, se tiene
        \begin{align*}
            \lim_{k\to \infty}\norm{f_{j_k}-f}_p^p=\lim_{k\to \infty}\int_{\Omega}|f_{j_k}(x)-f(x)|^p\, dx&=\int_{\Omega}\lim_{k\to \infty}|f_{j_k}(x)-f(x)|^p\, dx=0,
        \end{align*}
        es decir, $f_{j_k}\to f$ en $L^p(\Omega)$. Como la sucesión $(f_j)$ es de Cauchy, al tener una subsucesión convergente, la sucesión ``completa'' es convergente, con lo que se concluye el resultado.
     \end{itemize}
\end{proof}


%%%%%%%%%%%%%%%%%%%%%%%%%%%%%%%%%%%%%%%%%%%%%%%%%%%%%%%%%%%%%%%%%%%%%%%%%%


\textbf{Ejercicio 5.} Considere el espacio $L^p$, $1\leq p\leq \infty$. Sean
\begin{align*}
    f_0(x)=\begin{cases}
        |x|^{-\alpha}, \hspace{3mm} &\text{ si } |x|\leq 1,\\
        0, \hspace{3mm} &\text{ si } |x|>1.
    \end{cases} \hspace{5mm} f_1(x)=\begin{cases}
        0, \hspace{3mm} &\text{ si } |x|\leq 1\\
        |x|^{-\alpha}, \hspace{3mm} &\text{ si } |x|>1.
    \end{cases}
\end{align*}
\begin{enumerate}
    \item[(I)] ¿Para qué valores de $\alpha\in \mathbb{R}$, $f_0\in L^p(\mathbb{R}^n)$.
    \item[(II)] ¿Para qué valores de $\alpha\in \mathbb{R}$, $f_1\in L^p(\mathbb{R}^n)$.
    \item[(III)] ¿Para qué valores de $\alpha\in \mathbb{R}$, $f_2(x)=\dfrac{1}{1+|x|^\alpha}\in L^p(\mathbb{R}^n)$.
\end{enumerate}

\begin{proof}
\begin{enumerate}
    \item[(I)]Consideramos dos casos acá:\\
a) $1 \leq p < \infty$ \\
En primer lugar, haciendo el cambio a coordenadas polares se tiene que
\[
\int_{\mathbb{R}^n} f_0(x) d\lambda(x)=\int_0^1\int_{S^{n-1}}(r^{-\alpha})r^{n-1}dSdr=\int_0^1C(r^{-\alpha})r^{n-1}dr,
\]
en donde en la primera igualdad se utilizó el cambio a coordenadas polares, y en la segunda igualdad $C$ corresponde a la medida de la $S^{n-1}$ esfera respecto a la medida de Lebesgue. Luego
\begin{align*}
\norm{f_0}_{L^p}=\left( \int_{\mathbb{R}^n} (f_0(x))^p d\lambda(x) \right)^{1/p}&= \left( \int_0^1 \int_{S^{n-1}} (r^{-\alpha})^p r^{n-1} dS dr \right)^{1/p}\\&= \left( \int_0^1 C (r^{-\alpha})^p r^{n-1} dr \right)^{1/p}\\&= \left( \int_0^1 C r^{-\alpha p + n - 1} dr \right)^{1/p} 
\end{align*}
Veamos cuando $\displaystyle \int_0^1 r^{-\alpha p + n - 1} dr$ converge. Esto ocurre si y sólo si
\begin{align*}
-\alpha p + n - 1 &> -1 \\
-\alpha p + n &> 0 \\
n &> \alpha p \\
\frac{n}{p} &> \alpha
\end{align*}
b) $p = \infty$ \\ En este caso se tiene que

\[\|f_0\|_{L^\infty} = \inf \{ C \geq 0 : |f_0(x)| \leq C \quad \text{para casi todo } x \in \R^{n}\}.\]

Como $0 \leq C$ para todo $ C \in \mathbb{R}^+$, es suficiente considerar
cuándo$|x| \leq 1$. De esta forma
\[\|f_0(x)\|_{L^\infty} = \inf \{ C \geq 0 : |x|^{-\alpha} \leq C \quad \text{para casi todo } x \},\]

luego este ínfimo existe siempre que
$|x|^{-\alpha}$ sea acotado en $|x| \leq 1$ (para casi todo $x)$.
Esto se tiene si y solo si $\alpha \leq 0$. \\
Por tanto, $f_0 \in L^p(\mathbb{R}^n)$ con $1 \leq p < \infty$ si y solo si $\alpha< \dfrac{n}{p}$, y para $p=\infty$, $f_0 \in L^\infty(\mathbb{R}^n)$ si y sólo si $\alpha\leq 0$.
\item[(II)] Consideramos dos casos \\
a) $1 \leq p < \infty$ \\
Al igual que en el item (I), se considera el cambio a coordenadas polares de la misma forma. Así:
\[
\int_{\mathbb{R}^n} f_1(x) d\lambda(x)=\int_1^{\infty}\int_{S^{n-1}}(r^{-\alpha})r^{n-1}dSdr=\int_1^{\infty}C(r^{-\alpha)}r^{n-1}dr
\]
Esto nos lleva a:
\begin{align*}
\norm{f_1}_{L^p}=\left( \int_{\mathbb{R}^n} (f_1(x))^p d\lambda(x) \right)^{1/p}&= \left( \int_1^{\infty} \int_{S^{n-1}} (r^{-\alpha})^p r^{n-1} dS dr \right)^{1/p}\\&= \left( \int_1^{\infty} C (r^{-\alpha})^p r^{n-1} dr \right)^{1/p}\\&= \left( \int_1^{\infty} C r^{-\alpha p + n - 1} dr \right)^{1/p} 
\end{align*}
Veamos cuando $\displaystyle\int_1^{\infty} r^{-\alpha p + n - 1} dr$ converge. Esto ocurre si y sólo si
\begin{align*}
-\alpha p + n - 1 &< -1 \\
-\alpha p + n &< 0 \\
n &< \alpha p \\
\frac{n}{p} &< \alpha
\end{align*}
b) $p = \infty$ \\ En este caso se tiene que

\[\|f_1\|_{L^\infty} = \inf \{ C \geq 0 : |f_1(x)| \leq C \quad \text{para casi todo } x \in \R^{n}\}.\]

Como $0 \leq C$ para todo $ C \in \mathbb{R}^+$, es suficiente considerar
cuándo$|x| > 1$. De esta forma
\[\|f_1(x)\|_{L^\infty} = \inf \{ C \geq 0 : |x|^{-\alpha} \leq C \quad \text{para casi todo } x \},\]
luego este ínfimo existe siempre que
$|x|^{-\alpha}$ sea acotado en $|x| > 1$.
Esto se tiene siempre que $\alpha \geq 0$. \\
Por tanto, $f_1 \in L^p(\mathbb{R}^n)$ con $1 \leq p < \infty$ si y sólo si $\alpha> \frac{n}{p}$, y para $p=\infty$, $f_1 \in L^{\infty}(\R^n)$ si y sólo si $\alpha \geq 0$.\\
\item[(III)] Consideramos dos casos: \\
a) $1 \leq p < \infty$ \\
Queremos ver cuando $f_2 \in L^p$. Esto es, cuando
\begin{align*}
\norm{f_2}_{L^p}=\left( \int_{\mathbb{R}^n} (f_2(x))^p d\lambda(x) \right)^{1/p}&= \left( \int_0^{\infty} \int_{S^{n-1}} \left(\frac{1}{1+r^\alpha}\right)^p r^{n-1} dS dr \right)^{1/p}\\&= \left( \int_0^{\infty} C \left(\frac{1}{1+r^\alpha}\right)^p r^{n-1} dr \right)^{1/p}\\&< \infty 
\end{align*}
Para ello, veamos cuando $\displaystyle \left( \int_0^{\infty}  \left(\frac{1}{1+r^\alpha}\right)^p r^{n-1} dr \right)$ converge. Como
\begin{align*}
\int_0^{\infty}  \left(\frac{1}{1+r^\alpha}\right)^p r^{n-1} dr=\int_0^{1}  \left(\frac{1}{1+r^\alpha}\right)^p r^{n-1} dr+\int_1^{\infty}  \left(\frac{1}{1+r^\alpha}\right)^p r^{n-1} dr      
\end{align*}
y la primera integral converge por ser $\left(\dfrac{1}{1+r^\alpha}\right)^pr^{n-1}$ continua en el compacto $[0,1]$, es suficiente estudiar la convergencia de la segunda integral. Para ello, utilizaremos criterio de comparación por paso al límite. Sea $g(r)=\left(\dfrac{1}{1+r^\alpha}\right)^pr^{n-1}$, consideremos los siguientes casos: \\
\checkmark $\alpha>0$ \\
 Considerando $h(r)=r^{-\alpha p+n-1}$, se tiene que
 \begin{align*}
     \lim_{r\to\infty}\frac{g(r)}{h(r)}=\lim_{r\to\infty}\frac{\left(\dfrac{1}{1+r^\alpha}\right)^pr^{n-1}}{r^{-\alpha p+n-1}}=\lim_{r\to\infty}\left(\dfrac{r^{\alpha}}{1+r^\alpha}\right)^p=1
 \end{align*}
 donde la última igualdad se tiene pues al evaluar el límite, $\alpha>0$. Así, 
 \begin{align*}
     \int_1^{\infty}h(r)dr \quad \text{ converge si y sólo si} \quad \int_1^{\infty}g(r)dr \quad \text{converge}
 \end{align*}
 la primera converge si y sólo si
 \begin{align*}
      -\alpha p+n-1&<-1 \\
    -\alpha p +n&<0 \\
    \frac{n}{p}&<\alpha
 \end{align*}
\checkmark $\alpha=0$ \\
Para este caso, 
\begin{align*}
    \int_1^{\infty}g(r)dr=\int_1^{\infty}\frac{1}{2^p}r^{n-1}dr
\end{align*}
la cual diverge pues $n\geq 1$. \\
\checkmark $\alpha<0$ \\
Considerando $t(r)=r^{n-1}$, se tiene que
\begin{align*}
     \lim_{r\to\infty}\frac{g(r)}{t(r)}=\lim_{r\to\infty}\frac{\left(\dfrac{1}{1+r^\alpha}\right)^pr^{n-1}}{r^{n-1}}=\lim_{r\to\infty}\left(\dfrac{1}{1+r^\alpha}\right)^p=1
 \end{align*}
 donde la última igualdad se tiene pues al evaluar el límite, $\alpha<0$. Así, 
 \begin{align*}
     \int_1^{\infty}t(r)dr \quad \text{ converge si y sólo si} \quad \int_1^{\infty}g(r)dr \quad \text{converge}
 \end{align*}
 la primera converge si y sólo si $n-1<0$. Es decir, cuando $n<1$ lo cual no se tiene dado que $n\geq 0$. Por lo tanto, $\displaystyle\int_1^{\infty}g(r)dr$ diverge. \\
 b) $p=\infty$
 En este caso se tiene que
\begin{align*}
\|f_2\|_{L^\infty} &= \inf \{ C \geq 0 : |f_2(x)| \leq C \quad \text{para casi todo } x \in \R^{n}\} \\
&=\inf\{C\geq0: \dfrac{1}{1+|x|^\alpha}\leq C \quad \text{para casi todo $x \in \R^n$}\}
\end{align*}
Luego este ínfimo existe siempre pues:
\begin{align*}
    |x|^{\alpha}&\geq 0 \\
    1+|x|^{\alpha} &\geq 1 \\
    1 &\geq\frac{1}{1+|x|^{\alpha}}
\end{align*}
Por lo tanto, $f_2 \in L^{p}(\R^n)$ con $1\leq p < \infty$ si y sólo si $\dfrac{n}{p}<\alpha$, y para $p=\infty$, $f_2\in L^{\infty}(\R^n)$ para todo $\alpha \in \R$.
\end{enumerate}
\end{proof}


%%%%%%%%%%%%%%%%%%%%%%%%%%%%%%%%%%%%%%%%%%%%%%%%%%%%%%%%%%%%%%%%%%%%%%


\textbf{Ejercicio 8.}

\begin{enumerate}
    \item[(I)] Sea $1<p<\infty$. Considere las secuencias $x_n=\{x_n^j\}_{j=1}^\infty$, para cada $n \in \mathbb{N}$ y $x=\{x^j\}_{j=1}^\infty$. Asuma que $x_n,x\in l^p$, para todo $n \in \mathbb{N}$. Muestre que $x_n\rightharpoonup x$ en $l^p$ si y sólo si $\{x_n\}$ es acotada en $l^p$ y $x_n^j\to x^j$ para cada entero positivo $j$.

    \item[(II)] Considere la secuencia $x_n=\left(1,\dfrac{1}{2},\dfrac{1}{3},...,\dfrac{1}{n},0,0,0,....\right)$. ¿En cuales espacios $l^p$, con $1\leq p\leq \infty$ esta secuencia converge débilmente?
\end{enumerate}

\begin{proof}
    \begin{enumerate}
        \item[(I)] \begin{itemize}
            \item $(\Longrightarrow)$ Supongamos que $x_n\rightharpoonup x$ en $l^p$. Como $l^p$ es un espacio de Banach, sabemos que $\{x_n\}$ es acotada y además, $\norm{x}_{l^p}\leq \liminf\norm{x_n}_{l^p}$.

            Dado $j \in \mathbb{Z}^+$, definimos
            \begin{align*}
                \pi_j:l^p &\longrightarrow \mathbb{R}\\
                y &\longmapsto \pi_j(y)=y^j,
            \end{align*}
            donde $y=\{y^i\}_{i=1}^\infty\in l^p$. Veamos que $\pi_j\in (l^p)^{\bigstar}$ para todo $j \in \mathbb{Z}^+$. $\pi_j$ es claramente lineal como consecuencia de la suma y producto por escalar definida en $l^p$, además
            \begin{align*}
                |\pi_j(y)|=|y^j|=(|y^j|^p)^{1/p}\leq \left(\sum_{i=1}^\infty |y^{i}|^p\right)^{1/p}=\norm{y}_{l^p},
            \end{align*}
            por tanto, $\pi_j$ es continuo y $\norm{\pi_j}_{(l^p)^\bigstar}\leq 1$. Sabemos que $x_n\rightharpoonup x$ si y sólo si $\langle f;x_n\rangle\xrightarrow{n\to \infty} \langle f;x\rangle$ para todo $f\in (l^p)^\bigstar$, en particular para $\pi_j$. En este caso, se obtiene que 
            \begin{align*}
                \pi_j(x_n)=x_n^j\xrightarrow{n\to \infty}x^j=\pi_j(x),
            \end{align*}
            como $j\in \mathbb{Z}^+$ es arbitrario, se obtiene el resultado.

            \item $(\Longleftarrow)$ Supongamos ahora que $\{x_n\}$ es acotada y que $x_n^j\to x^j$ para todo $j\in \mathbb{Z}^+$. Sea $V$ una vecindad débil de $x$, que sin pérdida de generalidad, podemos expresar como
            \begin{align*}
                V=\{y \in l^p:|\langle f_i;y-x\rangle|<\epsilon \text{ para todo } i=1,...,k\},
            \end{align*}
            para algunos $f_1,...,f_k \in (l^p)^\bigstar$ y $\epsilon>0$. Queremos ver que existe $N\in \mathbb{Z}^+$ tal que si $n\geq N$, entonces $x_n\in V$. Sea $f_i$ con $1\leq i\leq k$, dado que $1<p<\infty$, por el Teorema de Representación de Riesz, existe una única $y_i=\{y_i^j\}_{j=1}^\infty\in l^{p'}$ con $\dfrac{1}{p}+\dfrac{1}{p'}=1$ tal que 
            \begin{align*}
                \langle f_i;z\rangle=\sum_{j=1}^\infty y_i^jz^j,
            \end{align*}
            para toda $z=\{z^j\}_{j=1}^\infty\in l^p$. Entonces, dado $n \in \mathbb{N}$
            \begin{align*}
                |\langle f_i;x_n-x\rangle|&=\left|\sum_{j=1}^\infty y_i^j(x_n^j-x^j)\right|\leq \sum_{j=1}^\infty |y_i^j||x_n^j-x^j|.
            \end{align*}
            Ahora, note que si $z=\{z^j\}_{j=1}^\infty\in l^q$ con $1\leq q<\infty$, entonces, dado $l\geq 1$, la sucesión 
            \begin{align*}
                z_l=(\underbrace{0,0,...,0}_{l-1 \text{ ceros}},z^l,z^{l+1},...),
            \end{align*}
            también está en $l^q$, dado que 
            \begin{align*}
                \norm{z_j}_{l^q}^q=\sum_{j=l}^\infty|z^j|^q\leq \sum_{j=1}^\infty |z^j|^q=\norm{z}_{l^q}^q<\infty,
            \end{align*}
            de esta manera, también vale la desigualdad de Hölder para estas sucesiones, en otras palabras, dadas $z=\{z^{j}\}_{j=1}^\infty\in l^q$ y $w=\{w^j\}_{j=1}^\infty \in l^{q'}$ con $\dfrac{1}{q}+\dfrac{1}{q'}=1$, dado $l\geq 1$
            \begin{align*}
                \sum_{j=l}^\infty |z^j||w^j|\leq \left(\sum_{j=l}^\infty|z^j|^q\right)^{1/q}\left(\sum_{j=l}^\infty |w^j|^{q'}\right)^{1/q'}.
            \end{align*}
            Como $y_i \in l^{p'}$, entonces
            \begin{align*}
                \norm{y_i}_{l^{p'}}=\left(\sum_{j=1}^\infty |y_i^j|^{p'}\right)^{1/p'}<\infty,
            \end{align*}
            además, como $\{x_n\}$ es una sucesión acotada, podemos definir
            \begin{align*}
                M=\sup_{n \in \mathbb{N}} \norm{x_n}+\norm{x}+1,
            \end{align*}
            de esta manera, por ser una serie absolutamente convergente, existe $J_i\in \mathbb{Z}^+$ con $J_i\geq 2$ tal que 
            \begin{align*}
                \norm{y_i}_{l^{p'}}=\left(\sum_{j=J_i}^\infty |y_i^j|^{p'}\right)^{1/p'}< \dfrac{\epsilon}{2M}.
            \end{align*}
            Entonces
            \begin{align*}
                |\langle f_i;x_n-x\rangle|\leq \sum_{j=1}^\infty|y_i^j||x_n^j-x^j|=\underbrace{\sum_{j=1}^{J_i-1}|y_i^j||x_n^j-x^j|}_{S_1}+\underbrace{\sum_{j=J_i}^\infty |y_i^j||x_n^j-x^j|}_{S_2},
            \end{align*}

            Acotemos primero $S_2$, por la observación que hicimos antes con las sucesiones $z_l$, la manera como escogimos $J_i$ y la definición de $M$, tenemos que 
            \begin{align*}
                S_2=\sum_{j=J_i}^\infty |y_i^j||x_n^j-x^j|&\leq \left(\sum_{j=J_i}^\infty |y_i^j|^{p'}\right)^{1/p'}\left(\sum_{j=J_i}^\infty|x_n^j-x^j|^p\right)^{1/p}\\
                &\leq  \left(\sum_{j=J_i}^\infty |y_i^j|^{p'}\right)^{1/p'}\norm{x_n-x}_{l^p}\\
                &\leq \left(\sum_{j=J_i}^\infty |y_i^j|^{p'}\right)^{1/p'}\left(\norm{x_n}_{l^p}+\norm{x}_{l^p}\right)\\
                &<\dfrac{\epsilon}{2M}(\norm{x_n}_{l^p}+\norm{x})\\
                &\leq \dfrac{\epsilon}{2}.
            \end{align*}
            Para $S_1$, definamos
            \begin{align*}
                N=\max_{1\leq j\leq J_i-1}|y_i^j|+1.
            \end{align*}
            Como $x_n^j\to x^j$ para todo $j \in \mathbb{Z}^+$, existe $n_i$ tal que si $n\geq n_i$, entonces 
            \begin{align*}
                |x_n^j-x^j|<\dfrac{\epsilon}{2NJ_i},
            \end{align*}
            para todo $j=1,...,J_i-1$, de esta manera
            \begin{align*}
                S_1=\sum_{j=1}^{J_i-1}|y_i^j||x_n^j-x^j|< N\dfrac{\epsilon}{2NJ_i}\sum_{j=1}^{J_i-1}1=\dfrac{\epsilon}{2J_i}\cdot(J_i-1)<\dfrac{\epsilon}{2}.
            \end{align*}
            Así
            \begin{align*}
                |\langle f_i;x_n-x\rangle|\leq S_1+S_2<\dfrac{\epsilon}{2}+\dfrac{\epsilon}{2}=\epsilon,
            \end{align*}
            para $n\geq n_i$. De esta manera, para $n\geq N=\max_{1\leq i\leq k}n_i$, se tiene que 
            \begin{align*}
                |\langle f_i;x_n-x\rangle|<\epsilon \text{ para todo } i=1,...,k,
            \end{align*}
            es decir, para $n\geq N$, $x_n\in V$, lo cuál prueba el resultado. 
        \end{itemize} 
        \item[(I)] Vamos a dividir la prueba en 3 casos
        \begin{enumerate}
            \item[(a)] \textbf{Caso $p=1$}. En este caso, note que 
            \begin{align*}
                \norm{x_n}_{l^1}=\sum_{j=1}^n\dfrac{1}{j},
            \end{align*}
            como la serie armónica no es convergente, tenemos que la sucesión $\{x_n\}$ es no acotada y, por tanto, no puede converger débilmente en $\sigma(l^1,(l^1)^\bigstar)$.

            \item[(b)] \textbf{Caso $1<p<\infty$} En este caso, tenemos que 
            \begin{align*}
                \norm{x_n}_{l^p}^p=\sum_{j=1}^n\dfrac{1}{j^p}\leq \sum_{j=1}^\infty\dfrac{1}{j^p}=C_p<\infty,
            \end{align*}
            la convergencia de esta serie se sigue de que $p>1$ y el criterio de la integral para convergencia de series. También se puede ver como la evaluación en $p$ de la función zeta de Riemann. De esta manera, la sucesión $\{x_n\}$ es acotada en $l^p$. Consideremos la sucesión $x=\left\{\dfrac{1}{j}\right\}_{j=1}^\infty$. Por lo que vimos anteriormente, $x \in l^p$ y $\norm{x_n}_{l^p}\leq \norm{x}_{l^p}$ para todo $n \in \mathbb{Z}^+$. Sea $j \in \mathbb{Z}^+$, entonces $x^j=\dfrac{1}{j}$ y
            \begin{align*}
                x_n^j=\begin{cases}
                    \dfrac{1}{j}, \hspace{3mm} &\text{ si } n\geq j,\\
                    0, &\text{ si } n<j,
                \end{cases}
            \end{align*}
            de manera que, para $n\geq j$ se tiene que $|x_n^j-x^j|=\left|\dfrac{1}{j}-\dfrac{1}{j}\right|=0$, es decir, $x_n^j\to x^j$ para todo $j \in \mathbb{Z}^+$. Usando el ítem (I), tenemos que $x_n\rightharpoonup x$ en $l^p$. 

            \item[(c)] \textbf{Caso $p=\infty$} Consideremos nuevamente la sucesión $x=\left\{\dfrac{1}{j}\right\}_{j=1}^\infty$, claramente $x \in l^\infty$. Sea $\epsilon>0$ y $n\in \mathbb{Z}^+$ tal que $\dfrac{1}{n+1}<\epsilon$, entonces, por la forma de los $x_n^j$'s
            \begin{align*}
                \norm{x-x_n}_{l^\infty}=\sup_{j \in \mathbb{Z}^+}|x^j-x_n^j|=\sup_{j\geq n+1}\dfrac{1}{j}=\dfrac{1}{n+1}<\epsilon,
            \end{align*}
            es decir, $x_n\to x$ fuertemente en $l^\infty$, y como la convergencia fuerte implica la convergencia débil, tenemos que $x_n\rightharpoonup x$ en $l^\infty$.
        \end{enumerate}
    \end{enumerate}
\end{proof}
