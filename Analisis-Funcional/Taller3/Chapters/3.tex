\thispagestyle{empty}

\begin{minipage}{0.3\textwidth}
  \includegraphics[scale=0.35]{logounal.png}
\end{minipage}%
\hfill
\begin{minipage}{0.65\textwidth}
  \begin{center}
    \scshape
    \Large \textsc{Universidad Nacional de Colombia} \\
    \textcolor{white}{\tiny.} \Large \textsc{Departamento de Matemáticas} \\
    \textcolor{white}{\tiny.} \large \textsc{Análisis Funcional} \\
    \textcolor{white}{\tiny.} \large \textsf{Taller 1: Espacios vectoriales normados} \normalsize (I-2025)
  \end{center}
\end{minipage}

\vspace{0.3cm}
\normalfont

\textbf{Profesor:} Oscar Guillermo Riaño Castañeda\\
\textbf{Integrantes:} Andrés David Cadena Simons \hspace{2.8cm}  Jairo Sebastián Niño Castro\hspace{2.8cm}
Iván Felipe Salamanca Medina\\
\hspace*{2.1cm}\hspace{2.25cm}\textbf{Fecha:} 03 de Junio del 2025\\
\vspace{0.25cm}\\


\textbf{Ejercicio 2.} Sea $E$ un espacio vectorial y $g,f_1,...,f_k$, $k+1$ funcionales lineales sobre $E$ tales que
\begin{align*}
    \langle f_i;x\rangle=0 \text{ para todo } i=1,...,k \Longrightarrow \langle g;x\rangle=0.
\end{align*}
Muestre que existen constantes $\lambda_1,...,\lambda_k\in \mathbb{R}$ tales que $\displaystyle g=\sum_{i=1}^k \lambda_if_i$. Es decir, $g$ es combinación lineal de los $f_i$'s.
\begin{proof}
Considere
\begin{align*}
    H:&E\longrightarrow\mathbb{R}^{k+1} \\&x \longrightarrow(g(x),f_1(x),...,f_k(x))
\end{align*}
Y sea el rango de $H$, $ran(H):=V$. En primer lugar, note que $V$ es un subespacio de $\mathbb{R}^{k+1}$, lo cual sigue de la linealidad de los $f_i$ y $g$. Para ello: \\
\checkmark Sea $u,v \in V$, $\lambda \in \mathbb{R}$. Así, existen $x,y \in E$ tales que $u=(g(x),f_1(x),...,f_k(x))$ y $v=(g(y),f_1(y))...,f_k(y))$. Luego, usando que $x+\lambda y \in E$ y la linealidad de los $f_i$ y $g$, se sigue que 
\begin{align*}
 u+\lambda v&=(g(x),f_1(x),...,f_k(x))+\lambda (g(y),f_1(y))...,f_k(y)) \\ &=(g(x+\lambda y),f_1(x+\lambda y),...,f_k(x+ \lambda y)) \in V.
\end{align*}
De esta forma, como $V$ es subespacio, se sigue que es conexo. Además, siendo $V$ subespacio de dimensión finita, se tiene que $V$ es cerrado. Por otro lado, se tiene que $u_0=(1,0,...0)\notin V$: en caso contrario, existiria $x \in E$ tal que $H(x)=(g(x),f_1(x),...,f_k(x))=u_0$ lo que querría decir que $g(x)=1$ y $f_i(x)=0$ para todo $i=1,..,k$, lo cual contradice la hipótesis. 
\\
Así, por Teorema Hahn- Banach, existe $h \in (\mathbb{R}^{k+1})^\bigstar$ tal que 
\[
h(x)<h(u_0) \qquad \forall x \in V
\]
Como $h\in (\mathbb{R}^{k+1})^\bigstar$, existe $r=(\lambda,\lambda_1,...,\lambda_k)\in \mathbb{R}^{k+1}$ tal que $h(x)=\langle r,x\rangle$ (producto interno usual en $\mathbb{R}^{k+1}$) para todo $x\in V$. Así:
\[
\lambda g(x)+\sum_{i=1}^{k}\lambda_if_i(x)<\lambda \qquad\forall x \in E
\]
Como $V$ es subespacio, tomando $t>0$
\begin{align*}
    \lambda g(x/t)+\sum_{i=1}^{k}\lambda_if_i(x/t)&<\lambda \\
    \lambda g(x)+\sum_{i=1}^{k}\lambda_if_i(x)&<\lambda/t
\end{align*}
y haciendo $t\to \infty$, se tiene que $\lambda g(x/t)+\sum_{i=1}^{k}\lambda_if_i(x/t)\leq 0$. Similarmente, tomando $t<0$
\begin{align*}
    \lambda g(x/t)+\sum_{i=1}^{k}\lambda_if_i(x/t)&<\lambda \\
    \lambda g(x)+\sum_{i=1}^{k}\lambda_if_i(x)&>\lambda/t
\end{align*}
y haciendo $t\to -\infty$, se tiene que $\lambda g(x/t)+\sum_{i=1}^{k}\lambda_if_i(x/t)\geq 0$. De modo que
\[
\lambda g(x/t)+\sum_{i=1}^{k}\lambda_if_i(x/t)= 0
\]
y como $\lambda>0$, se tiene que
\[
g(x)=\sum_{i=1}^{k}\tilde{\lambda_i}f_i(x) \quad \tilde{\lambda_i}=-\frac{\lambda_i}{\lambda}
\]
\end{proof}

%%%%%%%%%%%%%%%%%%%%%%%%%%%%%%%%%%%%%%%%%%%%%



\textbf{Ejercicio 9.} Sea $E$ un espacio de Banach de dimensión infinita. Muestre que cada vecindad débil$\empty^\bigstar$ del origen de $E^\bigstar$ es no acotada.

\begin{proof}
    Sea $V$ una vecindad del origen de $E^\bigstar$ con la topología $\sigma(E^\bigstar,E)$. Sin pérdida de generalidad, podemos suponer que
    \begin{align*}
        V=\{f \in E^\bigstar:|\langle f;x_i\rangle|<\varepsilon \text{ para todo } i=1,...,k\},
    \end{align*}
    para algunos $x_1,...,x_k\in E$ y $\varepsilon>0$. Si $x_i=\z$ para todo $i=1,..,k$, entonces $V=E^\bigstar$ y queda claro que $V$ es no acotado.
    
    Supongamos que no todos los $x_i$'s son nulos. Considere $W=\text{gen}\{x_1,...,x_k\}$. Como $E$ es de dimensión infinita, tenemos que $W\subset E$ estrictamente. Queremos usar la segunda forma geométrica del Teorema de Hahn-Banach para garantizar que existe $g\in E^\bigstar$ con $g\neq 0$ tal que $\langle g;v\rangle=0$ para todo $v \in W$, pero para esto, debemos ver que $W$ es cerrado.
    
    Sea $(w_n)_{n \in \mathbb{Z}^+}\subset W$ una sucesión convergente y sea $\{v_1,...,v_l\}$ una base de $W$. Como $(w_n)$ es convergente, en particular, es una sucesión de Cauchy, por tanto, dado $\delta>0$ existe $N\in \mathbb{Z}^+$ tal que si $n,m\geq N$, entonces
    \begin{align*}
        \norm{w_n-w_m}<\delta.
    \end{align*}
    Como $w_n,w_m\in W$, estos se pueden expresar de manera única como
    \begin{align*}
        w_n=\sum_{i=1}^l \lambda_{i,n} v_i, \hspace{5mm} w_m=\sum_{i=1}^l \lambda_{i,m}v_i 
    \end{align*}
    con $\lambda_{i,n},\lambda_{i,m}\in \mathbb{R}$ para todo $i=1,...,k$, luego
    \begin{align*}
        \norm{w_n-w_m}=\norm{\sum_{i=1}^l(\lambda_{i,n}-\lambda_{i,m})v_i}.
    \end{align*}
    Como $W$ es de dimensión finita, existe una constante $C_1>0$ tal que $C_1\norm{v}_\infty\leq \norm{v}$ para todo $v \in W$, donde $C_1$ no depende del vector $v$, con 

    \begin{align*}
        \norm{v}_\infty=\max_{1\leq i\leq l} |\lambda_i| \hspace{3mm} \text{ y } \hspace{3mm} v=\sum_{i=1}^l \lambda_i v_i,
    \end{align*}
    por tanto
    \begin{align*}
        C_1\norm{w_n-w_m}_\infty\leq \norm{w_n-w_m},
    \end{align*}
    
    de esta manera
    \begin{align*}
        C_1\norm{w_n-w_m}_\infty=C_1\max_{1\leq i\leq l}|\lambda_{i,n}-\lambda_{i,m}|\leq \norm{w_n-w_m}<\delta,
    \end{align*}
    así, para cada $i=1,...,l$, las sucesiones $(\lambda_{i,n})_{n \in \mathbb{Z}^+}$ son de Cauchy en $\mathbb{R}$ y, por ser $\mathbb{R}$ completo, son sucesiones convergentes. Así, $\lambda_{i,n}\to \lambda_i$ para algunos $\lambda_i \in \mathbb{R}$ para todo $i=1,...,l$. Entonces, $w_n\to w$ en la norma $\norm{\cdot}_\infty$ donde 
    \begin{align*}
        w=\sum_{i=1}^l \lambda_i v_i,
    \end{align*}
    en efecto
    \begin{align*}
        \norm{w_n-w}_\infty=\norm{\sum_{i=1}^l (\lambda_{i,n}-\lambda_i)v_i}_\infty=\max_{1\leq i \leq l}|\lambda_{n,i}-\lambda_i|\xrightarrow{n\to \infty} 0.
    \end{align*}
    Ahora, nuevamente por ser $W$ de dimensión finita, también existe $C_2>0$ tal que $\norm{v}\leq C_2\norm{v}_\infty$ para todo $v \in W$, entonces 
    \begin{align*}
        \norm{w_n-w}\leq C_2\norm{w_n-w}_\infty,
    \end{align*}
    así, $w_n\to w \in W$ en la norma $\norm{\cdot}$. Tenemos entonces que $W$ es cerrado.
    
    Sea $x_0 \in E\setminus W$, así, por la segunda forma geométrica del Teorema de Hahn-Banach, existe un hiperplano que separa estrictamente a $W$ y a $\{x_0\}$, más precisamente, existe $g \in E^\bigstar$ y $\alpha\in \mathbb{R}$ tal que 
    \begin{align*}
        \langle g;v\rangle<\alpha<\langle g;x_0\rangle,
    \end{align*}
    para todo $v \in W$. Como $W$ es un subespacio, tenemos que $|\langle g;v\rangle|<\alpha$ para todo $v \in W$, ya que podemos cambiar $v$ por $-v$. Además, para todo $n \in \mathbb{Z}^+$, tenemos
    \begin{align*}
        |\langle g;nv\rangle|=n|\langle g;v\rangle|<\alpha \Longrightarrow |\langle g;v\rangle|<\dfrac{\alpha}{n}
    \end{align*}
    de manera que $\langle g;v\rangle=0$ para todo $v \in W$, y $g\neq 0$, dado que $\langle g;x_0\rangle>\alpha>0$. Así, por la definición de $W$, $x_1,...,x_k\in W$ y tenemos que $|\langle g;x_i\rangle|=0<\varepsilon$ para todo $i=1,...,k$, es decir, $g \in V$, más aún para todo $n\in \mathbb{Z}^+$, $|\langle ng;v\rangle|=n|\langle g;v\rangle|=0$, es decir, $ng \in V$ para todo $n \in \mathbb{Z}^+$. Como $g\neq 0$, $\norm{g}_{E^\bigstar}>0$ y así, $\norm{ng}_{E^\bigstar}=n\norm{g}_{E^\bigstar}$. Así, por la propiedad Arquimediana de $\mathbb{R}$, $\norm{ng}_{E^\bigstar}$ es ``tan grande como se quiera'', para $n\in \mathbb{Z}^+$ suficientemente grande. Así, tenemos que $V$ no es acotado.
    
\end{proof}

%%%%%%%%%%%%%%%%%%%%%%%%%%%%%%%%%%%%%%%%%%%%%%%%%


\textbf{Ejercicio 11.} Sea $K$ un espacio métrico compacto infinito. Demuestre que $C(K)$ (con la norma del supremo $\norm{\cdot}_{L^\infty}$) no es reflexivo.
\begin{proof}
  Primero veamos que $(C(K),\norm{\cdot}_{L^{\infty}})$ es un espacio de Banach.\\
  Primero, note que como $K$ es un espacio métrico compacto infinito, las funciones continuas con dominio en $K$ alcanzan su máximo y su mínimo en $K$, por lo que sabemos que la norma $\norm{\cdot}_{L^{\infty}}$ se encuentra bien definida en el espacio.\\
  Por otro lado verificar que las propiedades de norma se dan, es suficiente con notar que estas se cumplen bajo las propiedades del $\max$ (recordando que $\norm{f}_{L^{\infty}}=\max_{x\in K}|f(x)|$) y de ser funciones continuas, por lo que solo nos centraremos en ver que es un espacio completo.\\
  Sea $\{f_{k}\}\subset C(K)$ una sucesión de Cauchy, es decir, dado $\epsilon>0$ sabemos que existe $N>0$ tal que si $n,m>N$, entonces
  \begin{align*}
    \norm{f_{n}-f_{m}}_{L^{\infty}}<\epsilon.
  \end{align*}
  Veamos que $f_{k}\to f\in C(K)$ cuando $k\to\infty$.\\
  Note que dado $x\in K$ se cumple que
  \begin{equation}\label{eq:puntual-f}
    |f_{n}(x)-f_{m}(x)|\leq \norm{f_{n}-f_{m}}_{L^\infty}<\epsilon. 
  \end{equation}
  Luego $\{f_{k}(x)\}\subset \mathbb{R}$ es una sucesión de Cauchy, por lo tanto existe $a\in \mathbb{R}$ tal que $f_{k}(x)\to a$ cuando $k\to \infty$, luego como se puede realizar el mismo razonamiento para todo $x\in K$, definamos
  \begin{align*}
    f:K&\to \mathbb{R},\\
      x\to\lim_{k \to \infty}f_{k}(x).
  \end{align*}
  Ahora veamos que $f\in C(K)$, entonces, dado $\epsilon>0$ existe $\delta>0$ (este $\delta$ depende de la continuidad de las funciones $f_{k}$) tal que si tomamos $x,y\in K$ que satisfacen
  \begin{align*}
    |x-y|<\delta,
  \end{align*}
  entonces, si tomamos un $k$ adecuado (de la condición $\label{eq:puntual-f}$) y aprovechando que las funciones $f_{k}$ con continuas en $K$, sabemos que
  \begin{align*}
    |f(x)-f(y)|&\leq|f(x)-f_{m}(x)+f_{m}(x)-f_{m}(y)+f_{m}(y)-f(y)|,\\
    &\leq |f(x)-f_{k}(x)|+|f_{k}(x)-f_{k}(y)|+|f_{k}(y)-f(y)|,\\
    &\leq \frac{\epsilon}{3}+\frac{\epsilon}{3}+\frac{\epsilon}{3},\\
    &\leq \epsilon.
  \end{align*}
  Lo que nos permite concluir que $f\in C(K)$.\\
  Ahora veamos que $f_{k}\to f$ en la norma de $L^{\infty}(K)$ cuando $k\to\infty$.\\
  Note que como $\{f_{k}\}$ es una sucesión de Cauchy, entonces dado $\epsilon>0$ existe $N>0$ tal que si $n,m>N$ se satisface que
  \begin{align*}
    \norm{f_{n}-f_{m}}_{L^{\infty}}=\max_{x\in K}|f_{n}(x)-f_{m}(x)|<\epsilon.
  \end{align*}
  Luego, si fijamos $n$ y hacemos que $m\to \infty$, entonces $f_{m}(x)\to f(x)$, por lo que podemos asegurar que
  \begin{align*}
    \norm{f_{n}-f}_{L^{\infty}}&=\max_{x\in K}|f_{n}(x)-f(x)|,\\
    &<\epsilon,
  \end{align*}
  lo que nos permite concluir que $f_{k}\to f$ cuando $k\to \infty$ en la norma de $L^{\infty}$, es decir, el espacio $(C(K),\norm{\cdot}_{L^{\infty}})$ es Banach.\\
  Ahora veamos que $(C(K),\norm{\cdot}_{L^{\infty}})$ no es un espacio reflexivo.\\
  Note que como $K$ es compacto, entonces podemos construir una sucesión $\{a_{k}\}\subset K$ tal que $a_{k}\to a$ con $a_{k}\neq a$ para todo $k$, esto ya que como $K$ es un espacio métrico compacto el teorema de Bolzano-Weierstrass nos afirma que en un espacio métrico compacto toda sucesión tiene una subsucesión convergente en la que se pueden tomar puntos distintos que convergen a un punto límite que no pertenece al conjunto de sucesión de puntos.\\
  Ahora, definamos $g_{n}\in C(K)$ tal que 
  \begin{align*}
    g_{n}(x)= 
    \begin{cases}
      1, &\text{ Si $x=a_{m}$ con } 1\leq m\leq n \text{,} \\
      0, &\text{ Si $x=a_{m}$ con } m>n,\\
      0, &\text{ Si $x=a$}.
    \end{cases}
  \end{align*}
  Con $\norm{g_{n}}_{\infty}=1$, esto se puede ya que como $K$ es un espacio normal (pues este es métrico y admite la propiedad de separación normal) y los conjuntos $A=\{a_{1},a_{2},\cdots,a_{n}\}$ y $B=\{a_{n+1},a_{n+2},\cdots\}\cup \{a\}$ son disyuntos y cerrados ya que $A$ es un conjunto finito y $B$ son los puntos de la sucesión convergente con su punto límite $a$, entonces por el lema de Urysohn si lo aplicamos en el subespacio normal $A\cup B$, sabemos que existe nuestra función continua $g_{n}$ tal que $g_{n}(A)=\{1\}$ y $g_{n}(B)=\{0\}$, luego $\norm{g_{n}}_{L^{\infty}(A\cup B)}=1$, además usando el teorema de extensión de Tietze–Urysohn–Brouwer esta se puede extender de forma continua y que preserve la norma en todo el espacio $K$ el cual es compacto y normal.\\
  Ahora, razonaremos por contradicción y vamos a suponer que $C(K)$ es un espacio reflexivo.\\
  Como anteriormente demostramos que $C(K)$ es de Banach y la sucesión $\{g_{n}\}\subset C(K)$ es acotada, entonces como $C(K)$ es un espacio reflexivo, debe de existir una subsucesión $\{g_{n_{k}}\}\subset C(K)$ débilmente convergente a una función $g\in C(K)$, es decir que para todo funcional $\phi\in C(K)^{*}$ se satisface que $\phi(g_{n_{k}})\to\phi(g)$. En particular, definamos para cada $a_{n}$
  \begin{align*}
    \pi_{a_{n}}:C(K)&\to \mathbb{R},\\
    f&\to f(a_{n}).
  \end{align*}
  Note que $\pi_{a_{n}}\in C(K)^{*}$ ya que la linealidad del operador se da por la suma y producto por escalar de evaluaciones en funciones continuas y la continuidad de que $|\pi_{a_{n}}(f)|\leq f(a_{n})\leq \norm{f}_{L^{\infty}}$.\\
  Siendo así, es fácil ver que dado $a_{n}$
  \begin{align*}
    \pi_{a_{n}}(g_{n_{k}})&=g_{n_{k}}(a_{n})= 
    \begin{cases}
      1, &\text{ si } n\leq n_{k} \text{,} \\
      0, &\text{ si } n>n_{k}.
    \end{cases}
  \end{align*}
  Luego se puede calcular que si tomamos $k\to \infty$, entonces eventualmente $n\leq n_{k}$, por lo que se puede afirmar que
  \begin{align*}
    \pi_{a_{n}}(g)=g(a_{n})=1. 
  \end{align*}
  para todo $a_{n}$ dado, en particular si tomamos $n\to\infty$, entonces se concluye que
  \begin{align*}
    \pi_{a}(g)=g(a)=1.
  \end{align*}
  \textbf{contradicción}, pues en un principio por construcción $g(a)=0$, lo que nos permite concluir que la sucesión acotada $\{g_{n}\}\subset C(K)$ no tiene una subsucesión débilmente convergente, es decir, el espacio $(C(K),\norm{\cdot}_{L^{\infty}})$ no es reflexivo. 
\end{proof}

%%%%%%%%%%%%%%%%%%%%%%%%%%%%%%%%%%%%%%%%%%%%%%%%


\textbf{Ejercicio 15.} Sea $E$ un espacio de Banach reflexivo. Sea $a:E\times E\to \mathbb{R}$ una forma bilineal continua, es decir, existe $M>0$ tal que $|a(x,y)|\leq M\norm{x}\norm{y}$ para todo $x,y \in E$. Asuma que $a$ es coerciva, esto es, existe $\alpha>0$ tal que para todo $x \in E$
\begin{align*}
    a(x,x)\geq \alpha\norm{x}^2.
\end{align*}
\begin{enumerate}
    \item[(a)] Dado $x \in E$, defina $A_x(y)=a(x,y)$ para todo $y \in E$. Muestre que $A_x\in E^\bigstar$ para cada $x \in E$. Además, concluya que la función $x\mapsto A(x)=A_x$ satisface que $A\in \mathcal{L}(E,E^\bigstar)$.
    \item[(b)] Muestre que $A$ definida como en (a) es una función sobreyectiva.
    \item[(c)] Deduzca que para $f \in E^\bigstar$, existe un único $x \in E$ tal que $a(x,y)=\langle f;y\rangle$, para todo $y \in E$. Esto es, la forma bilineal coerciva $a$ representa todo funcional lineal continuo.
\end{enumerate}


\begin{proof}
    \begin{enumerate}
        \item[(a)] Sea $x \in E$ cualquiera. Como $a$ es una forma bilineal, dados $y_1,y_2\in E$ y $\lambda\in \mathbb{R}$
        \begin{align*}
            A_x(y_1+y_2)&=a(x,y_1+y_2)=a(x,y_1)+a(x,y_2)=A_x(y_1)+A_x(y_2)\\
            A_x(\lambda y_1)&=a(x,\lambda y_1)=\lambda a(x,y_1)=\lambda A_x(y_1),
        \end{align*}
        es decir, $A_x$ es lineal para todo $x \in E$. Además, por hipótesis, dado $y \in E$
        \begin{align*}
            |A_x(y)|=|a(x,y)|\leq M\norm{x}\norm{y},
        \end{align*}
        es decir, $A_x\in E^\bigstar$ y $\norm{A_x}_{E^\bigstar}\leq M\norm{x}$. Consideremos ahora la aplicación
        \begin{align*}
            A:E &\longrightarrow E^\bigstar\\
             x &\longmapsto A(x)=A_x.
        \end{align*}
        Veamos que $A$ es lineal. Sean $x_1,x_2,y \in E$ y $\lambda\in \mathbb{R}$, entonces, dado que $a$ es bilineal
        \begin{align*}
            [A(x_1+x_2)](y)&=A_{x_1+x_2}(y)\\
            &=a(x_1+x_2,y)\\
            &=a(x_1,y)+a(x_2,y)\\
            &=A_{x_1}(y)+A_{x_2}(y)\\
            &=[A(x_1)+A(x_2)](y)
        \end{align*}
        \begin{align*}
            [A(\lambda x_1)](y)&=A_{\lambda x_1}(y)\\
            &=a(\lambda x_1,y)\\
            &=\lambda a(x_1,y)\\
            &=\lambda A_{x_1}(y)\\
            &=[\lambda A(\lambda x_1)](y),
        \end{align*}
        es decir, $A(x_1+x_2)=A(x_1)+A(x_2)$ y $A(\lambda x_1)=\lambda A(x_1)$ para todo $x_1,x_2 \in E$ y $\lambda \in \mathbb{R}$, así, $A$ es lineal. Veamos ahora que $A$ es acotada (y por tanto continua). Por lo que vimos anteriormente,
        \begin{align*}
            \norm{A(x)}_{E^{\bigstar}}=\norm{A_x}_{E^{\bigstar}}\leq M\norm{x},
        \end{align*}
        por tanto, $A\in \mathcal{L}(E,E^\bigstar)$ y $\norm{A}\leq M$ (la norma de operador).

        \item[(b)] Veamos que
        \begin{align}\label{eq: 1}
            \norm{A_x}_{E^\bigstar}\geq \alpha\norm{x} \hspace{5mm} \text{ para todo } x \in E.
        \end{align}
        Si $x=\z$, la desigualdad se tiene de manera inmediata. Si $x\neq \z$, como $a$ es bilineal y coerciva
        \begin{align*}
            \left|A_x\left(\dfrac{x}{\norm{x}}\right)\right|=\left|a\left(x,\frac{x}{\norm{x}}\right)\right|=\dfrac{1}{\norm{x}}|a(x,x)|\geq \dfrac{1}{\norm{x}}\alpha\norm{x}^2=\alpha\norm{x},
        \end{align*}
        de manera que
        \begin{align*}
            \norm{A_x}_{E^\bigstar}=\sup_{\substack{z \in E\\ \norm{z}=1}}|A_x(z)|\geq \left|A_x\left(\dfrac{x}{\norm{x}}\right)\right|\geq \alpha\norm{x}.
        \end{align*}
        Queremos probar que $F=A(E)$, la imagen del operador $A$, es cerrado, viendo que $\overline{F}=F$. Recordemos que, por ser $A$ lineal, $F$ es un subespacio de $E^\bigstar$.

        Sea $f \in \overline{F}$, entonces existe una sucesión $(f_n)_{n \in \mathbb{Z}^+}\subset F$ tal que $f_n\to f$. Por la definición de $F$, existen $x_n \in E$ tales que $f_n=A_{x_n}$ para todo $n\in \mathbb{Z}^+$. Como la sucesión $(A_{x_n})_{n \in \mathbb{Z}^+}$ es convergente, en particular, es de Cauchy, entonces, dado $\varepsilon>0$ existe $N\in \mathbb{Z}^+$ tal que si $n,m\geq N$, entonces $\norm{A_{x_n}-A_{x_m}}_{E^\bigstar}<\varepsilon$. Por la desigualdad (\ref{eq: 1}), tenemos que para $n,m\geq N$
        \begin{align*}
            \varepsilon>\norm{A_{x_n}-A_{x_m}}_{E^\bigstar}=\norm{A_{x_n-x_m}}_{E^\bigstar}\geq \alpha\norm{x_n-x_m},
        \end{align*}
        por tanto, la sucesión $(x_n)_{n \in \mathbb{Z}^+}$ es de Cauchy en $E$, y como $E$ es un espacio de Banach, existe $x \in E$ tal que $x_n\to x$ cuando $n\to \infty$. Consideremos entonces el operador $A_x$ y veamos que $A_{x_n}\to A_x$ en $E^\bigstar$. Por lo que probamos en la parte (a), se tiene que
        \begin{align*}
            \norm{A_{x_n}-A_x}_{E^\bigstar}=\norm{A_{x_n-x}}_{E^\bigstar}\leq M\norm{x_n-x},
        \end{align*}
        entonces, como $x_n\to x$, la desigualdad anterior nos garantiza que $A_{x_n}\to A_x \in E^{\bigstar}$. Por la unicidad del límite, se debe tener que $f=A_x$ y por tanto $f \in F$. De esta manera, $F=\overline{F}$, es decir, $F$  es cerrado. Supongamos por contradicción que $F\subset E^\bigstar$ estrictamente y sea $f_0 \in E^{\bigstar}\setminus F$, entonces, por la segunda forma geométrica del Teorema de Hahn-Banach, análogamente a lo hecho en el \textbf{Ejercicio 9}, existe $\xi \in E^{\bigstar\bigstar}$ con $\xi \neq 0$ tal que $\langle \xi;A_x\rangle=0$ para todo $x \in E$.
        
        Como $E$ es reflexivo, existe un único $x_0 \in E$ tal que $\xi=Jx_0$, por tanto
        \begin{align*}
            \langle \xi;A_x\rangle=\langle Jx_0;A_x\rangle=\langle A_x;x_0\rangle=A_x(x_0)=a(x,x_0)=0.
        \end{align*}
        para todo $x \in E$. Como $\xi\neq 0$, entonces $x_0\neq \z$, y por ser $a$ coerciva, para el caso en que $x=x_0$
        \begin{align*}
            a(x_0,x_0)\geq \alpha\norm{x_0}^2>0,
        \end{align*}
        lo cuál contradice que $a(x,x_0)=0$ para todo $x \in E$. De esta manera, concluimos que $F=E^\bigstar$.

        \item[(c)] Sea $x \in E$ tal que $A_x=0$, por la desigualdad (\ref{eq: 1}), tenemos
        \begin{align*}
            0=\norm{A_x}_{E^\bigstar}\geq \alpha\norm{x},
        \end{align*}
        luego $\norm{x}=0$, es decir, $x=\z$. Esto quiere decir que el núcleo del operador lineal $A$ es $\{\z\}$, o lo que es lo mismo, $A$ es inyectivo.
        
        Por el ítem (b) concluimos que $A:E\to E^\bigstar$ es una biyección, en otras palabras, para todo $f\in E^\bigstar$ existe un único $x \in E$ tal que 
        \begin{align*}
            \langle f;y\rangle=f(y)=A_x(y)=a(x,y),
        \end{align*}
        que es lo que se quería probar.
    \end{enumerate}
\end{proof}

%%%%%%%%%%%%%%%%%%%%%%%%%%%%%%%%%%%%%%%%%%%%%%%%%%%%%%%%%%%


\textbf{Ejercicio 18.} Sea $E$ un espacio de Banach.
\begin{enumerate}
    \item[(a)] Demuestre que existe un espacio topológico compacto $K$ y una isometría de $E$ en $(C(K),\norm{\cdot}_\infty)$. 

    \item[(b)] Asuma que $E$ es separable y muestre que existe una isometría de $E$ en $l^\infty$.
\end{enumerate}
\begin{proof}
    \begin{enumerate}
        \item[(a)] En primer lugar, se tiene que 
    \begin{align*}
        \{f\in E^{\bigstar}:\norm{f}\leq 1\}
    \end{align*}
    es compacto en la topología $\sigma(E^{\bigstar},E)$. Considere ahora $(C(K),\norm{\cdot}_{\infty}$. Ahora, note que $\norm{\cdot}_\infty$ está bien definida, pues dada
    \begin{align*}
        f\in C(K)=\{f:K\to \mathbb{R} |f \text{ es continua}\}
    \end{align*}
    Como $f$ es continua, $f(K)$ es compacto es $\mathbb{R}$, y por tanto, $f(K)$ es cerrado y acotado. Esto garantiza que
    \begin{align*}
        \norm{f}_{\infty}=\sup_{x\in K}|f(x)| \quad \text{existe}
    \end{align*}
    Ahora, defina la función $T:E\to C(K)$ dada por $x\mapsto Tx$, donde $(Tx)(f)=\langle f,x\rangle$ para $f\in K$. Así, dado $x\in E$
    \begin{align*}
        \norm{Tx}_{\infty}=\sup_{f\in K}|(Tx)(f)| =\sup_{f\in K}|\langle f,x\rangle|=\sup_{\substack{f \in E^{\bigstar}\\ \norm{f}\leq1}}|\langle f,x\rangle|=\norm{x}
    \end{align*}
    donde la última igualdad se sigue del coralario visto en clase: "Para todo $x\in E$, 
    \begin{align*}
        \max_{\substack{f \in E^{\bigstar}\\ \norm{f}\leq1}}|\langle f,x\rangle|=\sup_{\substack{f \in E^{\bigstar}\\ \norm{f}\leq1}}|\langle f,x\rangle|=\norm{x}
    \end{align*}
    \item[(b)] Recordamos que 
    \begin{align*}
        l^{\infty}=\{\{a_n\}_{n=1}^{\infty}:a_n\in \mathbb{R},\;\forall n\geq1,\;\sup_{n\geq 1}|a_n|<\infty\}
    \end{align*}
    con la norma $\norm{\{a_n\}_{n=1}^{\infty}}_{l^{\infty}}=\sup_{n\geq 1}|a_n|$. Con el objetivo de simplificar la notación, escribiremos $\{a_n\}:=\{a_n\}_{n=1}^{\infty}$. \\
    Ahora, como $E$ es separable, entonces $K$ es metrizable en $\sigma(E^{\bigstar},E)$ y además es compacto. Así, usando que "\textit{Todo espacio métrico compacto es separable}", se tiene que $K$ es separable, de modo que existe un subconjunto denso contable $\{f_n\}\subseteq K $. Definimos
    \begin{align*}
        F:&E\longrightarrow l^{\infty} \\ &x \longmapsto F(x)=\{\langle f_n,x\rangle\}
    \end{align*}
    En primer lugar, $F$ está bien definida pues como $f_n \in K$ para todo $n\geq1$, $\norm{f_n}\leq 1$ (norma en $E^{\bigstar}$) y por tanto
    \begin{align*}
        |\langle f_n,x\rangle|\leq \norm{f_n}\norm{x} \leq \norm{x} \qquad \forall n\geq1
    \end{align*}
    lo que garantiza que $\sup_{n\geq 1}|\langle f_n,x\rangle|<\infty$, o lo que es lo mismo, $\{\langle f_n,x\rangle\} \in l^{\infty}$. Además, $F$ resulta lineal, pues $f_n \in E^{\bigstar}$ para todo $n\geq 1$. Veamos que $F$ es una isometría. Por lo hecho en el numeral (a), $\norm{x}=\norm{Tx}_{\infty}$ para todo $x\in E$. Veremos que 
    \begin{align*}
        \norm{Tx}_{\infty}=\sup_{f\in K}|(Tx)(f)|=\sup_{n\geq 1}|(Tx)(f_n)|
    \end{align*}
    lo que permitirá concluir que $F$ es una isometría. Sea $x\in E$.
    \\
    \checkmark Como $\{f_n\}\subseteq K$, se tiene que
    \begin{align*}
        \sup_{n\geq 1}|(Tx)(f_n)| \leq \sup_{f\in K}|(Tx)(f)|
    \end{align*}
    \checkmark Por la definición de supremo, dado $\varepsilon>0$, existe $g \in K$ tal que
    \begin{align*}
    \sup_{f\in K}|(Tx)(f)|&<(Tx)(g)+\varepsilon \\
    \sup_{f\in K}|(Tx)(f)|-\varepsilon&<(Tx)(g)
    \end{align*}
    Como $Tx$ es continua (recuerde que $Tx \in C(K)$), existe $U\subseteq K$ vecindad de $g$ (en $\sigma(E^{\bigstar},E)$ tal que
    \begin{align*}
        \forall f \in U: \quad|(Tx)(f)-(Tx)(g)|<\varepsilon
    \end{align*}
    Como $\{f_n\}$ es denso en $K$, existe $f_k \in U\cap \{f_n\}$ de modo que,
    \begin{align*}
        |(Tx)(f_k)-(Tx)(g)|<\varepsilon
    \end{align*}
    Por lo tanto
    \begin{align*}
    \sup_{f\in K}|(Tx)(f)|-2\varepsilon<(Tx)(g)-\varepsilon<(Tx)(f_k)\leq \sup_{n\geq 1}|(Tx)(f_n)|
    \end{align*}
    Así, haciendo $\varepsilon \to 0$, se tiene 
    \begin{align*}
    \sup_{f\in K}|(Tx)(f)|\leq \sup_{n\geq 1}|(Tx)(f_n)| 
    \end{align*}
    Con lo cual,$\norm{Tx}_{\infty}=\sup_{f\in K}|(Tx)(f)|=\sup_{n\geq 1}|(Tx)(f_n)|$. Así:
    \begin{align*}
        \norm{Fx}_{l^{\infty}}=\norm{\{\langle f_n,x\rangle\}}&=\sup_{n\geq 1}|\langle f_n,x\rangle|\\&=\sup_{n\geq1}|(Tx)(f_n)|\\&=\sup_{f\in K}|(Tx)(f)|\\&=\norm{Tx}_{\infty}\\&=\norm{x}
    \end{align*}
    \end{enumerate}

\end{proof}

