\thispagestyle{empty}

\begin{minipage}{0.3\textwidth}
  \includegraphics[scale=0.35]{logounal.png}
\end{minipage}%
\hfill
\begin{minipage}{0.65\textwidth}
  \begin{center}
    \scshape
    \Large \textsc{Universidad Nacional de Colombia} \\
    \textcolor{white}{\tiny.} \Large \textsc{Departamento de Matemáticas} \\
    \textcolor{white}{\tiny.} \large \textsc{Análisis Funcional} \\
    \textcolor{white}{\tiny.} \large \textsf{Examen Final} \normalsize (I-2025)
  \end{center}
\end{minipage}

\vspace{0.3cm}
\normalfont

\textbf{Profesor:} Oscar Guillermo Riaño Castañeda\\
\textbf{Integrantes:} Andrés David Cadena Simons \hspace{2.8cm}  Jairo Sebastián Niño Castro\hspace{2.8cm}
Iván Felipe Salamanca Medina \hspace{5.05cm}\textbf{Fecha:} 22 de Julio del 2025\\
\vspace{0.25cm}\\


\section{Operadores Compactos}

\textbf{Problema 1.} Dado $u \in L^2((0,1))$, definimos el operador $T:L^2((0,1))\to L^2((0,1))$ por
\begin{align*}
    Tu(x)=\int_0^x tu(t)\, dt
\end{align*}
\begin{enumerate}
        \item[(a)] Demuestre que $T\in \mathcal{K}(L^2((0,1)))$.
        \item[(b)] Determine $EV(T)$ y $\sigma(T)$.
        \item[(c)] ¿Se puede escribir explícitamente $(T-\lambda I)^{-1}$ cuando $\lambda\in \rho(T)$?
        \item[(d)] Encuentre $T^\star$.
\end{enumerate}

\begin{proof}
    \begin{enumerate}
        \item[(a)] Por simplicidad, denotaremos $\norm{\cdot}_{L^2((0,1))}=\norm{\cdot}_2$. Veamos que $T$ es acotado. Sea $u \in L^2((0,1))$, usando la desigualdad de Cauchy-Schwarz
        \begin{align*}
            \norm{Tu}_2&=\left(\int_0^1\left|\int_0^x tu(t)\, dt\right|^2\, dx\right)^{1/2}\\
            &\leq \left(\int_0^1 \left(\int_0^x |t||u(t)|\ dt\right)^2\, dx\right)^{1/2}\\
            &\leq \left(\int_0^1 \left(\int_0^1|t||u(t)|\, dt\right)^2\, dx\right)^{1/2}\\
            &=\int_0^1 |t||u(t)|\, dt\\
            &\leq \left(\int_0^1 |t|^2\, dt\right)^{1/2}\norm{u}_2\\
            &=\dfrac{\sqrt{3}}{3}\norm{u}_2,
        \end{align*}
        probando así que $T$ es acotado. Para probar que $T$ es compacto, usamos el siguiente resultado.
        \begin{theorem}
            \textbf{(Kolmogorov. Riesz-Frechet).} Sea $\mathcal{F}$ un subconjunto acotado de $L^p(\mathbb{R}^n)$ con $1\leq p<\infty$. Para $f:\mathbb{R}^n\to \mathbb{R}$ y $h \in \mathbb{R}^n$, sea $\tau_h f(x)=f(x+h)$. Asuma que
            \begin{align*}
                \lim_{|h|\to 0}\norm{\tau_h f-f}_p=0 \text{ uniformemente en } f \in\mathcal{F},
            \end{align*}
            esto es, si para todo $\varepsilon>0$ existe $\delta>0$ tal que si $|h|<\delta$,\\ entonces $\norm{\tau_h f-f}_p<\varepsilon$ para toda $f \in \mathcal{F}$. Entonces la clausura de $\mathcal{F}\big|_{\Omega}$ es compacta en $L^p(\Omega)$, para cualquier $\Omega\subset \mathbb{R}^n$ con medida finita ($\mathcal{F}\big|_\Omega$ denota las restricciones a $\Omega$ de las funciones en $\mathcal{F}$).
        \end{theorem}
        Queremos ver que $\overline{T(B)}$ es compacto en $L^2((0,1))$, donde
        \begin{align*}
            B=\left\{f \in L^{2}((0,1)): \norm{f}_2\leq 1\right\}. 
        \end{align*}
        Para aplicar el \textbf{Teorema 1}, tenemos que ver que 
        \begin{align*}
            \lim_{|h|\to 0}\norm{\tau_h f-f}_2=0,
        \end{align*}
        para toda $f \in T(B)$. Sea $h\in \mathbb{R}$. Antes de realizar los cálculos, formalmente vemos a las funciones como extendidas a todo $\mathbb{R}$ por $0$, es decir, dada $u \in L^2((0,1))$, consideramos
        \begin{align*}
            \widetilde{u}(x)=\begin{cases}
                u(x) \hspace{3mm} &\text{ si } x \in (0,1)\\
                0 &\text{ si } x \in \mathbb{R}\setminus (0,1),
            \end{cases}
        \end{align*}
        aunque en la práctica trabajaremos con $u$. Esta aclaración se hace, dado que, dependiendo del valor de $h$, la expresión
        \begin{align*}
            \tau_hTu(x)=\int_0^{x+h}u(t)\, dt,
        \end{align*}
        podría no tener sentido si, en principio, $u$ está definida únicamente en $(0,1)$. Ahora, sí, procedemos con los cálculos: Sea $f \in T(B)$, es decir, existe $u \in B$, tal que $Tu=f$, 
        \begin{itemize}
            \item Si $h\geq 0$, tenemos
            \begin{align*}
                \norm{\tau_h f-f}_2^2&=\int_0^1 \left|\tau_h f(x)-f(x)\right|^2\, dx\\
                &=\int_0^1\left|\tau_h Tu(x)-Tu(x)\right|^2\, dx\\
                &=\int_0^1 \left|\int_0^{x+h}tu(t)\, dt-\int_0^x tu(t)\, dt\right|^2\, dx\\
                &=\int_0^1\left|\int_x^{x+h}tu(t)\, dt\right|^2\, dx\\
                &\leq \int_0^1 \left(\int_0^1 \chi_{(x,x+h)}(t)|t||u(t)|\, dt\right)^2\, dx,
            \end{align*}
            por la desigualdad de Cauchy-Schwarz y el hecho de que $u \in B$, tenemos 
            \begin{align*}
                \int_0^1 \chi_{(x,x+h)}(t)|t||u(t)|\, dt&\leq \left(\sup_{t\in (0,1)}|t|\right)\int_0^1 \chi_{(x,x+h)}(t)|u(t)|\, dt\\
                &\leq \norm{\chi_{(x,x+h)}}_2\norm{u}_2\\
                &\leq\left(\int_0^1 |\chi_{(x,x+h)}(t)|^2\, dt\right)^{1/2}\\
                &=\left(\int_x^{x+h}\, dt\right)^{1/2}\\
                &=h^{1/2}\\
                &=|h|^{1/2},
            \end{align*}
            de esta manera
            \begin{align*}
                \norm{\tau_hf-f}_2^2\leq &\int_0^1 \left(\int_0^1 \chi_{(x,x+h)}(t)|t||u(t)|\, dt\right)^2\, dx\\
                &\leq \int_0^1 \left(|h|^{1/2}\right)^2\, dx\\
                &=|h|,
            \end{align*}
            y por tanto
            \begin{align*}
                \norm{\tau_hf-f}_2\leq |h|^{1/2}.
            \end{align*}
            \item Si $h\leq 0$, tenemos
            \begin{align*}
                \norm{\tau_h f-f}_2^2&=\int_0^1 |\tau_hf(x)-f(x)|^2\, dx\\
                &=\int_0^1|\tau_h Tu(x)-Tu(x)|^2\, dx\\
                &=\int_0^1 \left|\int_0^{x+h}tu(t)\, dt-\int_0^x tu(t)\, dt\right|^2\, dx\\
                &=\int_0^1 \left|\int_{x+h}^xtu(t)\, dt\right|^2\, dx\\
                &\leq \int_0^1\left(\int_0^1 \chi_{(x+h,x)}(t)|t||u(t)|\, dt\right)^2\, dx,
            \end{align*}
            nuevamente, por la desigualdad de Cauchy-Schwarz y el hecho de que $u \in B$, tenemos
            \begin{align*}
                \int_0^1 \chi_{(x+h,x)}(t)|t||u(t)|\, dt&\leq \left(\sup_{t \in (0,1)}|t|\right)\int_0^1 \chi_{(x+h,x)}(t)|u(t)|\, dt\\
                &\leq \norm{\chi_{(x+h,x)}}_2\norm{u}_2\\
                &\leq \left(\int_0^1 |\chi_{(x+h,x)}(t)|^2\, dt\right)^{1/2}\\
                &=\left(\int_{x+h}^x \, dt\right)^{1/2}\\
                &=(-h)^{1/2}\\
                &=|h|^{1/2},
            \end{align*}
            de esta manera
            \begin{align*}
                \norm{\tau_hf-f}_2^2&\leq  \int_0^1\left(\int_0^1 \chi_{(x+h,x)}(t)|t||u(t)|\, dt\right)^2\, dx\\
                &\leq \int_0^1\left(|h|^{1/2}\right)^2\, dx\\
                &=|h|,
            \end{align*}
            y por tanto,
            \begin{align*}
                \norm{\tau_hf-f}_2\leq |h|^{1/2}.
            \end{align*}
        \end{itemize}
        En cualquier caso, tenemos que $\norm{\tau_hf-f}_2\leq |h|^{1/2}$ para toda $f \in T(B)$, así
        \begin{align*}
            0\leq \lim_{|h|\to 0}\norm{\tau_hf-f}_2\leq \lim_{|h|\to 0}|h|^{1/2}=0,
        \end{align*}
        para toda $f \in T(B)$. Así, el \textbf{Teorema 1} nos garantiza que $T(B)$ tiene clausura compacta en $L^2((0,1))$, es decir, $\overline{T(B)}$ es compacto en $L^2((0,1))$ y por tanto, $T$ es un operador compacto.

        \item[(b)] Como $T\in \mathcal{K}(L^2((0,1)))$, sabemos que $0 \in \sigma(T)$ y $\sigma(T)\setminus\{0\}=EV(T)\setminus\{0\}$. Sea $\lambda \in \mathbb{R}$ con $\lambda\neq 0$. Primero note que, si $f \in L^2((0,1))$, entonces, por la desigualdad de Cauchy-Schwarz
        \begin{align*}
            \norm{f}_1=\int_0^1 |f(x)|\, dx\leq \norm{1}_2\norm{f}_2=\norm{f}_2,
        \end{align*}
        es decir, $f \in L^1((0,1))$, además, como $g(t)=t$ es continua y acotada en $(0,1)$, tenemos $tf(t)\in L^1((0,1))$. De esta manera, podemos aplicar el Teorema de Diferenciación de Lebesgue para afirmar que si $f \in L^2((0,1))$
        \begin{align*}
            \lim_{h\to 0^+}\dfrac{1}{2h}\int_{x-h}^{x+h} tf(t)\, dt=\dfrac{1}{2h}\left[\int_{0}^{x+h}tf(t)\, dt-\int_0^{x-h}tf(t)\, dt\right]=xf(x),
        \end{align*}
        para casi todo $x \in (0,1)$, es decir, la función 
        \begin{align*}
            Tf(x)=\int_{0}^x tf(t)\, dt,
        \end{align*}
        es diferenciable en casi todo punto de $x \in (0,1)$ y, para los puntos donde esta sea diferenciable, vale que
        \begin{align*}
            \dfrac{d}{dx}(Tf(x))=xf(x).
        \end{align*}
        Sea $u \in L^2((0,1))$ tal que $Tu=\lambda u$, es decir
        \begin{align*}
            \int_0^x tu(t)\, dt=\lambda u(x),
        \end{align*}
        Note que, en este caso, podemos extender continuamente $u$ a $[0,1)$, definiendo
        \begin{align*}
            u(0):=\lim_{x\to 0^+}\int_0^x tu(t)\, dt=0.
        \end{align*}
        Por las observaciones que hicimos anteriormente, tenemos que si $Tu=\lambda u$, $u$ es diferenciable en casi toda parte, de manera que es válido expresar el problema de valor inicial dado por
        \begin{align*}
            \begin{cases}
                xu(x)=\lambda u'(x)\\
                u(0)=0,
            \end{cases}
        \end{align*}
        cuya solución general de la EDO asociada está dada por
        \begin{align*}
            u(x)=Ce^{x^2/2\lambda},
        \end{align*}
        donde $C \in \mathbb{R}$, de manera que, para que $u(0)=0$, se debe tener que $C=0$ y por tanto, $u=0$. De esta manera, $\lambda\notin EV(T)$, es decir, $\mathbb{R}\setminus \{0\}\subset \rho(T)$. Finalmente, si $\lambda=0$, la ecuación $Tu=\lambda u$ se transforma en
        \begin{align*}
            \int_0^x tu(t)\, dt=0,
        \end{align*}
        nuevamente, como $Tu$ es diferenciable en casi toda parte, tenemos que 
        \begin{align*}
            xu(x)=0,
        \end{align*}
        para casi todo $x \in (0,1)$, pero esto quiere decir que $u(x)=0$ para casi todo $x \in (0,1)$, es decir, $u=0$ y así, $0 \notin EV(T)$. De esta manera $\sigma(T)=\{0\}$ y $EV(T)=\emptyset$.

        \item[(c)] Sea $\lambda \in \rho(T)$, es decir, $\lambda\neq 0$. Sea $u \in L^2((0,1))$ y sea $f:=(Tu-\lambda u)$, de manera que $u=(T-\lambda I)^{-1}f$. Definimos 
        \begin{align*}
            v(x)=Tu(x)=\int_0^x tu(t)\, dt.
        \end{align*}
        y nuevamente, podemos extender $v$ a $[0,1)$ de manera continua con $v(0)=0$.
        Análogamente a lo hecho en el ítem anterior, tenemos que $v$ es diferenciable en casi toda parte y $v'(x)=xu(x)$ para casi todo $x \in (0,1)$, así, $v$ satisface el siguiente problema de valor inicial 
        \begin{align*}
            \begin{cases}
                v-\dfrac{\lambda}{x}v'=f\\
                v(0)=0,
            \end{cases}
        \end{align*}
        con $x \in (0,1)$. Esta es una ecuación diferencial lineal de primer orden, de manera que la única solución de el problema de valor inicial está dada por
        \begin{align*}
            v(x)=-\frac{1}{\lambda}e^{\frac{x^2}{2\lambda}}\int_0^x e^{-\frac{t^2}{2\lambda}}tf(t)\, dt, 
        \end{align*}
        Nuevamente, el Teorema de Diferenciación de Lebesgue nos garantiza que la función $e^{-\frac{t^2}{2\lambda}}tf(t)\in L^2((0,1))$ es diferenciable para casi todo $x \in (0,1)$, por tanto
        \begin{align*}
            v'(x)=xu(x)=-\frac{x}{\lambda^2}e^{\frac{x^2}{2\lambda}}\int_0^x e^{-\frac{t^2}{2\lambda}}tf(t)\, dt-\frac{1}{\lambda}xf(x),
        \end{align*}
        así, para $x \in (0,1)$, se tiene que
        \begin{align*}
            u(x)=-\frac{1}{\lambda^2}e^{\frac{x^2}{2\lambda}}\int_0^x e^{-\frac{t^2}{2\lambda}}tf(t)\, dt-\frac{1}{\lambda}f(x),
        \end{align*}
        de esta manera, para $f \in L^2((0,1))$ y $\lambda\neq 0$, tenemos que 
        \begin{align*}
            (T-\lambda I)^{-1}f(x)=-\frac{1}{\lambda^2}e^{\frac{x^2}{2\lambda}}\int_0^x e^{-\frac{t^2}{2\lambda}}tf(t)\, dt-\frac{1}{\lambda}f(x).
        \end{align*}

        \item[(d)] Vamos a calcular $T^\star$. Como $L^2((0,1))$ es un espacio de Hilbert con el producto interno
        \begin{align*}
            (f,g)=\int_0^1 f(x)g(x)\, dx,
        \end{align*}
        para toda $f,g \in L^2((0,1))$, queremos encontrar el operador $T^\star$ tal que
        \begin{align*}
            (Tf,g)=(f,T^\star g),
        \end{align*}
        para toda $f,g \in L^2((0,1))$. Por definición, dadas $f,g \in L^2((0,1))$ tenemos
        \begin{align*}
            (Tf,g)&=\int_0^1 Tf(x)g(x)\, dx\\
            &=\int_0^1 \left(\int_0^x tf(t)\, dt\right)g(x)\, dx
        \end{align*}
        Usando el Teorema de Fubini para cambiar el orden de integración en la siguiente región
        \begin{center}
            



% Pattern Info
 
\tikzset{
pattern size/.store in=\mcSize, 
pattern size = 5pt,
pattern thickness/.store in=\mcThickness, 
pattern thickness = 0.3pt,
pattern radius/.store in=\mcRadius, 
pattern radius = 1pt}
\makeatletter
\pgfutil@ifundefined{pgf@pattern@name@_go9zksz5u}{
\pgfdeclarepatternformonly[\mcThickness,\mcSize]{_go9zksz5u}
{\pgfqpoint{0pt}{-\mcThickness}}
{\pgfpoint{\mcSize}{\mcSize}}
{\pgfpoint{\mcSize}{\mcSize}}
{
\pgfsetcolor{\tikz@pattern@color}
\pgfsetlinewidth{\mcThickness}
\pgfpathmoveto{\pgfqpoint{0pt}{\mcSize}}
\pgfpathlineto{\pgfpoint{\mcSize+\mcThickness}{-\mcThickness}}
\pgfusepath{stroke}
}}
\makeatother
\tikzset{every picture/.style={line width=0.75pt}} %set default line width to 0.75pt        

\begin{tikzpicture}[x=0.75pt,y=0.75pt,yscale=-1,xscale=1]
%uncomment if require: \path (0,310); %set diagram left start at 0, and has height of 310

%Shape: Axis 2D [id:dp33092300198399116] 
\draw  (204,226.3) -- (458,226.3)(229.4,22) -- (229.4,249) (451,221.3) -- (458,226.3) -- (451,231.3) (224.4,29) -- (229.4,22) -- (234.4,29)  ;
%Shape: Square [id:dp6656203989122085] 
\draw   (229.4,66.7) -- (389,66.7) -- (389,226.3) -- (229.4,226.3) -- cycle ;
%Shape: Right Triangle [id:dp0038208365960288315] 
\draw  [pattern=_go9zksz5u,pattern size=15pt,pattern thickness=0.75pt,pattern radius=0pt, pattern color={rgb, 255:red, 208; green, 2; blue, 27}] (389,66.7) -- (228.65,226.3) -- (389,226.3) -- cycle ;
%Straight Lines [id:da8168646378355199] 
\draw [color={rgb, 255:red, 208; green, 2; blue, 27 }  ,draw opacity=1 ]   (389,66.7) -- (229.4,226.3) ;

% Text Node
\draw (214,27.4) node [anchor=north west][inner sep=0.75pt]    {$t$};
% Text Node
\draw (446,231.4) node [anchor=north west][inner sep=0.75pt]    {$x$};
% Text Node
\draw (217,58.4) node [anchor=north west][inner sep=0.75pt]    {$1$};
% Text Node
\draw (385,229.4) node [anchor=north west][inner sep=0.75pt]    {$1$};


\end{tikzpicture}

        \end{center}
        tenemos que 
        \begin{align*}
            (Tf,g)&=\int_0^1\int_0^x tf(t)g(x)\,  dt\, dx\\
            &=\int_0^1\int_{t}^1 tf(t)g(x)\, dx\, dt\\
            &=\int_0^1 f(t)\left(t\int_t^1g(x)\, dx\right)\, dt\\
        \end{align*}
        de manera que, si definimos $\displaystyle Ag(t)=t\int_t^1g(x)\, dx$, tenemos
        \begin{align*}
            (Tf,g)=\int_0^1 f(t)\left(t\int_t^1g(x)\, dx\right)\, dt=(f,Ag),
        \end{align*}
        es decir, $A=T^\star$, de esta manera, para $f \in L^2((0,1))$, tenemos que 
        \begin{align*}
            T^\star f(x)=x\int_x^1 f(y)\, dy.
        \end{align*}
    \end{enumerate}
\end{proof}

\section{Ecuaciones Diferenciales en Espacios de Hilbert}
\textbf{Problema 2.}

\textbf{Consideraciones preliminares.} Sea $H$ un espacio de Hilbert separable y $J \subseteq \mathbb{R}$ un intervalo abierto. $C(J;H)$ denota el espacio de todas las funciones $u : J \to H$ que son continuas, es decir, para todo $t \in J$ se tiene que
\begin{align*}
  \lim_{t' \to t} \|u(t) - u(t')\|_H = 0.
\end{align*}

Por otro lado, denotamos por $C^1(J;H)$ el conjunto de las funciones $u \in C(J;H)$ para las cuales
\begin{align*}
  u'(t) = \lim_{h \to 0} \frac{u(t+h) - u(t)}{h}
\end{align*}
existe para todo $t \in J$ (el límite anterior se toma en $H$) y $u' \in C(J;H)$. Luego, podemos definir $u \in C^2(J;H)$ como la clase de funciones $u$ para las cuales $u' \in C^1(J;H)$. De manera recursiva se define $C^k(J;H)$ para enteros $k \geq 1$.

Note que, definiendo derivadas laterales, podemos considerar el espacio $C^k(J;H)$ donde $J$ es un intervalo cerrado.

\begin{enumerate}
  \item[(a)] (1.5 puntos) Sea $k \geq 0$ entero. Suponga que el intervalo $J$ es cerrado y acotado. Muestre que $C^k(J;H)$ es un espacio de Banach con la norma
  \begin{align*}
    \|u\|_{C^k} = \sum_{l=0}^k \sup_{t \in J} \|u^{(l)}(t)\|_H,
  \end{align*}
  donde $u^{(l)}$ denota la $l$-ésima derivada de $u$, $l = 0, \dots, k$.
  
  \item[(b)] (1.5 puntos) Sean $a, b \in \mathbb{R}$ con $a < b$. Dada una función $F \in C([a,b];H)$, muestre que podemos definir la integral
  \begin{align*}
    \int_a^b F(\tau)\, d\tau \in H
  \end{align*}
  como límite de sumas de Riemann en $H$. Además, se sigue que
  \begin{align*}
    \left\| \int_a^b F(\tau)\, d\tau \right\|_H \leq \int_a^b \|F(\tau)\|_H\, d\tau.
  \end{align*}
  Más precisamente, sea $Z = \{t_0, t_1, \dots, t_n\}$ una partición del intervalo $[a,b]$ dada por $t_0 = a < t_1 < \cdots < t_n = b$. Muestre que las sumas de Riemann
  \begin{align*}
    S(F,Z) = \sum_{j=1}^n F(t_j^*)(t_j - t_{j-1}), \quad \text{donde } t_j^* \in [t_{j-1}, t_j],
  \end{align*}
  convergen a un límite en $H$ (la integral) cuando el tamaño de la partición
  \begin{align*}
    |Z| = \max_j |t_j - t_{j-1}|
  \end{align*}
  tiende a cero.

  \item[(c)] (4 puntos) Sea $A \in \mathcal{K}(H)$ un operador autoadjunto tal que $A \geq 0$ (es decir, $(Ax,x) \geq 0$ para todo $x \in H$). Sea $F \in C([0,\infty), H)$. Dado $u_0 \in H$, considere el problema de Cauchy para la ecuación del calor abstracta con término forzante
  \begin{align*}
    \begin{cases}
      u'(t) = -Au(t) + F(t), & t \in (0,\infty), \\
      u(0) = u_0.
    \end{cases}
  \end{align*}
  \begin{enumerate}
    \item[(c.1)] (2 puntos) Suponga que $F = 0$. Utilizando el cálculo funcional, que es válido por el teorema espectral (recuerde que $A$ es compacto y autoadjunto), defina el operador $e^{-tA}$ y muestre que
    \begin{align*}
      u(t) = e^{-tA} u_0, \quad t > 0,
    \end{align*}
    es solución de la ecuación anterior con $F = 0$ y que $u \in C^1((0,\infty), H)$. ¿Es posible concluir que $u \in C^k((0,\infty), H)$ para todo $k \geq 1$ y además
    \begin{align*}
      \sup_{t \geq 0} \|u(t)\|_H < \infty?
    \end{align*}

    \item[(c.2)] (2 puntos) Muestre que en el caso general (con $F$ no necesariamente nula), la función
    \begin{align*}
      u(t) = e^{-tA} u_0 + \int_0^t e^{-(t - \tau)A} F(\tau)\, d\tau,
    \end{align*}
    pertenece a $C^1((0,\infty), H)$, es solución de la ecuación. ¿Bajo qué condiciones sobre $F$ puede concluir que para un $k \geq 1$ entero dado, $u \in C^k((0,\infty), H)$ y además
    \begin{align*}
      \sup_{t \geq 0} \|u(t)\|_H < \infty?
    \end{align*}
  \end{enumerate}
\end{enumerate}
\begin{proof} 
  \begin{enumerate}
    \item[(a)] Veamos que si tomamos $k\geq 0$ entero, entonces $C^{k}(J;H)$ es un espacio de Banach con la norma
      \begin{align*}
        \norm{u}_{C^{k}}=\sum_{l=0}^{k}\sup_{t\in J}\norm{u^{(l)}(t)}_{H},
      \end{align*}
      Primero veamos que $\norm{\cdot}_{C^{k}}$ en efecto es una norma bien definida.\\
      Note que como $J$ es ujn intervalo cerrado y acotado, entonces dada $u\in C^{k}(J;H)$ se tiene que $u$ y todas sus derivadas alcanzan su máximo en $J$, por lo que en efecto la suma finita de los supremos de las derivadas de $u$ se encuentra bien definida.
      Ahora verifiquemos las condiciones de norma, note que dadas $u,v\in C^{k}(J;H)$ y $\lambda$ escalar se tiene que
      \begin{align*}
        \norm{u+\lambda v}_{C^{k}}&=\sum_{l=0}^{k}\sup_{t\in J}\norm{(u+\lambda v)^{(l)}(t)}_{H},\\
        &=\sum_{l=0}^{k}\sup_{t\in J}\norm{u^{(l)}(t)+\lambda v^{(l)}(t)}_{H},\\
        &\leq\sum_{l=0}^{k}\sup_{t\in J}\norm{u^{(l)}(t)}_{H}+|\lambda| \norm{v^{(l)}(t)}_{H},\\
        &\leq\sum_{l=0}^{k}\sup_{t\in J}\norm{u^{(l)}(t)}_{H}+|\lambda|\sup_{t\in J} \norm{v^{(l)}(t)}_{H},\\
        &\leq\sum_{l=0}^{k}\sup_{t\in J}\norm{u^{(l)}(t)}_{H}+|\lambda|\sum_{l=0}^{k}\sup_{t\in J} \norm{v^{(l)}(t)}_{H},\\
        &=\norm{u}_{C^{k}}+|\lambda|\norm{v}_{C^{k}}.
      \end{align*}
      Por otro lado note que $u=0$ sí y sólo si $\norm{u}_{H}=0$, lo que sucede si y sólo si el cociente $\dfrac{u(t+h)-u(t)}{h}=0$ en $H$ para todo $t$ y $h$, lo que a su vez se da si y sólo si $u'=0$, inductivamente se llega a que $u^{(l)}=0$ para todo $0\leq l\leq k$, lo que se cumple si y sólo si $\norm{u}_{C^{k}}=0$, lo que nos permite concluir que $\norm{\cdot}_{C^{k}}$ en efecto es una norma bien definida.\\
      Ahora veamos que el espacio antes mencionado es completo, es decir, dada $\{u_{m}\}\subset C^{k}(J:H)$ sucesión de Cauchy esta converge en $C^{k}(J;H)$.\\
      Note que dado $\epsilon>0$ existe $N>0$ tal que si $n,m>N$ entonces se satisface que
      \begin{align*}
        \norm{u_{n}-u_{m}}_{C^{k}}\leq \epsilon.
      \end{align*}
      Pero note que esto es lo mismo que
      \begin{align*}
        \norm{u_{n}-u_{m}}_{C^{k}}=\sum_{l=0}^{k}\sup_{t\in J}\norm{u_{n}(t)-u_{m}(t)}_{H}&\leq \epsilon.
      \end{align*}
      Lo que implica que para todo $t\in J$ y todo $0\leq l\leq k$ se satisface que
      \begin{align*}
        \norm{u_{n}^{(l)}(t)-u_{m}^{(l)}(t)}_{H}\leq \epsilon.
      \end{align*}
      Pero como $H$ es un espacio de Hilbert, sabemos que la sucesión $\{u_{m}^{(l)}(t)\}\subset H$ de Cauchy, converge a un $u^{(l)}(t)$ cuando $m\to\infty$.\\
      Por practicidad, veamos que en efecto $u'=u^{(1)}$, las demás derivadas se pueden razonar de forma inductiva.\\
      Note que como
      \begin{align*}
        \sup_{t\in J}\norm{u_{m}(t)-u(t)}_{H}\leq \epsilon,
      \end{align*}
      entonces $\{u_{m}\}$ converge uniformemente a $u$, por lo que podremos hacer el siguiente cálculo cambiando el orden de los límites 
      \begin{align*}
        u'(t)&=\lim_{h \to 0}\frac{u(t+h)-u(t)}{h},\\
        &=\lim_{h \to 0}\lim_{m \to \infty}\frac{u_{m}(t+h)-u_{m}(t)}{h},\\
        &=\lim_{m \to \infty}\lim_{h \to 0}\frac{u_{m}(t+h)-u_{m}(t)}{h},\\
        &=\lim_{m \to \infty}u_{m}^{(1)},\\
        &=u^{(1)}(t).
      \end{align*}
      Luego $u_{m}^{(l)}(t)\to u^{(l)}(t)$ en $H$ para todo $0\leq l\leq k$ y para cada $t\in J$.\\
      Veamos que esto implica convergencia en $C^k(J;H)$.\\
      Note que dado $\epsilon>0$ se puede tomar un $N>0$ adecuado para el cual si tomamos $n,m>N$ se cumple que 
      \begin{align*}
        \norm{u_{m}-u}_{C^{k}}&=\sum_{l=0}^{k}\sup_{t\in J}\norm{u_{m}^{(l)}(t)-u^{(l)}(t)}_{H},\\
        &=\sum_{l=0}^{k}\sup_{t\in J}\lim_{n \to \infty}\norm{u_{m}^{(l)}(t)-u_{n}^{(l)}(t)}_{H},\\
        &=\lim_{n \to \infty}\sum_{l=0}^{k}\sup_{t\in J}\norm{u_{m}^{(l)}(t)-u_{n}^{(l)}(t)}_{H},\\
        &\leq\lim_{n \to \infty}\epsilon,\\
        &\leq \epsilon.
      \end{align*}
      Lo que nos permite concluir que $(C^{k}(J;H),\norm{\cdot}_{C^{k}})$ es un espacio de Banach.
    \item[(b)] Sean $\mathcal{Z}=\{t_0,t_{1},\cdots,t_{n}\}$ y $\mathcal{Z}'=\{s_0,s_{1},\cdots,s_{m}\}$ particiones del intervalo $[a,b]$, veamos que dado $\epsilon>0$ existe $N>0$ tal que si $|\mathcal{Z}|,|\mathcal{Z}'|< N$, entonces 
    \begin{align*}
      \norm{S(f,\mathcal{Z})-S(f,\mathcal{Z}')}_{H}<\epsilon. 
    \end{align*}
    Para ver esto suponga $\mathcal{Z}''=\mathcal{Z}\cup\mathcal{Z}'=\{q_{0},q_{1},\cdots,q_{l}\}$, note que
    \begin{align*}
      \norm{S(f,\mathcal{Z})-S(f,\mathcal{Z}')}_{H}\leq\norm{S(f,\mathcal{Z})-S(f,\mathcal{Z}'')}_{H}+\norm{S(f,\mathcal{Z''})-S(f,\mathcal{Z}')}_{H}
    \end{align*}
    Luego
    \begin{align*}
      \norm{S(f,\mathcal{Z})-S(f,\mathcal{Z''})}_{H}&=\norm{\sum_{j=1}^{n}F(t^{*}_{j})(t_{j}-t_{j-1})-\sum_{j=1}^{l}F(q^{*}_{j})(q_{j}-q_{j-1})}_{H},
    \end{align*}
    Note que como $\mathcal{Z}\subset \mathcal{Z}''$, entonces sabemos que existe $r_{j}$ tal que
    \begin{align*}
      [t_{j-1},t_{j}]&=\bigcup_{i=0}^{r_{j}} [q_{j-1,i},q_{j,i}] &&\text{con $q_{j,i}\in \mathcal{Z}''$.}
    \end{align*}
    De lo que podemos computar que
    \begin{align*}
      F(t^{*}_{j})(t_{j}-t_{j-1})-\sum_{i=0}^{r_{j}}F(q^*_{j,i})(q_{j,i}-q_{j-1,i})&=\sum_{i=0}^{r_{j}}(F(t^*_{j})-F(q^*_{j,i}))(q_{j,i}-q_{j-1},i),
    \end{align*}
    además, recuerde que como $F$ es uniformemente continua en $[a,b]$, dado $\epsilon>0$ existe $N>0$ tal que si $|t-q|<N$, entonces
    \begin{align*}
      \norm{F(t)-F(q)}_{H}< \frac{\epsilon}{2(b-a)}. 
    \end{align*}
    Si suponemos que $|\mathcal{Z}''|<|\mathcal{Z}|<N$, entonces
    \begin{align*}
      \norm{S(f,\mathcal{Z})-S(f,\mathcal{Z''})}_{H}&=\norm{\sum_{j=1}^{n}F(t_{j}^{*})(t_{j}-t_{j-1})-\sum_{j=1}^{l}F(q^{*}_{j})(q_{j}-q_{j-1})}_{H},\\
      &=\norm{\sum_{j=1}^{n}\sum_{i=0}^{r_{j}}(F(t^{*}_{j})-F(q^{*}_{j,i}))(q_{j,i}-q_{j-1,i})}_{H},\\
      &\leq \sum_{j=1}^{n}\sum_{i=0}^{r_{j}}(q_{j,i}-q_{j-1,i})\norm{F(t^{*}_{j})-F(q^{*}_{j,i})}_{H},\\
      &\leq \frac{\epsilon}{2(b-a)} \sum_{j=1}^{n}(t_{j}-t_{j-1}),\\
      &\leq \frac{\epsilon}{2(b-a)} (b-a)=\frac{\epsilon}{2}.
    \end{align*}
    Análogamente, si suponemos $|\mathcal{Z}''|<|\mathcal{Z}'|<N$ podemos asegurar que
    \begin{align*}
      \norm{S(f,\mathcal{Z}'')-S(f,\mathcal{Z}')}_{H}\leq \frac{\epsilon}{2}.
    \end{align*}
    Luego podemos asegurar que dado $\epsilon>0$ existe $N>0$ tal que si $|\mathcal{Z}|,|\mathcal{Z}'|<N$, entonces
    \begin{align*}
      \norm{S(f,\mathcal{Z})-S(f,\mathcal{Z}')}_{H}&\leq\norm{S(f,\mathcal{Z})-S(f,\mathcal{Z}'')}_{H}+\norm{S(f,\mathcal{Z''})-S(f,\mathcal{Z}')}_{H},\\
      &\leq \frac{\epsilon}{2}+\frac{\epsilon}{2},\\
      &\leq \epsilon.
    \end{align*}
    Lo que nos permite concluir que $\{S(f,\mathcal{Z})\}\subset H$ es una sucesión de Cauchy, luego como $H$ es Hilbert (por ende completo) sabemos que converge a alguien que denotaremos $\int_{a}^{b}F(\tau)d\tau\in H$.\\
    Para ver la desigualdad note que
    \begin{align*}
      \norm{\int_{a}^{b}F(\tau)\, d\tau}_{H}&=\norm{\lim_{|\mathcal{Z}| \to 0}S(f,\mathcal{Z})},\\
      &\leq \lim_{|Z| \to 0}\sum_{j=1}^{n}\norm{F(t^{*})}_{H}(t_{j}-t_{j-1}),\\
      &\leq \int_{a}^{b}\norm{F(\tau)}_{H}\, d\tau.
    \end{align*}
  \item[(c.1)]Veamos que $u(t)=e^{-tA}u_{0}$, $t>0$ es solución de
\[
\begin{cases}
u^{\prime}(t)=-Au(t)\\
u(0)=u_{0}
\end{cases}
\]
En primer lugar, como $A\ge0$, se sigue que si $\lambda_n \in EV(A)$ con $v\ne0$ vector propio asociado (Esto es, $v\ne 0$ y $A~v=\lambda_{n}v)$ entonces
\[0\le(Av,v)=(\lambda_{n}v,v)=\lambda n||v||^{2}\]
De modo que $\lambda_{n}\ge0$. \\
Ahora, como H es de Hilbert, separable con $T\in K(H)$ y $T=T^{\star}$. Sea $\{\phi_n\}$ base de Hilbert de modo que
\[u(t)=e^{-tA}u_{0}=\sum_{n=0}^{\infty}(u_{0},\phi_{n})e^{-t\lambda_{n}}\phi_{n}\]

Queremos ver que
\[\lim_{h\rightarrow0}\norm{\frac{u(t+h)-u(t)}{h}+Au(t)}\]
Para ello, consideramos
\begin{align*}
\norm{\frac{u(t+h)-u(t)}{h}+Au(t)}^2&=\norm{\sum_{n=1}^{\infty}(u_{0},\phi_{n})\left(\frac{e^{-(t+h)\lambda_{n}}-e^{-t\lambda_{n}}}{h}+\lambda_{n}e^{-t\lambda_{n}}\right)\phi_{n}}^{2}\\&=\sum_{n=1}^{\infty}|(u_{0},\phi_{n})|^{2}\left|\frac{e^{-(t+h)\lambda_{n}}-e^{-t\lambda_{n}}}{h}+\lambda_{n}e^{-t\lambda_{n}}\right|^{2},
\end{align*}
donde la última igualdad se tiene por Bessel-Parseval.
Veamos que esta converge uniformemente.
\begin{align*}
  \left|\frac{e^{-(t+h)\lambda_{n}}-e^{-t\lambda_{n}}}{h}+\lambda_{n}e^{-t\lambda_{n}}\right|
&=\left|\frac{-\lambda_{n}}{h}\int_{t}^{t+h}e^{-\sigma\lambda_{n}}d\sigma+\lambda_{n}e^{-t\lambda_{n}}\right| \\
&\le\left|\frac{\lambda_{n}}{h}\right|\left|\int_{t}^{t+h}e^{-\sigma\lambda_{n}}d\sigma\right|+\left|\lambda_{n}\right|e^{-t\lambda_{n}}.  
\end{align*}


Como $t>0$, $\lambda_n \ge 0$, $-t\lambda_{n}\le0$. Así, $0\le e^{-t\lambda_n}\le1$.
Ahora consideremos los siguientes casos:
\begin{enumerate}
    \item[$\checkmark$] Si $h>0$, tomando $\sigma \in [t, t+h]$ se sigue que \[0\le e^{-\sigma\lambda_{n}}\le e^{-t\lambda n}\le1.\]
    \item[$\checkmark$] Si $h<0$, con $|h|$ suficientemente pequeño tal que $t+h>0$, tomando $\sigma \in [t+h, t]$ se sigue que \[0\le e^{-\sigma\lambda_{n}}\le e^{-(t+h)\lambda_{n}}\le1. \]
\end{enumerate}
Así, en cualquier caso, se tiene que
\begin{align*}
\left|\frac{\lambda_n}{h}\right|\left|\int_{t}^{t+h}e^{-\sigma\lambda_n}d\sigma\right|+|\lambda_n|e^{-t\lambda_n}
&\le\left|\frac{\lambda_{n}}{h}\right|\left|\int_{t}^{t+h}d\sigma\right|+|\lambda_{n}|
\\&=\left|\frac{\lambda_{n}(t+h-t)}{h}\right|+|\lambda_{n}|\\&=2\lambda_{n}\\&\le2\sup_{n\ge1}\lambda_{n}.
\end{align*}
De modo que
\begin{align}
|(u_{0},\phi_{n})|^{2}\left|\frac{e^{-(t+h)\lambda_{n}}-e^{-t\lambda_{n}}}{h}+\lambda_{n}e^{-t\lambda_{n}}\right|^{2}\le|(u_{0},\phi_{n})|^{2}(2\sup_{n\ge1}\lambda_{n})^{2}, \label{1}    
\end{align}

y como $\displaystyle\sum_{n=1}^{\infty}|(u_{0},\phi_{n})|^{2}(2\sup_{n\ge1}\lambda_{n})^{2}$ converge, pues $\displaystyle \sum_{n=1}^{\infty}|(u_{0},\phi_{n})|^{2}=||u_0||^2$, entonces por criterio M de Weierstrass,
\[\norm{\frac{u(t+h)-u(t)}{h}+Au(t)}^{2}=\sum_{n=1}^{\infty}|(u_{0},\phi_{n})|^{2}\left|\frac{e^{-(t+h)\lambda_{n}}-e^{-t\lambda_{n}}}{h}+\lambda_{n}e^{-t\lambda_{n}}\right|^{2}, \]
converge uniformemente. Por lo tanto
\begin{align*}
\lim_{h\rightarrow0}\norm{\frac{u(t+h)-u(t)}{h}+Au(t)}^{2}&=\lim_{h\rightarrow0}\sum_{n=1}^{\infty}|(u_{0},\phi_{n})|^{2}\left|\frac{e^{-(t+h)\lambda_{n}}-e^{-t\lambda_{n}}}{h}+\lambda_{n}e^{-t\lambda_{n}}\right|^{2} \\&=\sum_{n=1}^{\infty}|(u_{0},\phi_{n})|^{2}\left|\lim_{h\rightarrow0}\frac{e^{-(t+h)\lambda_{n}}-e^{-t\lambda_{n}}}{h}+\lambda_{n}e^{-t\lambda_{n}}\right|^{2}
\\&=\sum_{n=1}^{\infty}|(u_{0},\phi_{n})|^{2}|-\lambda_{n}e^{-t\lambda_{n}}+\lambda_{n}e^{-t\lambda_{n}}|^{2}=0. \end{align*}

Veamos ahora que $u(t)$ es continua. Para ello, mostremos que, 
\[\lim_{t'\rightarrow t}||u(t)-u(t')||=0.\]
Como
\begin{align*}
\norm{u(t)-u(t^{\prime})}^{2}&=\norm{\sum_{n=1}^{\infty}(\phi_{n},u_{0})(e^{-\lambda_{n}t}-e^{-\lambda_{n}t^{\prime}})\phi_{n}}^{2}
\\&=\sum_{n=1}^{\infty}|(\phi_{n},u_{0})|^{2}|e^{-\lambda_{n}t}-e^{-\lambda_{n}t^{\prime}}|^{2}
\\&=\sum_{n=1}^{\infty}|(\phi_{n},u_{0})|^{2}\left|\lambda_{n}e^{-\lambda_{n}\xi_{n}}|t-t^{\prime}|\right|^{2}.
\end{align*}
En donde en la última igualdad hemos usado teorema de valor medio, por lo que existe $\xi_n \in (t, t')$ (o $\xi_n \in (t',t)$ en el caso que $0<t^{\prime}<t)$ tal que
\[|\lambda_n e^{-\lambda_n \xi_n} (t-t')| = |e^{-\lambda_n t} - e^{-\lambda_n t'}|.\]

Ahora, como $0<\xi_n$ para todo $n$ y $\lambda_{n}\ge0$, se sigue entonces que \[e^{-\lambda_n \xi_n}\le1.\]
De modo que
\begin{align*}
\norm{u(t)-u(t^{\prime})}^{2}&=\sum_{n=1}^{\infty}|(\phi_{n},u_{0})|^{2}\left|\lambda_{n}e^{-\lambda_{n}\xi_{n}}|t-t^{\prime}|\right|^{2}
\\&\le\sum_{n=1}^{\infty}|(\phi_{n},u_{0})|^{2}|\lambda_{n}|t-t^{\prime}||^{2}
\\&=|t-t^{\prime}|^2\sum_{n=1}^{\infty}|(\phi_{n},u_{0})\lambda_{n}|^{2},
\end{align*}
y como $\displaystyle\sum_{n=1}^{\infty}|(\phi_n, u_0)\lambda_n|^2$ converge, entonces
\[\norm{u(t)-u(t^{\prime})}^{2}\le|t-t^{\prime}|^2\sum_{n=1}^{\infty}|(\phi_{n},u_{0})\lambda_n|^{2}\xrightarrow{t\rightarrow t^{\prime}}0,\]
por lo que $u(t)$ es continua.

Finalmente, dado que \[u^{\prime}(t)=\lim_{h\rightarrow0}\frac{u(t+h)-u(t)}{h}\] existe y $u^{\prime}(t)=-Au(t)$, se sigue que como $u:(0,\infty)\rightarrow H$ es continua, dado $\varepsilon>0$ existe $\delta>0$ tal que si $|t-t^{\prime}|<\delta$ entonces $||u(t)-u(t^{\prime})||<\varepsilon$.\\
Como $A$ es acotado, existe $M>0$ tal que $||Au||\le M||u||$.
Por lo tanto, \[\norm{-A~u(t)+A~u(t^{\prime})}=\norm{A(u(t)-u(t')}\le M||u(t)-u(t^{\prime})||<M\varepsilon,\] lo cual muestra que $u^{\prime}(t)=-Au(t)$ es continuo. \\
Por inducción, supongamos que $u^{(k)}(t)=(-1)^{k}A^{k}e^{-tA}u_0$. Veamos que \[\displaystyle u^{(k+1)}(t)=(-1)^{k+1}A^{k+1}e^{-tA}u_0 (t>0).\]
Para ello, tenemos
\begin{align*}   
&\norm{\frac{u^{(k)}(t+h)-u^{(k)}(t)}{h}-(-1)^{k+1}A^{k+1}e^{-tA}u_0}^{2}
\\&=\norm{\sum_{n=1}^{\infty}(u_{0},\phi_{n})\left[\left(\frac{(-1)^{k}\lambda_{n}^{k}e^{-(t+h)\lambda_{n}}-(-1)^{k}\lambda_{n}^{k}e^{-t\lambda_{n}}}{h}\right)-(-1)^{k+1}\lambda_{n}^{k+1}e^{-t\lambda_{n}}\right]\phi_{n}}^{2}
\\&=\sum_{n=1}^{\infty}|(u_{0},\phi_{n})|^{2}\left|\left(\frac{(-1)^{k}{\lambda_{n}}^{k}e^{-(t+h)\lambda_{n}}-(-1)^{k}\lambda_{n}^{k}e^{-t\lambda_{n}}}{h}\right)-(-1)^{k+1}\lambda_{n}^{k+1}e^{-t\lambda_{n}}\right|^{2}\\&=\sum_{n=1}^{\infty}|(u_{0},\phi_{n})\lambda_{n}^{k}|^{2}\left|\left(\frac{e^{-(t+h)\lambda_{n}}-e^{-t\lambda_n}}{h}\right)+\lambda_{n}e^{-t\lambda_n}\right|^{2}.
\end{align*}
Por \ref{1}, entonces 
\[|(u_{0},\phi_{n})\lambda_{n}^{k}|^{2}\left|\left(\frac{e^{-(t+h)\lambda_n}-e^{-t\lambda_n}}{h}\right)+\lambda_{n}e^{-t\lambda_n}\right|^{2}
\le\left|(u_{0},\phi_{n})\sup_{n\geq 1}\lambda_n^k\right|^{2}(2\sup_{n\ge1}\lambda_{n})^{2}.\]
Entonces por criterio M de Weierstrass,
\[\norm{\frac{u^{(k)}(t+h)-u^{(k)}(t)}{h}-(-1)^{k+1}A^{k+1}e^{-tA}u_0}^{2} \quad \text{converge uniformemente.}\] 
Por lo tanto,
\begin{align*}
&\lim_{h\rightarrow0}\norm{\frac{u^{(k)}(t+h)-u^{(k)}(t)}{h}-(-1)^{k+1}A^{k+1}e^{-tA}u_0}^{2}
\\&=\lim_{h\rightarrow0}\sum_{n=1}^{\infty}|(u_{0},\phi_{n})|^{2}\left|\left(\frac{(-1)^{k}{\lambda_{n}}^{k}e^{-(t+h)\lambda_{n}}-(-1)^{k}\lambda_{n}^{k}e^{-t\lambda_{n}}}{h}\right)-(-1)^{k+1}\lambda_{n}^{k+1}e^{-t\lambda_{n}}\right|^{2}\\&=\sum_{n=1}^{\infty}|(u_{0},\phi_{n})|^{2}\left|\lim_{h\rightarrow0}\left(\frac{(-1)^{k}{\lambda_{n}}^{k}e^{-(t+h)\lambda_{n}}-(-1)^{k}\lambda_{n}^{k}e^{-t\lambda_{n}}}{h}\right)-(-1)^{k+1}\lambda_{n}^{k+1}e^{-t\lambda_{n}}\right|^{2}\\&=\sum_{n=1}^{\infty}|(u_{0},\phi_{n})|^{2}\left|(-1)^{k}\lambda_n^k(-\lambda_n)e^{-t\lambda_n}-(-1)^{k+1}\lambda_n^{k+1}e^{-t\lambda_n}\right|^{2}=0.
\end{align*}

Ahora, veamos que $u^{(k)}(t)$ es continua, es decir, que \[\lim_{t^{\prime}\rightarrow t}||u^{(k)}(t)-u^{(k)}(t^{\prime})||=0.\]
Para ello
$\norm{u^{(k)}(t)-u^{(k)}(t^{\prime})}=\norm{(-1)^{k}A^{k}u(t)-(-1)^{k}A^{k}u(t^{\prime})}$
Como $u(t)$ es continuo, dado $\varepsilon>0$ existe $\delta>0$ tal que si $|t-t^{\prime}|<\delta$ entonces $||u(t)-u(t^{\prime})||<\varepsilon$.
Ahora como $A$ es acotado, existe $M>0$ tal que $\norm{Au}\le M\norm{u}$.
Por lo tanto
\begin{align*}
 \norm{(-1)^{k}A^{k}u(t)-(-1)^{k}A^{k}u(t^{\prime})}=\norm{A^{k}\left(u(t)-u(t^{\prime})\right)} \le M^k \norm{u(t)-u(t')} < M^k \varepsilon.   
\end{align*}
Con esto, tenemos que $u\in C^{k}((0,\infty),H)$ para todo $k\ge1$.

Como hemos visto
\[\sup_{t>0}\norm{u(t)}=\sup_{t>0}\norm{\sum_{n=0}^{\infty}(u_{0},\phi_{n})e^{-\lambda_{n}t}\phi_{n}}.\]
Ahora,
\begin{align*}
\norm{\sum_{n=0}^{\infty}(u_{0},\phi_{n})e^{-\lambda_{n}t}\phi_{n}}^{2}=\sum_{n=0}^{\infty}|(u_{0},\phi_{n})|^{2}\left|e^{-\lambda_{n}t}\right|^{2}
\le\sum_{n=0}^{\infty}|(u_{0},\phi_{n})|^{2}=||u_0||^2.
\end{align*}
Por lo tanto, $||u(t)||<\infty$ para todo $t\ge0$, de lo que se sigue que \[\sup_{t\ge0}\norm{u(t)}<\infty.\]



    \item[(c.2)] Debemos verificar que
    \begin{align*}
        \lim_{h\to 0}\norm{\dfrac{u(t+h)-u(t)}{h}+Au(t)-F(t)}_H=0,
    \end{align*}
    para todo $t>0$, donde
    \begin{align*}
        u(t)=e^{-tA}u_0+\int_0^t e^{-(t-\tau)}F(\tau)\, d\tau.
    \end{align*}
    Tomando $h>0$ tenemos estimar
    \begin{align*}
        S_h=\norm{\frac{1}{h}\left[e^{-(t+h)A}u_0+\int_0^{t+h}e^{-(t+h-\tau)A}F(\tau)\, d\tau-e^{-tA}u_0-\int_0^t e^{-(t-\tau)A}F(\tau)\, d\tau\right]+Au(t)-F(t)}_H,
    \end{align*}
    Por lo hecho en el ítem (b) y como $h>0$, se tiene que 
    \begin{align*}
        \int_{0}^{t+h}e^{-(t+h-\tau)A}F(\tau)\, d\tau=\int_{0}^{t}e^{-(t+h-\tau)A}F(\tau)\, d\tau+\int_{t}^{t+h}e^{-(t+h-\tau)A}F(\tau)\, d\tau,
    \end{align*}
    de manera que
    \begin{align*}
        \int_{0}^{t+h}e^{-(t+h-\tau)A}F(\tau)\, d\tau-\int_0^t e^{-(t-\tau)A}F(\tau)\, d\tau=\int_{0}^{t}\left[e^{-(t+h-\tau)A}F(\tau)-e^{-(t-\tau)A}F(\tau)\right]\, d\tau\\
        \hspace{25mm}+\int_{t}^{t+h}e^{-(t+h-\tau)A}F(\tau)\, d\tau,
    \end{align*}
    además, por la definición de $u(t)$ y la construcción de la integral por sumas de Riemann, obtenemos
    \begin{align*}
        Au(t)=Ae^{-tA}u_0+A\int_0^t e^{-(t-\tau)A}F(\tau)\, d\tau=Ae^{-tA}u_0+\int_0^t Ae^{-(t-\tau)A}F(\tau)\, d\tau
    \end{align*}
    Haciendo uso de la desigualdad triangular de la norma $\norm{\cdot}_H$, tenemos que 
    \begin{align*}
        S_h\leq S_h^1+S_h^2+S_h^3,
    \end{align*}
    donde
    \begin{align*}
        S_h^1&=\norm{\dfrac{1}{h}\left[e^{-(t+h)A}u_0-e^{-tA}u_0\right]+Aue^{-tA}u_0}_H\\
        S_h^2&=\norm{\dfrac{1}{h}\int_{0}^{t}\left[e^{-(t+h-\tau)A}F(\tau)-e^{-(t-\tau)A}F(\tau)\right]\, d\tau+\int_0^t Ae^{-(t-\tau)A}F(\tau)\, d\tau}_H\\
        S_h^3&=\norm{\dfrac{1}{h}\int_{t}^{t+h}e^{-(t+h-\tau)A}F(\tau)\, d\tau-F(t)}_H.
    \end{align*}
    Note que el término $S_h^1$ es exáctamente el término que estimamos cuando realizamos el sistema homogéneo, así, podemos afirmar que $\displaystyle\lim_{h\to 0}S_h^1=0$. Para $S_h^2$, note que, por la linealidad de la integral y la desigualdad triangular de la integral
    \begin{align*}
        S_h^2&=\norm{\int_0^t\left[\dfrac{e^{-(t+h-\tau)A}-e^{-(t-\tau)A}}{h}+Ae^{-(t-\tau)A}\right]F(\tau)\, d\tau}_H\\
        &\leq \int_0^t\norm{\left[\dfrac{e^{-(t+h-\tau)A}-e^{-(t-\tau)A}}{h}+Ae^{-(t-\tau)A}\right]F(\tau)}_H\, d\tau,
    \end{align*}
    así, por la identidad de Parseval y aplicando el cálculo funcional que nos permite definir el Teorema Espectral, tenemos
    \begin{align*}
        \norm{\left[\dfrac{e^{-(t+h-\tau)A}-e^{-(t-\tau)A}}{h}+Ae^{-(t-\tau)A}\right]F(\tau)}_H^2\\
        =\sum_{n=1}^\infty (F(\tau),\phi_n)^2\left|\dfrac{e^{-(t+h-\tau)\lambda_n}-e^{-(t-\tau)\lambda_n}}{h}+\lambda_ne^{-(t-\tau)\lambda_n}\right|^2, 
    \end{align*}
    donde $\{\phi_n\}$ es la base de Hilbert de vectores propios de $A$ y $\{\lambda_n\}$ es el espectro del operador $A$.
    Como $(Ax,x)\geq 0$ para todo $x \in H$, $\lambda_n\geq 0$ para todo $n\in \mathbb{Z}^+$. Note que
    \begin{align*}
        \dfrac{e^{-(t+h-\tau)\lambda_n}-e^{-(t-\tau)\lambda_n}}{h}=\dfrac{-\lambda_n}{h}\int_{t}^{t+h}e^{-(s-\tau)\lambda_n}\, ds,
    \end{align*}
    como $0\leq \tau\leq t$ y $t\leq s\leq t+h$, tenemos que $\tau\leq s$, es decir, $s-\tau \geq 0$, por lo que, como $\lambda_n\geq 0$, $-(s-\tau)\lambda_n$, de manera que $e^{-(s-\tau)\lambda_n}\leq 1$, de la misma manera, $t-\tau\geq 0$, entonces $e^{-(t-\tau)\lambda_n}\leq 1$, por tanto
    \begin{align*}
        \left|\dfrac{e^{-(t+h-\tau)\lambda_n}-e^{-(t-\tau)\lambda_n}}{h}+\lambda_ne^{-(t-\tau)\lambda_n}\right|&\leq \left|\dfrac{e^{-(t+h-\tau)\lambda_n}-e^{-(t-\tau)\lambda_n}}{h}\right|+\left|\lambda_ne^{-(t-\tau)\lambda_n}\right|\\
        &=\left|\dfrac{-\lambda_n}{h}\int_{t}^{t+h}e^{-(s-\tau)\lambda_n}\, ds\right|+\left|\lambda_ne^{-(t-\tau)\lambda_n}\right|\\
        &\leq \dfrac{|\lambda_n|}{h}\int_t^{t+h}\, ds+|\lambda_n|\\
        &=2|\lambda_n|\\
        &\leq 2\sup_{n \in \mathbb{Z}^+}|\lambda_n|<\infty,
    \end{align*}
    de manera que 
    \begin{align*}
        \sum_{n=1}^\infty (F(\tau),\phi_n)^2\left|\dfrac{e^{-(t+h-\tau)\lambda_n}-e^{-(t-\tau)\lambda_n}}{h}+\lambda_ne^{-(t-\tau)\lambda_n}\right|^2\leq \sum_{n=1}^\infty (F(\tau),\phi_n)^2\left(2\sup_{n \in \mathbb{Z}^+}|\lambda_n|\right)^2<\infty,
    \end{align*}
    dado que $F(\tau)\in H$ para todo $\tau \in \mathbb{R}$, así, por el criterio $M$ de Weierstrass, podemos ``meter'' el límite dentro de la serie. Como
    \begin{align*}
        \lim_{h\to 0}\dfrac{e^{-(t+h-\tau)\lambda_n}-e^{-(t-\tau)\lambda_n}}{h}=-\lambda_ne^{-(t-\tau)\lambda_n},
    \end{align*}
    tenemos que 
    \begin{align*}
        \lim_{h\to 0}\sum_{n=1}^\infty (F(\tau),\phi_n)^2\left|\dfrac{e^{-(t+h-\tau)\lambda_n}-e^{-(t-\tau)\lambda_n}}{h}+\lambda_ne^{-(t-\tau)\lambda_n}\right|^2\\
        =\sum_{n=1}^\infty (F(\tau),\phi_n)^2\lim_{h\to 0}\left|\dfrac{e^{-(t+h-\tau)\lambda_n}-e^{-(t-\tau)\lambda_n}}{h}+\lambda_ne^{-(t-\tau)\lambda_n}\right|^2=0,
    \end{align*}
    lo que nos permite concluir que $\displaystyle \lim_{h\to 0}S_h^2=0$. 

    Finalmente, para $S_h^3$, escribimos
    \begin{align*}
        \dfrac{1}{h}\int_{t}^{t+h}e^{-(t+h-\tau)A}F(\tau)\, d\tau-F(t)&=\dfrac{1}{h}\int_{t}^{t+h}e^{-(t+h-\tau)A}F(\tau)\, d\tau-\dfrac{1}{h}\int_{t}^{t+h} F(t)\, d\tau\\
        &=\dfrac{1}{h}\int_{t}^{t+h}[e^{-(t+h-\tau)A}F(\tau)-F(t)]\, d\tau,
    \end{align*}
    sumando y restando dentro de la integral $e^{-(t+h-\tau)A}F(t)$, tenemos
    \begin{align*}
        S_h^3&=\norm{\dfrac{1}{h}\int_t^{t+h}\left[e^{-(t+h-\tau)A}F(\tau)-e^{-(t+h-\tau)A}F(t)+e^{-(t+h-\tau)A}F(t)-F(t)\right]\, d\tau}_H\\
        &=\norm{\dfrac{1}{h}\int_t^{t+h}\left[e^{-(t+h-\tau)A}(F(\tau)-F(t))+(e^{-(t+h-\tau)A}-I)F(t)\right]\, d\tau}_H\\
        &\leq \dfrac{1}{h}\int_{t}^{t+h}\norm{\left[e^{-(t+h-\tau)A}(F(\tau)-F(t))+(e^{-(t+h-\tau)A}-I)F(t)\right]}_H\, d\tau\\
        &\leq \dfrac{1}{h}\int_{t}^{t+h}\left(\norm{e^{-(t+h-\tau)A}(F(\tau)-F(t))}_H+\norm{(e^{-(t+h-\tau)A}-I)F(t)}_H\right)\, d\tau\\
        &=\underbrace{\dfrac{1}{h}\int_{t}^{t+h}\norm{e^{-(t+h-\tau)A}(F(\tau)-F(t))}_H\, d\tau}_{I_1}+\underbrace{\dfrac{1}{h}\int_{t}^{t+h}\norm{(e^{-(t+h-\tau)A}-I)F(t)}_H\, d\tau}_{I_2}.
    \end{align*}
    Para $I_1$, acotamos la norma 
    \begin{align*}
        \norm{e^{-(t+h-\tau)A}(F(\tau)-F(t))}_H.
    \end{align*}
    Usando el cálculo funcional, y la identidad de Parseval, tenemos
    \begin{align*}
        \norm{e^{-(t+h-\tau)A}(F(\tau)-F(t))}_H^2&=\sum_{n=1}^\infty (F(\tau)-F(t),\phi_n)^2\left|e^{-(t+h-\tau)\lambda_n}\right|^2,
    \end{align*}
    como $t\leq \tau\leq t+h$, entonces $t+h-\tau\geq 0$, de manera que, como $\lambda_n\geq 0$ para todo $n \in \mathbb{Z}^+$, tenemos $e^{-(t+h-\tau)\lambda_n}\leq 1$, de manera que
    \begin{align*}
        \sum_{n=1}^\infty (F(\tau)-F(t),\phi_n)^2\left|e^{-(t+h-\tau)\lambda_n}\right|^2\leq \sum_{n=1}^\infty (F(\tau)-F(t),\phi_n)^2=\norm{F(\tau)-F(t)}_H^2<\infty,
    \end{align*}
    dado que $F(\tau),F(t)\in H$ y $H$ es espacio vectorial. Además, note que como $[t,t+h]$ es un compacto y $F$ es continua, $F$ es uniformemente continua en $[t,t+h]$, de manera que dado $\epsilon>0$ existe $\delta>0$ tal que si $|\tau-t|<\delta$, entonces $\norm{F(\tau)-F(t)}<\epsilon$ y $\delta$ no depende de $\tau$, de esta manera, para $h$ suficientemente pequeño para que $\tau-t\leq h<\delta$, se tiene que
    \begin{align*}
        I_1=\dfrac{1}{h}\int_{t}^{t+h}\norm{e^{-(t+h-\tau)A}(F(\tau)-F(t))}_H\, d\tau\leq \dfrac{1}{h}\int_{t}^{t+h}\norm{(F(\tau)-F(t))}_H\, d\tau<\dfrac{\epsilon}{h}\int_{t}^{t+h}\, d\tau=\epsilon,
    \end{align*}
    es decir, $\displaystyle \lim_{h\to 0}I_1=0$.

    Para $I_2$, escribimos
    \begin{align*}
        I_2&=\dfrac{1}{h}\int_{t}^{t+h}\norm{(e^{-(t+h-\tau)A}-I)F(t)}_H\, d\tau\\
        &=\int_{t}^{t+h}\norm{\dfrac{e^{-(t+h-\tau)A}-I}{h}F(t)}_H\, d\tau
    \end{align*}
    entonces acotamos la norma
    \begin{align*}
        \norm{\dfrac{e^{-(t+h-\tau)A}-I}{h}F(t)}_H
    \end{align*}
    usando el cálculo funcional y que definimos $A^0=I$, tenemos
    \begin{align*}
        \norm{\dfrac{e^{-(t+h-\tau)A}-I}{h}F(t)}_H^2=\sum_{n=1}^\infty(F(t),\phi_n)^2\left|\dfrac{e^{-(t+h-\tau)\lambda_n}-1}{h}\right|^2,
    \end{align*}
    como
    \begin{align*}
        e^{-(t+h-\tau)\lambda_n}-1=-\lambda_n\int_{0}^{t+h-\tau}e^{-s\lambda_n}\, ds,
    \end{align*}
    como $\tau\leq t+h$, entonces $t+h-\tau\geq 0$, de manera que $s\geq 0$ y como $\lambda_n\geq 0$ para todo $n \in \mathbb{Z}^+$ se tiene que $e^{-s\lambda_n}\leq 1$, así, tenemos
    \begin{align*}
        \left|\dfrac{e^{-(t+h-\tau)\lambda_n}-1}{h}\right|=\left|\dfrac{-\lambda_n}{h}\int_{0}^{t+h-\tau}e^{-s\lambda_n}\, ds\right|\leq \dfrac{|\lambda_n|}{h}\int_{0}^{t+h-\tau}\, ds=\dfrac{|\lambda_n|(t+h-\tau)}{h}
    \end{align*}
    Como $t\leq \tau\leq t+h$, tenemos $t-\tau\leq 0$ y por tanto, $t+h-\tau\leq h$, lo que nos garantiza que $\dfrac{t+h-\tau}{h}\leq 1$, así
    \begin{align*}
        \left|\dfrac{e^{-(t+h-\tau)\lambda_n}-1}{h}\right|\leq \dfrac{|\lambda_n|(t+h-\tau)}{h}\leq |\lambda_n|\leq \sup_{n\in \mathbb{Z}^+}|\lambda_n|,
    \end{align*}
    de esta manera
    \begin{align*}
        \sum_{n=1}^\infty(F(t),\phi_n)^2\left|\dfrac{e^{-(t+h-\tau)\lambda_n}-1}{h}\right|^2&\leq \sum_{n=1}^\infty(F(t),\phi_n)^2\left(\sup_{n\in \mathbb{Z}^+}|\lambda_n|\right)^2\\
        &=\left(\sup_{n\in \mathbb{Z}^+}|\lambda_n|\right)^2\norm{F(t)}_H^2<\infty,
    \end{align*}
    de esta manera
    \begin{align*}
        I_2&=\int_{t}^{t+h}\norm{\dfrac{e^{-(t+h-\tau)A}-I}{h}F(t)}_H\, d\tau\\
        &\leq \left(\sup_{n \in \mathbb{Z}^+}|\lambda_n|\right)\norm{F(t)}_H\int_{t}^{t+h}\, d\tau\\
        &= \left(\sup_{n \in \mathbb{Z}^+}|\lambda_n|\right)\norm{F(t)}_Hh,
    \end{align*}
    de manera que 
    \begin{align*}
        0\leq \lim_{h\to 0}I_2\leq \lim_{h\to 0}\left(\sup_{n \in \mathbb{Z}^+}|\lambda_n|\right)\norm{F(t)}_Hh=0,
    \end{align*}
    lo que nos garantiza que $\displaystyle \lim_{h\to 0}I_2=0$, concluyendo así que $\displaystyle \lim_{h\to 0}S_h^3=0$ y así
    \begin{align*}
        \lim_{h\to 0} S_h\leq \lim_{h\to 0}(S_h^1+S_h^2+S_h^3)=0,
    \end{align*}
    concluyendo así, que 
    \begin{align*}
        u(t)=e^{-tA}u_0+\int_0^t e^{-(t-\tau)}F(\tau)\, d\tau,
    \end{align*}
    es solución del problema de valor inicial. El caso para $h<0$ es análogo.

    %%%%%%%%%%%%%%%%%%%%%%%%%%%%%%%%%%%%%%%%%%%%%
    Veamos que si $F\in C^{k}[(0,\infty);H]$, entonces $u\in C^{k+1}[(0,\infty);H]$.\\
    Note que como vimos anteriormente, la solución $u$ satisface la ecuación:
    \begin{align*}
        u'(t)=-Au(t)+F(t).
    \end{align*}
    Ahora, como $F\in C^{k}[(0,\infty);H]$ sabemos que el cociente diferencial respectivo de las $k$ derivadas de $k$ en norma converge, de igual forma sabemos que $u\in C^{1}[(0,\infty);H]$ por lo anteriormente demostrado, por lo que sería válido afirmar que se satisface lo siguiente:
    \begin{align*}
        u''(t)=-Au'(t)+F'(t).
    \end{align*}
    Luego, si hacemos $v=u'$ y $G=F'$, entonces podemos ver que $v$ satisface
    \begin{align*}
        v'(t)=-Av(t)+G(t).
    \end{align*}
    al cual lo podemos tratar de la misma forma que a $u$, por lo que razonando de forma inductiva se deduce que
    \begin{align*}
        u^{(k+1)}(t)=-Au^{(k)}(t)+F^{(k)}(t),
    \end{align*}
    lo que nos permite concluir que si $F\in C^{k}[(0,\infty);H]$, entonces $u\in C^{k+1}[(0,\infty);H]$.
  \end{enumerate} 
\end{proof}
