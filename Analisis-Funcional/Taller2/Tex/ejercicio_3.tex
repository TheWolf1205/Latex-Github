\begin{homeworkProblem}
  Sea $(E,\norm{\cdot}_{E})$ y $(F,\norm{\cdot}_{F})$ espacios de Banach.
  \begin{enumerate}[(i)]
    \item Sea $K\subset E$ un subespacio cerrado de $E$. Definimos la relación sobre $E$ dada por $x\sim_{K} y$ si y solo si $x-y\in K$.
    \begin{enumerate}[(a)]
      \item Muestre que $\sim_{K}$ es una relación de equivalencia sobre $E$.
      \item Muestre que el espacio cociente $E/K$ es un espacio de Banach con la norma
        \begin{align*}
          \norm{x+K}_{E/K}=\inf_{k\in K}\norm{x-k},\quad x\in E.
        \end{align*}
        Es decir, debe verificar que el espacio cociente es un espacio vectorial normado, cuya norma lo hace completo.
    \end{enumerate}
    \item Sea $T\in L(E,W)$ tal que existe $c>0$ para el cual
      \begin{align*}
        \norm{Tx}_{F}\geq c\norm{x}_{E},
      \end{align*}
      para todo $x\in E$. Si $K$ denota el espacio nulo de $T$ y $R(T)$ el rango de $T$, muestre que $\overline{T}:E/K\to R(T)$ dada por $\tilde{T}(x+K)=T(x),x\in E$, está bien definida y es un isomorfismo. Esto es $\tilde{T}\in L(E/K,R(T))$ y $\tilde{T}^{-1}\in L(R(T),E/K)$.  
  \end{enumerate}
  \begin{solution}
    \begin{enumerate}[(i)]
      \item $\phantom{x}$\\
      \begin{enumerate}[(a)]
        \item Veamos que $\sim_{K}$ es una relación de equivalencia sobre $E$.
          \begin{itemize}
            \item Reflexiva.\\
              Note que $x\sim_{K}x$, ya que $x-x=0\in K$ por ser $K$ subespacio de $E$ para todo $x\in E$, lo que nos permite concluir la reflexividad.
            \item Simétrica.\\
              Note que si asumimos que $x\sim_{K}y$, entonces $x-y\in K$, pero como $K$ es subespacio, entonces $-(x-y)=y-x\in K$, por lo que podemos asegurar que $y\sim_{K}x$, lo que nos permite concluir la simetría en la relación. 
            \item Transitiva.\\
              Note que si asumimos que $x\sim_{K}y$ y $y\sim_{K}z$, entonces $x-y\in K$ y $y-z\in K$, pero como $K$ es un subespacio cerrado, entonces $(x-y)+(y-z)=x-z\in K$ y por ende $x\sim_{K}z$, lo que nos permite concluir la transitividad. 
          \end{itemize}
      \end{enumerate}
    \end{enumerate}
  \end{solution}
\end{homeworkProblem}
