\begin{homeworkProblem}
  Sea $(E,\norm{\cdot}_{E})$ y $(F,\norm{\cdot}_{F})$ espacios de Banach.
  \begin{enumerate}[(i)]
    \item Sea $K\subset E$ un subespacio cerrado de $E$. Definimos la relación sobre $E$ dada por $x\sim_{K} y$ si y solo si $x-y\in K$.
    \begin{enumerate}[(a)]
      \item Muestre que $\sim_{K}$ es una relación de equivalencia sobre $E$.
      \item Muestre que el espacio cociente $E/K$ es un espacio de Banach con la norma
        \begin{align*}
          \norm{x+K}_{E/K}=\inf_{k\in K}\norm{x-k},\quad x\in E.
        \end{align*}
        Es decir, debe verificar que el espacio cociente es un espacio vectorial normado, cuya norma lo hace completo.
    \end{enumerate}
    \item Sea $T\in L(E,W)$ tal que existe $c>0$ para el cual
      \begin{align*}
        \norm{Tx}_{F}\geq c\norm{x}_{E},
      \end{align*}
      para todo $x\in E$. Si $K$ denota el espacio nulo de $T$ y $R(T)$ el rango de $T$, muestre que $\overline{T}:E/K\to R(T)$ dada por $\tilde{T}(x+K)=T(x),x\in E$, está bien definida y es un isomorfismo. Esto es $\tilde{T}\in L(E/K,R(T))$ y $\tilde{T}^{-1}\in L(R(T),E/K)$.  
  \end{enumerate}
  \begin{solution}
    \begin{enumerate}[(i)]
      \item $\phantom{x}$\\
      \begin{enumerate}[(a)]
        \item Veamos que $\sim_{K}$ es una relación de equivalencia sobre $E$.
          \begin{itemize}
            \item Reflexiva.\\
              Note que $x\sim_{K}x$, ya que $x-x=0\in K$ por ser $K$ subespacio de $E$ para todo $x\in E$, lo que nos permite concluir la reflexividad.
            \item Simétrica.\\
              Note que si asumimos que $x\sim_{K}y$, entonces $x-y\in K$, pero como $K$ es subespacio, entonces $-(x-y)=y-x\in K$, por lo que podemos asegurar que $y\sim_{K}x$, lo que nos permite concluir la simetría en la relación. 
            \item Transitiva.\\
              Note que si asumimos que $x\sim_{K}y$ y $y\sim_{K}z$, entonces $x-y\in K$ y $y-z\in K$, pero como $K$ es un subespacio cerrado, entonces $(x-y)+(y-z)=x-z\in K$ y por ende $x\sim_{K}z$, lo que nos permite concluir la transitividad.
          \end{itemize}
        \item Veamos que el espacio $(E/K,\norm{\cdot}_{E/K})$ es Banach.
          \begin{itemize}
            \item Veamos que $(E/K,\norm{\cdot}_{E/K})$ es un espacio vectorial normado.\\
              Note que si $y-x=k\in K$, entonces $y=x+k$ con $x\in E$ y cualquier $k\in K$, por lo que escribiremos a $y=x+K$, luego los elementos de $E/K$ serán de la forma $[a]=\{a=x+k:k\in K, x\in E\}$, además podemos afirmar que $E/K$ es cerrado, ya que $K$ es cerrado, entonces $\lambda[a]+[b]=\lambda (x+k_1)+y+k_2=(\lambda x+y)+\tilde{k}=[\lambda a+b]$. 
              Veamos que la norma está bien definida, es decir, dados $x+K,y+K\in E/K$ y $\lambda$ escalar, entonces
              \begin{align*}
                \norm{\lambda(x+K)}_{E/K}&=\inf_{k\in K}\norm{\lambda(x-k)},\\
                &=\lambda\inf_{k\in K}\norm{x-k},\\
                &=\lambda \norm{x+K}_{E/K}.
              \end{align*}
              Además si tomamos $0\in E/K$, es decir, $x\in K$, como $K$ es subespacio cerrado, entonces $x-k,0\in K$ y por ende podemos afirmar que
              \begin{align*}
                \norm{x+K}_{E/K}&=\inf_{k\in K}\norm{x-k},\\
                &=\inf_{k\in K}\norm{k},\\
                &= 0.
              \end{align*}
              Y por último sabemos que la desigualdad triangular se cumple por las propiedades de la norma en $E$ y del ínfimo, primero note que dado $\epsilon>0$ existe $k_1,k_2\in K$ tales que
              \begin{align*}
                \norm{x+K}_{E/K}&=\inf_{k\in K}\norm{x-k},\\
                &\leq \norm{x-k_1},\\
                &\leq \norm{x+K}_{E/K}+\epsilon/2,\\
                \norm{y+K}_{E/K}&=\inf_{k\in K}\norm{y-k},\\
                &\leq \norm{y-k_2},\\
                &\leq \norm{y+K}_{E/K}+\epsilon/2.
              \end{align*}
              Luego
              \begin{align*}
                \norm{x+K+y+K}_{E/K}&=\norm{(x+y)+K}_{E/K},\\
                &\leq \inf_{k\in K}\norm{x+y-k},\\
                &\leq \norm{x+y-(k_1+k_2)},\\
                &\leq \norm{x-k_1}+\norm{y-k_2},\\
                &\leq \norm{x+K}_{E/K}+\norm{y+K}_{E/K}+\epsilon.
              \end{align*}
              Pero como la desigualdad se tiene para $\epsilon>0$ arbitrario, entonces
              \begin{align*}
                \norm{(x+y)+K}_{E/K}&\leq \norm{x+K}_{E/K}+\norm{y+K}_{E/K}.
              \end{align*}
              Lo que concluye la desigualdad triangular y a su vez nos permite afirmar que $(E/K,\norm{\cdot}_{E/K})$ es un espacio vectorial normado.
            \item Ahora veamos que $(E/K,\norm{\cdot}_{E/K})$ es Banach.\\
              Suponga $\{a_{n}\}\subset E/K$ sucesión de Cauchy, por facilidad tomaremos al representante $a_{n}=x_{n}+k$ con $k$ fijo, entonces note que dado $\epsilon > 0$ existe $N>0$ tal que si $n,m>N$ entonces
                \begin{align*}
                  \norm{a_{n}-a_{m}}_{E/K}&=\norm{(x_n-x_m)+K},\\
                  &=\inf_{k\in K}\norm{x_n-x_m-k}<\epsilon.
                \end{align*}
                Vamos a tomar una subsucesión $\{a_{j}\}$ tal que para todo $j\in \mathbb{Z}$ se tenga que
                \begin{align*}
                  \norm{a_{j}-a_{j+1}}_{E/K}<\frac{1}{2^{j}}.
                \end{align*}
                ¿Por qué se puede obtener esta subsucesión? Note que si tomamos $\epsilon=\frac{1}{2^{j}}$ existe $N_0>0$ tal que si $n_0,m_0>N_0$, entonces $\norm{x_{n_0}-x_{m_0}}<\frac{1}{2^{j}}$, luego si tomamos por otro lado $\epsilon=\frac{1}{2^{j-1}}$, note que para todo $n,m$ tal que $n,m>N_0$ se satisface la condición, ya que $\frac{1}{2^{j}}<\frac{1}{2^{j-1}}$, la idea es tomar $n,m$ adecuados que cumplan de forma consecutiva las condiciones.\\
                Siendo así, note que razonando de forma análoga a cuando demostramos la desigualdad triangular por propiedades del ínfimo se tiene que dado $\delta>0$ existe $k\in K$ tal que 
                \begin{align*}
                  \norm{a_{j}-a_{j+1}}_{E/K} &= \inf_{k\in K}\norm{(x_{j}-x_{j+1})-k},\\
                  &\leq \norm{(x_{j}-x_{j+1})-k},\\
                  &\leq \norm{a_{j}-a_{j+1}}_{E/K}+\delta,\\
                  &< \frac{1}{2^{j}}+\delta. 
                \end{align*}
                Tome $\delta>0$ tal que $\frac{1}{2^{j}}+\delta=\frac{1}{2^{j-1}}$, entonces tomando ese $k=k_j-k_{j+1}$ tenemos que para cada $j$ se cumple que tomando $y_j=x_j-k_j$ se tiene que 
                \begin{align*}
                  \norm{(x_j-x_{j+1}-k)}&=\norm{(x_{j}-k_{j})-(x_{j+1}-k_{j+1})},\\
                  &=\norm{y_j-y_{j+1}},\\
                  &\leq \frac{1}{2^{j-1}}.
                \end{align*}
                Luego, es claro que $\{y_j\}\subset E$ es una sucesión de Cauchy y por ende como $E$ es Banach, entonces existe $y\in E$ tal que $y_j\to y=x+k_0$ para algún $k_0\in K$ (esto porque $K$ es subespacio, entonces en el peor de los casos $k_0=0$.) cuando $j\to \infty$.
                Ahora, tome $a=x+K\in E/K$, y luego dadas las condiciones anteriores, dado $\epsilon>0$ existe $J>0$ tal que si $j>J$, entonces
                \begin{align*}
                  \norm{a_{j}-a}_{E/K}&=\inf_{k\in K}\norm{(x_j-x)-k},\\
                  &\leq \norm{(x_{j}-k_{j})-(x-k_{0})},\\
                  &\leq \norm{y_{j}-y},\\
                  &< \epsilon.
                \end{align*}
                Luego podemos afirmar que $a_j\to a$ cuando $j\to \infty$ y por ende $a_{n}\to a$ cuando $n\to\infty$, lo que nos permite concluir que $(E/K,\norm{\cdot}_{E/K})$ es un espacio de Banach. 
          \end{itemize}
        \item Veamos que $\tilde{T}$ está bien definido.\\
          Note que si tomamos $x\sim_{K}x'$ entonces $x-x'\in K$ por lo que se cumple que $x=x'+K$, luego
          \begin{align*}
            \tilde{T}(x+K)&=T(x),\\
            &=T(x'+K),\\
            &=T(x')+T(K) &&\text{como $K$ es el espacio nulo},\\
            &=T(x'),\\
            &=\tilde{T}(x'+K).
          \end{align*}
          Lo que nos asegura que $\tilde{T}$ está bien definida.\\
          Veamos que $\tilde{T}\in L(E/K,R(T))$.\\
          Veamos que es lineal gracias a la linealidad de $T$ ya que si tomamos $x+K,y+K\in E/K$ y $\lambda$ entonces
          \begin{align*}
            \tilde{T}(x+K+\lambda(y+K))&=\tilde{T}((x+\lambda y)+K),\\
            &=T(x+\lambda y),\\
            &=T(x)+\lambda T(y),\\
            &=\tilde{T}(x+K)+\lambda T(y+K).
          \end{align*}
          Veamos que $\tilde{T}$ es continua gracias a la continuidad de $T$ ya que se puede ver que $\norm{x+K}_{E/K}\leq 1$, entonces $\norm{x}\leq 1$ ya que $0\in K$, luego 
          \begin{align*}
            \norm{\tilde{T}}&=\sup_{\substack{x\in E,\\\norm{x+K}_{E/K}\leq 1}}\norm{\tilde{T}(x+K)}_{R(T)},\\
            &\leq \sup_{\substack{x\in E,\\ \norm{x}\leq 1}}\norm{T(x)}_{W},\\
            &\leq \norm{T}.
          \end{align*}
          Lo que concluye que $\tilde{T}\in L(E/K,R(T))$.\\
          Veamos ahora que $\tilde{T}^{-1}\in L(R(T),E/K)$, note que
          \begin{align*}
            \tilde{T}^{-1}:R(T)&\to E/K,\\
            y=T(x)&\to x+K.
          \end{align*}
          Sean $y_1,y_2\in R(T)$ y $\lambda$ escalar, entonces
          \begin{align*}
            \tilde{T}^{-1}(y_1+\lambda y_{2})&=\tilde{T}^{-1}\left( T(x_1)+\lambda T(x_2) \right),\\
            &=\tilde{T}^{-1}(T(x_1+\lambda x_2)),\\
            &=\tilde{T}^{-1}(\tilde{T}((x_1+\lambda x_2)+K),\\
            &=(x_1+\lambda x_{2})+ K,\\
            &=(x_1+K)+\lambda (x_{2}+K),\\
            &=\tilde{T}^{-1}(y_1)+\lambda \tilde{T}^{-1}(y_2).
          \end{align*}
          Y veamos que la hipótesis faltante nos da la continuidad, ya que si suponemos $\tilde{T}^{-1}(y)=x+K$, entonces
          \begin{align*}
            \norm{\tilde{T}^{-1}(y)}_{E/K}&=\norm{x+K}_{E/K},\\
            &\leq \norm{x}_{E},\\
            &\leq \frac{1}{c}\norm{T(x)}_{F},\\
            &\leq \frac{1}{c}\norm{y}_{R(T)}.
          \end{align*}
          Luego podemos afirmar que $\tilde{T}^{-1}\in L(R(T),E/K)$ y por ende es un isomorfismo. 
      \end{enumerate}
    \end{enumerate}
  \end{solution}
\end{homeworkProblem}
