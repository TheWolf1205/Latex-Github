\documentclass[10pt]{article}

% Paquetes necesarios
\usepackage[margin=1.5cm]{geometry} % márgenes pequeños
\usepackage{multicol}               % para múltiples columnas
\usepackage{amsmath, amssymb}       % para símbolos y fórmulas
\usepackage{graphicx}               % si deseas incluir imágenes
\usepackage{parskip}                % para evitar indentación

\pagestyle{empty}                   % sin encabezado ni pie de página

\begin{document}

\scriptsize % Tamaño de letra pequeño (puedes usar \tiny para aún más pequeño)

\begin{multicols}{3} % Cambia el número de columnas aquí

% ============================
% Empieza tu hoja de fórmulas
% ============================

\section*{Orientación y superficies}
\begin{itemize}
  \item Cambio de coordenadas. 
    \begin{align*}
      \begin{pmatrix}
        \overline{X}_{\overline{u}} \\
        \overline{X}_{\overline{v}}
      \end{pmatrix}
      =\begin{pmatrix}
        \frac{\partial u}{\partial \overline{u}} & \frac{\partial u}{\partial \overline{v}} \\
       \frac{\partial v}{\partial \overline{u}}& \frac{\partial v}{\partial \overline{v}}
      \end{pmatrix}\begin{pmatrix}
        X_{u} \\
        X_{v}
      \end{pmatrix}.
    \end{align*}
    \item grafos. 
      \begin{align*}
        N(u,v)=\frac{(f_{u},f_{v},1)}{\|(f_{u},f_{v},1)\|}
      \end{align*}
    \item Si $S$ es superficie de revolución
      \begin{align*}
        N(p)=\frac{\nabla f(p)}{\|\nabla f(p)\|}.
      \end{align*}
    \item 
      \begin{align*}
        A_{p}=\begin{bmatrix}
          \left[ -N_{u}\right]_{\{X_{u},X_{v}\}} & \left[ -N_{v}\right]_{\{X_{u},X_{v}\}}
        \end{bmatrix}
      \end{align*}
\end{itemize}


\end{multicols}

\end{document}

