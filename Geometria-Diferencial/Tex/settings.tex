% Paquetes esenciales
\usepackage[utf8]{inputenc}
\usepackage[T1]{fontenc}
\usepackage[spanish]{babel}
\usepackage{amsmath, amssymb, amsthm}
\usepackage{tikz}
\usepackage{xcolor}
\usepackage{csquotes}
\usepackage[most]{tcolorbox}
\usepackage{hyperref}  % Enlaces internos
\usepackage{graphicx}  % Para incluir imágenes
\usepackage{geometry}  % Márgenes más cómodos
\usepackage[numbers]{natbib} % Bibliografía
\usepackage{pdfpages}
\geometry{left=3cm, right=3cm, top=3cm, bottom=3cm}

% Comandos renombrados o nuevos
\newcommand{\x}{\textbf{x}}

% widehat y widecheck
\DeclareFontFamily{U}{mathx}{}
\DeclareFontShape{U}{mathx}{m}{n}{<-> mathx10}{}
\DeclareSymbolFont{mathx}{U}{mathx}{m}{n}
\DeclareMathAccent{\widecheck}{0}{mathx}{"71}
\renewcommand{\check}{\widecheck}
\renewcommand{\hat}{\widehat}
\renewcommand{\check}{\widecheck}

% norma 3 lineas
\newcommand{\seminorm}[1]{{\left\vert\kern-0.25ex\left\vert\kern-0.25ex\left\vert #1 
    \right\vert\kern-0.25ex\right\vert\kern-0.25ex\right\vert}}

% norma alargada
\newcommand{\norm}[1]{\left\lVert#1\right\rVert}

% Definir colores personalizados para las cajas
\definecolor{mygrayback}{RGB}{245, 245, 245}
\definecolor{mygrayframe}{RGB}{80, 80, 80}
\definecolor{mygrayframeproof}{RGB}{140, 140, 140}

% Redefinir el ambiente de Teorema
\newtcolorbox[auto counter, number within=section]{theorem}[2][]{%
  colback=mygrayback, colframe=mygrayframe, fonttitle=\bfseries,
  title=Teorema~\thetcbcounter: #2,#1, breakable}

% Redefinir el ambiente de Lema
\newtcolorbox[auto counter, number within=section]{lemma}[2][]{%
  colback=mygrayback, colframe=mygrayframe, fonttitle=\bfseries,
  title=Lema~\thetcbcounter: #2,#1, breakable}

% Redefinir el ambiente de Proposición
\newtcolorbox[auto counter, number within=section]{proposition}[2][]{%
  colback=mygrayback, colframe=mygrayframe, fonttitle=\bfseries,
  title=Proposición~\thetcbcounter: #2,#1, breakable}

% Redefinir el ambiente de Nota
\newtcolorbox[auto counter, number within=section]{note}[2][]{%
  colback=mygrayback, colframe=mygrayframe, fonttitle=\bfseries,
  title=Nota~\thetcbcounter: #2,#1, breakable}

% Redefinir el ambiente de Corolario
\newtcolorbox[auto counter, number within=section]{corollary}[2][]{%
  colback=mygrayback, colframe=mygrayframe, fonttitle=\bfseries,
  title=Corolario~\thetcbcounter: #2,#1, breakable}

% Redefinir el ambiente de Ejemplo
\newtcolorbox[auto counter, number within=section]{example}[2][]{%
  colback=mygrayback, colframe=mygrayframe, fonttitle=\bfseries,
  title=Ejemplo~\thetcbcounter: #2,#1, breakable}

% Redefinir el ambiente de Definición
\newtcolorbox[auto counter, number within=section]{definition}[2][]{%
  colback=mygrayback, colframe=mygrayframe, fonttitle=\bfseries,
  title=Definición~\thetcbcounter: #2,#1, breakable}

% Redefinir el ambiente de Notación
\newtcolorbox[auto counter, number within=section]{notation}[2][]{%
  colback=mygrayback, colframe=mygrayframe, fonttitle=\bfseries,
  title=Notación~\thetcbcounter: #2,#1, breakable}

% Eliminar la definición original de \proof
\let\proof\relax
\let\endproof\relax

% Redefinir el ambiente de Demostración (sin usar caracteres especiales en la clave)
\newtcolorbox{proof}[1][]{%
  colback=mygrayback, colframe=mygrayframeproof, fonttitle=\bfseries,
  title=Demostración, #1, breakable}

% Redefinir el ambiente de Solución
\newtcolorbox{solution}[1][]{%
  colback=mygrayback, colframe=mygrayframeproof, fonttitle=\bfseries, title=Solución,#1, breakable}

