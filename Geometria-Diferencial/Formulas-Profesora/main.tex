\documentclass[8pt]{article}
\usepackage[margin=1cm]{geometry}
\usepackage{multicol}
\usepackage{amsmath, amssymb}
\usepackage{mathrsfs}
\usepackage[utf8]{inputenc}
\usepackage[T1]{fontenc}
\usepackage{parskip}
\pagestyle{empty}

\begin{document}
\tiny
\begin{multicols}{3}

\section*{Curvatura normal}
\[
k_n(p, v) = \langle -dN_p(v), v \rangle = k_1 \cos^2(\theta) + k_2 \sin^2(\theta)
\]

\section*{Fórmulas de Weingarten}
\[
e = \langle -dN_p(X_u), X_u \rangle = - \langle N_u, X_u \rangle = \langle N, X_{uu} \rangle
\]
\[
f = \langle -dN_p(X_u), X_v \rangle = - \langle N_u, X_v \rangle = \langle N, X_{uv} \rangle
\]
\[
g = \langle -dN_p(X_v), X_v \rangle = - \langle N_v, X_v \rangle = \langle N, X_{vv} \rangle
\]
Si $v = aX_u + bX_v$,
\[
II_p(v) = a^2 e + 2abf + b^2 g
\]

\section*{Operador de forma}
\[
A_p(X_u) =
\frac{eG - fF}{EG - F^2} X_u +
\frac{fE - eF}{EG - F^2} X_v
\]
\[
A_p(X_v) =
\frac{fG - Fg}{EG - F^2} X_u +
\frac{gE - fF}{EG - F^2} X_v
\]

\section*{Curvaturas}
\[
K = \frac{eg - f^2}{EG - F^2}
\qquad
H = \frac{eG + gE - 2fF}{2(EG - F^2)}
\]
  si es de revolución y $g$ es de norma $1$ 
\begin{align*}
  K=\frac{-f''}{f} \qquad H=\frac{1}{2}\frac{-g'+f(g'f''-g''f')}{f}
\end{align*}
si son doblemente ortogonales
\begin{align*}
  k_1=\frac{e}{E}\qquad k_2=\frac{g}{G}
\end{align*}
\begin{align*}
  p(x)=x^2-2xH+K=(x-k_1)(x-k_2)
\end{align*}
\section*{Línea de curvatura}
Si $\alpha(t) = X(u(t), v(t))$,
\[
\begin{vmatrix}
v'^2 & -u'v' & u'^2 \\
E & F & G \\
e & f & g
\end{vmatrix} = 0
\]

\section*{Curvatura y torsión (si $\alpha$ es p.p.a.)}
\[
k_\alpha^2 = k_n^2 + k_g^2
\qquad
\tau_\alpha =
\frac{k_n' k_g - k_n k_g'}{k_n^2 + k_g^2} + \tau_g
\]

\section*{Triedro de Frenet}
\[
\vec{t}'(s) = k(s) \vec{n}(s)
\]
\[
\vec{b}'(s) = \tau(s) \vec{n}(s)
\]
\[
\vec{n}'(s) = -k(s)\vec{t}(s) - \tau(s)\vec{b}(s)
\]

\section*{Triedro de Darboux}
\[
\vec{t}'(s) = k_g(s) J\vec{t}(s) + k_n(s)\vec{N}(s)
\]
\[
(J\vec{t})'(s) = -k_g(s)\vec{t}(s) + \tau_g(s)\vec{N}(s)
\]
\[
\vec{N}'(s) = -k_n(s)\vec{t}(s) - \tau_g(s) J\vec{t}(s)
\]
\[
\tau_g(s) = \langle A_{\alpha(s)} \vec{t}(s), J\vec{t}(s) \rangle
\]

\section*{Símbolos de Christoffel}
\[
\Gamma^1_{11} E + \Gamma^2_{11} F = E_u
\quad
\Gamma^1_{11} F + \Gamma^2_{11} G = F_u - \tfrac{1}{2} E_v
\]
\[
\Gamma^1_{12} E + \Gamma^2_{12} F = \tfrac{1}{2} E_v
\quad
\Gamma^1_{12} F + \Gamma^2_{12} G = \tfrac{1}{2} G_u
\]
\[
\Gamma^1_{22} E + \Gamma^2_{22} F = F_v - \tfrac{1}{2} G_u
\quad
\Gamma^1_{22} F + \Gamma^2_{22} G = G_v
\]

\[
\begin{pmatrix}
\Gamma^1_{11} & \Gamma^1_{12} & \Gamma^1_{22} \\
\Gamma^2_{11} & \Gamma^2_{12} & \Gamma^2_{22}
\end{pmatrix}
=
\frac{1}{EG - F^2}
\begin{pmatrix}
G & -F \\
-F & E
\end{pmatrix}
\begin{pmatrix}
E_u \\ E_v \\ F_v - G_u \\
F_u - E_v \\ G_u \\ G_v
\end{pmatrix}
\]
Si es superficie de revolución
\begin{align*}
  \Gamma^{1}_{11}=0 \,& \Gamma^{2}_{11}=-\frac{ff'}{(f')^2+(g')^{2}} \, \Gamma^{1}_{12}=\frac{ff'}{f^2},\\
  \Gamma^{2}_{12}=0 \,& \Gamma^{1}_{22}=0 \, \Gamma_{22}^{2}=\frac{f'f''+g'g''}{(f')^{2}+(g')^2}
\end{align*}


\section*{Si $F = 0$}
\[
K = -\frac{1}{2\sqrt{EG}} \left[
\left( \frac{E_v}{\sqrt{EG}} \right)_v +
\left( \frac{G_u}{\sqrt{EG}} \right)_u
\right]
\]

\section*{Ecuación de Gauss}
\[
\Gamma^1_{11} \Gamma^2_{12} + (\Gamma^2_{11})_v + \Gamma^2_{11} \Gamma^2_{22}
- \Gamma^1_{12} \Gamma^2_{11} - (\Gamma^2_{12})_u - \Gamma^2_{12} \Gamma^2_{12}
= EK
\]

\section*{Ecs. de Mainardi-Codazzi}
\[
e_v - f_u = e\Gamma^1_{12} + f(\Gamma^2_{12} - \Gamma^1_{11}) - g\Gamma^2_{11}
\]
\[
f_v - g_u = e\Gamma^1_{22} + f(\Gamma^2_{22} - \Gamma^1_{12}) - g\Gamma^2_{12}
\]
Si es doblemente ortogonal
\begin{align*}
  \Gamma_{11}^{1}=\frac{1}{2}\frac{E_{u}}{E},  \, \Gamma_{11}^{2}=-\frac{1}{2}\frac{E_{v}}{G}, \,\Gamma_{12}^{1}=\frac{1}{2}\frac{E_{v}}{E},\\
  \Gamma_{12}^{2}=\frac{1}{2}\frac{G_{u}}{G}, \, \Gamma_{22}^{1}=-\frac{1}{2}\frac{G_u}{E}, \, \Gamma_{22}^{2}=\frac{1}{2}\frac{G_{v}}{E}.
\end{align*}
\begin{align*}
  e_{v}=\frac{E_{v}}{2}\left( \frac{e}{E}+\frac{g}{G} \right),\\
  g_{u}=\frac{G_{u}}{2}\left( \frac{e}{E}+\frac{g}{G} \right).
\end{align*}
\section*{Campos paralelos}
Si $V(t) = a(t)X_u + b(t)X_v$, con $V$ paralelo:
\[
a' + a u' \Gamma^1_{11} + (a v' + b u') \Gamma^1_{12} + b v' \Gamma^1_{12} = 0
\]
\[
b' + a u' \Gamma^2_{11} + (a v' + b u') \Gamma^2_{12} + b v' \Gamma^2_{22} = 0
\]

\section*{Geodésicas}
Si $\gamma(t) = X(u(t), v(t))$:
\[
u'' + (u')^2 \Gamma^1_{11} + 2u'v'\Gamma^1_{12} + (v')^2\Gamma^1_{22} = 0
\]
\[
v'' + (u')^2 \Gamma^2_{11} + 2u'v'\Gamma^2_{12} + (v')^2\Gamma^2_{22} = 0
\]

\section*{Curvatura geodésica}
\[
k^g_\alpha(t) =
\frac{\langle \alpha''(t), J\alpha'(t) \rangle}{|\alpha'(t)|^3}
=
\frac{\langle \alpha''(t), N(t) \times \alpha'(t) \rangle}{|\alpha'(t)|^3}
\]

% ...
\section*{Curvatura geodésica}
\[
k^g_\alpha(t) =
\frac{\langle \alpha''(t), J\alpha'(t) \rangle}{|\alpha'(t)|^3}
=
\frac{\langle \alpha''(t), N(t) \times \alpha'(t) \rangle}{|\alpha'(t)|^3}
\]
Curvatura normal
\begin{align*}
  k_{n}(v,p)=II_p(h,k)=\begin{pmatrix}
    h & k
  \end{pmatrix}\begin{pmatrix}
    k_{1} & 0 \\
    0 & k_2
  \end{pmatrix}\begin{pmatrix}
    h\\
    k
  \end{pmatrix}
\end{align*}
\section*{Orientación y superficies}
\begin{itemize}
  \item \textbf{Cambio de coordenadas}
  \[
  \begin{pmatrix}
    \overline{X}_{\overline{u}} \\
    \overline{X}_{\overline{v}}
  \end{pmatrix}
  =
  \begin{pmatrix}
    \frac{\partial u}{\partial \overline{u}} & \frac{\partial u}{\partial \overline{v}} \\
    \frac{\partial v}{\partial \overline{u}} & \frac{\partial v}{\partial \overline{v}}
  \end{pmatrix}
  \begin{pmatrix}
    X_{u} \\
    X_{v}
  \end{pmatrix}
  \]

  \item \textbf{Grafos} 
  \[
  N(u,v)=\frac{(f_{u}, f_{v}, 1)}{\|(f_{u}, f_{v}, 1)\|}
  \]

  \item \textbf{Superficie de revolución} 
  \[
  N(p)=\frac{\nabla f(p)}{\|\nabla f(p)\|}
  \]

  \item \textbf{Operador de forma vía derivadas de la normal} 
  \[
  A_p =
  \begin{bmatrix}
    \left[ -N_u \right]_{\{X_u, X_v\}} &
    \left[ -N_v \right]_{\{X_u, X_v\}}
  \end{bmatrix}
  \]
\end{itemize}
\section*{Derivada covariante}
\begin{align*}
  \frac{DV}{dt}(t):=V'(t)^{\perp}=V'(t)-<V'(t),N(t)>N(t).
\end{align*}
\section*{Tipos de puntos}
\begin{itemize}
  \item El\'iptico: $K > 0$ (ej: esfera)
  \item Hiperb\'olico: $K < 0$ (ej: silla de montar)
  \item Parab\'olico: $K = 0$, $H \neq 0$ (ej: cilindro)
  \item Plano: $K = 0$, $H = 0$ (ej: plano)
  \item Umb\'ilico: $k_1 = k_2$ (todos los puntos en esfera)
\end{itemize}

\section*{Tipos de curvas sobre superficies}
\begin{itemize}
  \item \textbf{L\'inea de curvatura:} $A_p(v) \parallel v$
  \item \textbf{Curva asint\'otica:} $k_n = 0$
  \item \textbf{Geod\'esica:} $\frac{D}{dt} T = 0$
  \item \textbf{Paralela:} derivada covariante cero
\end{itemize}

\end{multicols}
\end{document}

