\documentclass[a0,portrait]{a0poster}
\usepackage[spanish]{babel}
\usepackage[utf8]{inputenc}
\usepackage[T1]{fontenc}
\usepackage{times}
\usepackage{multicol}
\usepackage[pdftex]{graphicx}
\usepackage[pdftex,dvipsnames,usenames]{color}
\usepackage{epic}%cruces
%\renewcommand{\familydefault}{\sfdefault}
%\usepackage{colortbl}
%\usepackage[notcite,notref]{showkeys}
\usepackage[utf8]{inputenc}
\usepackage{amsfonts,amsmath,amstext,amssymb}
\usepackage{theorem,enumerate}
\usepackage{physics}
\usepackage{mathrsfs}
\usepackage{multirow}
\usepackage{hyperref}
\usepackage{cleveref}
\usepackage{csquotes}
\usepackage{verbatim}
\usepackage{float}
\usepackage{fancyhdr}

\usepackage{graphicx} % figuras
\usepackage{subfigure}%subfiguras
\usepackage{fancybox}
\usepackage{bibunits}
\usepackage[listings]{tcolorbox}
\advance\textwidth35mm %margen derecha
 \advance\hoffset-15mm %margen izqierda
 \advance\textheight11cm %margen abajo
 \advance\voffset-40mm %margen arriba
\parskip5pt plus1pt
\def\<{\left<}
 \def\>{\right>}
 \def\paren#1{\left(#1\right)}
 \let\fle\to
 \def\sen{\text{sin}}
 \def\senh{\text{sinh}}
 \def\arccosh{\text{arccosh}}
 \def\longi{\text{Length}}
 \def\e{\varepsilon}
 \def\eu{\e_1}
 \def\ed{\e_2}
 \def\et{\e_3}
  \def\diag{\text{\rm diag}}
 \def\sg{\text{sign}}
 \def\R{\mathbb{R}}
 \def\C{\mathbb{C}}
 \def\d{\text{\rm d}}
 \def\L{\mathbb{L}}
 \def\S{\mathbb{S}}
 \def\H{\mathbb{H}}
 \def\tr{\text{\rm tr}}
 \def\det{\text{\rm det}}
 \def\cm{{\mathcal C}^\infty(M^n_s)}
 \let\ds\displaystyle
 \def\div{\text{\rm div}}
 %\newcommand{\U}[1][k+1]{\mathcal{U}_{#1}}
 \def\card{\text{\rm card}}
 \newcommand{\ol}{\overline}
 \newcommand{\bin}[1]{\mbox{$\binom{n}{#1}$}}
\newcommand{\binm}[2]{\mbox{$\binom{#1}{#2}$}}
\def\hdashline#1#2{%
 \setlength{\unitlength}{1mm}
 \begin{picture}(0,0)
 \dashline[50]{2}(#1,0)(#2,0)
 \end{picture}}
\def\vdashline#1#2{%
 \setlength{\unitlength}{1mm}
 \begin{picture}(0,0)
 \dashline[50]{2}(0,#1)(0,#2)
 \end{picture}}
\def\Dashcruz#1#2#3#4{\setlength{\unitlength}{1mm}%
 \begin{picture}(0,0)
 \dashline[50]{2}(#1,0)(#2,0)
 \dashline[50]{2}(0,#3)(0,#4)
 \end{picture}}
\def\dashcruz#1#2#3#4{\relax}

\def\MatCero{\text{\large\bfseries 0}}
\def\ColSep{\kern2pt}
\def\Sep#1{\kern#1pt}
%\def\VSep{5pt}
\def\VSep{6pt}
\def\VSSep{3pt}
\def\VVSep{0.5cm}
\def\VVVSep{0.3cm}

\theorembodyfont{\slshape}
\newtheorem{theorem}{\color{White}Theorem}
\newtheorem{proposition}[theorem]{\color{white}Proposición}
\newtheorem{corollary}[theorem]{\color{Blue}Corolario}
\newtheorem{lemma}[theorem]{\color{Blue}Lemma} {\theorembodyfont{\slshape}
\newtheorem{definition}{\color{Blue}Definición}%\colorbox[cmyk]{0,0,1,0}
\newtheorem{conjecture}{\color{Blue}Conjetura}
\newtheorem{example}{\color{Black}Example}
\newtheorem{remark}{\color{Blue}Nota}
\newtheorem{problem}{\color{Blue}Problema}
}
\definecolor{MiColorw}{rgb}{.50,.90,.70}
\definecolor{MiColor}{rgb}{.90,.90,.90}
\definecolor{suave}{rgb}{.92,.92,1}
\definecolor{blue}{RGB}{0,12,55}
\definecolor{Gray}{RGB}{66,66,66}
\definecolor{Fondo}{RGB}{236,235,235}
\definecolor{gris}{RGB}{230,243,242}
\definecolor{rosa}{RGB}{255,236,232}
\definecolor{red}{RGB}{207,76,49}

\pagecolor{white}

\makeatletter
\renewcommand\section{\@startsection {section}{1}{\z@}%
                                   {-3.5ex \@plus -1ex \@minus -.2ex}%
                                  {2.3ex \@plus.2ex}%
                                 {\normalfont\LARGE\bfseries\textcolor{black}}}
\def\thesection{\arabic{section}}
\def\thesubsection{\thesection.\arabic{subsection}}
\renewcommand\subsection{\@startsection{subsection}{2}{\z@}%
                                     {-3.25ex\@plus -1ex \@minus -.2ex}%
                                     {1.5ex \@plus .2ex}%
                                     {\normalfont\Large\bfseries\textcolor{Green}}}
                                     %\renewcommand\subsubsection{\@startsection{subsubsection}{3}{\z@}%
 %                                    {-3.25ex\@plus -1ex \@minus -.2ex}%
  %                                   {1.5ex \@plus .2ex}%
  %                                   {\normalfont\normalsize\bfseries\color{Green}}}
\renewcommand\paragraph{\@startsection{paragraph}{4}{\z@}%
                                    {3.25ex \@plus1ex \@minus.2ex}%
                                    {-1em}%
                                    {\normalfont\normalsize\bfseries\color{Green}}}
\renewcommand\subparagraph{\@startsection{subparagraph}{5}{\parindent}%
                                       {3.25ex \@plus1ex \@minus .2ex}%
                                       {-1em}%
                                      {\normalfont\normalsize\bfseries\color{Green}}}
\makeatother
\newenvironment{proof}
 {\par\noindent\textbf{Proof.}\quad}
 {\hfill$\blacksquare$\par}
\newenvironment{proof*}
 {\par\noindent\textbf{Proof.}\quad}
 {}

\widowpenalty 10000 \clubpenalty 10000 \skip\footins=2\baselineskip
\def\baselinestretch{1.05}
\parindent0pt

\def\thepage{}

\renewcommand{\footnoterule}{\textcolor{blue}{\rule{0.587\columnwidth}{0.02in}}\vspace*{5mm}}
\title{\color{red}\bfseries {\Huge{Sobre una perturbación dispersiva de quinto orden de la ecuación de Benjamin}}\\\
\normalsize\textcolor{Black}{\huge{Iván Felipe Salamanca Medina}} \\
\textcolor{Black}{\Large{Trabajo desarrollado por el estudiante Diego Fernando Correa Castañeda bajo de dirección del profesor Ricardo Ariel Pastrán Ramírez }} \\\
\normalsize \textcolor{Black}{\Large{Semillero de Análisis Armónico y Ecuaciones Diferenciales Parciales, Departamento de Matemáticas, Universidad Nacional de Colombia}}\\\
\textcolor{Black}{\large{8 de Junio de 2023}}\\\    
}
\author{}
\date{}
\advance
\pdfpageheight8cm
\columnsep=2cm %Espacio entre las columnas

\begin{document}
	\pagestyle{empty}
	%\begin{minipage}{15cm}\ \\ \\ \\\includegraphics[width=9cm,height=10cm]{UNAL1.jpg}\end{minipage}
        \begin{center}
        \hspace{-3cm}
        \begin{minipage}{.20\columnwidth}
            \vspace{3.5cm} % Ajusta según sea necesario
            \includegraphics[width=9cm, height=4cm]{logo.png}
        \end{minipage}%
        \hspace{-7cm}
	\begin{minipage}{.80\columnwidth}
        \vspace{3.5cm}
            \fboxsep0.5mm
                \shadowsize5mm
                    \shadowbox{\fboxsep5mm
                        \colorbox{anita2}{
                            \begin{minipage}{\textwidth}
                            \vspace{-2cm}
                                \maketitle
                            \vspace{-5.5cm}
                            \end{minipage}%
                        }%
                    }
        \end{minipage}            
        \end{center}
	\begin{multicols}{2}
		
		{\fboxsep0.5mm\shadowsize4mm \shadowbox{\fboxsep5mm\colorbox{anita6}{\color{white}
					\begin{minipage}{.93\columnwidth}
						\begin{center}
							\normalsize{\color{white}\bf {\Large Resumen}}
						\end{center}
						El propósito de este póster es presentar uno de los resultados estudiados durante el \emph{Semillero de Análisis Armónico y Ecuaciones Diferenciales Parciales} del departamento de matemáticas de la Universidad Nacional de Colombia sede Bogotá, referentes al estudio de la \emph{función maximal de Hardy-Littlewood}.\\ \\
						En este mismo se presentarán resultados y teoremas preliminares como aproximaciones de la identidad,  convergencia en casi todo punto, desigualdades débiles/fuertes y el teorema de interpolación de Marcinkiewicz, esto con el fin de introducir la función maximal de Hardy-Littlewood y continuar su estudio enfocados a resultados como: mostrar que la función maximal establece un operador acotado en $\mathcal{L}^{\infty}$, un operador $(1,1)$-débil y deducir de esto el teorema de diferenciación de Lebesgue. Como trabajo futuro, se plantea estudiar para cuales espacios de Sobolev la función maximal de Hardy-Littlewood determina un operador continuo.

						
						
		\end{minipage}}}}
		\vspace{-5cm}
		
		\section{Conceptos y Definiciones}\vspace{-0.5cm}
		
		{\fboxsep0.5mm\shadowsize4mm \shadowbox{\fboxsep5mm\colorbox{anita3}{\color{white}
					\begin{minipage}{.93\columnwidth}
							\textbf{Aproximación de la identidad}\\
							Suponga $\phi$ como una función integrable en $\mathbb{R}^n$ tal que $\int_{\mathbb{R}^n}\phi=1$, y luego definamos para todo $t>0$ a 
							$$\phi_t(x)=t^{-n}\phi(t^{-1}x).$$ 
							Si $t\rightarrow 0$, $\phi_t$ converge en el sentido distribucional de $\mathcal{S}'$ a $\delta_0$ (la medida delta de Dirac en el origen), entonces diremos que  $\{\phi_t\, |\, t>0\}$ es una aproximación de la identidad.\\
							
						 \textbf{Desigualdades débiles y fuertes}\\
							Sean $(X,\mu)$ y $(Y, \nu)$ dos espacios de medida y sea $T$ un operador de $\mathcal{L}^{p}(X,\mu)$, en el espacio de funciones medibles de $Y$ en $\mathbb{C}$.
							$$T:\mathcal{L}^p(X,\mu)\rightarrow\mathcal{M}(Y,\mathbb{C})$$
							\begin{itemize}
								\item[i.] Se dice que $T$ es $(p,q)$-débil (con $q<\infty$) si para todo $\lambda>0$ existe $C>0$ tal que:
								$$\nu(\{y\in Y:|(Tf)(y)|>\lambda\})\leq \left(\frac{C||f||_p}{\lambda}\right)^q.$$
								\item[ii.] Se dice que $T$ es $(p,\infty)$-débil si está acotado de $\mathcal{L}^p(X,\mu)$ en $\mathcal{L}^{\infty}(Y,\nu)$.
								\item[iii.] Se dice que $T$ es $(p,q)$-fuerte si está acotado de $\mathcal{L}^p(X,\mu)$ en $\mathcal{L}^q(Y,\nu)$.
							\end{itemize}

       \vspace{0.3cm}
							
							\textbf{Función de distribución}\\
							Sea $(X,\mu)$ un espacio de medida y sea $f:X\rightarrow \mathbb{C}$ una función medible.\\
							Se llama función de distribución de $f$ asociadad a $\mu$ a la función:\\
							\begin{align*}
								a_f:(0,\infty)&\rightarrow [0,\infty]\\
								\lambda&\rightarrow \mu(\{x\in X: |f(x)|>\lambda\})	
							\end{align*}
							
							\textbf{Operador sublineal}\\
							Un operador $T$ de un espacio vectorial de funciones medibles en funciones medibles se dice \emph{sublineal} si
							\begin{itemize}
								\item $|T(f_1+f_2)(x)|\leq |T(f_1)(x)|+|T(f_2)(x)|$.\\
								\item $|T(\lambda f)(x)| = |\lambda||T(f)(x)|$ para todo $\lambda\in\mathbb{C}$
							\end{itemize}
       
							\vspace{1cm}
							\textbf{Función Maximal de Hardy-Littlewood}\\
							Sea $B_r$ la bola euclidea centrada en el origen y de radio $r$. Definiremos la función maximal de Hardy-Littlewood de una función localmente integrable $f$ en $\mathbb{R}^n$ como:
							$$\mathcal{M}f(x)=\sup_{r>0}\frac{1}{|B_r|}\int_{B_r}|f(x-y)|dy$$
						
		\end{minipage}}}}
		\vspace{-5cm}
		
		\section{Resultados}\label{s:Newton}
		
		{\fboxsep0.5mm\shadowsize4mm \shadowbox{\fboxsep5mm\colorbox{anita2}{\color{white}
					\begin{minipage}{.93\columnwidth}
						\textbf{Teorema.}  \emph{	Sea $\{\phi_t\, |\, t>0\}$ una aproximación de la identidad, entonces
							$$\lim_{t\rightarrow 0}||\phi_t*f-f||_p=0,$$
							para cualquier $f\in\mathcal{L}^p$, con $1\leq p < \infty$
							y uniformemente (caso $p=\infty$) si $f$ es continua y tiende a $0$ en infinito, esto es, $f\in C_0(\mathbb{R}^n)$. }\\

       \vspace{0.1cm}
							Como una consecuencia de este teorema, sabemos que existe una sucesión $\{t_k\}$ que depende de $f$ tal que $t_k\rightarrow 0$ y
							$$\lim_{k\rightarrow \infty}\phi_{t_k}*f(x)=f(x) \quad \text{c.t.p.}$$
							Es por esto que si el límite $\lim_{t\to 0} \phi_t*f(x)$ existe, debe de ser igual a $f(x)$ en casi todo punto. Más adelante se estudiar\'a la existencia de este límite en general, no solo para una secuencia $\{t_k\}$.

       \vspace{0.7cm}
						
						\textbf{Proposición.} \emph{Sea $T$ un operador $(p,q)$-fuerte, entonces $T$ es $(p,q)$-débil}.\\


{\bf Demostracón (idea).} Denotemos por 
$$E_\lambda=\{ y\in Y: |(Tf)(y)|>\lambda \},$$
luego se tiene que
						\begin{align*}
							\nu(E_\lambda)=\int_{E_\lambda}d_{\nu}&\leq\int_{E_\lambda}\Bigl|\frac{(Tf)(y)}{\lambda}\Bigr|^qd\nu \\
							&\leq\frac{1}{\lambda^q}\int_{Y}|(Tf)(y)|^qd\nu=\frac{1}{\lambda^q}||Tf||_q^q\\
							&\leq \frac{1}{\lambda^q}(C||f||_p)^q=\left(\frac{C||f||_p}{\lambda}\right)^q.
						\end{align*}
\hfill $\square$						
						
						
		\end{minipage}}}}
				
	{\fboxsep0.5mm\shadowsize4mm \shadowbox{\fboxsep5mm\colorbox{anita2}{\color{white}

 
	\begin{minipage}{.93\columnwidth}
  
						\textbf{Teorema.} \emph{Sea $\{T_t\}$ una familia de operadores lineales en $\mathcal{L}^p(X,\mu)$, es decir, }
						\begin{align*}
							T_t:\mathcal{L}^p(X,\mu)&\rightarrow\mathcal{L}^p(X,\mu)\\
							f&\rightarrow T_tf
						\end{align*}
						y definimos
						$$T^*f(x)=\sup_t|T_tf(x)|.$$
						\emph{Si $T^*$ es $(p,q)$-débil, el conjunto }
      $$\{f\in\mathcal{L}^p|\lim_{t\rightarrow t_0} T_tf(x)=f(x)\, \,  \text{c.t.p}\}$$
      
    \emph{  es cerrado en $\mathcal{L}^p$.}

\vspace{0.6cm}

{\bf Observaciones:}

      \begin{itemize}
          \item $T^*$ se llama operador maximal asociado a $\{T_t\}$.
          \item Como para las aproximaciones de la identidad conocemos la convergencia puntual hacía $f$ para funciones de $\mathcal{S}$, basta probar acotaciones débiles sobre el operador maximal $\sup_{t>0}|\phi_t*f(x)|$ para deducir la convergencia en casi todo punto para $f\in\mathcal{L}^p$, $1\leq p <\infty$, o para $f\in C^0$.
      \end{itemize}

      \vspace{0.6cm}
						
						\textbf{Proposición.} \emph{Sea $\varphi:[0,\infty)\rightarrow[0,\infty)$ una función derivable y creciente tal que $\varphi(0)=0$, entonces:
						$$\int_X \varphi(|f(x)|)d\mu=\int_0^\infty \varphi'(\lambda)a_f(\lambda)d\lambda.$$
						Si en particular, $\varphi(\lambda)=\lambda^p$, entonces podemos concluir que:
						$$||f||_{L^p}^p=p\int_0^\infty \lambda^{p-1}a_f(\lambda)\, d\lambda.$$}
	      \vspace{0.4cm}
       
						\textbf{Teorema (Teorema de interpolación de Marcinkiewicz).} \emph{Sean $(X, \mu)$ y $(Y, \nu)$ espacios medibles, $1 \leq p_0<p_1 \leq \infty$, y tome $T$ como un operador sublineal de $L^{p_0}(X, \mu)+L^{p_1}(X, \mu)$ a las funciones de medida de $Y$ que es débil $\left(p_0, p_0\right)$ y es débil $\left(p_1, p_1\right)$. Entonces $T$ es fuerte $(p, p)$ para $p_0<p<p_1$.}

\vspace{0.6cm}

{\bf Algunas consecuencias:}
    
La función maximal establece un operador fuerte $(\infty,\infty)$. Para ver esto, sea $f\in \mathcal{L}^{\infty}$, entonces para $x\in \mathbb{R}^N$ y $r>0$ arbitrarios se tiene
\begin{equation*}
    \frac{1}{|B_r|}\int_{B_r} |f(x-y)|\, dy \leq \|f\|_{\infty}.
\end{equation*}
Luego tomando el supremo en $r>0$, tenemos $\mathcal{M}f(x)\leq \|f\|_{\infty}$. Ahora, como $x\in \mathbb{R}^N$ es arbitrario, se deduce que
\begin{equation*}
  \|\mathcal{M}f\|_{\infty}\leq \|f\|_{\infty}.   
\end{equation*}
Luego $\mathcal{M}$ define un operador $(\infty,\infty)$. En trabajos futuros estudiaremos el siguiente resultado.

\vspace{0.6cm}

{\bf Teorema}. {El operador $\mathcal{M}$ es débil $(1,1)$.}

\vspace{0.3cm}

Como consecuencia del teorema de interpolación de Marcinkiewicz y lo anterior, tenemos que $\mathcal{M}$ es un operador fuerte $(p,p)$ para todo $1<p\leq \infty$.

\vspace{0.4cm}


Notemos que dada $f\in \mathcal{L}^1$, $\mathcal{M}f \in \mathcal{L}^1$ si y solo si $f=0$. Por lo que no se espera la acotación fuerte en $\mathcal{L}^1$. 



						
		\end{minipage}}}}			

    
				
		\vspace{-1.3cm}
		%%%%%%%%%%%%%%%%%%%%%%%%%%%%%%%%%%%%%%%%%%%%%%%%%%%%%%%%%%%%%%%%%%%%%%%%%%%%%%%%%%%%%%%%%%%%%%%%%%%%%%%%%%%%%%%%%%%%
		
		\section{Trabajo Futuro}
		
		{\fboxsep0.5mm\shadowsize4mm \shadowbox{\fboxsep5mm\colorbox{anita6}{\color{white}
					\begin{minipage}{.93\columnwidth}
						

      \begin{itemize}
        \item Demostrar que $\mathcal{M}$ es débil $(1,1)$. Para esto tendremos que estudiar conceptos como la función maximal diádica y la descomposición de Calderón-Zygmund.
          \item Una vez tengamos estos resultados, deduciremos el teorema de diferenciación de Lebesgue.  Para esto usaremos el resultado enunciado antes para ver que 
          $$\mathcal{L}^p=\{f\in \mathcal{L}^p\, |\, \lim_{r\to 0^{+}} \frac{1}{|B_r|}\int_{B_r}f(x-y)\, dy=f(x)\, \,  \text{c.t.p}\},$$
          para $1\leq p <\infty$.
          \item Mostraremos que si $\phi$ es una función positiva, radial, decreciente (como función de $(0,\infty)$), entonces $\sup_{t}|\phi_t\ast f(x)|\leq \|\phi\|_1\mathcal{M}f(x)$. Como consecuencia tenemos que la función maximal $\sup_{t}|\phi_t\ast f(x)|$ es débil $(1,1)$ y fuerte $(p,p)$, $1\leq p \leq \infty$. Este resultado permite generalizar (sin usar secuencias $\{t_n\}$) el resultado de convergencia mencionado anteriormente para aproximaciones de la identidad.
          \item Dado $1\leq p\leq \infty$. Recordamos que el \emph{espacio de Sobolev} $W^{1,p}(\mathbb{R}^N)$ comprende todas las funciones $f\in \mathcal{L}^p$ tales que $f$ tiene gradiente débil y $\nabla f\in \mathcal{L}^p$. A este espacio se le asigna la norma
          \begin{equation*}
              \|f\|_{W^{1,p}}=\|f\|_{L^p}+\|\nabla f\|_{L^p}.
          \end{equation*}

      La idea es estudiar artículos como el de Kinnunen (1997) donde se verifica  que
      \begin{equation*}
          \mathcal{M}:W^{1,p}(\mathbb{R}^N)\rightarrow W^{1,p}(\mathbb{R}^N) 
      \end{equation*}
      establece un operador acotado cuando $1<p\leq \infty$.


            \end{itemize}
            
						\vspace{-0.3cm}
						
		\end{minipage}}}}
		\vspace{-1.5cm}
		
		\section{Agradecimientos}
		
		{\fboxsep0.5mm\shadowsize4mm \shadowbox{\fboxsep5mm\colorbox{anita3}{\color{white}
					\begin{minipage}{.93\columnwidth}
						
					Agradezco al Semillero de Análisis Armónico y Ecuaciones Diferenciales Parciales de la Universidad Nacional de Colombia - Sede Bogotá, especialmente a los docentes Ricardo Ariel Pastrán Ramírez y Oscar Guillermo Riaño Castañeda, quienes orientaron el proceso de desarrollo de este póster, tanto en la parte teórica como en la presentación del mismo.
						
						\vspace{-0.3cm}
						
		\end{minipage}}}}
		\vspace{-1.5cm}
		
		
		\begin{thebibliography}{4}
			{\fboxsep0.5mm\shadowsize4mm \shadowbox{\fboxsep12mm\colorbox{anita2}{\color{white}
						\begin{minipage}{.87\columnwidth}
							
							\bibitem{J1} J. Duoandikoetxea, \textit{Fourier Analysis}, Grad. Stud. Math., vol. 29, American Mathematical Society, Providence, RI, 2001, translated and revised from the 1995 Spanish original by D. Cruz-Uribe.

   \bibitem{K1}    J. Kinnunen, \textit{The Hardy-Littlewood maximal function of a Sobolev function}, Israel J. Math. 100 (1997), 117–124.
  \bibitem{K2}  J. Kinnunen and P. Lindqvist, \textit{The derivative of the maximal function}, J. Reine Angew. Math. 503 (1998), 161–167.
							
			\end{minipage}}}}
		\end{thebibliography}
		
		%\textcolor{Cyan}{
			%\normalsize H. Fabian Ramírez (hectorfabian.ramirez@um.es) trabajo conjunto con Pascual Lucas (plucas@um.es) \normalsize Departamento de Matemáticas, Universidad de Murcia. Partially supported by MINECO and FEDER project MTM2012-34037.}
	\end{multicols}
\end{document}
