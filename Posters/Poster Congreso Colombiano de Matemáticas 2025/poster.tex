%!TEX program = lualatex
% Unofficial University of Cambridge Poster Template
% https://github.com/andiac/gemini-cam
% a fork of https://github.com/anishathalye/gemini
% also refer to https://github.com/k4rtik/uchicago-poster

\documentclass[final]{beamer}

% ====================
% Packages
% ====================

\usepackage[T1]{fontenc}
\usepackage{lmodern}
\usepackage[orientation=portrait,size=a2,scale=1.15]{beamerposter}
\usetheme{gemini}
\usecolortheme{nott}
\usepackage{graphicx}
\usepackage{booktabs}
\usepackage[numbers]{natbib} % o [authoryear] según prefieras
\usepackage{tikz}
\usepackage{pgfplots}
\pgfplotsset{compat=1.14}
\usepackage{anyfontsize}


% ====================
% Lengths
% ====================

% If you have N columns, choose \sepwidth and \colwidth such that
% (N+1)*\sepwidth + N*\colwidth = \paperwidth
\newlength{\sepwidth}
\newlength{\colwidth}
\setlength{\sepwidth}{0.025\paperwidth}
\setlength{\colwidth}{0.45\paperwidth}

\newcommand{\separatorcolumn}{\begin{column}{\sepwidth}\end{column}}

% ====================
% Title
% ====================

\title{Sobre el operador maximal de Hardy-Littlewood}

\author{Andrés David Cadena Simons \and Oscar Riaño}

\institute[shortinst]{Semillero de Análisis Armónico y Ecuaciones Diferenciales Parciales, Departamento de Matemáticas, Universidad Nacional de Colombia}

% ====================
% Footer (optional)
% ====================

\footercontent{
  Congreso Colombiano de Matemáticas --- 2025 \hfill
  \href{mailto:acadenas@unal.edu.co}{acadenas@unal.edu.co}}
% (can be left out to remove footer)


% ====================
% Logo (optional)
% ====================

% use this to include logos on the left and/or right side of the header:
\logoright{\includegraphics[height=2.5cm]{logos/logo.png}}
%\logoleft{\hspace{20ex}\includegraphics[height=3.5cm]{logos/ppgca-logo.png}}

% ====================
% Body
% ====================

\begin{document}

% Refer to https://github.com/k4rtik/uchicago-poster
% logo: https://www.cam.ac.uk/brand-resources/about-the-logo/logo-downloads
% \addtobeamertemplate{headline}{}
% {
%     \begin{tikzpicture}[remember picture,overlay]
%       \node [anchor=north west, inner sep=3cm] at ([xshift=-2.5cm,yshift=1.75cm]current page.north west)
%       {\includegraphics[height=7cm]{logos/unott-logo.eps}}; 
%     \end{tikzpicture}
% }

\begin{frame}[t]
\begin{columns}[t]
\separatorcolumn

\begin{column}{\colwidth}

  \begin{block}{Conceptos y definiciones}
    \textbf{Función de distribución}\\
		  Sea $(X,\mu)$ un espacio de medida y sea $f:X\rightarrow \mathbb{C}$ una función medible.\\
			Se llama función de distribución de $f$ asociadad a $\mu$ a la función:
			\begin{align*}
				a_f:(0,\infty)&\rightarrow [0,\infty]\\
				\lambda&\rightarrow \mu(\{x\in X: |f(x)|>\lambda\})	
				\end{align*}
    \textbf{Desigualdades débiles y fuertes}\\
			Sean $(X,\mu)$ y $(Y, \nu)$ dos espacios de medida y sea $T$ un operador de $\mathcal{L}^{p}(X,\mu)$, en el espacio de funciones medibles de $Y$ en $\mathbb{C}$.
			$$T:\mathcal{L}^p(X,\mu)\rightarrow\mathcal{M}(Y,\mathbb{C})$$
			\begin{itemize}
				\item[i.] Se dice que $T$ es $(p,q)$-débil (con $q<\infty$) si para todo $\lambda>0$ existe $C>0$ tal que:
				  $$\nu(\{y\in Y:|(Tf)(y)|>\lambda\})\leq \left(\frac{C||f||_p}{\lambda}\right)^q.$$
				\item[ii.] Se dice que $T$ es $(p,\infty)$-débil si está acotado de $\mathcal{L}^p(X,\mu)$ en $\mathcal{L}^{\infty}(Y,\nu)$.
				\item[iii.] Se dice que $T$ es $(p,q)$-fuerte si está acotado de $\mathcal{L}^p(X,\mu)$ en $\mathcal{L}^q(Y,\nu)$.
			\end{itemize}				
 		\textbf{Función Maximal de Hardy-Littlewood}\\
			Sea $B_r$ la bola euclidea centrada en el origen y de radio $r$. Definiremos la función maximal de Hardy-Littlewood de una función localmente integrable $f$ en $\mathbb{R}^n$ como:
			$$\mathcal{M}f(x)=\sup_{r>0}\frac{1}{|B_r|}\int_{B_r}|f(x-y)|dy$$
    \textbf{Espacios de Sobolev}\\
      Dado $1\leq p\leq \infty$. Recordamos que el \emph{espacio de Sobolev} $W^{1,p}(\mathbb{R}^N)$ comprende todas las funciones $f\in \mathcal{L}^p$ tales que $f$ tiene gradiente débil y $\nabla f\in \mathcal{L}^p$. A este espacio se le asigna la norma
          \begin{equation*}
              \|f\|_{W^{1,p}}=\|f\|_{L^p}+\|\nabla f\|_{L^p}.
          \end{equation*}
  \end{block}

  \begin{alertblock}{Resultados}
    \textbf{Teorema (Teorema de interpolación de Marcinkiewicz).} \emph{Sean $(X, \mu)$ y $(Y, \nu)$ espacios medibles, $1 \leq p_0<p_1 \leq \infty$, y tome $T$ como un operador sublineal de $L^{p_0}(X, \mu)+L^{p_1}(X, \mu)$ a las funciones de medida de $Y$ que es débil $\left(p_0, p_0\right)$ y es débil $\left(p_1, p_1\right)$. Entonces $T$ es fuerte $(p, p)$ para $p_0<p<p_1$.}\\
      \vspace{0.2cm}
      {\bf Algunas consecuencias:}\\
      La función maximal establece un operador fuerte $(\infty,\infty)$. Para ver esto, sea $f\in \mathcal{L}^{\infty}$, entonces para $x\in \mathbb{R}^N$ y $r>0$ arbitrarios se tiene
      \begin{equation*}
        \frac{1}{|B_r|}\int_{B_r} |f(x-y)|\, dy \leq \|f\|_{\infty}.
      \end{equation*}
      Luego tomando el supremo en $r>0$, tenemos $\mathcal{M}f(x)\leq \|f\|_{\infty}$. Ahora, como $x\in \mathbb{R}^N$ es arbitrario, se deduce que
      \begin{equation*}
        \|\mathcal{M}f\|_{\infty}\leq \|f\|_{\infty}.   
      \end{equation*}
      Luego $\mathcal{M}$ define un operador $(\infty,\infty)$.\\
      \vspace{0.2cm}
      {\bf Teorema}. {El operador $\mathcal{M}$ es débil $(1,1)$.} Para ver esto $\cdots$\\
      \vspace{0.2cm}
      Como consecuencia del teorema de interpolación de Marcinkiewicz y lo anterior, tenemos que $\mathcal{M}$ es un operador fuerte $(p,p)$ para todo $1<p\leq \infty$.\\
      \vspace{0.2cm}
      Notemos que dada $f\in \mathcal{L}^1$, $\mathcal{M}f \in \mathcal{L}^1$ si y solo si $f=0$. Por lo que no se espera la acotación fuerte en $\mathcal{L}^1$.\\
      \vspace{0.2cm}
    \textbf{Teorema (Teorema de diferenciación de Lebesgue).}\\
      \vspace{0.2cm}
    \textbf{Teorema} Si $\phi$ es una función positiva, radial, decreciente (como función de $(0,\infty)$), entonces $\sup_{t}|\phi_t\ast f(x)|\leq \|\phi\|_1\mathcal{M}f(x)$. Como consecuencia tenemos que la función maximal $\sup_{t}|\phi_t\ast f(x)|$ es débil $(1,1)$ y fuerte $(p,p)$, $1\leq p \leq \infty$. Este resultado permite generalizar (sin usar secuencias $\{t_n\}$) el resultado de convergencia mencionado anteriormente para aproximaciones de la identidad.\\ 
      \vspace{0.2cm}
      \textbf{Teorema \cite{MR1469106}} Se verifica  que
      \begin{equation*}
          \mathcal{M}:W^{1,p}(\mathbb{R}^n)\rightarrow W^{1,p}(\mathbb{R}^n) 
      \end{equation*}
      establece un operador acotado cuando $1<p\leq \infty$.
  \end{alertblock}

\end{column}

\separatorcolumn

\begin{column}{\colwidth}

  \begin{exampleblock}{Trabajo Futuro}

    A different kind of highlighted block.

    $$
    \int_{-\infty}^{\infty} e^{-x^2}\,dx = \sqrt{\pi}
    $$

    Interdum et malesuada fames $\{1, 4, 9, \ldots\}$ ac ante ipsum primis in
    faucibus. Cras eleifend dolor eu nulla suscipit suscipit. Sed lobortis non
    felis id vulputate.

    \heading{A heading inside a block}

    Praesent consectetur mi $x^2 + y^2$ metus, nec vestibulum justo viverra
    nec. Proin eget nulla pretium, egestas magna aliquam, mollis neque. Vivamus
    dictum $\mathbf{u}^\intercal\mathbf{v}$ sagittis odio, vel porta erat
    congue sed. Maecenas ut dolor quis arcu auctor porttitor.

    \heading{Another heading inside a block}

    Sed augue erat, scelerisque a purus ultricies, placerat porttitor neque.
    Donec $P(y \mid x)$ fermentum consectetur $\nabla_x P(y \mid x)$ sapien
    sagittis egestas. Duis eget leo euismod nunc viverra imperdiet nec id
    justo.

  \end{exampleblock}

  \begin{block}{Agradecimientos}
    Agradezco al Semillero de Análisis Armónico y Ecuaciones Diferenciales Parciales de la Universidad Nacional de Colombia - Sede Bogotá, especialmente a los docentes Ricardo Ariel Pastrán Ramírez y Oscar Guillermo Riaño Castañeda, quienes orientaron el proceso de desarrollo de este póster, tanto en la parte teórica como en la presentación del mismo.
  \end{block}

  \begin{block}{Referencias}

    \nocite{*}
    \footnotesize{
      \bibliographystyle{plainnat}
      \bibliography{poster}
    }

  \end{block}

\end{column}
\separatorcolumn



\end{columns}
\end{frame}

\end{document}
