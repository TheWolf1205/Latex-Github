%!TEX program = lualatex
% Unofficial University of Cambridge Poster Template
% https://github.com/andiac/gemini-cam
% a fork of https://github.com/anishathalye/gemini
% also refer to https://github.com/k4rtik/uchicago-poster

\documentclass[final]{beamer}

% ====================
% Packages
% ====================

\usepackage[T1]{fontenc}
\usepackage{lmodern}
\usepackage[orientation=portrait,size=a2,scale=1]{beamerposter}
\usetheme{gemini}
\usecolortheme{nott}
\usepackage{graphicx}
\usepackage{booktabs}
\usepackage[numbers]{natbib} % o [authoryear] según prefieras
\usepackage{tikz}
\usepackage{pgfplots}
\pgfplotsset{compat=1.14}
\usepackage{anyfontsize}

\providecommand{\abs}[1]{\left|#1\right|}
\providecommand{\norm}[1]{\left\|#1\right\|}


% ====================
% Lengths
% ====================

% If you have N columns, choose \sepwidth and \colwidth such that
% (N+1)*\sepwidth + N*\colwidth = \paperwidth
\newlength{\sepwidth}
\newlength{\colwidth}
\setlength{\sepwidth}{0.025\paperwidth}
\setlength{\colwidth}{0.45\paperwidth}

\newcommand{\separatorcolumn}{\begin{column}{\sepwidth}\end{column}}

% ====================
% Title
% ====================

\title{Sobre la función maximal de Hardy-Littlewood en espacios de Sobolev}

\author{Andrés David Cadena Simons}

\institute[shortinst]{Semillero de Análisis Armónico y Ecuaciones Diferenciales Parciales, Departamento de Matemáticas, Universidad Nacional de Colombia sede Bogotá.}

% ====================
% Footer (optional)
% ====================

\footercontent{
  Congreso Colombiano de Matemáticas --- 2025 \hfill
  \href{mailto:acadenas@unal.edu.co}{acadenas@unal.edu.co}}
% (can be left out to remove footer)


% ====================
% Logo (optional)
% ====================

% use this to include logos on the left and/or right side of the header:
\logoright{\includegraphics[height=3.5cm]{logos/logo.png}}
%\logoleft{\hspace{20ex}\includegraphics[height=3.5cm]{logos/ppgca-logo.png}}

% ====================
% Body
% ====================

\begin{document}

% Refer to https://github.com/k4rtik/uchicago-poster
% logo: https://www.cam.ac.uk/brand-resources/about-the-logo/logo-downloads
% \addtobeamertemplate{headline}{}
% {
%     \begin{tikzpicture}[remember picture,overlay]
%       \node [anchor=north west, inner sep=3cm] at ([xshift=-2.5cm,yshift=1.75cm]current page.north west)
%       {\includegraphics[height=7cm]{logos/unott-logo.eps}}; 
%     \end{tikzpicture}
% }

\begin{frame}[t]
\begin{columns}[t]
\separatorcolumn

\begin{column}{\colwidth}

  \begin{block}{Conceptos y definiciones}
    \textbf{Función de distribución}\\
		  Sea $(X,\mu)$ un espacio de medida y sea $f:X\rightarrow \mathbb{C}$ una función medible.\\
			Se llama función de distribución de $f$ asociada a $\mu$ a la función:
			\begin{align*}
				a_f:(0,\infty)&\rightarrow [0,\infty],\\
				\lambda&\rightarrow \mu(\{x\in X: |f(x)|>\lambda\}).	
				\end{align*}
    \textbf{Desigualdades débiles y fuertes}\\
			Sean $(X,\mu)$ y $(Y, \nu)$ dos espacios de medida y sea $T$ un operador de $\mathcal{L}^{p}(X,\mu)$, en el espacio de funciones medibles de $Y$ en $\mathbb{C}$.
			$$T:\mathcal{L}^p(X,\mu)\rightarrow\mathcal{M}(Y,\mathbb{C}).$$
			\begin{itemize}
				\item[i.] Se dice que $T$ es $(p,q)$-débil (con $q<\infty$) si para todo $\lambda>0$ existe $C>0$ tal que:
				  $$\nu(\{y\in Y:|(Tf)(y)|>\lambda\})\leq \left(\frac{C||f||_p}{\lambda}\right)^q.$$
				\item[ii.] Se dice que $T$ es $(p,\infty)$-débil si está acotado de $\mathcal{L}^p(X,\mu)$ en $\mathcal{L}^{\infty}(Y,\nu)$.
				\item[iii.] Se dice que $T$ es $(p,q)$-fuerte si está acotado de $\mathcal{L}^p(X,\mu)$ en $\mathcal{L}^q(Y,\nu)$.
			\end{itemize}				
 		\textbf{Función Maximal de Hardy-Littlewood}\\
			Sea $B_r$ la bola euclídea centrada en el origen y de radio $r$. Definiremos la función maximal de Hardy-Littlewood de una función localmente integrable $f$ en $\mathbb{R}^n$ como:
			$$\mathcal{M}f(x)=\sup_{r>0}\frac{1}{|B_r|}\int_{B_r}|f(x-y)|dy.$$
    \textbf{Espacios de Sobolev}\\
      Dado $1\leq p\leq \infty$. Recordamos que el \emph{espacio de Sobolev} $W^{1,p}(\mathbb{R}^N)$ comprende todas las funciones $f\in \mathcal{L}^p$ tales que $f$ tiene gradiente débil y $\nabla f\in \mathcal{L}^p$. A este espacio se le asigna la norma
          \begin{equation*}
              \|f\|_{W^{1,p}}=\|f\|_{L^p}+\|\nabla f\|_{L^p}.
          \end{equation*}
  \end{block}

  \begin{alertblock}{Resultados}
    \textbf{Teorema (Teorema de interpolación de Marcinkiewicz).} \emph{Sean $(X, \mu)$ y $(Y, \nu)$ espacios medibles, $1 \leq p_0<p_1 \leq \infty$, y tome $T$ como un operador sublineal de $L^{p_0}(X, \mu)+L^{p_1}(X, \mu)$ a las funciones de medida de $Y$ que es débil $\left(p_0, p_0\right)$ y es débil $\left(p_1, p_1\right)$. Entonces $T$ es fuerte $(p, p)$ para $p_0<p<p_1$.}\\
      \vspace{0.2cm}
      {\bf Algunas consecuencias:}\\
      La función maximal establece un operador fuerte $(\infty,\infty)$. Para ver esto, sea $f\in \mathcal{L}^{\infty}$, entonces para $x\in \mathbb{R}^N$ y $r>0$ arbitrarios se tiene
      \begin{equation*}
        \frac{1}{|B_r|}\int_{B_r} |f(x-y)|\, dy \leq \|f\|_{\infty}.
      \end{equation*}
      Luego tomando el supremo en $r>0$, tenemos $\mathcal{M}f(x)\leq \|f\|_{\infty}$. Ahora, como $x\in \mathbb{R}^N$ es arbitrario, se deduce que
      \begin{equation*}
        \|\mathcal{M}f\|_{\infty}\leq \|f\|_{\infty}.   
      \end{equation*}
      Luego $\mathcal{M}$ define un operador $(\infty,\infty)$.\\
      \vspace{0.2cm}
      {\bf Teorema}. {El operador $\mathcal{M}$ es débil $(1,1)$.} \emph{Es decir, sea $f\in L^{1}(\mathbb{R}^{n})$, entonces
      \begin{align*}
        a_{\mathcal{M}f}(\lambda)\leq \frac{C\norm{f}_{1}}{\lambda}.
      \end{align*}}
      \vspace{0.2cm}
      Como consecuencia del teorema de interpolación de Marcinkiewicz y lo anterior, tenemos que $\mathcal{M}$ es un operador fuerte $(p,p)$ para todo $1<p\leq \infty$.\\
      \vspace{0.2cm}
      Notemos que dada $f\in \mathcal{L}^1$, $\mathcal{M}f \in \mathcal{L}^1$ si y solo si $f=0$. Por lo que no se espera la acotación fuerte en $\mathcal{L}^1$.\\
      \vspace{0.2cm}
    \textbf{Teorema (Teorema de diferenciación de Lebesgue).} \emph{Sea $f\in L^{1}_{loc}(\mathbb{R}^{n})$, entonces
      \begin{align*}
        \lim_{r \to 0^{+}}\frac{1}{|B_{r}(x)|}\int_{B_{r}(x)}|f(y)|\, dy=f(x)\quad c.t.p.
      \end{align*}}
      \vspace{0.2cm}
    \textbf{Teorema.} \emph{Si $\phi$ es una función positiva, radial, decreciente (como función de $(0,\infty)$), entonces tomando $\phi_{t}(x)=t^{-n}\phi(t^{-1}x)$ se cumple que $\sup_{t}|\phi_t\ast f(x)|\leq \|\phi\|_1\mathcal{M}f(x)$.}\\
      Supongamos $\phi$ función simple, entonces
      \begin{align*}
        \left|\phi*f(x)\right|&=\left|\left(\sum_{j=1}^{\infty}c_j\chi_{B_{r_j}(x_j)}*f\right)(x)\right|,\\
        &=\left|\sum_{j=1}^{\infty}c_{j}|B_{r_j}(x_j)|\left( \frac{1}{|B_{r_{j}}(x_j)|}\chi_{B_{r_j}(x_j)}*f \right)(x)\right|,\\
        &\leq\norm{\phi}_{1}\mathcal{M}f(x).
      \end{align*}
      Luego como $\phi_{t}$ es una dilatación de $\phi$ se puede hacer el mismo procedimiento y concluir el mismo resultado.\\
      \vspace{0.2cm}
      Como consecuencia tenemos que la función maximal $\sup_{t}|\phi_t\ast f(x)|$ es débil $(1,1)$ y fuerte $(p,p)$, $1\leq p \leq \infty$.
      Este resultado permite generalizar (sin usar secuencias $\{t_n\}$) el resultado de convergencia mencionado anteriormente para aproximaciones de la identidad.\\ 
    \textbf{Teorema \cite{MR1469106}.} \emph{Se verifica  que
      \begin{equation*}
          \mathcal{M}:W^{1,p}(\mathbb{R}^n)\rightarrow W^{1,p}(\mathbb{R}^n). 
      \end{equation*}
      establece un operador acotado cuando $1<p\leq \infty$.}\\
      \vspace{0.2cm}
      \textbf{(Demostración (idea))} Es suficiente con mostrar que si $u\in W^{1,p}(\mathbb{R}^{n})$ con $1<p< \infty$, entonces $\mathcal{M}f(x)$ tiene derivada débil y se satisface que
      \begin{align*}
        |\partial_{x_i}(\mathcal{M}u)|\leq \mathcal{M}\partial_{x_i}u.
      \end{align*}
      Por otro lado, cuando $p=\infty$ vamos a mostrar que $\mathcal{M}f$ es Lipschitz. 
  \end{alertblock}
\end{column}

\separatorcolumn

\begin{column}{\colwidth}

  \begin{alertblock}{}
      Para ver la primera desigualdad, es suficiente con ver que si $u\in W^{1,p}(\mathbb{R}^{n})$, entonces $|u|\in W^{1,p}(\mathbb{R}^{n})$ y que se satisface
      \begin{align*}
        \partial_{x_i}\left( |u|*\chi_{r} \right)=\partial_{x_i}|u|*\chi_{r}.
      \end{align*}
      con $\chi_{r}(x)=\frac{\chi_{B_{r}(0)}}{|B_{r}(0)|}(x)$, para todo $i=1,\cdots,n$.\\
      Tome una sucesión de $r_{j}>0$ con $j=1,2,\cdots$ como una enumeración de racionales positivos y definamos las funciones $v_{k}:\mathbb{R}^{n}\to\mathbb{R}$, con $k=1,2\cdots$ por 
      \begin{align*}
        v_{k}(x)=\max_{1\leq j\leq k}\left( |u|*\chi_{r_{j}} \right)(x).
      \end{align*}
      Note que la sucesión de funciones $\{v_{k}\}\subset W^{1,p}(\mathbb{R}^{n})$ es creciente y acotada, luego por construcción se tiene la convergencia puntual a $\mathcal{M}\partial_{x_i}u$ ya que
      \begin{align*}
        \sup_{k}|\partial_{x_i}v_{k}|\leq\sup_{k}\max_{1\leq j\leq k}|\partial_{x_i}|u|*\chi_{r_{j}}|\leq \mathcal{M}\partial_{x_i}u.
      \end{align*}
      Así, tomando norma en $L^{p}(\mathbb{R}^{n})$ se cumple que
      \begin{align*}
        \norm{\partial v_{k}}_{p}\leq\sum_{i=1}^{n}\norm{\partial_{x_{i}}v_k}_{p}\leq\sum_{i=1}^{n}\norm{\mathcal{M}\partial_{x_i} u}_{p}.
      \end{align*}
      Luego, usando el teorema de Hardy-Littlewood se satisface que
      \begin{align*}
        \norm{v_k}_{1,p}&\leq \norm{\mathcal{M}u}_{p}+\sum_{i=1}^{n}\norm{\mathcal{M}\partial_{x_{i}}u}_{p}\leq c_p\norm{u}_{p}+c_{p}\sum_{i=1}^{n}\norm{\partial_{x_{i}}u}_{p}<\infty,
      \end{align*}
      de esta manera, para todo $k=1,2,\cdots$, se cumple que la sucesión $\{v_{k}\}\subset W^{1,p}(\mathbb{R}^{n})$ es una sucesión creciente y acotada en $W^{1,p}(\mathbb{R}^{n})$. Por tanto, como el espacio $W^{1,p}(\mathbb{R}^{n})$ es reflexivo, entonces existe una subsucesión $\{v_{k_{i}}\}$ débilmente convergente, pero como la sucesión $\{v_{k}\}$ es creciente, entonces la sucesión completa converge débilmente a $\mathcal{M}|u|$, por la misma razón $\{\partial_{x_{i}}v_{k}\}$ converge débilmente a $\mathcal{M}\partial_{x_i}|u|$. Lo que nos permite concluir que
      \begin{align*}
        |\partial_{x_i}\mathcal{M}u|\leq \mathcal{M}\partial_{x_i}u.
      \end{align*}
      \emph{ Sea $f\in W^{1,\infty}(\mathbb{R}^n)$. Veamos que $\mathcal{M}(f)$ es Lipschitz continua.} Para esto, sean $x,y\in \mathbb{R}^n$ y $r>0$. Haciendo un cambio de variables para centrar las bolas en $0$ y algunas propiedades del valor absoluto, tenemos que
      \begin{align*}
        \frac{1}{|B_{r}(x)|}\int_{B_r(x)} |f(z)|\, dz-\mathcal{M}f(y)\leq \frac{1}{|B_{r}(0)|}\int_{B_{r}(0)} \Big||f(x+w)-f(y+w)|\Big|\, dw.
      \end{align*}
      Como $f\in W^{1,\infty}(\mathbb{R}^n)$, por el teorema del valor intermedio se sigue que
      \begin{equation*}
      \big|f(x+w)-f(y+w)\big|\leq \|\nabla f\|_{L^{\infty}} |x-y|.   
      \end{equation*}
      Concluimos que 
      \begin{equation*}
      \begin{aligned}
      \frac{1}{|B(x,r)|}&\int_{B(x,r)} |f(z)|\, dz-\mathcal{M}f(y)\leq & \|\nabla f\|_{L^{\infty}} |x-y|.
      \end{aligned}
      \end{equation*}
      Tomando el supremo sobre $r>0$, se sigue de la desigualdad anterior 
      \begin{equation*}
      \begin{aligned}
      \mathcal{M}f(x)-\mathcal{M}f(y)\leq & \|\nabla f\|_{L^{\infty}} |x-y|.
      \end{aligned}
      \end{equation*}
      Y cambiando los papeles de $x$ y $y$, se llega a
      \begin{equation*}
      \big|\mathcal{M}f(y)-\mathcal{M}f(x)\big| \leq \|\nabla f\|_{L^{\infty}} |x-y|.    
      \end{equation*}
      De lo anterior, se sigue que $\mathcal{M}f$ es Lipschitz continua. Por tanto, la caracterización de $W^{1,\infty}(\mathbb{R}^{n})$ muestra que $\mathcal{M}f\in W^{1,\infty}(\mathbb{R}^n)$ y
      \begin{equation*}
        \|\nabla \mathcal{M}f\|_{L^{\infty}}\leq \|\nabla f\|_{L^{\infty}}.  
      \end{equation*}
  \end{alertblock}

  \begin{exampleblock}{Trabajo Futuro}
   \begin{itemize}
     \item Estudiar propiedades adicionales de la función maximal de Hardy–Littlewood en espacios de Sobolev. Por ejemplo, consultar los resultados presentados en \cite{Mazya2009}, páginas 25–67.
     \item Analizar otros tipos de funciones maximales. Por ejemplo, considerar el núcleo del calor
     \begin{equation*}
        K_t(x)=\frac{1}{(4\pi t)^{\frac{d}{2}}}e^{-\frac{|x|^2}{4t}}. 
     \end{equation*}
     Estudiar el operador maximal asociado,
     \begin{equation*}
         \mathcal{K} f(x):=\sup_{t>0}\big(f\ast K_t\big)(x),
     \end{equation*}
     en espacios de Sobolev $W^{1,p}(\mathbb{R}^n)$. Para ello, se pueden consultar los artículos \cite{CarneiroFinderSousa2018,CarneiroBenar2013}.
  \end{itemize}
  \end{exampleblock}

  \begin{block}{Agradecimientos}
    Agradezco al Semillero de Análisis Armónico y Ecuaciones Diferenciales Parciales de la Universidad Nacional de Colombia - Sede Bogotá, especialmente a los docentes Ricardo Ariel Pastrán Ramírez y Oscar Guillermo Riaño Castañeda, quienes orientaron el proceso de desarrollo de este póster, tanto en la parte teórica como en la presentación del mismo.
  \end{block}

  \begin{block}{Referencias}

    \nocite{*}
    \footnotesize{
      \bibliographystyle{plainnat}
      \bibliography{poster}
    }

  \end{block}

\end{column}
\separatorcolumn



\end{columns}
\end{frame}

\end{document}
