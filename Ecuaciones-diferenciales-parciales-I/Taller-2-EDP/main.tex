\documentclass{article}
\usepackage{tikz}
\usetikzlibrary{positioning, automata}
\usepackage{fancyhdr}
\usepackage{extramarks}
\usepackage[plain]{algorithm}
\usepackage{algpseudocode}
\usepackage[utf8]{inputenc} 
\usepackage[T1]{fontenc}    
\usepackage{hyperref}       
\usepackage{url}        
\usepackage{booktabs}
\usepackage{pdfpages}
\usepackage{amsfonts}   
\usepackage{nicefrac}       
\usepackage{microtype}      
\usepackage{amsmath}
\usepackage{graphicx}
\usepackage{float}
\usepackage{caption}
\usepackage{ragged2e}
\usepackage{amssymb}
\usepackage{mathtools}
\usepackage{xcolor}
\usepackage[spanish]{babel} % ¡Solo una vez!
\usepackage{array}
\usepackage{multirow}
\usepackage{multicol}
\usepackage{manfnt}
\usepackage{layout}
\usepackage{manfnt}
\usepackage{phaistos}
\usepackage{polynom}
\usepackage[inkscapeformat=png]{svg}
\usepackage[most]{tcolorbox}
\usepackage{listings}

\lstset{
    language=Python,                        % Lenguaje de programación
    basicstyle=\ttfamily\small\color[HTML]{F8F8F2}, % Fuente monoespaciada y texto blanco claro
    backgroundcolor=\color[HTML]{282A36},   % Fondo oscuro (Dracula)
    keywordstyle=\color[HTML]{FF79C6}\bfseries, % Palabras clave en rosa y negrita
    commentstyle=\color[HTML]{6272A4}\itshape, % Comentarios en gris azulado y cursiva
    stringstyle=\color[HTML]{F1FA8C},       % Cadenas en amarillo
    numberstyle=\tiny\color[HTML]{BD93F9}, % Números en lila
    identifierstyle=\color[HTML]{8BE9FD},  % Identificadores en azul claro
    emph={xlabel, ylabel, title, legend, grid}, % Destaca palabras específicas
    emphstyle=\color[HTML]{50FA7B},        % Color verde brillante para palabras destacadas
    numbers=left,                           % Números de línea a la izquierda
    stepnumber=1,                           % Mostrar número en cada línea
    numbersep=10pt,                         % Separación de los números de línea
    showstringspaces=false,                 % No marcar los espacios en las cadenas
    frame=note,                           % Marco alrededor del código
    framexleftmargin=20pt,                  % Espacio a la izquierda del marco para los números
    rulecolor=\color[HTML]{F8F8F2},         % Color del borde del marco
    breaklines=true,                        % Rompe líneas largas
    breakatwhitespace=true,                 % Rompe en los espacios
    tabsize=4,                              % Tamaño de tabuladores
    escapeinside={\%*}{*)}                  % Permite la inserción de LaTeX dentro del código
}

\lstset{literate=
  {á}{{\'a}}1 {é}{{\'e}}1 {í}{{\'i}}1 {ó}{{\'o}}1 {ú}{{\'u}}1
  {Á}{{\'A}}1 {É}{{\'E}}1 {Í}{{\'I}}1 {Ó}{{\'O}}1 {Ú}{{\'U}}1
  {à}{{\`a}}1 {è}{{\`e}}1 {ì}{{\`i}}1 {ò}{{\`o}}1 {ù}{{\`u}}1
  {À}{{\`A}}1 {È}{{\'E}}1 {Ì}{{\`I}}1 {Ò}{{\`O}}1 {Ù}{{\`U}}1
  {ä}{{\"a}}1 {ë}{{\"e}}1 {ï}{{\"i}}1 {ö}{{\"o}}1 {ü}{{\"u}}1
  {Ä}{{\"A}}1 {Ë}{{\"E}}1 {Ï}{{\"I}}1 {Ö}{{\"O}}1 {Ü}{{\"U}}1
  {â}{{\^a}}1 {ê}{{\^e}}1 {î}{{\^i}}1 {ô}{{\^o}}1 {û}{{\^u}}1
  {Â}{{\^A}}1 {Ê}{{\^E}}1 {Î}{{\^I}}1 {Ô}{{\^O}}1 {Û}{{\^U}}1
  {ã}{{\~a}}1 {ẽ}{{\~e}}1 {ĩ}{{\~i}}1 {õ}{{\~o}}1 {ũ}{{\~u}}1
  {Ã}{{\~A}}1 {Ẽ}{{\~E}}1 {Ĩ}{{\~I}}1 {Õ}{{\~O}}1 {Ũ}{{\~U}}1
  {œ}{{\oe}}1 {Œ}{{\OE}}1 {æ}{{\ae}}1 {Æ}{{\AE}}1 {ß}{{\ss}}1
  {ű}{{\H{u}}}1 {Ű}{{\H{U}}}1 {ő}{{\H{o}}}1 {Ő}{{\H{O}}}1
  {ç}{{\c c}}1 {Ç}{{\c C}}1 {ø}{{\o}}1 {å}{{\r a}}1 {Å}{{\r A}}1
  {€}{{\euro}}1 {£}{{\pounds}}1 {«}{{\guillemotleft}}1
  {»}{{\guillemotright}}1 {ñ}{{\~n}}1 {Ñ}{{\~N}}1 {¿}{{?`}}1 {¡}{{!`}}1 
}

\RequirePackage{algorithm}
\RequirePackage{algpseudocode}

%% Integral promedio
\def\Xint#1{\mathchoice
{\XXint\displaystyle\textstyle{#1}}%
{\XXint\textstyle\scriptstyle{#1}}%
{\XXint\scriptstyle\scriptscriptstyle{#1}}%
{\XXint\scriptscriptstyle\scriptscriptstyle{#1}}%
\!\int}
\def\XXint#1#2#3{{\setbox0=\hbox{$#1{#2#3}{\int}$ }
\vcenter{\hbox{$#2#3$ }}\kern-.57\wd0}}
\def\ddashint{\Xint=}
\def\dashint{\Xint-}


\setcounter{page}{0}
%
% Basic Document Settings
%

\topmargin=-0.45in
\evensidemargin=0in
\oddsidemargin=0in
\textwidth=6.5in
\textheight=9.0in
\headsep=0.25in

\linespread{1.1}
%|#*******************************************************************#|
\pagestyle{fancy}
\lhead{Taller}
\chead{\hmwkClass\ : \hmwkTitle}
\rhead{\firstxmark}
\lfoot{\lastxmark}
\cfoot{\thepage}

\renewcommand\headrulewidth{0.4pt}
\renewcommand\footrulewidth{0.4pt}

\setlength\parindent{0pt}

%
% Create Problem Sections
%

\newcommand{\enterProblemHeader}[1]{
    \nobreak\extramarks{}{Problema \arabic{#1} continúa en la siguiente página\ldots}\nobreak{}
    \nobreak\extramarks{Problema \arabic{#1} (continuación}{Problema \arabic{#1} continúa en la siguiente página\ldots}\nobreak{}
}

\newcommand{\exitProblemHeader}[1]{
    \nobreak\extramarks{Problema \arabic{#1} (continuación)}{Problema \arabic{#1} continúa en la siguiente página\ldots}\nobreak{}
    \stepcounter{#1}
    \nobreak\extramarks{Problema \arabic{#1}}{}\nobreak{}
}



\setcounter{secnumdepth}{0}
\newcounter{partCounter}
\newcounter{homeworkProblemCounter}
\setcounter{homeworkProblemCounter}{1}
\nobreak\extramarks{Problema \arabic{homeworkProblemCounter}}{}\nobreak
%margen de una pagina
\newenvironment{changemargin}[2]{%
\begin{list}{}{%
\setlength{\topsep}{0pt}%
\setlength{\leftmargin}{#1}%
\setlength{\rightmargin}{#2}%
\setlength{\listparindent}{\parindent}%
\setlength{\itemindent}{\parindent}%
\setlength{\parsep}{\parskip}%
}%
\item[]}{\end{list}}

%
% Homework Problem Environment
%
% This environment takes an optional argument. When given, it will adjust the
% problem counter. This is useful for when the problems given for your
% assignment aren't sequential. See the last 3 problems of this template for an
% example.
%
\newenvironment{homeworkProblem}[1][-1]{
    \ifnum#1>0
        \setcounter{homeworkProblemCounter}{#1}
    \fi
    \section{Problema \arabic{homeworkProblemCounter}:}
    \setcounter{partCounter}{1}
    \enterProblemHeader{homeworkProblemCounter}
}{
    \exitProblemHeader{homeworkProblemCounter}
}


%|#*******************************************************************#|

\newcommand{\hmwkTitle}{Taller 2}
\newcommand{\hmwkDueDate}{\today}
\newcommand{\hmwkClass}{Análisis Númerico}
\newcommand{\hmwkUniversity}{Universidad Nacional de Colombia}
\newcommand{\hmwkAuthorName}{Andrés David Cadena Simons \and acadenas@unal.edu.co \and Sandra Natalia Florez Garcia \and sflorezga@unal.edu.co}
\newcommand{\hmwkInstructor}{Jorge Mauricio Ruiz Vera}

%
% Title Page
%

\title{
    \vspace{2in}
    \textmd{\textbf{\hmwkClass:\ \hmwkTitle}}\\
    \normalsize\vspace{0.1in}\small{23 de enero del 2025}\\
    \vspace{0.1in}\large{\textit{\hmwkUniversity}}\\
    \vspace{1.5in} \textrm{\hmwkInstructor}
    \vspace{1.5in}
}

\author{\hmwkAuthorName}
\date{}

\renewcommand{\part}[1]{\textbf{\large Part \Alph{partCounter}}\stepcounter{partCounter}\\}
%
%   Nuevos tipos de columna
%
\newcolumntype{C}{>{$}c<{$}}
\newcolumntype{L}{>{$}l<{$}}
\newcolumntype{R}{>{$}r<{$}}
% ---------------------------
%|  Various Helper Commands  |
% ---------------------------

% Useful for algorithms
\newcommand{\alg}[1]{\textsc{\bfseries \footnotesize #1}}

% -For derivatives-
\newcommand{\deriv}[2]{\frac{\mathrm{d #1 }}{\mathrm{d} #2} }

%For derivatives of degree >1
\newcommand{\mderiv}[3]{\frac{\mathrm{d^{#3} #1 }}{\mathrm{d} #2} }

% -For partial derivatives-
\newcommand{\pderiv}[2]{\frac{\partial}{\partial #1} (#2)}

% -Integral dx-
\newcommand{\dx}{\hspace{3pt}\mathrm{d}}

% Alias for the Solution section header
\newcommand{\solution}{\textbf{\\\\\large Solución:\\ \hspace*{5pt}}}

%parentesis automatico
\newcommand{\parauto}[1]{\ensuremath{\left( #1 \right)}}

% Probability commands: Expectation, Variance, Covariance, Bias
\newcommand{\E}{\mathrm{E}}
\newcommand{\Var}{\mathrm{Var}}
\newcommand{\Cov}{\mathrm{Cov}}
\newcommand{\Bias}{\mathrm{Bias}}


%TcolorBox

% Definir colores y la tcolorbox de la solución
\definecolor{myDColor}{HTML}{101010} 

\definecolor{myLColor}{RGB}{153,204,255} 

\definecolor{LinkColor}{HTML}{9669d9} 


\newtcolorbox{solucion}[1][]{%
    enhanced,
    skin first=enhanced,
    skin middle=enhanced,
    skin last=enhanced,
    before upper={\parindent15pt},
    breakable,
    boxrule = 0pt,
    frame hidden,
    borderline west = {4pt}{0pt}{myDColor},
    colback = myLColor!5,
    coltitle = myLColor!5,
    sharp corners,
    rounded corners = southeast,
    arc is angular,
    arc = 3mm,
    attach boxed title to top left,
    boxed title style = {%
        enhanced,
        colback = myDColor,
        colframe = myDColor,
        top = 0pt,
        bottom = 0pt,
        sharp corners,
        rounded corners = northeast,
        arc is angular,
        arc = 2mm,
        rightrule = 0pt,
        bottomrule = 0pt,
        toprule = 0pt,
    },
    title = {\bfseries\large Solución:}, 
    overlay unbroken={%
        \node[anchor=west, color=black!70] at (title.east) {#1};
        \path[fill = tcbcolback!80!black] 
            ([yshift = 3mm]interior.south east) -- ++(-0.4,-0.1) -- ++(0.1,-0.2);
    },
    overlay first = {%
        \node[anchor=west, color=black!70] at (title.east) {#1};
        \path[fill = tcbcolback!80!black] 
            ([yshift = 3mm]interior.south east) -- ++(-0.4,-0.1) -- ++(0.1,-0.2);
    },
    overlay middle={%
        \path[fill = tcbcolback!80!black] 
            ([yshift = -3mm]interior.north east) -- ++(-0.4,0.1) -- ++(0.1,0.2);
        \path[fill = tcbcolback!80!black] 
            ([yshift = 3mm]interior.south east) -- ++(-0.4,-0.1) -- ++(0.1,-0.2);
    },
    overlay last={%
        \path[fill = tcbcolback!80!black] 
            ([yshift = -3mm]interior.north east) -- ++(-0.4,0.1) -- ++(0.1,0.2);
        \path[fill = tcbcolback!80!black] 
            ([yshift = 3mm]interior.south east) -- ++(-0.4,-0.1) -- ++(0.1,-0.2);
    },
    extras middle and last = { rounded corners = northeast }
}

% Ambientes de teorema, nota, etc.
\newtheorem{teorema}{Teorema}[section] % Vinculado a la sección
\newtheorem{proposicion}{Proposición}[section] % Vinculado a la sección
\newtheorem{nota}{Nota}[section] % Vinculado a la sección
\newtheorem{ejemplo}{Ejemplo}[section] % Vinculado a la sección
\newtheorem{lema}{Lema}[section] % Vinculado a la sección
\newtheorem{axioma}{Axioma}[section] % Vinculado a la sección
\newtheorem{definicion}{Definición}[section] % Vinculado a la sección




%Cuadrito de Demostración
\newcommand{\heart}{\begin{tikzpicture}[scale=0.001cm,rotate=180]
\fill[black] (0,0) 
        .. controls (0,-0.5) and (0.3,-1.8) .. (2,-2)
        .. controls (4.2,-2) and (5.5,0) .. (5.5,3)
        .. controls (5.5,5.5) and (3.5,7.5) .. (0,10)
        .. controls (-3.5,7.5) and (-5.5,5.5) .. (-5.5,3)
        .. controls (-5.5,0) and (-4.2,-2) .. (-2,-2)
        .. controls (-0.3,-1.8) and (0,-0.5) .. (0,0);
\end{tikzpicture}}

\newcommand{\demostrado}[0]{ \begin{flushright} $\heart{}$ \end{flushright}}

\usepackage{amsmath}
\usepackage{geometry}
\usepackage{tikz}
\usepackage{float}
\usepackage{graphics}
\usepackage{cancel}
\usepackage{enumerate}

\providecommand{\abs}[1]{\lvert#1\rvert}
\providecommand{\norm}[1]{\lVert#1\rVert}


\tikzset{every picture/.style={line width=0.75pt}} %set default line width to 0.75pt        

\begin{document}
\maketitle
\thispagestyle{empty}
\newpage

\begin{homeworkProblem}
Considere el siguiente problema de valor inicial:
\begin{align*}
    \begin{cases}
        \partial_{x}^{2}u-u&=0,\hspace{2cm}(x,y)\in\mathbb{R}^2\\
        u(0,y)&=f(y),\hspace{2cm}y\in\mathbb{R}\\
        \partial_{x}u(0,y)&=g(y),\hspace{2cm}y\in\mathbb{R}
    \end{cases}
\end{align*}
Donde $f,g\in C^{2}(\mathbb{R})$ y se busca una solución $u\in C^{2}(\mathbb{R}^2)$.
\begin{enumerate}[i)]
    \item Encuentre una solución del problema anterior y muestre su unicidad.\\
    \textbf{Sugerencia:} note que como la EDP no depende de y, podemos buscar soluciones como en EDO, es decir, $u(x,y)=c(y)e^{\lambda x}$ para algún $\lambda$ para la unicidad, muestre que si $w$ es una solución $C^{2}(\mathbb{R}^{2})$ del problema $(1)$ con $f=g=0$, entonces intgrando la ecuación $\partial_{x}^{2}w-w=0$ se sigue que:
    \begin{align*}
        w(x,y)=\int_{0}^{x}(x-s)w(s,y)ds
    \end{align*}
    Itere la formula anterior (reemplace el valor de $w$ nuevamente por la integral) para mostrar que $w(x,y)=0$ para cualquier punto arbitrario $(x,y)$.
    \begin{solucion}
        Suponga que $u(x,y)=c(y)e^{\lambda x}$, de aquí tenemos que:
        \begin{align*}
            \partial^{2}_{x}u-u&=c(y)\lambda^2e^{\lambda x}-c(y)e^{\lambda x} &&\text{Como no deseamos que $u$ sea la solución trivial, entonces}\\
            &=c(y)e^{\lambda x}(\lambda^2-1)\\
            &=c(y)e^{\lambda x}(\lambda-1)(\lambda + 1)\\
            &=0
        \end{align*}
        De aquí se puede deducir que $\lambda=\pm 1$, luego $u(x,y)=C_1(y)e^{x}+C_2e^{-x}$.\\
        Ahora, para cálcular $C_1$ y $C_2$ usaremos los datos iniciales, así:
        \begin{align*}
            u(0,y)&=C_1+C_2=f(y)\\
            \partial_{x}u(0,y)&=C_1-C_2=g(y)
        \end{align*}
        De aquí se puede deducir que:
        \begin{align*}
            C_1=\parauto{\frac{f(y)+g(y)}{2}}\\
            C_2=\parauto{\frac{f(y)-g(y)}{2}}
        \end{align*}
        Luego:
        \begin{align*}
            u(x,y)&=C_1(y)e^{x}+C_2e^{-x}\\
            &=\parauto{\frac{f(y)+g(y)}{2}}e^{x}+\parauto{\frac{f(y)-g(y)}{2}}e^{-x}\\
            &=f(y)\parauto{\frac{e^{x}+e^{-x}}{2}}+g(y)\parauto{\frac{e^{x}-e^{-x}}{2}}\\
            &=f(y)cosh(x)+g(y)sinh(x)
        \end{align*}
        Ahora, comprobemos la solución:
        \begin{align*}
            \partial_{x}^2u-u&=\partial_{x}^2(f(y)cosh(x)+g(y)sinh(x))-(f(y)cosh(x)+g(y)sinh(x))\\
            &=\partial_{x}(f(y)sinh(x)+g(y)cosh(x))-(f(y)cosh(x)+g(y)sinh(x))\\
            &=(f(y)cosh(x)+g(y)sinh(x))-f(y)cosh(x)+g(y)sinh(x)\\
            &=0
        \end{align*}
        Ahora, para demostrar la unicidad de la solución, vamos a suponer que existe $v\in C^{2}(\mathbb{R}^2)$ tal que $v$ es solución de $(1)$ con $f=g=0$, así tome $w=u-v$ y tome $y\in \mathbb{R}$ fijo, entonces:
        \begin{align*}
            w&=C_1e^{x}+C_2e^{-x}\\
            w(0)&=C_1+C_2=0\\
            w'(0)&=C_1-C_2=0
        \end{align*}
        De aquí podemos deducir que:
        \begin{align*}
            C_1&=-C_2\\
            C_1&=C_2
        \end{align*}
        Luego $C_1=\pm C_2$, lo que implica que $C_1=C_2=0$.\\
        Así:
        \begin{align*}
            w(x)&=C_1e^{x}+C_2e^{-x}\\
            &=(0)e^{x}+(0)e^{-x}\\
            &=0
        \end{align*}
        Luego como dado $y\in \mathbb{R}$ arbitrario $w=0$, entonces podemos concluir en que $w(x,y)=u(x,y)-v(x,y)=0$, lo que implica que $u(x,y)=v(x,y)$.\\
        Así, podemos concluir que $u$ es solución única del problema de valor inicial $(1)$ con $f=g=0$
        \demostrado
    \end{solucion}
    \item Sean $f,g,\widetilde{f},\widetilde{g}\in C^{2}(\mathbb{R})$ funciones acotadas, tales que existe $\epsilon>0$ para el cual
    \begin{align*}
        \|f-\widetilde{f}\|_{\infty}<\epsilon, \hspace{2cm} \|g-\widetilde{g}\|_{\infty}<\epsilon,
    \end{align*}
    Donde $\|h\|_{\infty}=sup_{x\in\mathbb{R}}|h(x)|$. Sean $u(x,y)$ y $\widetilde{u}(x,y)$ soluciones del problema de Cauchy (1) con datos iniciales $f,g,\widetilde{f},\widetilde{g}$, respectivamente. Muestre que:
    \begin{align*}
        |u(x,y)-\widetilde{u}(x,y)|\leq\epsilon(|sinh(x)|+|cosh(x)|),
    \end{align*}
    Para cualquier $(x,y)\in\mathbb{R}^2$. ¿Qué puede decir de la dependencia continua de los datos? Es decir, si los datos iniciales están cercanos en la distancia $\| \cdot \|_{\infty}$, ¿sucede lo mismo con las soluciones del problema de Cauchy (1)?
    \begin{solucion}
        Note que:
        \begin{align*}
            |u(x,y)-\widetilde{u}(x,y)|&\leq|f(y)cosh(x)+g(y)sinh(x)-\widetilde{f}(y)cosh(x)-\widetilde{g}(y)sinh(x)|\\
            &\leq |(f(y)-\widetilde{f}(y))cosh(x)+(g(y)-\widetilde{g}(y))sinh(x)|\\
            &\leq |(f(y)-\widetilde{f}(y))cosh(x)|+|(g(y)-\widetilde{g}(y))sinh(x)|\\
            &\leq sup_{y\in\mathbb{R}}|f(y)-\widetilde{f}(y)||cosh(x)|+sup_{y\in\mathbb{R}}|g(y)-\widetilde{g}(y)||sinh(x)|\\
            &\leq \|f-\widetilde{f}\|_{\infty}|cosh(x)|+\|g-\widetilde{g}\|_{\infty}|sinh(x)|\\
            &\leq \epsilon |cosh(x)|+ \epsilon |sinh(x)|\\
            &\leq \epsilon(|cosh(x)|+|sinh(x)|)
        \end{align*}
        \demostrado
        Sobre la dependencia continua no se puede decir mucho al respecto, pues, aunque los datos iniciales estén cercanos en la distancia $\| \cdot\|_{\infty}$ tendríamos problemas en la siguiente expresión:
        \begin{align*}
            \|u-\widetilde{u}\|_{\infty}=sup_{x\in\mathbb{R}}|u(x,y-\widetilde{u}(x,y))|&\leq \epsilon (sup_{x\in\mathbb{R}}(|cosh(x)|+|sinh(x)|))\\
            &\leq \epsilon \| |cosh|+|sinh| \|_{\infty}=\infty
        \end{align*}
        Por ende, en un principio no podemos concluir nada sobre la cercanía de las soluciones, respecto a la cercanía de los datos iniciales.
    \end{solucion}
\end{enumerate}
\end{homeworkProblem}
\newpage

\begin{homeworkProblem}
Clasifique las siguientes ecuaciones según su tipo y llévelas a su forma canónica.
\begin{enumerate}[(i)]
    \item $4u_{xx}+12u_{xy}+5u_{yy}=6u_{x}-u_{y}$ 
\begin{solucion}
    \textbf{Respuesta:} Hiperbólica $v_{\xi\eta}=\frac{4}{9}v_{\xi}+\frac{1}{9}v_{\eta}$\\
    Note que:
    \begin{itemize}
        \item $a=4$
        \item $b=6$
        \item $c=5$
        \item $g=6u_{x}-u_{y}$
    \end{itemize}
    De aquí podemos determinar $\delta$, así:
    \begin{align*}
        \delta(x,y)&=b^2-ac\\
        &=(6)^2-(4)(5)\\
        &=16\\
        &>0
    \end{align*}
    Luego podemos concluir en que la EDP es \textbf{Hiperbólica}.
    \begin{enumerate}
        \item Paso 1) Solucionar:\\
        Note que en este caso:
        \begin{align*}
            \frac{dy}{dx}&=\frac{b \pm \sqrt{b^{2}-ac}}{a}\\
            &=\frac{(6) \pm \sqrt{(6)^{2}-(4)(5)}}{(4)}\\
            &=\frac{6\pm 4}{4}\\
            &=\frac{3 \pm 2}{2}
        \end{align*}
        Luego:
        \begin{align*}
            y&=\frac{5}{2}x+C_1\\
            y&=\frac{1}{2}x+C_2\\
        \end{align*}
        \item Paso 2) Despejar las constantes:\\
        \begin{align*}
            \xi&=C_1=y-\frac{5}{2}x\\
            \eta&=C_2=y-\frac{1}{2}x
        \end{align*}
        \item Paso 3) Verificar:\\
        Suponga $u(x,y)=v(\xi,\eta)$, luego:
        \begin{itemize}
            \item \begin{align*}
                u_{x}&=v_{\xi}\xi_{x}+v_{\eta}\eta_{x}\\
                &=v_{\xi}\parauto{-\frac{5}{2}}+v_{\eta}\parauto{-\frac{1}{2}}
            \end{align*}
            \item \begin{align*}
                u_{y}&=v_{\xi}\xi_{y}+v_{\eta}\eta_{y}\\
                &=v_{\xi}+v_{\eta}
            \end{align*}
            \item \begin{align*}
                u_{xx}&=\parauto{-\frac{5}{2}}(v_{\xi\xi}\xi_{x}+v_{\xi\eta}\eta_{x})+\parauto{-\frac{1}{2}}(v_{\xi\eta}\xi_{x}+v_{\eta\eta}\eta_{x})\\
                &=\parauto{-\frac{5}{2}}\parauto{v_{\xi\xi}\parauto{-\frac{5}{2}}+v_{\xi\eta}\parauto{-\frac{1}{2}}}+\parauto{-\frac{1}{2}}\parauto{v_{\xi\eta}\parauto{-\frac{5}{2}}+v_{\eta\eta}\parauto{-\frac{1}{2}}}\\
                &=\parauto{\frac{25}{4}}v_{\xi\xi}+\parauto{\frac{10}{4}}v_{\xi\eta}+\parauto{\frac{1}{4}}v_{\eta\eta}
            \end{align*}
            \item \begin{align*}
                u_{xy}&=v_{\xi\xi}\xi_{x}+v_{\xi\eta}\eta_{x}+v_{\xi\eta}\xi_{x}+v_{\eta\eta}\eta_{x}\\
                &=v_{\xi\xi}\parauto{-\frac{5}{2}}+v_{\xi\eta}\parauto{-\frac{1}{2}}+v_{\xi\eta}\parauto{-\frac{5}{2}}+v_{\eta\eta}\parauto{-\frac{1}{2}}\\
                &=\parauto{-\frac{5}{2}}v_{\xi\xi}+(-3)v_{\xi\eta}+\parauto{-\frac{1}{2}}v_{\eta\eta}
            \end{align*}
            \item \begin{align*}
                u_{yy}&=v_{\xi\xi}\xi_{y}+v_{\xi\eta}\eta_{y}+v_{\xi\eta}\xi_{y}+v_{\eta\eta}\eta_{y}\\
                &=v_{\xi\xi}+v_{\xi\eta}+v_{\xi\eta}+v_{\eta\eta}\\
                &=v_{\xi\xi}+2v_{\xi\eta}+v_{\eta\eta}\\
            \end{align*}
        \end{itemize}
        Luego, reemplazando en $4u_{xx}+12u_{xy}+5u_{yy}=6u_{x}-u_{y}$ tenemos:
        \begin{align*}
            4u_{xx}+12u_{xy}+5u_{yy}&=6u_{x}-u_{y}\\
            v_{\xi\eta}&=\frac{4}{9}v_{\xi}+\frac{1}{9}v_{\eta}            
        \end{align*}
        
    \end{enumerate}
\end{solucion}
    \item $u_{xx}-4u_{xy}+4u_{yy}=4+2u_{x}$ 
\begin{solucion}
    \textbf{Respuesta:} Parabólica $v_{\xi\xi}=2v_{\xi}+4v_{\eta}+4$\\
    Note que:
    \begin{itemize}
        \item $a=1$
        \item $b=-2$
        \item $c=4$
        \item $g=4-2u_{x}$
    \end{itemize}
    De aquí podemos determinar $\delta$, así:
    \begin{align*}
        \delta(x,y)&=b^2-ac\\
        &=(-2)^2-(1)(4)\\
        &=0
    \end{align*}
    Luego podemos concluir en que la EDP es \textbf{Parabólica}.
    \begin{enumerate}
        \item Paso 1) Solucionar:\\
        Note que en este caso:
        \begin{align*}
            \frac{dy}{dx}&=\frac{b}{a}\\
            &=-\frac{2}{1}\\
            &=-2
        \end{align*}
        Luego:
        \begin{align*}
            y&=-2x+C_1\\
        \end{align*}
        \item Paso 2) Despejar las constantes:\\
        \begin{align*}
            \eta&=C_1=y+2x
        \end{align*}
        Ahora, para calcular $\xi$ usaremos el determinante de la matrix Jacobiana asegurandonos que este mismo sea distinto de $0$, así podrá notar que $\xi=y+x$ satisface:
        \begin{align*}
            \eta_{x}-2\eta_{y}&=1-2\\
            &=-1\\
            &\neq 0
        \end{align*}
        \item Paso 3) Verificar:\\
        Suponga $u(x,y)=v(\xi,\eta)$, luego:
        \begin{itemize}
            \item \begin{align*}
                u_{x}&=v_{\xi}\xi_{x}+v_{\eta}\eta_{x}\\
                &=v_{\xi}+2v_{\eta}
            \end{align*}
            \item \begin{align*}
                u_{y}&=v_{\xi}\xi_{y}+v_{\eta}\eta_{y}\\
                &=v_{\xi}+v_{\eta}
            \end{align*}
            \item \begin{align*}
                u_{xx}&=(v_{\xi\xi}\xi_{x}+v_{\xi\eta}\eta_{x})+2(v_{\xi\eta}\xi_{x}+v_{\eta\eta}\eta_{x})\\
                &=v_{\xi\xi}+4v_{\xi\eta}+4v_{\eta\eta}
            \end{align*}
            \item \begin{align*}
                u_{xy}&=v_{\xi\xi}\xi_{x}+v_{\xi\eta}\eta_{x}+v_{\xi\eta}\xi_{x}+v_{\eta\eta}\eta_{x}\\
                &=v_{\xi\xi}+3v_{\xi\eta}+2v_{\eta\eta}
            \end{align*}
            \item \begin{align*}
                u_{yy}&=v_{\xi\xi}\xi_{y}+v_{\xi\eta}\eta_{y}+v_{\xi\eta}\xi_{y}+v_{\eta\eta}\eta_{y}\\
                &=v_{\xi\xi}+v_{\xi\eta}+v_{\xi\eta}+v_{\eta\eta}\\
                &=v_{\xi\xi}+2v_{\xi\eta}+v_{\eta\eta}\\
            \end{align*}
        \end{itemize}
        Luego, reemplazando en $u_{xx}-4u_{xy}+4u_{yy}=4+2u_{x}$ tenemos:
        \begin{align*}
            u_{xx}-4u_{xy}+4u_{yy}&=4+2u_{x}\\
            v_{\xi\xi}&=2v_{\xi}+4v_{\eta}+4            
        \end{align*}
        
    \end{enumerate}
\end{solucion}
    \item $(1+x^{2})^{2}u_{xx}-(1+y^2)^2u_{yy}=0$ 
\begin{solucion}
    \textbf{Respuesta:} Hiperbólica\\
    $$v_{\xi\eta}=\parauto{\frac{tan\parauto{\frac{\eta-\xi}{2}}}{2}+\frac{tan\parauto{\frac{\xi+\eta}{2}}}{2}}v_{\xi}+\parauto{\frac{tan\parauto{\frac{\xi+\eta}{2}}}{2}-\frac{tan\parauto{\frac{\eta-\xi}{2}}}{2}}v_{\eta}$$
    Note que:
    \begin{itemize}
        \item $a=(1+x^2)^2$
        \item $b=0$
        \item $c=-(1+y^2)^2$
        \item $g=0$
    \end{itemize}
    De aquí podemos determinar $\delta$, así:
    \begin{align*}
        \delta(x,y)&=b^2-ac\\
        &=(0)^2-(1+x^2)^2(-(1+y^2)^2)\\
        &=(1+x^2)^2(1+y^2)^2\\
        &>0
    \end{align*}
    Luego podemos concluir en que la EDP es \textbf{Hiperbólica}.
    \begin{enumerate}
        \item Paso 1) Solucionar:\\
        Note que en este caso:
        \begin{align*}
            \frac{dy}{dx}&=\frac{b \pm \sqrt{b^{2}-ac}}{a}\\
            &=\pm\frac{ (1+x^2)(1+y^2)}{(1+x^2)^2}\\
            &=\pm\frac{1+y^2}{1+x^2}
        \end{align*}
        Luego:
        \begin{align*}
            arctan(y)&=arctan(x)+C_1\\
            arctan(y)&=-arctan(x)+C_2
        \end{align*}
        \item Paso 2) Despejar las constantes:\\
        \begin{align*}
            \xi&=C_1=arctan(y)-arctan(x)\\
            \eta&=C_2=arctan(y)+arctan(x)
        \end{align*}
        \item Paso 3) Verificar:\\
        Suponga $u(x,y)=v(\xi,\eta)$, luego:
        \begin{itemize}
            \item \begin{align*}
                u_{x}&=v_{\xi}\xi_{x}+v_{\eta}\eta_{x}\\
                &=\parauto{\frac{1}{1+x^2}}(-v_{\xi}+v_{\eta})
            \end{align*}
            \item \begin{align*}
                u_{y}&=v_{\xi}\xi_{y}+v_{\eta}\eta_{y}\\
                &=\parauto{\frac{1}{1+y^2}}(v_{\xi}+v_{\eta})            \end{align*}
            \item \begin{align*}
                u_{xx}&=\parauto{\frac{1}{1+x^2}}(-v_{\xi\xi}\xi_{x}-v_{\xi\eta}\eta_{x}+v_{\xi\eta}\xi_{x}+v_{\eta\eta}\eta_{x})-\parauto{\frac{2x}{(1+x^2)^2}}(-v_{\xi}+v_{\eta})\\
                &=\parauto{\frac{1}{(1+x^2)^2}}(v_{\xi\xi}-2v_{\xi\eta}+v_{\eta\eta}\eta_{x})-\parauto{\frac{2x}{(1+x^2)^2}}(-v_{\xi}+v_{\eta})
            \end{align*}
            \item \begin{align*}
                u_{yy}&=\parauto{\frac{1}{1+y^2}}(v_{\xi\xi}\xi_{x}+v_{\xi\eta}\eta_{x}+v_{\xi\eta}\xi_{x}+v_{\eta\eta}\eta_{x})-\parauto{\frac{2y}{(1+y^2)^2}}(v_{\xi}+v_{\eta})\\
                &=\parauto{\frac{1}{(1+y^2)^2}}(v_{\xi\xi}+2v_{\xi\eta}+v_{\eta\eta}\eta_{x})-\parauto{\frac{2y}{(1+y^2)^2}}(v_{\xi}+v_{\eta})
            \end{align*}
            \item \begin{align*}
                arctan(y)&=\frac{\xi+\eta}{2}\\
                y&=tan\parauto{\frac{\xi+\eta}{2}}\\
                arctan(x)&=\frac{\eta-\xi}{2}\\
                x&=tan\parauto{\frac{\eta-\xi}{2}}
                \end{align*}
        \end{itemize}
        Luego, reemplazando en $(1+x^{2})^{2}u_{xx}-(1+y^2)^2u_{yy}=0$ tenemos:
        \begin{align*}
            (1+x^{2})^{2}u_{xx}-(1+y^2)^2u_{yy}&=0\\
            -4v_{\xi\eta}&=-2xv_{\xi}+2xv_{\eta}-2yv_{\xi}-2yv_{\eta}\\
            v_{\xi\eta}&=\parauto{\frac{x}{2}+\frac{y}{2}}v_{\xi}+\parauto{\frac{y}{2}-\frac{x}{2}}v_{\eta}\\
            v_{\xi\eta}&=\parauto{\frac{tan\parauto{\frac{\eta-\xi}{2}}}{2}+\frac{tan\parauto{\frac{\xi+\eta}{2}}}{2}}v_{\xi}+\parauto{\frac{tan\parauto{\frac{\xi+\eta}{2}}}{2}-\frac{tan\parauto{\frac{\eta-\xi}{2}}}{2}}v_{\eta}
        \end{align*}
        
    \end{enumerate}
\end{solucion}\end{enumerate}
\end{homeworkProblem}
\newpage

\begin{homeworkProblem}
Considere la ecuación 
\begin{equation}
    \label{eq: original p4}
    u_{xx}+4u_{xy}+u_{x} = 0.
\end{equation}
\begin{enumerate}[(i)]
    \item Lleve la ecuación a su forma canónica.
    \begin{solucion}
        Primero calculemos su determinante para identificar si es de tipo hiperbólico, elíptico o parabólico. Tenemos $a=1, b=2, c=0$. Luego $\Delta = b^{2}-ac = 2^{2} -(1)(0) = 4>0$, por lo tanto la EDP es de tipo hiperbólico en el plano $xy$.\\
        Tenemos entonces que el cambio de variables está dado por las ecuaciones características.
        \begin{align}
            \dfrac{dy}{dx} &= \dfrac{b \pm \sqrt{b^2-ac}}{a} = \dfrac{2 \pm \sqrt{2^2-(1)(0)}}{1} =2 \pm 2,\\
            \dfrac{dy}{dx} &=(2 \pm 2), \hspace{4cm}\text{solucionando la EDO tenemos,}\\
            y &= (2 \pm 2)x + c,
        \end{align}
        al igualar a la constante obtenemos los cambios de variables $\xi \text{ y } \eta$. De este modo, nos queda $\xi = y - 4x$, $\eta = y$. Ahora veamos las derivadas de $u$ en términos de $\xi$ y $\eta$.
        \begin{align*}
            u_{x} &= u_{\xi}\xi_{x} + u_{\eta}\eta_{x} = u_{\xi}(-4) + u_{\eta}(0)\\ 
            &= -4u_{\xi},\\
            u_{xx} &= -4(u_{\xi})_{x} = -4(u_{\xi \xi} \xi_{x} + u_{\xi \eta} \eta_{x}) = -4(u_{\xi \xi}(-4) + u_{\eta \xi}(0))\\
            &= 16u{_\xi \xi}. \hspace{5cm} \text{y}\\
            u_{xy} &= -4(u_{\xi})_{y} = -4((u_{\xi \xi} \xi_{y} + u_{\xi \eta} \eta_{y}) = -4 (u_{\xi\xi}(1) + u_{\xi \eta}(1))\\ 
            &= -4 (u_{\xi\xi} + u_{\xi \eta}).
        \end{align*}
        Por lo que al reemplazar, ~\ref{eq: original p4} nos queda:
        \begin{align*}
            16u_{\xi \xi} + 4(-4)(u_{\xi \eta}) -4u{\xi}&= 0\\
            16u_{\xi \xi} - 16u_{\xi \xi} -16u_{\xi \eta} - 4u_{\xi} &=0\\
            16u_{\xi \eta} + 4u_{\xi} = 0
        \end{align*}
            Al dividir la igualdad por $4$ tenemos que la forma canónica de ~\ref{eq: original p4} es:
            \begin{equation}
                \label{eq: canonica p4}
                4u_{\xi \eta} + u_{\xi} = 0.
            \end{equation}   
        \demostrado
\end{solucion}
\newpage
    \item Encuentre la solución específica $u(x,8x) = 0, u_{x}(x,8x)=4e^{-2x}$.
\end{enumerate}
\begin{solucion}
    Si tomamos $w(\eta) = u_{\xi}$ en la ecuación ~\ref{eq: canonica p4} nos queda una EDO de orden 1 que podemos solucionar por variables separables como sigue:
    \begin{align*}
        4w' &= -w,\\
        \frac{4dw}{w} &= -d\eta,\\
        4log(w) &= -\eta + h(\xi),\\
        w^{4} &= h(\xi)e^{-\eta},
        \end{align*}
        luego,
        \begin{equation}
            \label{eq: kja}
            w = u_{\xi} = h(\xi)e^{-\frac{\eta}{4}}.
        \end{equation}
    Integrando respecto a la variable $\xi$ nos queda
    \begin{align*}
        u(\xi, \eta)  =e^{\frac{-\eta}{4}}(H(\xi) + C(\eta)),
    \end{align*}
    con $H$ siendo una antiderivada de $h$. Luego, al regresar a $x$ e $y$ obtenemos
    \begin{align*}
        u(x,y) &= e^{-\frac{-y}{4}}(H(y-4x) + C(y)), \text{  al derivar respecto a } x \text{ y usar las condiciones iniciales para $u_{x} $,}\\
        u_{x}(x,8x) &= -4e^{-2x}H'(4x) = 4e^{-2x}H'(4x), \text{ por lo que }\\
        H'(4x) &= -1, \text{ y al integrar respecto a x llegamos a }\\
        H(4x) &= -4x, \text{ de modo que}\\
        H\left(4 (\frac{y}{4} - x)\right) &= 4x - y = H(y-4x).
    \end{align*}
    Y al usar a condición inicial sobre $u$ tenemos
    \begin{align*}
        u(x,8x) &= e^{-2x}(H(4x) + C(8x)) = 0
    \end{align*}
    Como el factor exponencial nunca es cero, entonces $(H(4x) + C(8x)) = 0$, luego $4x = C(8x)$, por lo que $C\left(8(\dfrac{x}{8})\right) = 4\left(\dfrac{x}{8}\right) = \dfrac{x}{2} = C(x)$.\\

    Por lo tanto la solución específica $u(x,y) = e^{-\frac{y}{4}}(4x -  \frac{y}{2})$
    \demostrado
\end{solucion}
\end{homeworkProblem}
\newpage

\begin{homeworkProblem}
Sea $U\subset \mathbb{R}^2$ un abierto. Considere la ecuación de segundo orden:
\begin{align*}
    a(x,y)u_{xx}+2bu_{xy}+cu_{yy}=0
\end{align*}
    Donde $u:U\rightarrow \mathbb{R}$ pertenece a $C^{2}(U)$ y los términos $a,b,c$ son funciones $C^{1}(U)$.\\
    Definimos el determinante de la ecuación por:
    \begin{align*}
        \delta(x,y)=b^2(x,y)-a(x,y)c(x,y)
    \end{align*}
    Considere el cambio de variable:
    \begin{align*}
        \xi&=\xi(x,y)\\
        \eta&=\eta(x,y)
    \end{align*}
    Donde las funciones son de clase $C^{2}(U)$ y el Jacobiano $det\parauto{(\xi\eta)_{(x,y)}}\neq 0$ en todo punto $(x,y)\in U$.\\
    Muestre que el signo del determinante $\delta(x,y)$ es invariante por el cambio de variables $(\xi,\eta)$. Es decir, considere $u(x,y)=v(\xi,\eta)$, usando que $u$ satisface la ecuación  muestre que $v$ satisface la siguiente ecuación:
    \begin{align*}
        A(\xi,\eta)v_{\xi\xi}+2B(\xi,\eta)v_{\xi\eta}+C(\xi,\eta)=G(\xi,\eta,v,v_{\xi},v_{\eta})
    \end{align*}
    Para agunas funciones $A,B,C,G$. Luego muestre que el signo de $\delta (\xi,\eta)=B^2(\xi,\eta)-A(\xi,\eta)C(\xi,\eta)$ es igual al signo de $\delta (x,y)=b^2(x,y)-a(x,y)c(x,y)$. 
\begin{solucion}
    Suponiendo que $u(x,y)=v(\xi,\eta)$ y que $v(\xi,\eta)$ es $C^{2}(U)$ vamos a calcular sus derivadas de primer y segundo orden, por ende:
    \begin{itemize}
        \item $u_{x}=v_{\xi}\xi_{x}+v_{\eta}\eta_{x}$
        \item $u_{y}=v_{\xi}\xi_{y}+v_{\eta}\eta_{y}$
        \item $u_{xx}=\xi_{x}^2v_{\xi\xi}+2\xi_{x}\eta_{x}v_{\xi\eta}+\eta_{x}^2v_{\eta\eta}$
        \item $u_{yy}=\xi_{y}^2v_{\xi\xi}+2\xi_{y}\eta_{y}v_{\xi\eta}+\eta_{y}^2v_{\eta\eta}$
        \item $u_{xy}=\xi_{x}\xi_{y}v_{\xi\xi}+(\xi_{x}\eta_{y}+\xi_{y}\eta_{x})v_{\xi\eta}+\eta_{x}\eta_{y}v_{\eta\eta}$
        \end{itemize}
        Ahora, reemplazando:
        \begin{align*}
            au_{xx}+2bu_{xy}+cu_{yy}&=0\\
            &=a(\xi_{x}^2v_{\xi\xi}+2\xi_{x}\eta_{x}v_{\xi\eta}+\eta_{x}^2v_{\eta\eta})+2b(\xi_{x}\xi_{y}v_{\xi\xi}+(\xi_{x}\eta_{y}+\xi_{y}\eta_{x})v_{\xi\eta}+\eta_{x}\eta_{y}v_{\eta\eta})\\
            &\hspace{0.3cm}+c(\xi_{y}^2v_{\xi\xi}+2\xi_{y}\eta_{y}v_{\xi\eta}+\eta_{y}^2v_{\eta\eta})\\
            &=v_{\xi\xi}(a\xi_{x}^{2}+2b\xi_{x}\xi_{y}+c\xi_{y}^{2})+2v_{\xi\eta}(a\xi_{x}\eta_{x}+b(\xi_{x}\eta_{y}+\xi_{y}\eta_{x})+c\xi_{y}\eta_{y})\\
            &\hspace{0.3cm}+v_{\eta\eta}(a\eta_{x}^{2}+2b\eta_{x}\eta_{y}+c\eta_{y}^{2})
        \end{align*}
        De aquí podemos concluir en que:
        \begin{itemize}
            \item $A=a\xi_{x}^{2}+2b\xi_{x}\xi_{y}+c\xi_{y}^{2}$
            \item $B=a\xi_{x}\eta_{x}+b(\xi_{x}\eta_{y}+\xi_{y}\eta_{x})+c\xi_{y}\eta_{y}$
            \item $C=a\eta_{x}^{2}+2b\eta_{x}\eta_{y}+c\eta_{y}^{2}$
        \end{itemize}
        Ahora para calcular el determinante calcularemos las expresiones $B^2$ y $-AC$ (con mucha paciencia en el trabajo borrador), así:
        \begin{align*}
            B^2=&(a\xi_{x}\eta_{x}+b(\xi_{x}\eta_{y}+\xi_{y}\eta_{x})+c\xi_{y}\eta_{y})\\
            =&a^2\xi_{x}^2\eta_{x}^2+2ab\xi_{x}^2\eta_{x}\eta_{y}+2ab\xi_{x}\xi_{y}\eta_{x}^2+2av\xi_{x}\xi_{y}\eta_{x}\eta_{y}+b^2\xi_{x}^2\eta_{y}^2\\
            &+2b\xi_{x}\xi_{y}\eta_{x}\eta_{y}+b^2\xi_{y}^2\eta_{x}^2+2bc\xi_{x}\xi_{y}\eta_{y}^2+2bc\xi_{y}^2\eta_{x}\eta_{y}+c^2\xi_{y}^2\eta_{y}^2
        \end{align*}
        Luego:
        \begin{align*}
            -AC=&-(a\xi_{x}^{2}+2b\xi_{x}\xi_{y}+c\xi_{y}^{2})(a\eta_{x}^{2}+2b\eta_{x}\eta_{y}+c\eta_{y}^{2})\\
            &-a^2\xi_{x}^2\eta_{x}^2-2ab\xi_{x}^2\eta_{x}\eta_{y}-ac\xi_{x}^2\eta_{y}^2-2ab\xi_{x}\xi_{y}\eta_{x}^2\\
            &-4b^2\xi_{x}\xi_{y}\eta_{x}\eta_{y}-2bc\xi_{x}\xi_{y}\eta_{y}^2-ac\xi_{y}^2\eta_{x}^2-2bc\xi_{y}^2\eta_{x}\eta_{y}-c^2\xi_{y}^2\eta_{y}^2
        \end{align*}
        Luego de esto podremos calcular $\delta (\xi,\eta)=B^2-AC$, así:
        \begin{align*}
            B^2-AC&=2(ac-b^2)\xi_{x}\xi_{y}\eta_{x}\eta_{y}+(b^2-ac)\xi_{y}^2\eta_{x}^2+(b^2-ac)\xi_{x}^2\eta_{y}^2\\
            &=(b^2-ac)\xi_{x}^2\eta_{y}^2-2(b^2-ac)\xi_{x}\xi_{y}\eta_{x}\eta_{y}+(b^2-ac)\xi_{y}^2\eta_{x}^2\\
            &=(b^2-ac)(\xi_{x}^2\eta_{y}^2-2\xi_{x}\xi_{y}\eta_{x}\eta_{y}+\xi_{y}^2\eta_{x}^2)\\
            &=(b^2-ac)(\xi_{x}\eta_{y}-\xi_{y}\eta_{x})^2
        \end{align*}
        Luego como $(\xi_{x}\eta_{y}-\xi_{y}\eta_{x})^2$ es una función siempre positiva (distinta de cero ya que el determinante del Jacobiano es distinto de $0$ en todo punto $(x,y)\in U$), entonces el signo de $\delta (\xi,\eta)$ está totalmente determinado y es igual al signo de $\delta (x,y)=b^2-ac$.
        \demostrado
\end{solucion}
\end{homeworkProblem}
\newpage

%%%%%%%%%%%%%%%%%%%%%%%%%%%%%%%%%%%%%%%%%%%%%%%%%%%%%%%
\end{document}