\begin{homeworkProblem}
  Una onda esférica es una solución $u$ de la ecuación de onda en tres dimensiones que es radial en espacio, es decir, $u(x,t) = u(r,t), r=|x|$.
  \begin{enumerate}[i)]
    \item Encuentre la ecuacion que satisfacen las ondas esféricas.
      \begin{solucion}
        Note que si $u(x,t)=u(|x|,t)=u(r,t)$, se cumple que:
        \begin{align*}
          \Delta u&=\frac{2}{r} \frac{du}{dr} + \frac{d^2 u}{dr^2},
        \end{align*}
        por lo que sabemos que $u(r,t)$ debe de solucionar:
        \begin{align*}
          u_{tt}(r,t)-\frac{2}{r} u_r(r,t) - u_{rr}(r,t)=0
        \end{align*}
        \demostrado
      \end{solucion}
    \item Para la ecuación encontrada en $I)$, realice el cambio de variables $v = ru$  para determinar y solucionar la ecuación que satisface $v$. Con esto encuentre una solución de la ecuación de onda radial con datos iniciales $u(r,0) = g(r)$, $u_t(r,0) = h(r)$, donde $g$ y $h$ son funciones pares de $r$.
      \begin{solucion}
        Suponga $v=ru$, es decir:
        \begin{align*}
          u(x,t)&=\frac{v(x,t)}{r}
        \end{align*}
        Note que:
        \begin{align*}
          u_r(r,t) &= \frac{1}{r} v_r(r,t) - \frac{1}{r^2}v(r,t)\\
          u_{rr}(r,t) &= \frac{1}{r} v_{rr}(r,t) - \frac{2}{r^2} v_r(r,t) + \frac{2}{r^3}v(r,t)\\
          u_{tt}(r,t)&=\frac{v_{tt}(r,t)}{r}
        \end{align*}
        sustituyendo en la ecuación del punto $I)$, nos queda:
        \begin{align*}
          \frac{1}{r}v_{tt}(x,t)-\frac{2}{r}\left( \frac{1}{r} v_r(r,t) - \frac{1}{r^2}v(r,t) \right)-\left( \frac{1}{r} v_{rr}(r,t) - \frac{2}{r^2} v_r(r,t) + \frac{2}{r^3}v(r,t) \right)&=0\\
        \end{align*}
        realizando los cálculos:
        \begin{align*}
          v_{tt}-v_{rr}=0
        \end{align*}
        luego, si proponemos el siguiente problema de Cauchy:
        \begin{align*}
          \begin{cases}
            v_{tt}-v_{rr}=0, &\text{ en } \mathbb{R}\times (0,\infty) \text{,} \\
            v(r,0)=rg(r), &\text{ en }\mathbb{R} \text{,}\\
            v_{t}(r,0)=rh(r), &\text{ en }\mathbb{R} \text{,}
          \end{cases}
        \end{align*}
        por lo realizado en el punto problema $1$, sabemos que:
        \begin{align*}
          v(r,t)&=\frac{1}{2}((r+t)g(r+t)+(r-t)g(r-t))+\frac{1}{2}\int_{r-t}^{r+t}yh(y)dy
        \end{align*}
        luego como $u=v/r$, entonces:
        \begin{align*}
          u(x,t)&=\frac{1}{2r}((r+t)g(r+t)+(r-t)g(r-t))+\frac{1}{2r}\int_{r-t}^{r+t}yh(y)dy
        \end{align*}
        es solución del problema de Cauchy:
        \begin{align*}
          \begin{cases}
            u_{tt}-\frac{2}{r} u_r(r,t) - u_{rr}(r,t)=0, &\text{ en } \mathbb{R}\times (0,\infty) \text{,} \\
            u(r,0)=g(r), &\text{ en }\mathbb{R} \text{,}\\
            u_{t}(r,0)=h(r), &\text{ en }\mathbb{R} \text{,}
          \end{cases}
        \end{align*}
        \demostrado
      \end{solucion}
  \end{enumerate}
\end{homeworkProblem}
