\begin{homeworkProblem}
  (\textbf{Buena colocación de la ecuación de onda en una dimensión}). Considere el problema de Cauchy:
  \begin{align*}
    (1)\hspace{3cm}\begin{cases}
      u_{tt}-u_{xx}=0, &\text{ en } \mathbb{R}\times (0,\infty) \text{,} \\
      u(x,0)=g(x), &\text{ en }\mathbb{R} \text{,}\\
      u_{t}(x,0)=h(x), &\text{ en }\mathbb{R} \text{,}
    \end{cases}
  \end{align*}
  donde suponemos que $g\in C^2(\mathbb{R})$ y $h\in C^{1}(\mathbb{R})$.
  \begin{enumerate}[i]
    \item Sea $u$ solución de $(1)$ de clase $C^2$. Muestre que si $g(x)$ y $h(x)$ son nulas para $|x|>R$, entonces $u(x,t)=0$ para todo $|x|>R+t$.\\
    \textbf{Sugerencia.} Utilice el principio de propagación finita.
      \begin{solucion}
        Note que como $u$ es solución de la ecuación de onda, entonces se satisface que para cada $x_0\in\mathbb{R}$ y $t_0>0$ podemos definir el siguiente cono de onda:
        $$K(x_0,t_0):=\{(x,t):0\leq t\leq t_0, |x-x_0|\leq t_0-t\}$$
        Ahora, por hipótesis tenemos que $Supp\phantom{x}g, Supp\phantom{x}h \subseteq [-R,R]$, por lo que usando el principio el principio de propagación finita tenemos que si suponemos $x_0\notin [-R,R]$ y $t_0>0$ tal que $(x_0-t_0,x_0+t_0)\cap(-R,R)=\emptyset$, entonces $u\equiv u_t\equiv 0$ en $(x_0-t_0,x_0+t_0)\times (t=0)$, y por ende $u\equiv 0$ en el cono $K(x_0,t_0)$.\\
        Luego, si tomamos $(x,t)$ tal que $|x|>R+t$, entonces $x\notin [-R-t,R+t]$, luego como $[-R,R]\subset[-R-t,R+t]$, entonces $x\notin [-R,R]$ y por ende se satisface que $u(x,t)=0$.
        \demostrado
      \end{solucion}
    \item \textbf{(Existencia).} Muestre que existe una solución de clase $C^2$ del problema $(1)$.\\
    \textbf{Sugerencia.} Verifique que la fórmula de d'Alemberd es en efecto una solución de $(1)$.
      \begin{solucion}
        Suponga:
        \begin{align*}
          u(x,t)=\frac{1}{2}\left( g(x+t)+g(x-t) \right)+\frac{1}{2}\int_{x-t}^{x+t}h(y)dy.
        \end{align*}
        Note que:
        \begin{align*}
          \lim_{t \rightarrow 0}u(x,t)&=\lim_{t \rightarrow 0}\frac{1}{2}\left( g(x+t)+g(x-t) \right)+\frac{1}{2}\int_{x-t}^{x+t}h(y)dy,\\
          &=\frac{1}{2}\left( g(x)+g(x) \right)+\frac{1}{2}\int_{x}^{x}h(y)dy,\\
          &=g(x),
        \end{align*}
        además:
        \begin{align*}
          \lim_{t \rightarrow 0}u_t(x,0)&=\lim_{t \rightarrow 0}\frac{1}{2}\left( g'(x+t)-g'(x-t) \right)+\frac{1}{2}\int_{-1}^{1}h'(ut+x)ut+h(ut+x)du,\\
          &=\frac{1}{2}\int_{-1}^{1}h(x)du,\\
          &=h(x).
        \end{align*}
        Ahora veamos que $u$ es solución de la ecuación de onda.\\
        Calculando:
        \begin{align*}
          u_x(x,t)&=\frac{1}{2}\left( g'(x+t)+g'(x-t) \right)+\frac{1}{2}\left( h(x+t)+h(x-t) \right).\\
          u_t(x,t)&=\frac{1}{2}\left( g'(x+t)-g'(x-t) \right)+\frac{1}{2}\left( h(x+t)-h(x-t) \right),
        \end{align*}
        luego:
        \begin{align*}
          u_{xx}(x,t)&=\frac{1}{2}\left( g''(x+t)+g''(x-t) \right)+\frac{1}{2}\left( h'(x+t)+h'(x-t) \right).\\
          u_{tt}(x,t)&=\frac{1}{2}\left( g''(x+t)+g''(x-t) \right)+\frac{1}{2}\left( h'(x+t)+h'(x-t) \right),\\
          &=u_{xx}(x,t).
        \end{align*}
        De lo que se concluye que $u_{tt}-u_{xx}=0$.\\
        Por lo que se puede asegurar que el problema $(1)$ tiene solución.
        \demostrado
      \end{solucion}
      \newpage
    \item \textbf{(Unicidad).} Muestre que existe una única solución del problema de Cauchy $(1)$ en la clase $C^2(\mathbb{R}\times [0,\infty))$.\\
    \textbf{Sugerencia.} Es suficiente con mostrar que el problema:
    \begin{align*}
      \begin{cases}
        w_{tt}-w_{xx}=0, &\text{ en } \mathbb{R}\times (0,\infty) \text{,} \\
        w(x,0)=0, &\text{ en }\mathbb{R} \text{,}\\
        w_{t}(x,0)=0, &\text{ en }\mathbb{R} \text{,}
      \end{cases}
    \end{align*}
    tiene como única solución $w=0$. Para esto, defina la energía:
    \begin{align*}
      E(t)=\int_{-\infty}^{\infty}(w_{t}^2(x,t)+w_{x}^2(x,t))dx.
    \end{align*}
    Utilice $I)$ para justificar que la energía anterior está bien definida (como integral en todo $\mathbb{R}$). Luego muestre que $\frac{d}{dt}E(t)=0$.
      \begin{solucion}
        Suponga que existen $u_1(x,t),u_2(x,t)$ soluciones del problema $(1)$, entonces sabemos que $w(x,t)=u_1(x,t)-u_2(x,t)$ es solución del problema de Cauchy:
            \begin{align*}
              \begin{cases}
                w_{tt}-w_{xx}=0, &\text{ en } \mathbb{R}\times (0,\infty) \text{,} \\
                w(x,0)=0, &\text{ en }\mathbb{R} \text{,}\\
                w_{t}(x,0)=0, &\text{ en }\mathbb{R} \text{,}
              \end{cases}
            \end{align*}
            Luego, note que como $w(x,0)=w_t(x,0)=0$, entonces por $I)$ se satisface que $w(x,t)=0$ si $|x|>t$, por lo que podemos asegurar que dado $t\in[0,\infty)$ si definimos $g_t(x)=w(x,t)$, entonces $g_t$ es de soporte compacto y por ende sus derivadas también. Luego, como $w(x,t)=0$ si $|x|>t$, entonces bajo las mismas condiciones sabemos que $w_t(x,t)=0$, luego podemos asegurar que $E(t)$ se encuentra bien definida y que además el conjunto de integración se puede reducir a un conjunto compacto.\\
            \newpage
            Ahora veamos que $\frac{d}{dt}E(t)=0$, para esto usaremos la derivación bajo el signo de la integral aprovechándonos de que el dominio de integración es un conjunto compacto.
            \begin{align*}
              \frac{d}{dt}E(t)&=\frac{d}{dt}\int_{-\infty}^{\infty}(w_t^2(x,t)+w_x^2(x,t))dx\\
              &=\int_{-\infty}^{\infty}\frac{d}{dt}(w_t^2(x,t)+w_x^2(x,t))dx\\
              &=\int_{-\infty}^{\infty}(2w_t(x,t)w_{tt}(x,t)+2w_x(x,t)w_{tx}(x,t))dx\\
              &=2\int_{-\infty}^{\infty}w_t(x,t)w_{tt}(x,t)dx + 2\int_{-\infty}^{\infty}w_x(x,t)w_{tx}(x,t)dx\\
              &=2\int_{-\infty}^{\infty}w_t(x,t)w_{tt}(x,t)dx + 2\left[ w_x(x,t)w_{t}(x,t)\Big|_{-\infty}^{\infty}-\int_{-\infty}^{\infty}w_t(x,t)w_{xx}(x,t)dx\right]\\
              &=2\int_{-\infty}^{\infty}w_t(x,t)(w_{tt}(x,t)-w_{xx}(x,t))dx\\
              &=0
            \end{align*}
            Por lo que podemos concluir que $E(t)=c$ para alguna constante $c$, ahora calculemos quien es $E(0)$:
            \begin{align*}
              E(0)&=\int_{-\infty}^{\infty}(w_t^2(x,0)+w_x^2(x,0))dx\\
              &=\int_{-\infty}^{\infty}(0^2+0^2)dx\\
              &=0
            \end{align*}
            Por lo que $E(0)=0=E(t)$ para todo $t>0$, por lo que podemos asegurar que $w_x^2(x,t)=w_t^2(x,t)=0$ y por ende $w_t(x,t)=w_x(x,t)=0$, luego $w(x,t)=c$ y como $w_(x,0)=0$, se sigue por continuidad que $w(x,t)=0$, por lo que podemos asegurar que $u_1(x,t)-u_2(x,t)=0$ y por ende $u_1(x,t)=u_2(x,t)$.
            \demostrado
      \end{solucion}
      \newpage
    \item \textbf{(Tipo de dependencia continua).} Sea $u_j$ solución de $(1)$ con datos iniciales $g_j\in C^2(\mathbb{R})\cap L^{\infty}(\mathbb{R})$ y $h_j\in C^{1}(\mathbb{R})\cap L^{\infty}(\mathbb{R})$, para $j=1,2$. Dado $T>0$ fijo, muestre que:
    \begin{align*}
      \sup_{(x,t)\in \mathbb{R}\times [0,T]}|u_1(x,t)-u_2(x,t)|\leq \|g_1-g_2\|_{L^{\infty}}+T\|h_1-h_2\|_{L^{\infty}}.
    \end{align*}
    En particular, concluya que datos iniciales próximos en la norma $L^{\infty}(\mathbb{R})$ general soluciones de $(1)$ próximas en la norma $L^{\infty}(\mathbb{R}\times [0,T])$.
      \begin{solucion}
        Note que:
        \begin{align*}
          |u_1(x,t)-u_2(x,t)|&\leq \frac{1}{2}\left|g_1(x+t)+g_1(x-t)+\int_{x-t}^{x+t}h_1(y)dy-g_2(x+t)-g_2(x-t)-\int_{x-t}^{x+t}h_2(y)dy\right|,\\
          &\leq \frac{1}{2}\left|g_1(x+t)+g_1(x-t)-(g_2(x+t)+g_2(x-t))\right|+\frac{1}{2}\left| \int_{x-t}^{x+t}h_1(y)-h_2(y)dy \right|,\\
          &\leq \frac{1}{2}\left|[g_1(x+t)-g_2(x+t)]+[g_1(x-t)-g_2(x-t)]\right|+\frac{1}{2}\left| \int_{x-T}^{x+T}h_1(y)-h_2(y)dy \right|,\\
          &\leq \frac{1}{2}\left|2\|g_1-g_2\|_{\infty}\right|+\frac{1}{2}\left| \|h_1-h_2\|_{\infty}2T \right|,\\
          &\leq \|g_1-g_2\|+T\|h_1-h_2\|_{\infty},
        \end{align*}
        de lo que se sigue que si tomamos $(x,t)\in \mathbb{R}\times [0,T]$, entonces:
        \begin{align*}
          \|u_1-u_2\|_{\infty}\leq \|g_1-g_2\|+T\|h_1-h_2\|_{\infty}
        \end{align*}
        \demostrado
      \end{solucion}
  \end{enumerate}
\end{homeworkProblem}
