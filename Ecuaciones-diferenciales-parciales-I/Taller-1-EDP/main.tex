\documentclass{article}
\usepackage{tikz}
\usetikzlibrary{positioning, automata}
\usepackage{fancyhdr}
\usepackage{extramarks}
\usepackage[plain]{algorithm}
\usepackage{algpseudocode}
\usepackage[utf8]{inputenc} 
\usepackage[T1]{fontenc}    
\usepackage{hyperref}       
\usepackage{url}        
\usepackage{booktabs}
\usepackage{pdfpages}
\usepackage{amsfonts}   
\usepackage{nicefrac}       
\usepackage{microtype}      
\usepackage{amsmath}
\usepackage{graphicx}
\usepackage{float}
\usepackage{caption}
\usepackage{ragged2e}
\usepackage{amssymb}
\usepackage{mathtools}
\usepackage{xcolor}
\usepackage[spanish]{babel} % ¡Solo una vez!
\usepackage{array}
\usepackage{multirow}
\usepackage{multicol}
\usepackage{manfnt}
\usepackage{layout}
\usepackage{manfnt}
\usepackage{phaistos}
\usepackage{polynom}
\usepackage[inkscapeformat=png]{svg}
\usepackage[most]{tcolorbox}
\usepackage{listings}


\lstset{
    language=Python,                        % Lenguaje de programación
    basicstyle=\ttfamily\small\color[HTML]{F8F8F2}, % Fuente monoespaciada y texto blanco claro
    backgroundcolor=\color[HTML]{282A36},   % Fondo oscuro (Dracula)
    keywordstyle=\color[HTML]{FF79C6}\bfseries, % Palabras clave en rosa y negrita
    commentstyle=\color[HTML]{6272A4}\itshape, % Comentarios en gris azulado y cursiva
    stringstyle=\color[HTML]{F1FA8C},       % Cadenas en amarillo
    numberstyle=\tiny\color[HTML]{BD93F9}, % Números en lila
    identifierstyle=\color[HTML]{8BE9FD},  % Identificadores en azul claro
    emph={xlabel, ylabel, title, legend, grid}, % Destaca palabras específicas
    emphstyle=\color[HTML]{50FA7B},        % Color verde brillante para palabras destacadas
    numbers=left,                           % Números de línea a la izquierda
    stepnumber=1,                           % Mostrar número en cada línea
    numbersep=10pt,                         % Separación de los números de línea
    showstringspaces=false,                 % No marcar los espacios en las cadenas
    frame=note,                           % Marco alrededor del código
    framexleftmargin=20pt,                  % Espacio a la izquierda del marco para los números
    rulecolor=\color[HTML]{F8F8F2},         % Color del borde del marco
    breaklines=true,                        % Rompe líneas largas
    breakatwhitespace=true,                 % Rompe en los espacios
    tabsize=4,                              % Tamaño de tabuladores
    escapeinside={\%*}{*)}                  % Permite la inserción de LaTeX dentro del código
}

\lstset{literate=
  {á}{{\'a}}1 {é}{{\'e}}1 {í}{{\'i}}1 {ó}{{\'o}}1 {ú}{{\'u}}1
  {Á}{{\'A}}1 {É}{{\'E}}1 {Í}{{\'I}}1 {Ó}{{\'O}}1 {Ú}{{\'U}}1
  {à}{{\`a}}1 {è}{{\`e}}1 {ì}{{\`i}}1 {ò}{{\`o}}1 {ù}{{\`u}}1
  {À}{{\`A}}1 {È}{{\'E}}1 {Ì}{{\`I}}1 {Ò}{{\`O}}1 {Ù}{{\`U}}1
  {ä}{{\"a}}1 {ë}{{\"e}}1 {ï}{{\"i}}1 {ö}{{\"o}}1 {ü}{{\"u}}1
  {Ä}{{\"A}}1 {Ë}{{\"E}}1 {Ï}{{\"I}}1 {Ö}{{\"O}}1 {Ü}{{\"U}}1
  {â}{{\^a}}1 {ê}{{\^e}}1 {î}{{\^i}}1 {ô}{{\^o}}1 {û}{{\^u}}1
  {Â}{{\^A}}1 {Ê}{{\^E}}1 {Î}{{\^I}}1 {Ô}{{\^O}}1 {Û}{{\^U}}1
  {ã}{{\~a}}1 {ẽ}{{\~e}}1 {ĩ}{{\~i}}1 {õ}{{\~o}}1 {ũ}{{\~u}}1
  {Ã}{{\~A}}1 {Ẽ}{{\~E}}1 {Ĩ}{{\~I}}1 {Õ}{{\~O}}1 {Ũ}{{\~U}}1
  {œ}{{\oe}}1 {Œ}{{\OE}}1 {æ}{{\ae}}1 {Æ}{{\AE}}1 {ß}{{\ss}}1
  {ű}{{\H{u}}}1 {Ű}{{\H{U}}}1 {ő}{{\H{o}}}1 {Ő}{{\H{O}}}1
  {ç}{{\c c}}1 {Ç}{{\c C}}1 {ø}{{\o}}1 {å}{{\r a}}1 {Å}{{\r A}}1
  {€}{{\euro}}1 {£}{{\pounds}}1 {«}{{\guillemotleft}}1
  {»}{{\guillemotright}}1 {ñ}{{\~n}}1 {Ñ}{{\~N}}1 {¿}{{?`}}1 {¡}{{!`}}1 
}

\RequirePackage{algorithm}
\RequirePackage{algpseudocode}

%% Integral promedio
\def\Xint#1{\mathchoice
{\XXint\displaystyle\textstyle{#1}}%
{\XXint\textstyle\scriptstyle{#1}}%
{\XXint\scriptstyle\scriptscriptstyle{#1}}%
{\XXint\scriptscriptstyle\scriptscriptstyle{#1}}%
\!\int}
\def\XXint#1#2#3{{\setbox0=\hbox{$#1{#2#3}{\int}$ }
\vcenter{\hbox{$#2#3$ }}\kern-.57\wd0}}
\def\ddashint{\Xint=}
\def\dashint{\Xint-}


\setcounter{page}{0}
%
% Basic Document Settings
%

\topmargin=-0.45in
\evensidemargin=0in
\oddsidemargin=0in
\textwidth=6.5in
\textheight=9.0in
\headsep=0.25in

\linespread{1.1}
%|#*******************************************************************#|
\pagestyle{fancy}
\lhead{Taller}
\chead{\hmwkClass\ : \hmwkTitle}
\rhead{\firstxmark}
\lfoot{\lastxmark}
\cfoot{\thepage}

\renewcommand\headrulewidth{0.4pt}
\renewcommand\footrulewidth{0.4pt}

\setlength\parindent{0pt}

%
% Create Problem Sections
%

\newcommand{\enterProblemHeader}[1]{
    \nobreak\extramarks{}{Problema \arabic{#1} continúa en la siguiente página\ldots}\nobreak{}
    \nobreak\extramarks{Problema \arabic{#1} (continuación}{Problema \arabic{#1} continúa en la siguiente página\ldots}\nobreak{}
}

\newcommand{\exitProblemHeader}[1]{
    \nobreak\extramarks{Problema \arabic{#1} (continuación)}{Problema \arabic{#1} continúa en la siguiente página\ldots}\nobreak{}
    \stepcounter{#1}
    \nobreak\extramarks{Problema \arabic{#1}}{}\nobreak{}
}



\setcounter{secnumdepth}{0}
\newcounter{partCounter}
\newcounter{homeworkProblemCounter}
\setcounter{homeworkProblemCounter}{1}
\nobreak\extramarks{Problema \arabic{homeworkProblemCounter}}{}\nobreak
%margen de una pagina
\newenvironment{changemargin}[2]{%
\begin{list}{}{%
\setlength{\topsep}{0pt}%
\setlength{\leftmargin}{#1}%
\setlength{\rightmargin}{#2}%
\setlength{\listparindent}{\parindent}%
\setlength{\itemindent}{\parindent}%
\setlength{\parsep}{\parskip}%
}%
\item[]}{\end{list}}

%
% Homework Problem Environment
%
% This environment takes an optional argument. When given, it will adjust the
% problem counter. This is useful for when the problems given for your
% assignment aren't sequential. See the last 3 problems of this template for an
% example.
%
\newenvironment{homeworkProblem}[1][-1]{
    \ifnum#1>0
        \setcounter{homeworkProblemCounter}{#1}
    \fi
    \section{Problema \arabic{homeworkProblemCounter}:}
    \setcounter{partCounter}{1}
    \enterProblemHeader{homeworkProblemCounter}
}{
    \exitProblemHeader{homeworkProblemCounter}
}


%|#*******************************************************************#|

\newcommand{\hmwkTitle}{Parcial 3}
\newcommand{\hmwkDueDate}{\today}
\newcommand{\hmwkClass}{Análisis Númerico}
\newcommand{\hmwkUniversity}{Universidad Nacional de Colombia}
\newcommand{\hmwkAuthorName}{Andrés David Cadena Simons \and acadenas@unal.edu.co \and Sandra Natalia Florez Garcia \and sflorezga@unal.edu.co}
\newcommand{\hmwkInstructor}{Jorge Mauricio Ruiz Vera}

%
% Title Page
%

\title{
    \vspace{2in}
    \textmd{\textbf{\hmwkClass:\ \hmwkTitle}}\\
    \normalsize\vspace{0.1in}\small{\today}\\
    \vspace{0.1in}\large{\textit{\hmwkUniversity}}\\
    \vspace{1.5in} \textrm{\hmwkInstructor}
    \vspace{1.5in}
}

\author{\hmwkAuthorName}
\date{}

\renewcommand{\part}[1]{\textbf{\large Part \Alph{partCounter}}\stepcounter{partCounter}\\}
%
%   Nuevos tipos de columna
%
\newcolumntype{C}{>{$}c<{$}}
\newcolumntype{L}{>{$}l<{$}}
\newcolumntype{R}{>{$}r<{$}}
% ---------------------------
%|  Various Helper Commands  |
% ---------------------------

% Useful for algorithms
\newcommand{\alg}[1]{\textsc{\bfseries \footnotesize #1}}

% -For derivatives-
\newcommand{\deriv}[2]{\frac{\mathrm{d #1 }}{\mathrm{d} #2} }

%For derivatives of degree >1
\newcommand{\mderiv}[3]{\frac{\mathrm{d^{#3} #1 }}{\mathrm{d} #2} }

% -For partial derivatives-
\newcommand{\pderiv}[2]{\frac{\partial}{\partial #1} (#2)}

% -Integral dx-
\newcommand{\dx}{\hspace{3pt}\mathrm{d}}

% Alias for the Solution section header
\newcommand{\solution}{\textbf{\\\\\large Solución:\\ \hspace*{5pt}}}

%parentesis automatico
\newcommand{\parauto}[1]{\ensuremath{\left( #1 \right)}}

% Probability commands: Expectation, Variance, Covariance, Bias
\newcommand{\E}{\mathrm{E}}
\newcommand{\Var}{\mathrm{Var}}
\newcommand{\Cov}{\mathrm{Cov}}
\newcommand{\Bias}{\mathrm{Bias}}


%TcolorBox

% Definir colores y la tcolorbox de la solución
\definecolor{myDColor}{HTML}{101010} 

\definecolor{myLColor}{RGB}{153,204,255} 

\definecolor{LinkColor}{HTML}{9669d9} 


\newtcolorbox{solucion}[1][]{%
    enhanced,
    skin first=enhanced,
    skin middle=enhanced,
    skin last=enhanced,
    before upper={\parindent15pt},
    breakable,
    boxrule = 0pt,
    frame hidden,
    borderline west = {4pt}{0pt}{myDColor},
    colback = myLColor!5,
    coltitle = myLColor!5,
    sharp corners,
    rounded corners = southeast,
    arc is angular,
    arc = 3mm,
    attach boxed title to top left,
    boxed title style = {%
        enhanced,
        colback = myDColor,
        colframe = myDColor,
        top = 0pt,
        bottom = 0pt,
        sharp corners,
        rounded corners = northeast,
        arc is angular,
        arc = 2mm,
        rightrule = 0pt,
        bottomrule = 0pt,
        toprule = 0pt,
    },
    title = {\bfseries\large Solución:}, 
    overlay unbroken={%
        \node[anchor=west, color=black!70] at (title.east) {#1};
        \path[fill = tcbcolback!80!black] 
            ([yshift = 3mm]interior.south east) -- ++(-0.4,-0.1) -- ++(0.1,-0.2);
    },
    overlay first = {%
        \node[anchor=west, color=black!70] at (title.east) {#1};
        \path[fill = tcbcolback!80!black] 
            ([yshift = 3mm]interior.south east) -- ++(-0.4,-0.1) -- ++(0.1,-0.2);
    },
    overlay middle={%
        \path[fill = tcbcolback!80!black] 
            ([yshift = -3mm]interior.north east) -- ++(-0.4,0.1) -- ++(0.1,0.2);
        \path[fill = tcbcolback!80!black] 
            ([yshift = 3mm]interior.south east) -- ++(-0.4,-0.1) -- ++(0.1,-0.2);
    },
    overlay last={%
        \path[fill = tcbcolback!80!black] 
            ([yshift = -3mm]interior.north east) -- ++(-0.4,0.1) -- ++(0.1,0.2);
        \path[fill = tcbcolback!80!black] 
            ([yshift = 3mm]interior.south east) -- ++(-0.4,-0.1) -- ++(0.1,-0.2);
    },
    extras middle and last = { rounded corners = northeast }
}

% Ambientes de teorema, nota, etc.
\newtheorem{teorema}{Teorema}[section] % Vinculado a la sección
\newtheorem{proposicion}{Proposición}[section] % Vinculado a la sección
\newtheorem{nota}{Nota}[section] % Vinculado a la sección
\newtheorem{ejemplo}{Ejemplo}[section] % Vinculado a la sección
\newtheorem{lema}{Lema}[section] % Vinculado a la sección
\newtheorem{axioma}{Axioma}[section] % Vinculado a la sección
\newtheorem{definicion}{Definición}[section] % Vinculado a la sección




%Cuadrito de Demostración
\newcommand{\heart}{\begin{tikzpicture}[scale=0.001cm,rotate=180]
\fill[black] (0,0) 
        .. controls (0,-0.5) and (0.3,-1.8) .. (2,-2)
        .. controls (4.2,-2) and (5.5,0) .. (5.5,3)
        .. controls (5.5,5.5) and (3.5,7.5) .. (0,10)
        .. controls (-3.5,7.5) and (-5.5,5.5) .. (-5.5,3)
        .. controls (-5.5,0) and (-4.2,-2) .. (-2,-2)
        .. controls (-0.3,-1.8) and (0,-0.5) .. (0,0);
\end{tikzpicture}}

\newcommand{\demostrado}[0]{ \begin{flushright} $\heart{}$ \end{flushright}}

\usepackage{amsmath}
\usepackage{geometry}
\usepackage{tikz}
\usepackage{float}
\usepackage{graphics}
\usepackage{cancel}
\providecommand{\abs}[1]{\lvert#1\rvert}
\providecommand{\norm}[1]{\lVert#1\rVert}


\tikzset{every picture/.style={line width=0.75pt}} %set default line width to 0.75pt        

\begin{document}
\maketitle
\thispagestyle{empty}
\newpage

\begin{homeworkProblem}
Demuestre el teorema multinomial. Sea $x=\left(x_1, \ldots, x_n\right) \in \mathbb{R}^n$, entonces
$$
\left(x_1+\cdots+x_n\right)^k=\sum_{|\alpha|=k}\left(\begin{array}{c}
|\alpha| \\
\alpha
\end{array}\right) x^\alpha,
$$
donde $\left(\begin{array}{c}|\alpha| \\ \alpha\end{array}\right):=\dfrac{|\alpha| !}{\alpha !}$ con $\alpha !=\alpha_{1} ! \ldots \alpha_{n} !$ y $x^\alpha=x_1^\alpha \ldots x_n^{\alpha_n}$. La suma es considerada sobre todos los multi-índices $\alpha=\left(\alpha_1, \ldots, \alpha_n\right)$ con $|\alpha|=k$.
Sugerencia. Para cada $k \geq 0$, muestre la igualdad buscada por inducción sobre en número de variables $n$.\\
    \begin{solucion}
        Dado $k \geq 0 $, razonemos por inducción sobre el número de variables $n$. Para $n = 1$ el resultado es trivial. \\
        Ahora, supongamos que el resultado se tiene para todo $s<n+1$, esto es:
        \begin{align*}
            (x_{1}+...+x_{n})^{k} = \displaystyle \sum_{\abs{\alpha_{1} + ... + \alpha_{n}} = k} \binom{\abs{\alpha}}{\alpha} x^{\alpha}.
        \end{align*}
        Ahora, procedamos con $n+1:$
        \begin{align*}
            (x_{1}+...+x_{n}+x_{n+1})^{k}  &= (x_{1}+...+x_{n}+x_{n+1})^{k}, \hspace{1cm} \text{teorema del binomio de Newton}\\
                                        & = \displaystyle \sum_{s=0}^{k} \dfrac{k!}{s!(k-s)!} (x_{1}+...+x_{n})^s x_{n+1}^{k-s}, \hspace{1cm} \text{ hipótesis de inducción}\\
                                        & = \displaystyle \sum_{s=0}^{k}\dfrac{k!}{s!(k-s)!} \left( \displaystyle \sum_{\abs{\alpha_{1} + ... +\alpha_{n} = s}} \dfrac{s!}{\alpha_{1}!...\alpha_{n}!}x_{1}^{\alpha_{1}}...x_{n}^{\alpha_{n}} \right)x_{n+1}^{k-s} ,\\ 
                                         & = \displaystyle\sum_{s=0}^{k} \displaystyle \sum_{\abs{\alpha_{1} + ... +\alpha_{n} } = s} \dfrac{k!}{s!(k-s)!} \dfrac{s!}{\alpha_{1}!...\alpha_{n}!}x_{1}^{\alpha_{1}}...x_{n}^{\alpha_{n}}x_{n+1}^{k-s}\\
                                         & = \displaystyle\sum_{s=0}^{k} \displaystyle \sum_{\abs{\alpha_{1} + ... +\alpha_{n} } = s} \dfrac{k!}{\alpha_{1}!...\alpha_{n}!(k-s)!}x_{1}^{\alpha_{1}}...x_{n}^{\alpha_{n}}x_{n+1}^{k-s}, \hspace{0,5cm}  \alpha_{n+1} = k-s\\
                                         &=\displaystyle \sum_{\abs{\alpha_{1}...+\alpha_{n}+\alpha_{n+1}}} \dfrac{k!}{\alpha_{1}!...\alpha_{n}!(k-s)!}x_{1}^{\alpha_{1}}...x_{n}^{\alpha_{n}}x_{n+1}^{\alpha_{n+1}}.\\
                                         &= \displaystyle \sum_{\abs{\alpha}} \binom{\abs{\alpha}}{\alpha}x^{\alpha}
        \end{align*}
        \demostrado
    \end{solucion}
\end{homeworkProblem}
\newpage

\begin{homeworkProblem}
    Para cada una de las siguientes EDPs diga su orden y determine si es lineal, semilineal, cuasilineal o completamente no lineal.\\
    \textbf{Nota:} Por simplicidad asumiremos que $u=u(x,t)$.
\begin{itemize}
    \item  \textbf{Ecuación de Schrödinger:} $i u_t+\Delta u+V(x) u=0$.
    \begin{solucion}
        \textbf{Lineal de segundo orden.}\\
        Veamos que la siguiente expresión es válida:
        \begin{align*}
            iu_t+\Delta u+V(x)u&=iu_t+u_{xx}+u_{tt}+V(x)u\\
            &=u_{xx}+u_{tt}+iu_{t}+V(x)u\\
            &=0
        \end{align*}
        Luego los coeficientes de la EDP son $(1,1,i,V(x))$, por lo cuál sabemos que la ecuación es \textbf{Lineal} y como tenemos  a $u_{xx}$, entonces podemos concluir en que la EDP es de \textbf{segundo orden}. 
    \end{solucion}
    
\item \textbf{Ecuación del telégrafo:} $u_{tt}+2du_{t}-u_{x}x=0$ donde $d \in \mathbb{R}$.
    \begin{solucion}
        \textbf{Lineal de segundo orden.}\\
        Veamos que la siguiente expresión es válida:
        \begin{align*}
            u_{tt}+2du_{t}-u_{x}x&=u_{tt}+2du_{t}-xu_{x}\\
            &=0
        \end{align*}
        Luego los coeficientes de la EDP son $(1,2d,-x)$, por lo cuál sabemos que la ecuación es \textbf{Lineal} y como tenemos a $u_{tt}$, entonces podemos concluir en que la EDP es de \textbf{segundo orden}.
    \end{solucion}
\newpage
\item \textbf{Ecuación de superficie mínima:} $\operatorname{div}\left(\dfrac{\nabla u}{\left(1+\|\nabla u\|^2\right)^{\frac{1}{2}}}\right)=0$, donde $\operatorname{div}(F)=$ $\nabla \cdot F$, es el operador divergencia del campo $F$.
    \begin{solucion}
        \textbf{Cuasilineal de segundo orden.}\\
        Veamos que la siguiente expresión es válida:
        \begin{align*}
            \dfrac{\nabla u}{\left(1+\|\nabla u\|^2\right)^{\frac{1}{2}}}&=\dfrac{\left(u_x,u_t\right)}{(1+u_x^2+u_t^2)^{\frac{1}{2}}}\\
            &=\left(u_x(1+u_x^2+u_t^2)^{-\frac{1}{2}},u_t(1+u_x^2+u_t^2)^{-\frac{1}{2}}\right)
        \end{align*}
        Luego:
        \begin{align*}
            \operatorname{div}\left(\dfrac{\nabla u}{\left(1+\|\nabla u\|^2\right)^{\frac{1}{2}}}\right)&=\operatorname{div}\left(u_x(1+u_x^2+u_t^2)^{-\frac{1}{2}},u_t(1+u_x^2+u_t^2)^{-\frac{1}{2}}\right)\\
            &=\frac{u_{xx}}{\sqrt{1+u_x^2+u_t^2}}+\left(-\frac{2(u_x)^2(u_{xx})+2(u_t)(u_x)(u_{tx})}{2\sqrt{(1+u_x^2+u_t^2)^3}}\right)+\\
            &\phantom{=.}\frac{u_{tt}}{\sqrt{1+u_x^2+u_t^2}}+\left(-\frac{2(u_x)(u_t)(u_{xt})+2(u_t)^2(u_{tt})}{2\sqrt{(1+u_x^2+u_t^2)^3}}\right)\\
            &\hspace{-1cm}=\frac{u_{xx}(1+u_x^2+u_t^2)+u_{tt}(1+u_x^2+u_t^2)-u_{xx}(u_x^2)-u_{tx}(u_x)(u_t)-u_{tt}(u_t^2)-u_{xt}(u_x)(u_t)}{\sqrt{(1+u_x^2+u_t^2)^3}}\\
            &=\frac{u_{xx}(1+u_t^2)+u_{tt}(1+u_x^2)-u_{tx}(u_x)(u_t)-u_{xt}(u_x)(u_t)}{\sqrt{(1+u_x^2+u_t^2)^3}}\\
            &=0
        \end{align*}
        Luego por el coeficiente $\frac{(1+u_t^2)}{\sqrt{(1+u_x^2+u_t^2)^3}}$ que acompaña al término $u_{xx}$ de la EDP sabemos que la ecuación es \textbf{Cuasilineal} y como tenemos a $u_{xx}$, entonces podemos concluir en que la EDP es de \textbf{segundo orden}.
    \end{solucion}
\newpage
\item \textbf{Ecuación de Monge-Ampere:} $\operatorname{det}(\mathrm{Hu})=f$, donde $\mathrm{Hu}$ denota la matriz Hessiana de $u$.
    \begin{solucion}
        \textbf{Completamente no lineal de segundo orden.}\\
        En nuestro caso $u=u(x,t)$, es válida la siguiente expresión:
        \begin{align*}
            \operatorname{det}(Hu)&=u_{xx}u_{tt}-u_{tx}u_{xt}=u_{xx}u_{tt}-(u_{xt})^2\\
            &=f
        \end{align*}
        Luego, si asumimos $f=f(x)$, por el termino $(u_{xt})^2$ de la EDP sabemos que la ecuación es \textbf{Completamente no lineal} y como tenemos a $u_{xx}$, entonces podemos concluir en que la EDP es de \textbf{segundo orden}.\\ 
    \end{solucion}

\end{itemize}
\end{homeworkProblem}
\newpage
\begin{homeworkProblem}
    Solucione utilizando características:
\begin{itemize}
    \item[(a)] $u_x+u_y=u^2$, con $u(x, 0)=h(x)$.
    \begin{solucion}
     Tendiendo en cuenta el problema de valor inicial tomaremos la curva \\$C=\{(r,0,h(r))\}$. Note que la EDP puede ser reescrita de la siguiente forma:
     \begin{align*}
         u_x+u_y-u^2&=0\\
         (1,1,u^2)\cdot(u_x,u_y,-1)&=0
     \end{align*}
     De esta forma con la información de la curva $C$ y la reescritura podemos plantear el siguiente sistema de EDOs 
    \begin{align*}
        \begin{cases}
            \dfrac{dx}{ds}=1,& x(0,r)=r\\
            \\
            \dfrac{dy}{ds}=1,& y(0,r)=0\\
            \\
            \dfrac{dz}{ds}=z^2,& z(0,r)=h(r).\\
        \end{cases}
    \end{align*}
    De esta manera tenemos tres EDO que pueden ser resueltas por separación de variables obteniendo así:
    \begin{align*}
        x&=s+c_1(r)\\
        y&=s+c_2(r)\\
        z&=-\frac{1}{s+c_3(r)}.
    \end{align*}
    y usando las condiciones iniciales para cada EDO obtenemos que:
    \begin{align*}
        c_1(r)&=r\\
        c_2(r)&=0\\
        c_3(r)&=-\frac{1}{h(r)}.
    \end{align*}
    Ahora verificaremos si podemos devolver el cambio de variable por medio del Jacobiano:
    \begin{align*}
       \left. \det\begin{pmatrix}
            \dfrac{\partial x}{\partial s} & \dfrac{\partial x}{\partial r}\\
            & \\
            \dfrac{\partial y}{\partial s} & \dfrac{\partial y}{\partial r}
        \end{pmatrix} \right|_{(0,r)}=\det\begin{pmatrix}
            1 & 1\\
            1 & 0
        \end{pmatrix}=-1\neq 0
    \end{align*}
    Como es distinto de 0 por el teorema de la función inversa sabemos que podemos despejar nuestras ecuaciones para dejar a $z$ en términos de $x$ y $y$. Como $c_2(r)=0$ tenemos que $y=s$. Además como $c_1(r)=r$ en la ecuación de $x$ tenemos que $x=y+r$ es decir que $r=x-y$ y reemplazando en $z$ obtenemos que nuestra solución es:
    $$u(x,y)=z=-\frac{1}{y-\frac{1}{h(x-y)}}=\frac{h(x-y)}{yh(x-y)-y}.$$
    \end{solucion}
    \item[(b)] Sea $\vec{b}=\left(b_1, \ldots, b_n\right) \in \mathbb{R}^n$ con $b_n \neq 0$. Considere la EDP: $\vec{b} \cdot \nabla u=e^{-u}$, con $u\left(x_1, \ldots, x_{n-1}, 0\right)=h\left(x_1, \ldots, x_{n-1}\right)$.
    \begin{solucion}
        Apoyándonos en el dato inicial nuestra superficie esta dada por: $$C=\{(r_1,\dots,r_{n-1},0,h(r_1\dots,r_{n-1}))\}$$
        y realizando una reescritura de la EDP obtenemos que:
        $$(b_1,\dots,b_n,e^{-u})\cdot(u_{x_1},\dots,u_{x_n},-1)=0$$
         De esta manera notamos que el sistema de EDOs es el siguiente:
    \begin{align*}
        \begin{cases}
            \dfrac{dx_i}{ds}=b_i,&x_i(0,r_1\dots,r_{n-1})=r_i\text{ con }1\leq i\leq n-1,\\
            \\
            \dfrac{dx_n}{ds}=b_n,&x_n(0,r_1\dots,r_{n-1})=0,\\
            \\
            \dfrac{dz}{ds}=e^{-z},&z(0,r_1\dots,r_{n-1})=h(r_1\dots,r_{n-1}).
        \end{cases}
    \end{align*}
    Así resolviendo cada EDO tenemos que:
    \begin{align*}
        x_i&=b_is+c_i(r)\text{ con }1\leq i\leq n,\\
        z&=\log(s+c(r)).
    \end{align*}
    y usando las condiciones iniciales tenemos que:
    \begin{align*}
        c_i(r)&=r_i\text{ con }1\leq i\leq n-1,\\
        c_n(r)&=0,\\
        c(r)&=e^{h(r_1,\dots,r_{n-1})}.
    \end{align*}
    Ahora observemos si el Jacobiano es diferente de 0 para poder aplicar el teorema de la función inversa:
    \begin{align*}
       \left. \det\begin{pmatrix}
            \dfrac{\partial x_1}{ \partial s} & \dfrac{ \partial x_1}{\partial r_{1}} & \cdot & \cdot & \cdot & \dfrac{\partial x_1}{\partial r_{n-1}}\\
            \dfrac{\partial x_2}{\partial s} & \dfrac{\partial x_2}{
            \partial r_{1}} & \cdot & \cdot & \cdot & \dfrac{\partial x_2}{\partial r_{n-1}}\\
            \cdot&\cdot & & & &\cdot \\
            \cdot&\cdot & & & &\cdot \\
            \cdot&\cdot & & & &\cdot \\
            \dfrac{\partial x_n}{\partial s} & \dfrac{\partial x_n}{
            \partial r_{1}} & \cdot & \cdot & \cdot & \dfrac{\partial x_n}{\partial r_{n-1}}
        \end{pmatrix}\right|_{(0,r_1,\dots,r_{n-1})}&=\det\begin{pmatrix}
            b_1& 1 & 0 & \cdot& \cdot& \cdot& 0\\
            b_2& 0 & 1 &\cdot& \cdot& \cdot& 0\\
            \cdot& \cdot& & & & & \\
            \cdot& \cdot& & & & & \\
            \cdot& \cdot& & & & & \\
            b_n & 0 & 0 & \cdot& \cdot& \cdot& 0
        \end{pmatrix}\\&=(-1)^{n+1}b_n\neq0
    \end{align*}
    Ahora si con seguridad podemos despejar. Empecemos notando que como $c_n(r)=0$ $x_n=b_ns$ y como $b_n\neq0$ tenemos que $s=\dfrac{x_n}{b_n}$ de esta manera como $c_i(r)=r_i$ para $1\leq i\leq n-1$ tenemos que $x_i=\dfrac{b_i}{b_n}x_n+r_i$ y por tanto $r_i=x_i-\dfrac{b_i}{b_n}x_n$. De esta manera reemplazando en $z$ obtenemos que la solución es:
    $$u(x,y)=z=\log\left(\frac{x_n}{b_n}+e^{h\left(x_1-\frac{b_1}{b_n}x_n,\dots,x_{n-1}-\frac{b_{n-1}}{b_n}x_n\right)}\right)$$
    \end{solucion}
   
    \item[(c)] $x_1 u_{x_1}+2 x_2 u_{x_2}+u_{x_3}=3 u$, con $u\left(x_1, x_2, 0\right)=g\left(x_1, x_2\right)$.
    \begin{solucion}
        Inspirado por el dato inicial tenemos que la superficie es $C=\{(r_1,r_2,0,g(r_1,r_2))\}$ y si reescribimos la EDP de la siguiente forma:
        $$(x_1,2x_2,1,3u)\cdot(u_{x_1},u_{x_2},u_{x_3},-1)=0$$
        De esta forma el sitema de EDOs queda planteado de la siguiente manera:
        $$\begin{cases}
            \dfrac{dx_1}{ds}=x_1,&x_1(0,r_1,r_2)=r_1,\\
            \\
            \dfrac{dx_2}{ds}=2x_2,&x_2(0,r_1,r_2)=r_2,\\
            \\
            \dfrac{dx_3}{ds}=1,&x_3(0,r_1,r_2)=0\\
            \\
            \dfrac{dz}{ds}=3z,&z(0,r_1,r_2)=g(r_1,r_2)
        \end{cases}$$
        Asi resolviendo cada EDO tenemos que:
        \begin{align*}
            x_1&=c_1(r)e^s\\
            x_2&=c_2(r)e^{2s}\\
            x_3&=s+c_3(r)\\
            z&=c(r)e^{3s}
        \end{align*}
        y por las condiciones iniciales tenemos que:
        \begin{align*}
            c_1(r)&=r_1\\
            c_2(r)&=r_2\\
            c_3(r)&=0\\
            c(r)&=g(r_1,r_2)\\
        \end{align*}
        Ahora verifiquemos por medio del Jacobiano si podemos aplicar el teorema de la función inversa:
        \begin{align*}
            \left.\det\begin{pmatrix}
                \dfrac{\partial x_1}{\partial s} & \dfrac{\partial x_1}{\partial r_1} & \dfrac{\partial x_1}{\partial r_2} \\ 
                & & \\
                \dfrac{\partial x_2}{\partial s} & \dfrac{\partial x_2}{\partial r_1} & \dfrac{\partial x_2}{\partial r_2} \\
                & & \\
                \dfrac{\partial x_3}{\partial s} & \dfrac{\partial x_3}{\partial r_1} & \dfrac{\partial x_3}{\partial r_2} \\
            \end{pmatrix}\right|_{(0,r_1,r_2)}=\det\begin{pmatrix}
                r_1 & 1 & 0 \\
                2r_2& 0 & 1\\
                1 & 0 & 0 \\
            \end{pmatrix}=1\neq 0
        \end{align*}
        Esto nos indica que si podemos despejar entonces de entrada sabemos que $x_3=s$ ya que $c_3(r)=0.$ Luego como $c_1(r)=r_1$ tenemos que $x_1=r_1e^x_3$ y por tanto $r_1=x_1e^{-x_3}$. De manera similar obtenemos que $r_2=x_2e^{-2x_3}$ y juntando todo y reemplazando en $z$ obtenemos la solución que es:
        $$u(x,y)=z=g(x_1e^{-x_3},x_2e^{-2x_3})e^{3x_3}.$$
    \end{solucion}
    \item[(d)] $u u_x+u_y=1$, con $u(x, x)=\frac{1}{2} x$.
    \begin{solucion}
        Tomando como referencia el dato inicial nuestra curva sera $C=\{(r,r,\frac{1}{2}r)\}$ y con una reescritura de la EDP tenemos que:
        $$(u,1,1)\cdot(u_x.u_y,-1)=0$$
        De esta manera obtenemos el siguiente sistema de EDOs:
        $$\begin{cases}
            \dfrac{dx}{ds}=z,&x(r,r)=r\\
            \\
            \dfrac{dy}{ds}=1,&y(r,r)=r\\
            \\
            \dfrac{dz}{ds}=1,&z(r,r)=\frac{1}{2}r\\  
        \end{cases}$$
        Resolviendo la ultima ecuación tenemos que $z=s+c_3(r)$ pero por el valor inicial sabemos que $c_3(r)=-\frac{1}{2}r$ es decir $z=s-\frac{1}{2}r$. Si reemplazamos $z$ por la expresión podemos resolver la primera ecuación y obtenemos que:
        \begin{align*}
            x&=\dfrac{s^2}{2}-\dfrac{rs}{2}+c_1(r)\\
            y&=s+c_2(r)
        \end{align*}
        y utilizando los valores iniciales obtenemos que $c_1(r)=r$ y $c_2(r)=0$ así llegamos a las 3 ecuaciones:
        \begin{align*}
            x&=\dfrac{s^2}{2}-\dfrac{rs}{2}+r\\
            y&=s\\
            z&=s-\frac{r}{2}
        \end{align*}
        Antes de intentar despejar, aseguremos que podemos por medio del Jacobiano:
        \begin{align*}
            \left.\det\begin{pmatrix}
                \dfrac{\partial x}{\partial s} & \dfrac{\partial x}{\partial r}\\
            & \\
            \dfrac{\partial y}{\partial s} & \dfrac{\partial y}{\partial r}
            \end{pmatrix}\right|_{(r,r)}=\det\begin{pmatrix}
                \dfrac{r}{2}& 1\\
                1& 1\\
            \end{pmatrix}=\dfrac{r}{2}-1\neq 0
        \end{align*}
        esto quiere decir que podemos invertir solo si $r\neq 2$ pero suponiendo esto tenemos por las ecuaciones que $x=\dfrac{y^2}{2}+r(1-\dfrac{y}{2})$ es decir que $r=\dfrac{2x-y^2}{2-y}$ de esta manera reemplazando en $z$ obtenemos que la solución es:
        $$u(x,y)=z=y-\dfrac{2x-y^2}{4-2y}=\dfrac{4y-2x-y^2}{4-2y}.$$
    \end{solucion}
    \item[(e)] $u_t+u_x^2=t$, con $u(x, 0)=0$.
    \begin{solucion}
        Inspirado en los datos iniciales tenemos que nuestra curva es $C=\{(r,0,0)\}$ Notemos que si reescribimos nuestra EDP en términos de una función $F(x,t,z,p,q)=0$ donde $z=u$, $p=u_x$ y $q=u_t$ tenemos que:
        $$F(x,t,z,p,q)=p^2+q-t=0$$
        De aquí podemos concluir inmediatamente que $F_p=2p$, $F_q=1$, $F_x=0$, $F_t=-1$ y $F_z=0$. Por lo tanto tendríamos las siguientes EDOs:
        \begin{align*}
            \dfrac{dx}{ds}=2p&&
            \dfrac{dt}{ds}=1&&
            \dfrac{dz}{ds}=2p^2+q&&
            \dfrac{dp}{ds}=0&&
            \dfrac{dq}{ds}=1
        \end{align*}
        Ahora por la curva ya tenemos tres datos iniciales $x(0,r)=r$, $t(0,r)=0$ y $z(0,r)=0$ nos faltan los datos iniciales para las EDOs de $p$ y $q$ intentemos determinarlas. Nuestra primera ecuación para esto es:
        $$F(r,0,0,\psi_1,\psi_2)=\psi_1^2+\psi_2=0$$
        y la segunda esta dada por:
        $$0=\psi_1\cdot1+\psi_2\cdot0$$
        De esto obtenemos que $\psi_1=0=\psi_2$. De esta manera ya podemos plantear el sistema de EDOs
        $$\begin{cases}
           \dfrac{dx}{ds}=2p,&x(0,r)=r,\\
           \\
            \dfrac{dt}{ds}=1,&t(0,r)=0,\\
            \\
            \dfrac{dz}{ds}=2p^2+q,&z(0,r)=0,\\
            \\
            \dfrac{dp}{ds}=0,&p(0,r)=0,\\
            \\
            \dfrac{dq}{ds}=1,&q(0,r)=0.\\
        \end{cases}$$
        De las dos ultimas ecuaciones obtenemos que $p=c_4(r)$ pero por el valor inicial sabemos entonces que $p=0$ y que $q=s+c_5(r)$ pero como $q(0,r)=0$ tenemos que $q=s$ y con esta información podemos resolver las tres primeras fácilmente ya que si reemplazamos obtenemos que:
        $$\begin{cases}
           \dfrac{dx}{ds}=0,&x(0,r)=r,\\
           \\
            \dfrac{dt}{ds}=1,&t(0,r)=0,\\
            \\
            \dfrac{dz}{ds}=s,&z(0,r)=0.\\
        \end{cases}$$
        Entonces por razonamientos similares tenemos que $t=s$, $x=c_1(r)$ pero por el valor inicial concluimos que $x=r$ y por ultimo $z=\dfrac{s^2}{2}+c_3(r)$ y por el valor inicial $c_3(r)=0$ es decir $z=\dfrac{s^2}{2}$ de esta manera reemplazando la solución es:
        $$u(x,y)=z=\dfrac{t^2}{2}.$$
        Note que no hubo necesidad de hacer ningún despeje pero solo para sentirnos seguros aquí realizaremos el Jacobiano:
        \begin{align*}
            \left.\det\begin{pmatrix}
                \dfrac{\partial x}{\partial s} & \dfrac{\partial x}{\partial r}\\
            & \\
            \dfrac{\partial t}{\partial s} & \dfrac{\partial t}{\partial r}
            \end{pmatrix}\right|_{(0,r)}=\det\begin{pmatrix}
                0 & 1\\
                1 & 0
            \end{pmatrix}=1\neq 0.
        \end{align*}
    \end{solucion}
\end{itemize}
\end{homeworkProblem}
\newpage

\begin{homeworkProblem}
    Sean $a_i, c, i=1, \ldots, n$ funciones en $C\left(\mathbb{R}^n\right)$ (el espacio de funciones continuas a valores reales con domino $\mathbb{R}^n$ ). Considere el problema de Cauchy

\begin{equation}
    \left\{\begin{array}{l}
\displaystyle\sum_{i=1}^n a_i(x) u_{x_i}=c(x), \quad x \in \mathbb{R}^n, \\
\left.u\right|_{\Gamma}=g
\end{array}\right.
\end{equation}

donde $\Gamma$ es la superficie (o $n-1$ variedad) en $\mathbb{R}^n$ dada por
$$
\Gamma=\left\{\vec{r}=\left(r_1, r_2 \ldots, r_{n-1}\right) \in \mathbb{R}^{n-1}: \Phi(\vec{r})=\left(\phi_1(\vec{r}), \ldots, \phi_n(\vec{r})\right)\right\},
$$
cada una de las funciones $g, \phi_j, j=1, \ldots, n$ está en $C^1\left(\mathbb{R}^{n-1}\right.$ ) (es decir, son funciones continuamente diferenciables con dominio $\mathbb{R}^{n-1}$ ). Además, suponga que $\Gamma$ no es característica en el sentido
$$
\operatorname{det}\left(\begin{array}{cccc}
a_1(\Phi(\vec{r})) & \partial_{r_1} \phi_1(\vec{r}) & \ldots & \partial_{r_{n-1}} \phi_1(\vec{r}) \\
\vdots & \vdots & & \\
a_n(\Phi(\vec{r})) & \partial_{r_1} \phi_n(\vec{r}) & \ldots & \partial_{r_{n-1}} \phi_n(\vec{r})
\end{array}\right) \neq 0
$$
para todo punto $\vec{r}=\left(r_1, \ldots, r_{n-1}\right) \in \mathbb{R}^{n-1}$.
Muestre que existe una única solución $u \in C^1\left(\mathbb{R}^n\right)$ del problema de Cauchy (1).
\begin{solucion}
    Note que el problema nos invita a plantear el siguiente sistema de ecuaciones ordinarias:
    \begin{align*}
        \frac{dx_i}{ds}=a_i(x) &\hspace{2cm} x_i(0,\vec{r})=\phi_i(\vec{r})\\
        \frac{dz}{ds}=c(x) &\hspace{2cm} z(o,\vec{r})=g(\vec{r})
    \end{align*}
    Con $1\leq i\leq n$.\\
    Note que podemos describir el problema acoplandolo de la siguiente forma:
    \begin{align*}
        \frac{d(x_1,x_2,\cdots,x_n,z)}{ds}=(a_1,a_2,\cdots,a_n,c)(x) \hspace{2 cm} f(0,\vec{r})=(\phi_1,\phi_2,\cdots,\phi_n,g)(\vec{r})
    \end{align*}
    Luego, como $a_i, c\in C^1(\mathbb{R}^n$) y $\phi_i, g \in C^1(\mathbb{R}^{n-1})$ con $1\leq i\leq n$, entonces por la desigualdad del valor medio se tiene que todas estas son localmente Lipschitz y por ende el acoplamiento de estás, luego $(a_1,a_2,\cdots,a_n,c)(x) \in C^1(\mathbb{R}^n)$ y es localmente Lipschitz, por lo que el sistema de ecuaciones diferenciales ordinarias tiene una única solución $z(s,\vec{r})$, además como por hipótesis tenemos que $\Gamma$ es no característica en todo punto  $(0,\vec{r})$ tal que $\vec{r}\in\mathbb{R}^{n-1}$, entonces existe una vecindad $V$ al rededor de $0$ en $\mathbb{R}$, tal que el cambio de variable de  $x$ a $(s,\vec{r})$ es invertible, lo que nos asegura que el problema de Cauchy tiene una única solución $\mu(x)\in C^{1}(\mathbb{R}^{n})$ en $U\times \mathbb{R}^{n-1}$ en donde $U$ es una vecindad de $\mathbb{R}$.
    \demostrado
 \end{solucion}
\end{homeworkProblem}

%%%%%%%%%%%%%%%%%%%%%%%%%%%%%%%%%%%%%%%%%%%%%%%%%%%%%%%
\end{document}