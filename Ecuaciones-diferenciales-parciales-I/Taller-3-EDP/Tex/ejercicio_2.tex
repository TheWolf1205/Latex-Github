\begin{homeworkProblem}
  Considere la función $x\in\mathbb{R}^{n}\rightarrow \eta(x)$ dada por:
  \begin{align*}
    \eta(x)= 
    \begin{cases}
      Ce^{\frac{1}{|x|^2-1}}, &\text{ Si } |x|<1,\\
      0, &\text{ Si } |x|\geq 1.
    \end{cases}
  \end{align*}
  Donde $C$ es tal que $\int_{\mathbb{R}}\eta(x)dx=1$. La función anterior se le llama un mollifier. Dado $\epsilon>0$ definimos $\eta_{\epsilon}(x)=\frac{1}{\epsilon}\eta(\frac{x}{\epsilon})$.
  \begin{enumerate}
    \item Muestre que $\eta_{\epsilon}\in C^{\infty}(\mathbb{R}^{n})$. Además muestre que $supp(\eta_{\epsilon})=\overline{\{x\in\mathbb{R}^{n}:\eta_{\epsilon}(x)\neq 0\}}=\overline{B(0,\epsilon)}$.
      \begin{solucion}
        Primero veamos que $\eta_{\epsilon}\in C^{\infty}(\mathbb{R}^{n})$, para esto note $\eta_{\epsilon}=\frac{C}{\epsilon}(e^{x}\circ \frac{1}{x} \circ |x|^2-1 \circ \frac{x}{\epsilon})$ cuando $|x|<|\epsilon|$ y $0$ en el caso contrario, además, como $\epsilon>0$ sabemos que $\frac{x}{\epsilon}\in C^{\infty}(\mathbb{R}^{n})$, luego también sabemos que $|x|^2-1\in C^{\infty}(\mathbb{R}^{n})$ y además sabemos que $\left|\frac{x}{\epsilon}\right|^2-1\neq 0$ ya que $|x|<|\epsilon|$, por lo que sería correcto afirmar que $\frac{1}{\left| \frac{x}{\epsilon} \right|^2-1}\in C^{\infty}(\{x:|x|<|\epsilon|\})$, luego como $e^{x}\in C^{\infty}(\mathbb{R}^{n})$ se puede concluir usando la regla de la cadena que $ \eta_{\epsilon} \in C^{\infty}(\mathbb{R}^{n})$.\\
        Ahora veamos que $supp(\eta_{\epsilon})=\overline{\{x\in\mathbb{R}^{n}:\eta_{\epsilon}(x)\neq 0\}}=\overline{B(0,\epsilon)}$.\\
        Para ver esto, note que:
        \begin{align*}
          \eta_{\epsilon}(x)= 
          \begin{cases}
            \frac{C}{\epsilon}e^{\frac{1}{\left| \frac{x}{\epsilon} \right|^2-1}}, &\text{ Si } \left| \frac{x}{\epsilon} \right|< 1,\\
            0, &\text{ Si } \left| \frac{x}{\epsilon} \right|\geq 1.
          \end{cases}
        \end{align*}
        Luego, si $|x|<|\epsilon|$, se tiene el primer caso, que como $C$ no puede ser $0$ ya que la integral de $\eta$ es $1$ y la exponencial no se anula en el rango de valores dados cuando $|x|<|\epsilon|$, entonces sabemos que el $supp(\eta_{\epsilon})=\overline{B(0,\epsilon)}$. 
      \end{solucion}
      \newpage
    \item Sea $f\in C(\mathbb{R}^{n})$. Utilizando la convolución definimos $f_{\epsilon}=\eta_{\epsilon}*f$, con $\epsilon > 0$.\\
      Muestre que $f_{\epsilon}\in C^{\infty}(\mathbb{R}^{n})$ y que $f_{\epsilon}\rightarrow f$ uniformemente en compactos $K\subset \mathbb{R}^{n}$ cuando $\epsilon\rightarrow 0^+$.
      \begin{solucion}
        Veamos que $f_{\epsilon}\in C^{\infty}(\mathbb{R}^{n})$.
        Para esto note que:
        \begin{align*}
          \frac{\partial f_{\epsilon}}{\partial x_i}(x)&=\frac{\partial }{\partial x_i}(\eta_{\epsilon}*f)(x)\\
          &=\frac{\partial }{\partial x_i}\int_{\mathbb{R}^{n}}\eta_{\epsilon}(x-y)f(y)dy\\
          &=\frac{\partial }{\partial x_i}\int_{B(0,\epsilon)}\frac{C}{\epsilon}e^{\frac{1}{\left| \frac{x-y}{\epsilon} \right|-1}}f(y)dy
        \end{align*}
        Ahora, note que como $f\in C(\mathbb{R}^{n})$, y es de soporte compacto, esta alcanza su máximo y mínimo en $\overline{B(0,\epsilon)}$, por lo que podemos afirmar que $f$ es Riemman Integrable en $\overline{B(0,\epsilon)}$.\\
        Ahora, suponga $K=\overline{B(0,\epsilon)}\cap \overline{B(x,\epsilon)}$ compacto, luego note que $e^{\frac{1}{\left| \frac{x-y}{\epsilon} \right|^2-1}}f(y)\in C^{\infty}(K)$ respecto a $x$, por lo que es válido realizar el siguiente cálculo:
        \begin{align*}
          \frac{\partial f_{\epsilon}}{\partial x_i}(x)&=\frac{\partial }{\partial x_i}\int_{K}\frac{C}{\epsilon}e^{\frac{1}{\left| \frac{x-y}{\epsilon} \right|^2-1}}f(y)dy\\
          &=\int_{K}\frac{\partial }{\partial x_i}\left( \frac{C}{\epsilon} e^{\frac{1}{\left| \frac{x-y}{\epsilon} \right|^2-1}} \right)f(y)dy\\
          &=\int_{K}\frac{\partial \eta_{\epsilon}}{\partial x_i}(x-y)f(y)dy\\
          &=\int_{\mathbb{R}^{n}}\frac{\partial \eta_{\epsilon}}{\partial x_i}(x-y)f(y)dy\\
          &=\left(\frac{\partial \eta_{\epsilon}}{\partial x_i}*f\right)(x)
        \end{align*}
        Luego utilizando un argumento inductivo podemos llegar a que si suponemos $\alpha$ multi-índice, entonces $\partial^{\alpha}f_{\epsilon}=(\partial^{\alpha}\eta_{\epsilon}*f)$ y como $\eta_{\epsilon}\in C^{\infty}(\mathbb{R}^{n})$, entonces podemos concluir que $f_{\epsilon}\in C^{\infty}(\mathbb{R}^{n})$.\\
        \newpage
        Ahora, veamos que $f_{\epsilon}\rightarrow f$ uniformemente en compactos $K\subset \mathbb{R}^{n}$ cuando $\epsilon\rightarrow 0^+$.\\
        Es decir, que dado $e>0$ existe $P>0$ tal que si $p<P$, entonces:
        \begin{align*}
          |f_n(x)-f(x)|<e,
        \end{align*}
        para todo $x\in K$.\\
        Para esto, note que existe $r>0$ tal que $K\subseteq \overline{B(0,r)}$, además como $f\in C(\mathbb{R}^{n})$, entonces es uniformemente continua en $\overline{B(0,r)}$, por lo cuál dado $e>0$ se cumple existe $\delta>0$ tal que si $|x-y|<\delta$, entonces $|f(x)-f(y)|<e$.\\
        Ahora, si tomamos $x\in K\subset \overline{B(0,r)}$ y $p\in (0,\delta)$ (Aquí suponemos $P=\delta$) podemos afirmar que:
        \begin{align*}
          |f_p(x)-f(x)|&\leq \left| \int_{\mathbb{R}^{n}}\eta_p(t)f(x-t)dt-f(x) \right|\\
          &\leq \left| \int_{\mathbb{R}^{n}}\eta_p(t)f(x-t)dt-\int_{\mathbb{R}^{n}}\eta_p(t)f(x)dt \right|\\
          &\leq \left| \int_{\mathbb{R}^{n}}\eta_p(t)(f(x-t)-f(x))dt \right|\\
          &\leq \int_{\mathbb{R}^{n}}\eta_p(t)|f(x-t)-f(x)|dt\\
          &< \int_{\mathbb{R}^{n}}\eta_p(t)e dt\\
          &< e\int_{\mathbb{R}^{n}}\eta_p(t)dt\\
          &< e. 
        \end{align*}
        Por lo que se cumple que $f_{\epsilon}\rightarrow f$ uniformemente en compactos $K\subset \mathbb{R}^{n}$ cuando $\epsilon \rightarrow 0^+$. 
      \end{solucion}
  \end{enumerate}
\end{homeworkProblem}
