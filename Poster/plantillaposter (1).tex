\documentclass[a0,portrait]{a0poster}
\input{configuración}

\begin{document}
\pagestyle{empty}
%\begin{minipage}{15cm}\ \\ \\ \\\includegraphics[width=9cm,height=10cm]{UNAL1.jpg}\end{minipage}
\kern-5cm\begin{minipage}{80cm}\maketitle \vspace*{-5cm}
\end{minipage}
\begin{multicols}{2}

{\fboxsep0.5mm\shadowsize4mm \shadowbox{\fboxsep5mm\colorbox{gris}{\color{black}
\begin{minipage}{.93\columnwidth}
\begin{center}
\normalsize{\color{black}\bf {\Large Resumen}}
\end{center}
Presentaremos resultados referentes al buen planteamiento local y global en $L^2(\mathbb{R})$ del problema de valor inicial

\vspace{-1.5cm}
{\Large\begin{align}
     \left \{ 
    \begin{array}{lcc}
            \partial_t u+\partial_x^5 u-\partial_x^3 u+\mathcal{H} \partial_x^2 u +u\partial_x u=0,\\
             u(0,x)=\phi(x) \in H^{s}(\mathbb{R}), \\
    \end{array}
   \right .
\end{align}}
donde $\mathcal{H}$ representa la transformada de Hilbert, definida mediante la transformada de Fourier como $\widehat{\mathcal{H}\varphi}(\xi)=-isgn(\xi)\widehat{\varphi}(\xi)$, $\xi \in \mathbb{R}$, para toda $\varphi \in \mathcal{S}(\mathbb{R})$. Esta ecuación se puede considerar como una perturbación dispersiva de quinto orden de la ecuación de Benjamin y apareció en un trabajo de Gleeson, Hammerton, Papageorgiou y Vanden-Broeck en el que estudian fenómenos de electrohidrodinámica, en 2007.\\

Los resultados que aquí se presentan hacen parte del trabajo de grado de maestría del estudiante Diego F. Correa C., que tuvo su origen en las sesiones del Semillero de Análisis Armónico y Ecuaciones Diferenciales Parciales de la Universidad Nacional de Colombia, sede Bogotá. Las técnicas usadas combinan una versión simplificada de las ideas introducidas por Bourgain y algunas estimativas establecidas por Kenig, Ponce y Vega en el estudio del problema de valor inicial asociado a la ecuación KdV.


\bigskip
\end{minipage}}}}
\vspace{-1.5cm}
\section{Conceptos y Definiciones}\vspace{-0.5cm}

{\fboxsep0.5mm\shadowsize4mm \shadowbox{\fboxsep5mm\colorbox{rosa}{\color{black}
\begin{minipage}{.93\columnwidth}

Sea $s \in \mathbb{R}$. El espacio de Sobolev $H^s(\mathbb{R})$ se define como $H^s(\mathbb{R}):=\{f \in L^2(\mathbb{R}) \hspace{0.3cm} | \hspace{0.3cm} (1+|\xi|)^{s}\widehat{f}(\xi) \in L^2(\mathbb{R})\}$, donde $\displaystyle{\widehat{f}(\xi):=\int_{-\infty}^\infty f(x)e^{-i\xi x}dx}$. El espacio $H^s$ está dotado con la norma 
\begin{align*}
    ||f||_{H_x^s}=||\langle \cdot \rangle^s\widehat{f}(\cdot)||_{L_\xi^2}=\left(\int_{-\infty}^\infty \langle \xi \rangle^{2s} |\widehat{f}(\xi)|^2d\xi \right)^{\frac{1}{2}}
\end{align*}
donde $\langle \cdot \rangle:=1+|\cdot|$. Note que para el caso particular $s=0$, $H^0(\mathbb{R})=L^2(\mathbb{R})$.\\

\textbf{¿Qué es el buen Planteamiento?} Consideremos el problema de Cauchy 
{\large
\begin{align}
     \left \{ 
    \begin{array}{lcc}
            \partial_t u=F(t,u(t))\\
             u(0)=u_0 \in H^{s}(\mathbb{R}) \\
    \end{array}
   \right .
\end{align}
}
Seguimos la noción de buen planteamiento local en $H^s(\mathbb{R})$ introducida por Kato, la cual consiste
\begin{itemize}
    \item \textit{Existencia y unicidad:} Para todo dato inicial $u_0 \in H^s(\mathbb{R})$ existe un $T>0$ y una única solución $u \in C([0,T]:H^s(\mathbb{R}))$ del problema de Cauchy con dato inicial $u_0$.

    \item \textit{Dependencia continua:} El flujo dato inicial - solución es continua en el siguiente sentido: dado $u_0 \in H^s(\mathbb{R})$ existe una vecindad $V$ de $u_0$ en $H^s(\mathbb{R})$ y $T>0$ tal que la aplicación $v_0 \in V \longmapsto v \in C([0,T]: H^s(\mathbb{R}))$ donde $v$ soluciona el problema de Cauchy (2), con dato inicial $v_0$, es continua.
\end{itemize}

Si lo anterior es válido para cualquier tiempo arbitrario $T > 0$ diremos que el problema de Cauchy (2) está globalmente bien planteado en $H^s(\mathbb{R})$.\\


\textbf{Espacios de Bourgain:} Sean $s,b \in \mathbb{R}$ y $h$ una función de crecimiento lento. \textbf{El espacio de Bourgain (o espacio de Sobolev dispersivo)} asociado a la curva $\tau=h(\xi)$, denotado por $X_{s,b}^{\tau=h(\xi)}=X_{s,b}(\mathbb{R}^2)$ cuando no hay ambigüedad en cuanto a la función $h$, se define como el completado del espacio $\mathcal{S}(\mathbb{R}^2)$ respecto a la norma
\begin{align*}
    ||u||_{X_{s,b}}:=||\langle \tau-h(\xi)\rangle^b\langle \xi \rangle^s\widetilde{u}(\tau,\xi)||_{L_\tau^2 L_\xi^2},
\end{align*}
 donde $u=u(t,x)$ y $\displaystyle{\widetilde{u}(\tau,\xi):=\int_{-\infty}^\infty\int_{-\infty}^\infty u(t,x)e^{-i\tau t}e^{-i\xi x}dtdx.}$\\

Cabe anotar que la Transformada de Hilbert, antes definida, también se puede ver como
\begin{align*}
    (\mathcal{H}f)(x)=v.p.\hspace{0.5cm}\dfrac{1}{\pi} \int_{-\infty}^\infty \dfrac{f(y)}{y-x}dy, \hspace{1cm} f \in H^s(\mathbb{R})
\end{align*}
\vspace{0.3cm}
\bigskip
\end{minipage}}}}
\vspace{-1.5cm}\\

\vspace{-1.4cm}
\section{El problema}\label{s:Newton}

{\fboxsep0.5mm\shadowsize4mm \shadowbox{\fboxsep5mm\colorbox{gris}{\color{black}
\begin{minipage}{.93\columnwidth}

En este caso, el espacio de Bourgain asociado a nuestra ecuación es el asociado a la curva $\tau=h(\xi)=-\xi^5-\xi^3-\xi|\xi|$.\\

Para estudiar (1) usamos su formulación integral asociada
\begin{align}
    u(t)=e^{ith(D)}\phi-\int_0^t e^{i(t-t')h(D)}(uu_x)(t')dt', 
\end{align}
donde $D=-i\partial_x$, $e^{ith(D)}$ es el grupo unitario asociado a la ecuación lineal, dado por $e^{ith(D)}\phi=\{e^{ith(\xi)}\widehat{\phi}(\xi)\}${\huge $\check{}$} para toda $\phi \in H^s(\mathbb{R})$.\\


\textbf{Estimativas para la demostración del resultado}
\begin{itemize}
    \item Sean $s \in \mathbb{R}$, $a,b \in (0,\frac{1}{2})$ con $a<b$ y $0<\delta<1$. Entonces para $F \in X_{s,-a}$ se tiene
    \begin{align*}
        ||\psi(\delta^{-1}t)F||_{X_{s,-b}} \lesssim \delta^{(b-a)/4(1-a))}||F||_{X_{s,-a}}
    \end{align*}

    \item Sean $s \geq 0$, $b \in (\frac{1}{2},\frac{3}{4}]$, $u\in X_{s,b}$ y $\phi \in H^s(\mathbb{R})$. Entonces para todo $\delta \in (0,1)$,
    
    \begin{multicols}{2}
    {\small
        \begin{align*}
            ||\psi(\delta^{-1}t)e^{ith(D)}\phi||_{X_{s,b}} \lesssim \delta^{(1-2b)/2}||\phi||_{H^s}
        \end{align*}
        \begin{align*}
            ||\psi(\delta^{-1}t)u||_{X_{s,b}} \lesssim \delta^{(1-2b)/2}||u||_{X_{s,b}}
        \end{align*}
        \begin{align*}
          \left|\left|\psi(\delta^{-1}t)\int_0^t e^{i(t-t')h(D)}u(t')dt'\right|\right|_{X_{s,b}} \lesssim \delta^{(1-2b)/2}||u||_{X_{s,b-1}}
        \end{align*}
        \begin{align*}
          \sup_{t\in (0,T)} \left|\left| \psi(\delta^{-1}t)\int_0^te^{i(t-t')h(D)}u(t')dt'\right|\right|_{H_x^s} \lesssim \delta^{(1-2b)/2}||u||_{X_{s,b-1}}
        \end{align*}
        }
    \end{multicols}
    \item \textbf{Estimativa Bilineal.} Sea $b\in (\frac{1}{2},\frac{3}{4})$. Si $u,v \in X_{0,b}$, entonces $\partial_x(uv) \in X_{0,b-1}$. Específicamente,
    \begin{align}
        ||\partial_x(uv)||_{X_{0,b-1}} \lesssim ||u||_{X_{0,b}}||v||_{X_{0,b}}.
    \end{align}
\end{itemize}

\bigskip
\end{minipage}}}}\\
\vspace{-1.5cm}

{\fboxsep0.5mm\shadowsize4mm \shadowbox{\fboxsep5mm\colorbox{gris}{\color{black}
\begin{minipage}{.93\columnwidth}
\begin{tcolorbox}
\textbf{Teorema 1}  Si $\phi \in L^2(\mathbb{R})$ y $b \in (\frac{1}{2},\frac{3}{4})$, entonces existe un tiempo $T = T(||\phi||_{L^2})>0$ y una única solución de la ecuación integral asociada al problema (1) tal que 
\begin{align*}
   \bullet \hspace{5} u \in C([0,T]:L^2(\mathbb{R}))\hspace{50}  \bullet \hspace{5} u \in X_{0,b} \hspace{50} \bullet \hspace{5} u u_x \in X_{0,b-1}
\end{align*}
Más aún, para todo $T' \in (0,T)$ la aplicación dato-solución es continua de $L^2(\mathbb{R})$ en $C([0,T]:L^2(\mathbb{R}))$.
\end{tcolorbox}

Usamos el principio de contracción de Banach. Definimos el operador
\begin{align*}
    \Phi_{\phi}(u)=\psi(t)e^{ith(D)}\phi-\psi(t)\int_0^t \psi(\delta^{-1}t')e^{i(t-t')h(D)}(uu_x)(t')dt', \hspace{1cm} \phi \in L^2(\mathbb{R})
\end{align*}
y el conjunto $\mathscr{X}_a:=\{u \in X_{0,b} \hspace{5}: \hspace{5}||u||_{X_{0,b}}\leq a\}$, donde $a:=2C||\phi||_{L^2(\mathbb{R})}$, $\delta>0$ y $\psi \in C_0^\infty(\mathbb{R})$ tal que $\phi \equiv 1$ en el intervalo $[0,\frac{1}{2}]$ y $\psi \equiv 0$ en $\mathbb{R}\setminus[-1,1]$.\\

Haciendo $\delta=(4aC)^{-4b'/(b'-b)}$, con $b' \in (b,\frac{3}{4})$, dada $u \in \mathscr{X}_a$
\begin{align*}
    ||\Phi_\phi(u)||_{X_{0,b}} &\leq ||\psi(t)e^{ith(D)}\phi||_{X_{0,b}}+\left|\left| \psi(t)\int_0^t \psi(\delta^{-1}t')e^{i(t-t')h(D)}(uu_x)(t')dt'\right|\right|_{X_{0,b}}\\
    &\leq C||\phi||_{L^2(\mathbb{R})}+C||\psi(\delta^{-1}t)(uu_x)||_{X_{0,b-1}}\\
    &\leq C||\phi||_{L^2(\mathbb{R})}+C\delta^{\frac{1-b-(1-b')}{4(1-(1-b'))}}||uu_x||_{X_{0,b'-1}}=C||\phi||_{L^2(\mathbb{R})}+C\delta^{(b'-b)/4b'}||uu_x||_{X_{0,b'-1}}\\
    &\leq C||\phi||_{L^2(\mathbb{R})}+C\delta^{(b'-b)/4b'}||u||_{X_{0,b}}^2 \leq C||\phi||_{L^2(\mathbb{R})}+C\delta^{(b'-b)/4b'}a^2\\
    &\leq \frac{a}{2}+\frac{a}{4} \leq \frac{a}{2}+\frac{a}{2}=a
\end{align*}

Luego, la aplicación $\Phi_\phi: \mathscr{X}_a \longrightarrow \mathscr{X}_a$ está bien definida. Sean $u,v \in \mathscr{X}_a$
\begin{align*}
    ||\Phi_\phi(u)-\Phi_\phi(v)||_{X_{0,b}} &\leq C\left|\left|\psi(t)\int_0^t \psi(\delta^{-1}t')e^{i(t-t')h(D)}\partial_x(u^2-v^2)(t')dt'\right|\right|_{X_{0,b}}\\
    &\leq C||\psi(\delta^{-1}t)\partial_x[(u+v)(u-v)]||_{X_{0,b-1}}\\
    &\leq C \delta^{(b-b')/4b'}||u+v||_{X_{0,b}}||u-v||_{X_{0,b}}\\
    &\leq 2Ca\delta^{(b'-b)/4b'}||u-v||_{X_{0,b}}\leq \frac{1}{2}||u-v||_{X_{0,b}}.\\
\end{align*}

\vspace{-1.5cm}

Así, la aplicación $u \longmapsto \Phi_\phi(u)$ es una contracción en $\mathscr{X}_a$ y por lo tanto, existe un único punto fijo de esta aplicación. De esta manera, se tiene la existencia de solución de (1).


\vspace{-0.3cm}
\bigskip
\end{minipage}}}}

\vspace{-2cm}

\section{Resultados}\label{s:examples}\vspace{-0.5cm}

{\fboxsep0.5mm\shadowsize4mm \shadowbox{\fboxsep5mm\colorbox{rosa}{\color{black}
\begin{minipage}{.93\columnwidth}

Haciendo uso de las estimativas, se encontró buen planteamiento local en $H^0(\mathbb{R})=L^2(\mathbb{R})$ para el problema de valor inicial (1).\\

\vspace{-0.5cm}
\begin{tcolorbox}
\textbf{Corolario}  Para cualquier $s\geq 0$ y $u_0 \in H^s(\mathbb{R})$, las conclusiones del Teorema 1 son válidas para $u \in C([0,T]:H^s(\mathbb{R}))$.
\end{tcolorbox}

Como $\displaystyle{I(u)=\int_{-\infty}^\infty u^2(x,t)dx}$ es una cantidad conservada establecida para soluciones suaves de la ecuación en (1), entonces las soluciones de (3) también la satisfacen. Esto implica el siguiente Teorema.
\begin{tcolorbox}
    \textbf{Teorema 2} Las soluciones obtenidas en el Teorema 1 pueden ser extendidas para cualquier $T>0$.
\end{tcolorbox}

\vspace{-0.9cm}
\bigskip
\end{minipage}}}}\\
\vspace{-1.5cm}

\vspace{-1.3cm}
%%%%%%%%%%%%%%%%%%%%%%%%%%%%%%%%%%%%%%%%%%%%%%%%%%%%%%%%%%%%%%%%%%%%%%%%%%%%%%%%%%%%%%%%%%%%%%%%%%%%%%%%%%%%%%%%%%%%

\section{Trabajo Futuro}

{\fboxsep0.5mm\shadowsize4mm \shadowbox{\fboxsep5mm\colorbox{gris}{\color{black}
\begin{minipage}{.93\columnwidth}

En un futuro cercano, se espera estudiar el buen planteamiento del problema (1) en ciertos espacios definidos por: dados $\sigma >0$ y $s \in \mathbb{R}$, se define el espacio $G^{\sigma,s}$ como el subespacio de $L^2(\mathbb{R})$ definido por la norma
\begin{align*}
    ||f||_{G^{\sigma,s}}^2=\int_{-\infty}^\infty \langle\xi\rangle^{2s}e^{2\sigma\langle \xi \rangle}|\widehat{f}(\xi)|^2d\xi
\end{align*}
el objetivo es encontrar $\sigma>0$ y $s \in \mathbb{R}$ tales que el problema (1) esté bien planteado en el espacio $G^{\sigma,s}$.Luego, con esta información buscaremos cotas inferiores para el radio de analiticidad de las soluciones de la ecuación, todo esto motivado por trabajos anteriores realizados para la ecuación KdV. La relación con el presente trabajo viene dada por el hecho de que, una herramienta fundamental para estos resultados es estudiar el espacio de restricción de la transformada dado por
\begin{align*}
    ||f||_{X_{s,b,\sigma}}=||\langle \xi \rangle^{s}\langle \tau-h(\xi) \rangle^b e^{\sigma\langle \xi \rangle}\widetilde{f}(\tau,\xi)||_{L_\xi^2L_\tau^2}
\end{align*}
donde $h(\xi)=-\xi^5-\xi^3-\xi|\xi|$.

\vspace{-0.3cm}
\bigskip
\end{minipage}}}}
\vspace{-1.5cm}

\section{Agradecimientos}

{\fboxsep0.5mm\shadowsize4mm \shadowbox{\fboxsep5mm\colorbox{rosa}{\color{black}
\begin{minipage}{.93\columnwidth}

Agradecimiento al Semillero de Análisis Armónico y Ecuaciones Diferenciales Parciales de la Universidad Nacional de Colombia - Sede Bogotá, especialmente a los docentes Ricardo Ariel Pastrán Ramírez, Omar Duque Gomez y Oscar Guillermo Riaño Castañeda, quienes orientaron el proceso de desarrollo de este póster, tanto en la parte teórica como en la presentación del mismo.

\vspace{-0.3cm}
\bigskip
\end{minipage}}}}
\vspace{-1.5cm}


\begin{thebibliography}{4}
{\fboxsep0.5mm\shadowsize4mm \shadowbox{\fboxsep12mm\colorbox{Fondo}{\color{black}
\begin{minipage}{.87\columnwidth}

\bibitem{K1} Kenig, Ponce y Vega,\textit{ The Cauchy problem for the Korteweg de Vries equation in Sobolev spaces of negative indices}, Duke Math. J. 71 (1993), 1 - 21. MR
94g:35196.

\bibitem{K2} Kenig, Ponce y Vega, \textit{A bilinear estimate with applications to the KdV equation}, Journal of the american mathematical society, Volume 9, Number 2, April 1996.

\bibitem{F1} Gleeson H., Hammerton P., Papageorgiou D. T., Vanden-Broeck J.-M., \textit{A new application of the Korteweg–de Vries Benjamin-Ono equation in interfacial electrohydrodynamics}, American Institute of Physics, 2007.

\bibitem{TK1} T. Kato, K. Masuda. \textit{Nonlinear evolution equations and analyticity}. I. Ann. Inst. H. Poincaré Anal. Non Lin ́eaire, 3(6):455–467, 1986.

\bibitem{I1} Iorio, Jr R. J. y Valéria de Magalhaes Iorio, \textit{Fourier Analysis and Partial Differential Equations}, Cambridge University Press, 2001.

\vspace{-0.7cm}
\bigskip
\end{minipage}}}}
\end{thebibliography}

%\textcolor{Cyan}{
%\normalsize H. Fabian Ramírez (hectorfabian.ramirez@um.es) trabajo conjunto con Pascual Lucas (plucas@um.es) \normalsize Departamento de Matemáticas, Universidad de Murcia. Partially supported by MINECO and FEDER project MTM2012-34037.}
\end{multicols}
\end{document}
