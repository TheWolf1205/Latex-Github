\usepackage[spanish]{babel}
\usepackage[utf8]{inputenc}
\usepackage[T1]{fontenc}
\usepackage{times}
\usepackage{multicol}
\usepackage[pdftex]{graphicx}
\usepackage[pdftex,dvipsnames,usenames]{xcolor}
\usepackage{epic}%cruces
%\renewcommand{\familydefault}{\sfdefault}
%\usepackage{colortbl}
%\usepackage[notcite,notref]{showkeys}
\usepackage[utf8]{inputenc}
\usepackage{amsfonts,amsmath,amstext,amssymb}
\usepackage{theorem,enumerate}
\usepackage{physics}
\usepackage{mathrsfs}
\usepackage{multirow}
\usepackage{hyperref}
\usepackage{cleveref}
\usepackage{csquotes}
\usepackage{verbatim}
\usepackage{float}
\usepackage{fancyhdr}

\usepackage{graphicx} % figuras
\usepackage{subfigure}%subfiguras
\usepackage{fancybox}
\usepackage{bibunits}
\usepackage[listings]{tcolorbox}
\advance\textwidth35mm %margen derecha
\advance\hoffset-15mm %margen izqierda
\advance\textheight11cm %margen abajo
\advance\voffset-40mm %margen arriba
\parskip5pt plus1pt
\def\<{\left<}
\def\>{\right>}
\def\paren#1{\left(#1\right)}
\let\fle\to
\def\sen{\text{sin}}
\def\senh{\text{sinh}}
\def\arccosh{\text{arccosh}}
\def\longi{\text{Length}}
\def\e{\varepsilon}
\def\eu{\e_1}
\def\ed{\e_2}
\def\et{\e_3}
\def\diag{\text{\rm diag}}
\def\sg{\text{sign}}
\def\R{\mathbb{R}}
\def\C{\mathbb{C}}
\def\d{\text{\rm d}}
\def\L{\mathbb{L}}
\def\S{\mathbb{S}}
\def\H{\mathbb{H}}
\def\tr{\text{\rm tr}}
\def\det{\text{\rm det}}
\def\cm{{\mathcal C}^\infty(M^n_s)}
\let\ds\displaystyle
\def\div{\text{\rm div}}
%\newcommand{\U}[1][k+1]{\mathcal{U}_{#1}}
\def\card{\text{\rm card}}
\newcommand{\ol}{\overline}
\newcommand{\bin}[1]{\mbox{$\binom{n}{#1}$}}
\newcommand{\binm}[2]{\mbox{$\binom{#1}{#2}$}}
\def\hdashline#1#2{%
	\setlength{\unitlength}{1mm}
	\begin{picture}(0,0)
		\dashline[50]{2}(#1,0)(#2,0)
\end{picture}}
\def\vdashline#1#2{%
	\setlength{\unitlength}{1mm}
	\begin{picture}(0,0)
		\dashline[50]{2}(0,#1)(0,#2)
\end{picture}}
\def\Dashcruz#1#2#3#4{\setlength{\unitlength}{1mm}%
	\begin{picture}(0,0)
		\dashline[50]{2}(#1,0)(#2,0)
		\dashline[50]{2}(0,#3)(0,#4)
\end{picture}}
\def\dashcruz#1#2#3#4{\relax}

\def\MatCero{\text{\large\bfseries 0}}
\def\ColSep{\kern2pt}
\def\Sep#1{\kern#1pt}
%\def\VSep{5pt}
\def\VSep{6pt}
\def\VSSep{3pt}
\def\VVSep{0.5cm}
\def\VVVSep{0.3cm}

\theorembodyfont{\slshape}
\newtheorem{theorem}{\color{White}Theorem}
\newtheorem{proposition}[theorem]{\color{white}Proposición}
\newtheorem{corollary}[theorem]{\color{Blue}Corolario}
\newtheorem{lemma}[theorem]{\color{Blue}Lemma} {\theorembodyfont{\slshape}
	\newtheorem{definition}{\color{Blue}Definición}%\colorbox[cmyk]{0,0,1,0}
	\newtheorem{conjecture}{\color{Blue}Conjetura}
	\newtheorem{example}{\color{Black}Example}
	\newtheorem{remark}{\color{Blue}Nota}
	\newtheorem{problem}{\color{Blue}Problema}
}
\definecolor{MiColorw}{rgb}{.50,.90,.70}
\definecolor{MiColor}{rgb}{.90,.90,.90}
\definecolor{suave}{rgb}{.92,.92,1}
\definecolor{blue}{RGB}{0,12,55}
\definecolor{Gray}{RGB}{66,66,66}
\definecolor{Fondo}{RGB}{236,235,235}
\definecolor{gris}{RGB}{230,243,242}
\definecolor{rosa}{RGB}{255,236,232}
\definecolor{red}{RGB}{207,76,49}
\definecolor{moradoclaro}{RGB}{221, 160, 221}
\definecolor{verdeagua}{RGB}{173, 216, 230}
\definecolor{azulcielo}{RGB}{ 135, 206, 250}
\definecolor{anita1}{HTML}{A62522}
\definecolor{anita2}{HTML}{80221E}
\definecolor{anita3}{HTML}{AD7C59}
\definecolor{anita4}{HTML}{CABCAB}
\definecolor{anita5}{HTML}{D4B4A1}
\definecolor{anita6}{HTML}{B85C48}
\pagecolor{anita4}

\makeatletter
\renewcommand\section{\@startsection {section}{1}{\z@}%
	{-3.5ex \@plus -1ex \@minus -.2ex}%
	{2.3ex \@plus.2ex}%
	{\normalfont\LARGE\bfseries\textcolor{black}}}
\def\thesection{\arabic{section}}
\def\thesubsection{\thesection.\arabic{subsection}}
\renewcommand\subsection{\@startsection{subsection}{2}{\z@}%
	{-3.25ex\@plus -1ex \@minus -.2ex}%
	{1.5ex \@plus .2ex}%
	{\normalfont\Large\bfseries\textcolor{Green}}}
%\renewcommand\subsubsection{\@startsection{subsubsection}{3}{\z@}%
	%                                    {-3.25ex\@plus -1ex \@minus -.2ex}%
	%                                   {1.5ex \@plus .2ex}%
	%                                   {\normalfont\normalsize\bfseries\color{Green}}}
\renewcommand\paragraph{\@startsection{paragraph}{4}{\z@}%
	{3.25ex \@plus1ex \@minus.2ex}%
	{-1em}%
	{\normalfont\normalsize\bfseries\color{Green}}}
\renewcommand\subparagraph{\@startsection{subparagraph}{5}{\parindent}%
	{3.25ex \@plus1ex \@minus .2ex}%
	{-1em}%
	{\normalfont\normalsize\bfseries\color{Green}}}
\makeatother
\newenvironment{proof}
{\par\noindent\textbf{Proof.}\quad}
{\hfill$\blacksquare$\par}
\newenvironment{proof*}
{\par\noindent\textbf{Proof.}\quad}
{}

\widowpenalty 10000 \clubpenalty 10000 \skip\footins=2\baselineskip
\def\baselinestretch{1.05}
\parindent0pt

\def\thepage{}

\renewcommand{\footnoterule}{\textcolor{blue}{\rule{0.587\columnwidth}{0.02in}}\vspace*{5mm}}
\title{\color{White}\bfseries {\huge{Sobre la función máximal de Hardy-Littlewood y un vistazo al trabajo futuro}}\\\
	\normalsize\textcolor{White}{\huge{Andrés David Cadena Simons}} \\
	\textcolor{White}{\large{Trabajo realizado bajo la dirección del profesor Oscar Guillermo Riaño Castañeda y Ricardo Ariel Pastrán Ramírez.}} \\\
	\normalsize \textcolor{White}{\large{Semillero de Análisis Armónico y Ecuaciones Diferenciales Parciales, Departamento de Matemáticas, Universidad Nacional de Colombia}}\\\
	\textcolor{White}{\large{01 de diciembre de 2023}}\\\    
}
\author{}
\date{}
\advance
\pdfpageheight8cm
\columnsep=2cm %Espacio entre las columnas
