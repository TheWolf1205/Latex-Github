\documentclass{beamer}
\usepackage{amsmath, amssymb, graphicx}
\usepackage{tikz}
\usepackage{hyperref}

\title{Aplicaci\'on de los Polinomios de Zernike en la Resoluci\'on de EDPs}
\author{Expositor: [Tu Nombre]}
\date{[Fecha de la exposición]}

\begin{document}

\begin{frame}
    \titlepage
\end{frame}

\begin{frame}{Introducción}
    \begin{itemize}
        \item ¿Por qué resolver EDPs en regiones circulares?
        \item Aplicaciones en óptica, fluidos, electromagnetismo.
        \item Métodos tradicionales y sus limitaciones.
        \item Se propone el uso de los polinomios de Zernike, que son ortogonales en el disco unitario y permiten una mejor representación de soluciones en regiones circulares.
    \end{itemize}
\end{frame}

\begin{frame}{Polinomios de Zernike}
    \begin{itemize}
        \item Definición y propiedades.
        \item Expansión en términos de \textbf{parte radial} y \textbf{parte azimutal}.
        \item Comparación con bases ortogonales en el disco unitario.
        \item \textbf{Animación}: Expansión de una función con polinomios de Zernike.
    \end{itemize}
    \centering
    \includegraphics[width=0.7\textwidth]{zernike_expansion.png} % Incluir imagen de la animación
\end{frame}

\begin{frame}{Resolución de EDPs de Primer Orden}
    \begin{itemize}
        \item Forma general: $\alpha(x, y) \frac{\partial u}{\partial x} + \beta(x, y) \frac{\partial u}{\partial y} + \gamma(x, y) u = f(x, y)$.
        \item Transformación a coordenadas polares.
        \item Uso de matrices operacionales de integración.
        \item Comparación con métodos tradicionales.
    \end{itemize}
\end{frame}

\begin{frame}{Resolución de EDPs de Segundo Orden}
    \begin{itemize}
        \item Caso especial: ecuación de Laplace en el disco unitario.
        \item Transformación a un sistema de ecuaciones lineales $Ax = b$.
        \item Comparación de métodos $l_1$ y $l_2$.
        \item Estudio de errores y convergencia.
        \item \textbf{Animación}: Solución numérica de una EDP con polinomios de Zernike.
    \end{itemize}
    \centering
    \includegraphics[width=0.7\textwidth]{zernike_edp_solution.png} % Incluir imagen de la animación
\end{frame}

\begin{frame}{Conclusiones}
    \begin{itemize}
        \item Ventajas del método: manejo de discontinuidades, convergencia rápida.
        \item Desafíos computacionales y posibles mejoras.
        \item Aplicaciones futuras en modelado numérico.
    \end{itemize}
\end{frame}

\end{document}

