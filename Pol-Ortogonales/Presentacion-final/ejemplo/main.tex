\documentclass{article}
\usepackage{amsmath, amssymb, amsfonts}
\usepackage{graphicx}
\usepackage{geometry}
\geometry{a4paper, margin=1in}

\begin{document}

\title{Solución de Ecuaciones Diferenciales Parciales de Segundo Orden con Polinomios de Zernike}
\author{}
\date{}
\maketitle

\section{Método de solución}

Consideramos la ecuación diferencial parcial de segundo orden en coordenadas cartesianas:
\begin{equation}
    a \frac{\partial^2 u}{\partial x^2} + b \frac{\partial^2 u}{\partial x \partial y} + c \frac{\partial^2 u}{\partial y^2} + a_1 \frac{\partial u}{\partial x} + a_2 \frac{\partial u}{\partial y} + a_0 u = f(x, y).
\end{equation}
Para problemas en regiones circulares, se usa una forma rotacionalmente invariante:
\begin{equation}
    \Delta u + \alpha \left( x \frac{\partial}{\partial x} + y \frac{\partial}{\partial y} \right)^2 u + \beta \left( x \frac{\partial}{\partial x} + y \frac{\partial}{\partial y} \right) u + \gamma u = f.
\end{equation}
En coordenadas polares $ (r, \phi) $ la ecuación se reescribe como:
\begin{equation}
    (1 + \alpha r^2) \frac{\partial^2 u}{\partial r^2} + \left( \frac{1}{r} + (\alpha + \beta) r \right) \frac{\partial u}{\partial r} + \frac{1}{r^2} \frac{\partial^2 u}{\partial \phi^2} + \gamma u = f.
\end{equation}

\subsection{Transformación a un sistema lineal}

Se usa una base de polinomios de Zernike $ R_n^m(r) e^{im\phi} $ para aproximar $ u(r,\phi) $. Luego, se integran las ecuaciones dos veces respecto a $ r $ y dos veces respecto a $ \phi $, lo que permite expresar la ecuación en términos de matrices operacionales $ E $, transformando la ecuación en un sistema lineal:
\begin{equation}
    A x = b,
\end{equation}
donde $ A $ es una matriz dispersa de tamaño $ MN \times MN $, $ x $ es el vector de coeficientes de la solución y $ b $ representa los términos forzantes y las condiciones de frontera.

\section{Ejemplo: Ecuación de Laplace}

Consideremos la ecuación de Laplace:
\begin{equation}
    r^2 \frac{\partial^2 u}{\partial r^2} + r \frac{\partial u}{\partial r} + \frac{\partial^2 u}{\partial \phi^2} = 0.
\end{equation}

Supongamos que la condición de frontera en $ r = r_0 $ es $ u(r_0, \phi) = g(\phi) $ y el valor inicial $ u(0,\phi) = 1 $. Expandiendo $ u(r,\phi) $ en polinomios de Zernike hasta grado 3:
\begin{equation}
    u(r,\phi) \approx 1 + r \cos\phi + \frac{1}{4}r^2(1 + \cos 2\phi) - \frac{3}{2}r^3 \cos\phi.
\end{equation}

El sistema $ A x = b $ se resuelve con dos métodos:
\begin{itemize}
    \item \textbf{Mínimos cuadrados ($\ell_2$)} usando la pseudo-inversa de Moore-Penrose.
    \item \textbf{Minimización $\ell_1$}, obteniendo mejor precisión.
\end{itemize}

La solución por minimización $ \ell_1 $ es:
\begin{equation}
    u(r,\phi) = 1 + \frac{1}{4} r^2 (1 + \cos 2\phi) - \frac{3}{2} r^3 \cos\phi.
\end{equation}

Este método muestra que la representación en términos de polinomios de Zernike es eficaz para resolver ecuaciones diferenciales parciales en dominios circulares.

\end{document}

