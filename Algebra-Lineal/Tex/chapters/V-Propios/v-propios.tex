\section{Valores y vectores propios.}
A partir de este momento solo hablaremos de matrices cuadradas $n\times x$. Recordemos que si tomamos una matriz $A$ podemos definir $L(\x)=A\x$ con $\x\in\mathbb{R}^{n}$ como una transformación lineal. Una buena pregunta será ver si existen vectores $\x$ tales que $A\x$ y $\x$ son paralelos, esto nace de varias aplicaciones en distintas áreas del saber.
\begin{definition}
  Sea $A$ una matriz $n\times n$. El número real $\lambda$ es un \textbf{valor propio} (o \textbf{valor característico o autovalor}) de $A$ si existe un vector $\x$ distinto de $0$ en $\mathbb{R}^{n}$ tal que
  \begin{equation}\label{eq:5-valor-propio}
    A\x=\lambda\x.
  \end{equation}
  A su vez todo vector que cumpla $(\ref{eq:5-valor-propio})$ es un \textbf{vector propio} asociado al valor propio $\lambda$, su nombre puede variar de la misma forma que los valores propios. 
\end{definition}
\begin{note}
  En otros contextos $\lambda$ y $\x$ se pueden tomar tanto reales como complejos. 
\end{note}
\begin{note}
  Note que un mismo valor propio de $A$ puede tener asociados muchos vectores propios, pues si suponemos que $\x$ es un vector propio asociado a $\lambda$, entonces 
  \begin{align*}
    A(r\x)&=rA\x,\\
    &=r\lambda\x,\\
    &=\lambda(r\x).
  \end{align*}
\end{note}
\begin{example}
  Sea
  \begin{align*}
    A = \begin{pmatrix}
      0 & 0 \\
      0 & 1
    \end{pmatrix}
    .
  \end{align*}
  Note que
  \begin{align*}
    A \begin{pmatrix}
      1 \\
      0
    \end{pmatrix}
    &=\begin{pmatrix}
      0 & 0 \\
      0 & 1
    \end{pmatrix}
    \begin{pmatrix}
      1 \\
      0
    \end{pmatrix},\\
    &=\begin{pmatrix}
      0 \\
      0
    \end{pmatrix}
    ,\\
    &=0 \begin{pmatrix}
      1\\
      0
    \end{pmatrix}
    .
  \end{align*}
  De lo que podemos ver que $\lambda=0$ es un valor propio de $A$ (verifique esto con la definición). 
\end{example}
A groso modo, note que cuando queremos hallar los valores propios de una matriz $A$ en general nos cruzamos con una ecuación como la siguiente.
\begin{align*}
  \begin{pmatrix}
    a & b \\
    c & d
  \end{pmatrix}
  \begin{pmatrix}
    x_1 \\
    x_2
  \end{pmatrix}
  =\lambda \begin{pmatrix}
    x_1 \\
    x_2
  \end{pmatrix}
  .
\end{align*}
Lo que en general se puede ver como
\begin{align*}
  \begin{cases}
    ax_1+bx_2=\lambda x_1, \\
    cx_1+dx_2=\lambda x_2.
  \end{cases}
\end{align*}
lo cual se puede llevar a
\begin{align*}
  \begin{cases}
    (a-\lambda)x_1+bx_2=0, \\
    cx_1+(d-\lambda)x_2=0.
  \end{cases}
\end{align*}
Luego como esto es un sistema homogéneo de 2 ecuaciónes con 2 incógnitas podemos afirmar que este tiene solución no trivial si y sólo si el determinante respectivo es distinto de $0$, es decir
\begin{align*}
  \begin{bmatrix}
    a-\lambda & b \\
    c & d-\lambda
  \end{bmatrix}\neq 0.
\end{align*}
Lo que de puede escribir como $det(A-\lambda I)$, vamos a extender esto en general para cualquier matriz $A$ $n\times n$.
\begin{definition}
  Sea $A$ una matriz $n\times n$, definimos $f(\lambda)=det(A-\lambda I)$ como el \textbf{polinomio característico} de $A$.\\
  Así mismo definiremos $f(\lambda)=0$ como la \textbf{ecuación característica} de $A$. 
\end{definition}
\begin{theorem}
  Una matriz $n\times n$ es singular si y sólo si $0$ es un valor propio. 
\end{theorem}
