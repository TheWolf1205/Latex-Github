El Álgebra Lineal es una de las ramas fundamentales de las matemáticas, con aplicaciones en múltiples disciplinas, desde la física y la computación hasta la economía y la ingeniería. Estas notas han sido elaboradas con el propósito de servir como una herramienta de estudio, proporcionando una exposición estructurada y accesible de los conceptos esenciales de la materia.

El contenido de estas notas se basa en la obra Álgebra Lineal de Bernard Kolman y David R. Hill \citep{kolman2006algebra}, un texto ampliamente reconocido en el ámbito académico por su claridad y profundidad en la exposición de los temas. No obstante, este documento no pretende reemplazar dicho material, sino complementarlo, ofreciendo una presentación que pueda servir de apoyo para el aprendizaje y la comprensión de los conceptos fundamentales del Álgebra Lineal.

Se espera que este material sea útil para estudiantes que inician su estudio del Álgebra Lineal, así como para aquellos que desean reforzar sus conocimientos en la materia. Además, el documento está estructurado de manera que permita su expansión y adaptación, incorporando ejemplos adicionales, demostraciones y ejercicios según las necesidades del lector.
