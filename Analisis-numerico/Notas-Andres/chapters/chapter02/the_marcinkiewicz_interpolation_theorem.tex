The following theorem will allow us to deduce in which spaces our operator is $(p,p)$-strong just by analyzing in which other spaces it is $(p,p)$-weak.
\begin{theorem}{}
  Let $(X, \mu)$ and $(Y, \nu)$ be measure spaces, $1 \leq p_0 < p_1 \leq \infty$, and let $T$ be a sublinear operator from $L^{p_0}(X, \mu) + L^{p_1}(X, \mu)$ to measurable functions on $Y$ that is weakly $\left(p_0, p_0\right)$ and weakly $\left(p_1, p_1\right)$. Then $T$ is strongly $(p, p)$ for $p_0 < p < p_1$.
\end{theorem}
\begin{proof}{}
  Let $f \in L^p$. For each $\lambda$, we decompose $f = f_0 + f_1$:
  \begin{itemize}
    \item $f_0 = f \mathcal{X}_{\{ x \in X: |f(x)| > c\lambda \}}$\\
    \item $f_1 = f \mathcal{X}_{\{ x \in X: |f(x)| \leq c\lambda \}}$
  \end{itemize}
  where $c$ is a fixed constant.
  Then $f_0 \in L^{p_0}$:
  \begin{align*}
    \int_X |f_0(x)|^{p_0} d\mu & = \int_{\{ x \in X: |f(x)| > c\lambda \}} |f(x)|^{p_0} d\mu\\
    & < \int_{\{ x \in X: |f(x)| > c\lambda \}} (c\lambda)^{p_0 - p} |f(x)|^{p} d\mu\\
    & \leq (c\lambda)^{p_0 - p} ||f(x)||_p^p
  \end{align*}   
  And $f_1 \in L^{p_1}$:
  \begin{align*}
    \int_X |f_1(x)|^{p_1} d\mu & = \int_{\{ x \in X: |f(x)| \leq c\lambda \}} |f(x)|^{p_1} d\mu\\
    & < \int_{\{ x \in X: |f(x)| \leq c\lambda \}} (c\lambda)^{p_1 - p} |f(x)|^{p} d\mu\\
    & \leq (c\lambda)^{p_1 - p} ||f(x)||_p^p
  \end{align*}
  Furthermore, since $T$ is sublinear, we have:
  \begin{align*}
    \lambda & < |Tf(x)| \\
    & \leq |Tf_0(x)| + |Tf_1(x)| \leq \begin{cases}
      2|Tf_0(x)| & \text{if } x \in \{|Tf_1(x)| \leq |Tf_0(x)|\} \\
      2|Tf_1(x)| & \text{if } x \in \{|Tf_0(x)| \leq |Tf_1(x)|\}
    \end{cases}
  \end{align*}
  Then:
  $$\mu(\{x \in X: |Tf(x)| > \lambda\}) \leq \mu(\{x \in X: |Tf_0(x)| > \frac{\lambda}{2}\}) + \mu(\{x \in X: |Tf_1(x)| > \frac{\lambda}{2}\})$$
  Thus:
  $$a_Tf(\lambda) \leq a_{Tf_0}\left(\frac{\lambda}{2}\right) + a_{Tf_1}\left(\frac{\lambda}{2}\right)$$
  For the case $p_1 = \infty$, consider $c = \frac{1}{2A_1}$ with $A_1$ such that $||Tg(x)||_{\infty} \leq A_1 ||g||_{\infty}$, so:
  \begin{align*}
    |Tf_1(x)| \leq A_1 ||f||_{\infty} \leq A_1 c\lambda - \frac{\lambda}{2}
  \end{align*}
  Almost everywhere, therefore:
  $$a_{Tf_1}\left(\frac{\lambda}{2}\right) = 0$$
  Using the weak inequality $\left(p_0, p_0\right)$-weak: \\
  $$a_{Tf_0}\left(\frac{\lambda}{2}\right) \leq\left(\frac{2 A_0}{\lambda}\left\|f_0\right\|_{p_0}\right)^{p_0} \quad (f_0 \in L^{p_0})$$  
\end{proof}
The following theorem will allow us to deduce in which spaces our operator is $(p,p)$-strong just by analyzing in which others it is $(p,p)$-weak.
\begin{theorem}{The Marcinkiewicz Interpolation Theorem (Diagonal Case)}\label{theorem:marcienkiewickz-theorem}
  Let $(X, \mu)$ and $(Y, \nu)$ be measure spaces, $1 \leq p_0 < p_1 \leq \infty$, and let $T$ be a sublinear operator from $L^{p_0}(X, \mu) + L^{p_1}(X, \mu)$ to measurable functions on $Y$ that is weak $\left(p_0, p_0\right)$ and is weak $\left(p_1, p_1\right)$. Then $T$ is strong $(p, p)$ for $p_0 < p < p_1$.
\end{theorem}
\begin{proof} 
  Let $f \in L^p$, for each $\lambda$ we decompose $f = f_0 + f_1$\\
  \begin{itemize}
    \item $f_0 = f \mathcal{X}_{\{ x \in X: |f(x)| > c\lambda \}}$\\
    \item $f_1 = f \mathcal{X}_{\{ x \in X: |f(x)| \leq c\lambda \}}$
  \end{itemize}
  where $c$ is a fixed constant.\\
  Then $f_0 \in L^{p_0}$:
  \begin{align*}
    \int_X |f_0(x)|^{p_0} d\mu & = \int_{\{ x \in X: |f(x)| > c\lambda \}} |f(x)|^{p_0} d\mu\\
    & < \int_{\{ x \in X: |f(x)| > c\lambda \}} (c\lambda)^{p_0 - p}|f(x)|^{p} d\mu\\
    & \leq (c\lambda)^{p_0 - p} ||f(x)||_p^p
  \end{align*}   
  And $f_1 \in L^{p_1}$:
  \begin{align*}
    \int_X |f_1(x)|^{p_1} d\mu & = \int_{\{ x \in X: |f(x)| \leq c\lambda \}} |f(x)|^{p_1} d\mu\\
    & < \int_{\{ x \in X: |f(x)| \leq c\lambda \}} (c\lambda)^{p_1 - p}|f(x)|^{p} d\mu\\
    & \leq (c\lambda)^{p_1 - p} ||f(x)||_p^p
  \end{align*}
  Moreover, since $T$ is sublinear, then:
  \begin{align*}
    \lambda & < |Tf(x)| \\
    & \leq |Tf_0(x)| + |Tf_1(x)| \leq \begin{cases}
      2|Tf_0(x)| & \text{if } x \in \{|Tf_1(x)| \leq |Tf_0(x)|\} \\
      2|Tf_1(x)| & \text{if } x \in \{|Tf_0(x)| \leq |Tf_1(x)|\}
    \end{cases}
  \end{align*}
  Thus:
  $$\mu(\{x \in X: |Tf(x)| > \lambda\}) \leq \mu(\{x \in X: |Tf_0(x)| > \frac{\lambda}{2}\}) + \mu(\{x \in X: |Tf_1(x)| > \frac{\lambda}{2}\})$$
  Hence:
  $$a_Tf(\lambda) \leq a_{Tf_0}(\frac{\lambda}{2}) + a_{Tf_1}(\frac{\lambda}{2})$$
  For the case $p_1 = \infty$, consider $c = \frac{1}{2A_1}$ with $A_1$ such that $||Tg(x)||_{\infty} \leq A_1 ||g||_{\infty}$, that is:
  \begin{align*}
    |Tf_1(x)| \leq A_1 ||f||_{\infty} \leq A_1 c\lambda - \frac{\lambda}{2}
  \end{align*}
  Almost everywhere, hence:
  $$a_{Tf_1}(\frac{\lambda}{2}) = 0$$
  Using the weak $\left(p_0, p_0\right)$ inequality: \\
  $$a_{Tf_0}\left(\frac{\lambda}{2}\right) \leq \left(\frac{2 A_0}{\lambda} ||f_0||_{p_0}\right)^{p_0} \left(f_0 \in L^{p_0}\right).$$ Then
  \begin{align*}
    \|T f\|_p^p & = p \int_0^{\infty} \lambda^{p - 1} a_Tf(\lambda) d\lambda\\
    & \leq p \int_0^{\infty} \lambda^{p - p_0 - 1} \left(2 A_0\right)^{p_0} ||f_0||_{p_0}^{p_0} d\lambda \\
    & \leq p \int_0^{\infty} \lambda^{p - p_0 - 1} \left(2 A_0\right)^{p_0} \int_{\{x \in X: |f(x)| > c\lambda\}} |f(x)|^{p_0} d\mu d\lambda \\
    & \leq p \left(2 A_0\right)^{p_0} \int_X \int_0^{\frac{|f(x)|}{c}} \lambda^{p - p_0 - 1}|f(x)|^{p_0} d\lambda d\mu \\
    & \leq p \left(2 A_0\right)^{p_0} \int_X \frac{|f(x)|^{p - p_0}}{c^{p - p_0}(p - p_0)} |f(x)|^{p_0} d\mu \\
    & \leq \frac{p(2A_0)^{p_0}}{c^{p - p_0}(p - p_0)} \int_X |f(x)|^p d\mu\\
    & \leq \frac{p(2A_0)^{p_0}}{c^{p - p_0}(p - p_0)} \|f(x)\|_p^p
  \end{align*}
  Since $c = \frac{1}{2A_1}$, we obtain:
  $$\|Tf\|_p^p \leq \frac{p(2A_0)^{p_0}}{p - p_0}(2A_1)^{p - p_0}\|f(x)\|_p^p$$
  From which it is shown that $T$ is $(p,p)$-strong.
  \newpage
  Now let's consider the case where $p_1 < \infty$, for this it will be useful to remember the weak inequalities in $p_0$ and $p_1$:
  $$a_{Tf_i}\left(\frac{\lambda}{2}\right) \leq \left( \frac{2A_i}{\lambda} ||f_i||_{p_i}\right)^{p_i} \hspace{1cm}\text{for } i=0,1$$
  Now:
  \begin{align*}
    \|Tf\|_p^p & = p \int_0^\infty \lambda^{p - 1} a_{Tf}(\lambda) d\lambda\\
    & \leq p \int_0^\infty \lambda^{p - 1} \left(a_{Tf_0}\left(\frac{\lambda}{2}\right) + a_{Tf_1}\left(\frac{\lambda}{2}\right)\right) d\lambda\\
    & \leq p \int_0^\infty \lambda^{p - 1} \left(\left( \frac{2A_0}{\lambda} ||f_0||_{p_0}\right)^{p_0} + \left( \frac{2A_1}{\lambda} ||f_1||_{p_1}\right)^{p_1}\right) d\lambda\\
    & \leq p \int_0^\infty \lambda^{p - p_0 - 1} (2A_0)^{p_0} ||f_0||_{p_0}^{p_0} d\lambda + p \int_0^\infty \lambda^{p - p_1 - 1} (2A_1)^{p_1} ||f_1||_{p_1}^{p_1} d\lambda\\
    & \leq p \int_0^\infty \lambda^{p - p_0 - 1} (2A_0)^{p_0} \int_{\{x \in X: |f(x)| > c\lambda\}} |f(x)|^{p_0} d\mu d\lambda\\
    & \hspace{0.5cm} + p \int_0^\infty \lambda^{p - p_1 - 1} (2A_1)^{p_1} \int_{\{x \in X: |f(x)| \leq c\lambda\}} |f(x)|^{p_1} d\mu d\lambda\\
    & \leq p \int_0^\infty \int_{\{x \in X: |f(x)| > c\lambda\}} \lambda^{p - p_0 - 1} (2A_0)^{p_0} |f(x)|^{p_0} d\mu d\lambda\\
    & \hspace{0.5cm} + p \int_0^\infty \int_{\{x \in X: |f(x)| \leq c\lambda\}} \lambda^{p - p_1 - 1} (2A_1)^{p_1} |f(x)|^{p_1} d\mu d\lambda\\
    & \leq p \int_X \int_0^{\frac{|f(x)|}{c}} \lambda^{p - p_0 - 1} (2A_0)^{p_0} |f(x)|^{p_0} d\lambda d\mu\\
    & \hspace{0.5cm} + p \int_X \int_{\frac{|f(x)|}{c}}^\infty \lambda^{p - p_1 - 1} (2A_1)^{p_1} |f(x)|^{p_1} d\lambda d\mu\\
    & \leq \left(\frac{p(2A_0)^{p_0}}{c^{p - p_0}(p - p_0)} + \frac{p(2A_1)^{p_1}}{c^{p - p_1}(p - p_1)}\right) \|f\|_p^p
  \end{align*}
  From which we can conclude that $T$ is $(p,p)$-strong.\\
  \textbf{Comment:} If $A_0$ and $A_1$ are the constants of the weak inequalities from the previous theorem, we have more precisely:
  \begin{align*}
    \|Tf\|_p \leq 2p^{1/p} \left(\frac{1}{p - p_0} + \frac{1}{p_1 - p}\right)^{1/p} A_0^{1 - \theta} A_1^{\theta} ||f||_p
  \end{align*}
  Where $\frac{1}{p} = \frac{\theta}{p_1} + \frac{1 - \theta}{p_0}$ with $0 < \theta < 1$.\\
  When $p_1 = \infty$, it is enough to take $c$ as $\frac{1}{2A_1}$, in the opposite case, it is enough to take $c$ such that $(2A_0 c)^{p_0} = (2A_1 c)^{p_1}$.
\end{proof}
