In many problems and applications of analysis, questions arise such as the following:\\
What is the asymptotic behavior of $I(\lambda)$ as $\lambda\rightarrow\infty$, where:
\begin{align*}
  I(\lambda)&=\int_{a}^{b}e^{i\lambda\phi(x)}f(x)dx
\end{align*}
and $\phi$ is a smooth real-valued function called "the phase function," and $f$ is a smooth complex-valued function? 
Let us see an example where this concern arises.  
\begin{example}{Airy Equation (Linear KdV)}
  \begin{align*}
    \begin{cases}
      \partial_t\mu +\partial_x^3\mu=0\\
      \mu(x,0)=f(x)\in\mathcal{S}(\mathbb{R}^n)
    \end{cases}
  \end{align*}
  Applying the Fourier transform:
  \begin{align*}
    0&=\hat{\partial_t\mu + \partial_x^3\mu}\\
    &=\hat{\partial_t\mu} + \hat{\partial_x^3\mu}\\
    &=\partial_t\hat{\mu} + (2\pi i\xi)^3 \hat{\mu}\\
    &=\partial_t\hat{\mu} - 8i(\pi\xi)^3 \hat{\mu}\\
  \end{align*}
  Thus, we have:
  \begin{align*}
    \begin{cases}
      \partial_t\hat{\mu} - 8i(\pi\xi)^3 \hat{\mu}=0\\
      \hat{\mu}(\xi,0)=\hat{f}(\xi)\in\mathcal{S}(\mathbb{R}^n)
    \end{cases}
  \end{align*}
  Viewing it as an ODE with respect to the variable $t$, we can deduce that:
  $$\hat{\mu}(\xi,t)=e^{8i(\pi\xi)^3t}\hat{f}(\xi)$$
  Then, using the inversion formula, we have:
  \begin{align*}
    \mu(x,t) &= \int_{\mathbb{R}}\hat{\mu}e^{2\pi ix\xi}d\xi\\
    &= \int_{\mathbb{R}} e^{8i(\pi\xi)^3t+2i\pi x\xi}\hat{f}(\xi)d\xi\\
    &= \int_{\mathbb{R}} e^{it(8(\pi\xi)^3+2\pi\frac{\xi}{t}x)}\hat{f}(\xi)d\xi\\
    &= \int_{\mathbb{R}} e^{it\phi(\xi)}\hat{f}(\xi)d\xi
  \end{align*}
  where $\phi(\xi)=8(\pi\xi)^3+2\pi\frac{\xi}{t}x$, which motivates us to see what happens as $t\rightarrow \infty$. 
\end{example}
\begin{proposition}{}
  Let $f\in C^{\infty}_{0}([a,b])$ and $\phi'(x)\neq 0$ for all $x\in[a,b]$, then:
  $$I(\lambda)=\int_{a}^{b}e^{i\lambda \phi(x)}f(x)dx=O(\lambda^{-k})$$
  as $\lambda \rightarrow \infty$ for any $k\in\mathbb{Z}^+$.
\end{proposition} 
\begin{proof}{}  
  Note that since $f\in C^{\infty}_{0}([a,b])$, we can assert that $f(a)=0=f(b)$, since $supp \subseteq (a,b)$.
  \begin{itemize}
    \item Case $k=1$.\\
      Note that:
      \begin{align*}
        \frac{d}{dx}e^{i\lambda\phi(x)}&=i\lambda\phi'(x)e^{i\lambda\phi(x)}\\
        &\rightarrow e^{i\lambda\phi(x)}=\frac{1}{i\lambda\phi'(x)}\frac{d(e^{i\lambda\phi(x)})}{dx}
      \end{align*}
      Using this, we can see that:
      \begin{align*}
        \int_{a}^{b} e^{i\lambda\phi(x)}f(x)dx &=\int_{a}^{b}\frac{1}{i\lambda\phi'(x)}\frac{d(e^{i\lambda\phi(x)})}{dx}f(x)dx\\
        &= \frac{1}{i\lambda}\int_{a}^{b}\frac{d(e^{i\lambda\phi(x)})}{dx}\frac{f}{\phi'}(x)dx\\
        &= \frac{1}{i\lambda}\left(e^{i\lambda\phi(x)}\frac{f}{\phi'}(x)\mid^{b}_{a}-\int_{a}^{b}e^{i\lambda\phi(x)}\frac{d}{dx}(\frac{f}{\phi'})(x)dx\right)\\
        &= -\frac{1}{i\lambda}\int_{a}^{b}e^{i\lambda\phi(x)}\frac{f'\phi'-f\phi''}{\phi'^2}(x)dx
      \end{align*}
      This implies that:
      \begin{align*}
        |I(\lambda)|&\leq \left|-\frac{1}{i\lambda}\int_{a}^{b}e^{i\lambda\phi(x)}\frac{f'\phi'-f\phi''}{\phi'^2}(x)dx\right|\\
        &\leq \left|\frac{1}{i\lambda}\right|\int_{a}^{b}\left|\frac{f'\phi'-f\phi''}{\phi'^2}(x)\right|dx\\
      \end{align*}
      Now, note that since $\phi',\phi'',f,f'\in C([a,b])$ and $\phi'\neq 0$ on the entire interval $[a,b]$, we have:
      \begin{align*}
        \sup_{x\in[a,b]}|\phi''(x)|&=M\\
        \sup_{x\in[a,b]}|\phi'(x)|&=m\neq0\\
        \inf_{x\in[a,b]}|\phi'(x)|&=n\neq0\\
        \sup_{x\in[a,b]}|f(x)|&=p\\
        \sup_{x\in[a,b]}|f'(x)|&=q
      \end{align*}
      Thus:
      \begin{align*}
        |I(\lambda)|&\leq \left| \frac{1}{i\lambda} \right|\int_{a}^{b}\left|\frac{f'\phi'-f\phi''}{\phi'^2}(x) \right|dx\\
        &\leq \left| \frac{1}{i\lambda} \right|\int_{a}^{b}\left|\frac{qm-pM}{n^2}(x) \right|dx\\
        &\leq \frac{c}{\lambda}
      \end{align*}
    \item Case $k\geq 1$.\\
      Let us define the operator $\mathcal{L}(f)=\frac{1}{i\lambda\phi'}\frac{df}{dx}$.\\
      Note that the following holds:
      $$\mathcal{L}^t(f)=-\frac{d}{dx}\left(\frac{f}{i\lambda\phi'}\right)\hspace{0.5cm}\text{and}\hspace{0.5cm}\mathcal{L}^k(e^{i\lambda\phi})=e^{i\lambda\phi}$$
      where $\mathcal{L}^t$ is the adjoint operator of $\mathcal{L}$.\\
      Now, using integration by parts, we can see that:
      \begin{align*}
        \int_{a}^{b}e^{i\lambda\phi(x)}f(x)dx&=\int_{a}^{b}\mathcal{L}(e^{i\lambda\phi(x)})f(x)dx\\
        &=\int_{a}^{b}\frac{1}{i\lambda\phi'(x)}\frac{d}{dx}(e^{i\lambda\phi(x)})f(x)dx\\
        &=\frac{1}{i\lambda}\left(e^{i\lambda\phi(x)}\frac{f}{\phi'}(x)\mid_{a}^{b}-\int e^{i\lambda\phi(x)}\frac{d}{dx}\left(\frac{f}{\phi'}\right)\right)dx\\
        &=\int_{a}^{b}e^{i\lambda\phi(x)}\mathcal{L}^t(f)dx\\
        &=\int_{a}^{b}e^{i\lambda\phi(x)}(\mathcal{L}^t)^k(f)dx\\
        &=O(\lambda^{-k})
      \end{align*}
      for all $k\in\mathbb{Z}^+$.
  \end{itemize}
\end{proof}
\begin{proposition}{}
  Given $k\in\mathbb{Z}^+$, let $|\phi^{(k)}(x)|\geq 1$ for all $x\in[a,b]$ with $\phi'(x)$ monotonic for the case $k=1$. Then:
  \begin{align*}
    \left| \int_{a}^{b} e^{i\lambda\phi(x)}dx\right|\leq c_k\lambda^{-\frac{1}{k}}
  \end{align*}
  where the constant $c_k$ is independent of $a,b$.
\end{proposition}
\begin{proof}{}
  \begin{itemize}
    \item Case $k=1$.
      Note that:
      \begin{align*}
        \int_{a}^{b}e^{i\lambda\phi(x)}dx&=\int_{a}^{b}\mathcal{L}(e^{i\lambda\phi(x)})dx\\
        &=\int_{a}^{b}\frac{1}{i\lambda\phi'(x)}\frac{d}{dx}(e^{i\lambda\phi(x)})dx\\
        &=\frac{1}{i\lambda\phi'(x)}e^{i\lambda\phi(x)}\mid_{a}^{b}-\int_{a}^{b}e^{i\lambda\phi(x)}\frac{d}{dx}\left(\frac{1}{i\lambda\phi'(x)}\right)dx
      \end{align*}
      Let's look at this in parts. Since $|\phi'(x)|\geq 1$, we have:
      \begin{align*}
        \left|\frac{1}{i\lambda\phi'(x)}e^{i\lambda\phi(x)}\mid_{a}^b\right|&\leq \left| \frac{1}{i\lambda\phi'(b)}e^{i\lambda\phi(b)} - \frac{1}{i\lambda\phi'(a)}e^{i\lambda\phi(a)}\right|\\
        &\leq\left| \frac{1}{i\lambda\phi'(b)}e^{i\lambda\phi(b)}\right| + \left|\frac{1}{i\lambda\phi'(a)}e^{i\lambda\phi(a)}\right|\\
        &\leq\frac{1}{\lambda} + \frac{1}{\lambda}\\
        &\leq \frac{2}{\lambda}
      \end{align*}
      On the other hand:
      \begin{align*}
        \left| \int_{a}^{b}e^{i\lambda\phi(x)}\frac{d}{dx}\left(\frac{1}{i\lambda\phi'(x)}\right) dx\right|&\leq \int_{a}^{b}\left|e^{i\lambda\phi(x)}\frac{d}{dx}\left(\frac{1}{i\lambda\phi'(x)}\right)\right|dx\\
        &=\leq \frac{1}{\lambda}\int_{a}^{b}\left|-\frac{\phi''(x)}{\phi'^2(x)}\right|dx
      \end{align*}
      Now, since $\phi'$ is monotonic, then $\phi''(x)\geq 0$ for all $x\in[a,b]$ or $\phi''(x)\geq 0$ for all $x\in[a,b]$.\\
      Suppose $\phi''(x)\geq 0$ for all $x\in[a,b]$ (the reasoning in the opposite case is analogous).\\
      \begin{align*}
        \left| \int_{a}^{b}e^{i\lambda\phi(x)}\frac{d}{dx}\left(\frac{1}{i\lambda\phi'(x)}\right) dx\right|&\leq \frac{1}{\lambda}\int_{a}^{b}\left|-\frac{\phi''(x)}{\phi'^2(x)}\right|dx\\
        &\leq \frac{1}{\lambda}\int_{a}^{b}\frac{\phi''(x)}{\phi'^2(x)}dx\\
        &\leq \frac{1}{\lambda}\int_{a}^{b}\frac{d}{dx}\left(\frac{1}{\phi'(x)}\right)dx\\
        &\leq \frac{1}{\lambda}\left(\frac{1}{\phi'(b)}-\frac{1}{\phi'(a)}\right)\\
        &\leq \frac{1}{\lambda}\left|\frac{1}{\phi'(b)}-\frac{1}{\phi'(a)}\right|\\
        &\leq \frac{2}{\lambda}
      \end{align*}
      Then:
      \begin{align*}
        \left| \int_{a}^{b} e^{i\lambda\phi(x)}dx\right|&\leq \left| \frac{1}{i\lambda\phi'(x)}e^{i\lambda\phi(x)}\mid_{a}^{b}-\int_{a}^{b}e^{i\lambda\phi(x)}\frac{d}{dx}\left(\frac{1}{i\lambda\phi'(x)}\right)dx\right|\\
        &\leq \left|\frac{2}{\lambda}\right|+\left|\frac{2}{\lambda}\right|\\
        &\leq c_1\lambda^{-\frac{1}{1}}
      \end{align*}
    \item Case $k\geq 2$.\\
      Let's reason by induction. If $|\phi^{(k)}(x)|\geq 1$ and $|\phi^{(k+1)(x)}|\geq 1$ for all $x\in[a,b]$ and:
      \begin{align*}
        \left|\int_{a}^{b}e^{i\lambda\phi(x)}dx\right|\leq c_k\lambda^{-\frac{1}{k}}
      \end{align*}
      Then:
      \begin{align*}
        \left|\int_{a}^{b}e^{i\lambda\phi(x)}dx\right|\leq c_{k+1}\lambda^{-\frac{1}{k+1}}
      \end{align*}
      Let $x_0\in[a,b]$ such that:
      \begin{align*}
        |\phi^{(k+1)}(x_0)|=\min_{x\in[a,b]}|\phi^{(k+1)}(x)|
      \end{align*}
      Let's look at this in cases:
      \begin{itemize}
        \item $|\phi^{(k+1)}(x_0)|=0$ and $x_0\in(a,b)$.\\
          \textbf{Claim:} $|\phi^{(k)}(x)|\geq \delta$ for all $x\in[a,x_0-\delta]\cup[x_0+\delta,b]$.\\
          Let's see that this holds since as $|\phi^{(k+1)}(x)|\geq 1$ for all $x\in[a,b]$, then $\phi^{(k+1)}(x) \geq 1$ for all $x\in[a,b]$ or $\phi^{(k+1)} \leq 1$ for all $x\in[a,b]$.\\
          Suppose $\phi^{(k+1)}(x)\geq 1$ for all $x\in[a,b]$ (the reasoning is analogous for the opposite case).\\
          Let $\psi(x)=\phi^{(k)}-(x-x_0)$, so:
          \begin{align*}
            \psi(x_0)&=\phi^{(k)}(x_0)-(x_0-x_0)=0\\
            \psi'(x)&=\phi^{(k+1)}(x)-1
          \end{align*}
          If we take $x>x_0+\delta$, then:
          \begin{align*}
            0&=\psi(x_0)\\
            &\leq\psi(x)\\
            &\leq\phi^{(k)}(x)-(x-x_0)
          \end{align*}
          Then as $x>x_0+\delta$, we have $x-x_0>\delta$ and since $0<\phi^{(k)}(x)-(x-x_0)$, then:
          \begin{align*}
            \delta &\leq \phi^{(k)}(x) 
          \end{align*}
          Similarly, it can be verified that if $x<x_0-\delta$ then:
          \begin{align*}
            \delta \leq \phi^{(k)}(x)
          \end{align*}
          Therefore, we conclude that $|\phi^{(k)}(x)|\geq\delta$ for all $x\in[a,x_0-\delta]\cup[x_0+\delta,b]$.\\
          \textbf{Conclusion:} Therefore:
          \begin{align*}
            \left| \int_{a}^{b} e^{i\lambda\phi(x)}dx\right|&\leq\left|\int_{a}^{x_0-\delta}e^{i\lambda\phi(x)}dx\right|+\left|\int_{x_0+\delta}^{b}e^{i\lambda\phi(x)}dx\right|+\left|\int_{x_0-\delta}^{x_0+\delta}e^{i\lambda\phi(x)}dx\right|\\
            &\leq c_k\lambda^{-\frac{1}{k}}+c_k\lambda^{-\frac{1}{k}}+\mathcal{O}(\lambda^{-k-1})\\
            &\leq c_{k+1}\lambda^{-\frac{1}{k+1}}.
          \end{align*}
      \end{itemize}
  \end{itemize}
\end{proof}
\begin{corollary}{van der Corput}
  Under the hypotheses of the previous proposition:
  \begin{align*}
    \left| \int_{a}^{b} e^{i\lambda\phi(x)}dx\right|&\leq c_k\lambda^{\frac{1}{k}}(\|f\|_{\infty}+\|f'\|_{1}) 
  \end{align*}
\end{corollary}
\begin{proof}{} 
  Let:
  \begin{align*}
    G(x)&=\int_{a}^{x}e^{i\lambda\phi(y)}dy 
  \end{align*}
  Then:
  \begin{align*}
    |G(x)|&\leq \tilde{c_k}\lambda^{-\frac{1}{k}} 
  \end{align*}
  For all $x\in[a,b]$, then:
  \begin{align*}
    \left|\int_{a}^{b}e^{i\lambda\phi(x)}f(x)dx\right|&=\left|\int_{a}^{b}G'(x)f(x)dx\right|\\
    &=\left|G(x)f(x)\mid_{a}^{b}-\int_{a}^{b}G(x)f'(x)dx\right|\\
    &\leq\left|G(b)f(b)\right|+\left|G(a)f(a)\right|+\int_{a}^{b}\left|G(x)f'(x)\right|dx\\
    &\leq\left|G(b)f(b)\right|+\left|G(a)f(a)\right|+c_k\lambda^{-\frac{1}{k}}\int_{a}^{b}\left|f'(x)\right|dx\\
    &\leq\left|G(b)f(b)\right|+\left|G(a)f(a)\right|+c_k\lambda^{-\frac{1}{k}}\|f'\|_{1}\\
    &\leq 2\tilde{c_k}\lambda^{-\frac{1}{k}}\|f\|_{\infty}+\tilde{c_k}\lambda^{-\frac{1}{k}}\|f'\|_{1}\\
    &\leq c_k\lambda^{-\frac{1}{k}}(\|f\|_{\infty}+\|f'\|_{1})
  \end{align*}
\end{proof}
\begin{proposition}{}
  Let $\beta\in [0,1/2]$ and $I_{\beta}(x)$ be the oscillatory integral:
  \begin{align*}
    I_{\beta}(x)&=\int_{-\infty}^{\infty}e^{i(x\eta+\eta^3)}|\eta|^{\beta}d\eta
  \end{align*}
  Then, $I_{\beta}\in L^{\infty}(\mathbb{R})$.
\end{proposition}
\begin{proof}{} 
  First, let’s take $\phi_{0}\in C^{\infty}(\mathbb{R})$ such that:
  \begin{align*}
    \phi_0(\eta)= 
    \begin{cases}
      1, &\text{ If } |\eta|>3 \text{,}\\
      0, &\text{ If } |\eta|<2 .
    \end{cases}
  \end{align*}
  Note that $\phi_0'\in C^{\infty}_{c}(\mathbb{R})$.\\
  Let’s first see that $[(1-\phi_0)(\eta)]e^{i(x\eta+\eta^3)}|\eta|^{\beta}\in L^1(\mathbb{R})$:
  \begin{align*}
    \|[(1-\phi_0)(\eta)]e^{i(x\eta+\eta^3)}|\eta|^{\beta}\|_{L^{1}(\mathbb{R})}&\leq\int_{-\infty}^{\infty}\left| [(1-\phi_0)(\eta)]e^{i(x\eta+\eta^3)}|\eta|^{\beta} \right|d\eta\\
    &\leq \int_{-3}^{3}\left| e^{i\eta^3}|\eta|^{\beta} \right|d\eta \\
    &\leq \int_{-3}^{3}|\eta|^{\beta}d\eta\\
    &\leq 2\int_{0}^{3}\eta^{\beta}d\eta\\
    &\leq 2\left(\frac{\eta^{\beta+1}}{\beta+1}\Big|_{0}^{3}\right)\\
    &\leq 2\frac{3^{\beta+1}}{\beta+1}
  \end{align*}
  So we can ensure that its Fourier transform belongs to $L^{\infty}(\mathbb{R})$.\\
  Now consider:
  \begin{align*}
    \tilde{I}_{\beta}(x)&=\int_{-\infty}^{\infty}e^{i(x\eta+\eta^3)}|\eta|^{\beta}\phi_0(\eta)d\eta
  \end{align*}
  Note that for $x\geq -3$, the phase function $\phi(x)=x\eta+\eta^3$, in the $supp(\phi_0)=\overline{\{\eta: \phi_0(\eta)\neq 0\}}\subseteq (-\infty,-2]\cup[2,\infty)$ satisfies that:
  \begin{align*}
    |\phi_x'(\eta)|\geq |x+3\eta^2|\geq |x|+|\eta|^2\geq |\eta|^2
  \end{align*}
  Now, note that $\tilde{I}_{\beta}(x)$ can be partitioned as follows:
  \begin{align*}
    \tilde{I}_{\beta}(x)&=\int_{-\infty}^{\infty}e^{i(x\eta+\eta^3)}|\eta|^{\beta}\phi_0(\eta)d\eta\\
    &=\int_{-\infty}^{\infty}\frac{d}{d\eta}\left( e^{i(x\eta+\eta^3)} \right)\frac{|\eta|^{\beta}\phi_0(\eta)}{i(x+3\eta^2)}d\eta\\
    &=\lim_{R\rightarrow \infty}\int_{-R}^{R}\frac{d}{d\eta}\left( e^{i(x\eta+\eta^3)} \right)\frac{|\eta|^{\beta}\phi_0(\eta)}{i(x+3\eta^2)}d\eta\\
    &=\lim_{R\rightarrow \infty}\left[ e^{i(x\eta+\eta^3)}\frac{|\eta|^{\beta}\phi_0(\eta)}{i(x+3\eta^2)} \right]_{-R}^{R}+ \lim_{R\rightarrow \infty} -\int_{-R}^{R}e^{i(x\eta+\eta^3)}\frac{d}{d\eta}\left( \frac{|\eta|^{\beta}\phi_0(\eta)}{i(x+3\eta^{2})} \right)d\eta\\
    &=\lim_{R\rightarrow \infty} A + B
  \end{align*}
  Now let’s examine this by parts:
  \begin{align*}
    |A|&\leq \lim_{R\rightarrow \infty}\left| e^{i(xR+R^3)}\frac{|R|^{\beta}\phi_0(R)}{i(x+3R^2)} - e^{i(x(-R)+(-R)^3)}\frac{|(-R)|^{\beta}\phi_0(-R)}{i(x+3(-R)^2)} \right|\\
    &\leq \lim_{R\rightarrow\infty} \left| \frac{|R|^{\beta}\phi_0(R)}{x+3R^2} \right| + \left| \frac{|-R|^{\beta}\phi_0(-R)}{x+3(-R)^2} \right|\\
    &\leq \|\phi_0\|_{L^{\infty}(\mathbb{R})}\lim_{R\rightarrow\infty} \left| \frac{|R|^{\beta}}{R^2} \right| + \left| \frac{|-R|^{\beta}}{(-R)^2} \right|\\
    &\leq \|\phi_0\|_{L^{\infty}(\mathbb{R})}\lim_{R\rightarrow\infty} \left| \frac{1}{|R|^{2-\beta}} \right| + \left| \frac{1}{|-R|^{2-\beta}} \right|\\
    &\leq 0
  \end{align*}
  Now let’s look at part $B$:
  \begin{align*}
    |B|&\leq \lim_{R\rightarrow \infty}\int_{-R}^{R}\left| e^{i(x\eta+\eta^3)}\frac{d}{d\eta}\left( \frac{|\eta|^{\beta}\phi_0(\eta)}{i(x+3\eta^2)} \right) \right|d\eta\\
    &\leq \lim_{R\rightarrow\infty}\int_{-R}^{R}\left| \frac{(\beta|\eta|^{\beta-1}\phi_0(\eta)+|\eta|^{\beta}\phi_0'(\eta))(x+3\eta^2)-(|\eta|^{\beta}\phi_0(\eta))(6\eta)}{(x+3\eta^2)^2} \right|d\eta\\
    &\leq \lim_{R\rightarrow\infty}\int_{-R}^{R}\left| \frac{(\beta|\eta|^{\beta-1}\|\phi_0\|_{L^{\infty}(\mathbb{R})}+|\eta|^{\beta}\|\phi_0'\|_{L^{\infty}(\mathbb{R})})(x+3\eta^2)-(|\eta|^{\beta}\|\phi_0\|_{L^{\infty}(\mathbb{R})})(6\eta)}{\eta^4} \right|d\eta\\
    &\leq \lim_{R\rightarrow \infty}\int_{-R}^{R} \beta\|\phi_0\|_{L^{\infty}(\mathbb{R})}|\eta|^{\beta-5}+\|\phi_0'\|_{L^{\infty}(\mathbb{R})}(x|\eta|^{\beta-4}+3|\eta|^{\beta-2})-6\|\phi_0\|_{L^{\infty}(\mathbb{R})}|\eta|^{\beta-3}d\eta\\
    &\leq \lim_{R\rightarrow \infty} \beta\|\phi_0\|_{L^{\infty}(\mathbb{R})}\frac{|\eta|^{\beta-4}}{\beta-4}+\|\phi_0'\|_{L^{\infty}(\mathbb{R})}\left(x\frac{|\eta|^{\beta-3}}{\beta-3}+3\frac{|\eta|^{\beta-1}}{\beta-1}\right)-6\|\phi_0\|_{L^{\infty}(\mathbb{R})}\frac{|\eta|^{\beta-2}}{\beta-2}\Bigg|_{-R}^{R}\\
    &\leq 0
  \end{align*}
  So we can see that:
  \begin{align*}
    \|I_{b}\|_{L^{\infty}(\mathbb{R})}&\leq \|e^{i(x\eta+\eta^3)}|\eta|^{\beta}(1-\phi_0)\|_{L^{1}(\mathbb{R})} + \|\tilde{I}_{\beta}\|_{L^{\infty}(\mathbb{R})}\\
    &\leq \|e^{i(x\eta+\eta^3)}|\eta|^{\beta}(1-\phi_0)\|_{L^{1}(\mathbb{R})} + |A| + |B|\\
    &\leq \|e^{i(x\eta+\eta^3)}|\eta|^{\beta}(1-\phi_0)\|_{L^{1}(\mathbb{R})}\\
    &\leq 2\left( \frac{3^{\beta+1}}{\beta+1} \right)
  \end{align*}
  Then if $x\geq -3$, we can ensure that $I_{\beta}\in L^{\infty}(\mathbb{R})$.\\
  Now let’s consider the case $x<-3$.\\
  For this, suppose $(\phi_1,\phi_2)\in C^{\infty}_{0}(\mathbb{R})\times C^{\infty}(\mathbb{R})$ such that $\phi_1(\eta)+\phi_2(\eta)=1$ with:
  \begin{enumerate}
    \item $supp(\phi_1)\subseteq A=\{\eta:|x+3\eta^2|\leq \frac{|x|}{2}\}$.
    \item $\phi_2=0$ in $B=\{\eta:|x+3\eta^2|<\frac{|x|}{3}\}$. 
  \end{enumerate}
  Now, with this, we can split $I_{\beta}$ into 2 pieces:
  \begin{align*}
    |\tilde{I}_{\beta}|\leq |\tilde{I}_{\beta}^{1}(x)|+|\tilde{I}_{\beta}^{2}(x)|
  \end{align*}
  Where:
  \begin{align*}
    \tilde{I}_{\beta}^{j}(x)&=\int_{-\infty}^{\infty}e^{i(x\eta+\eta^3)}|\eta|^{\beta}\phi_0(\eta)\phi_j(\eta)d\eta && \text{For } j=1,2.
  \end{align*}
  When $\phi_2\neq 0$, a rather strange inequality is obtained :c.
  \begin{align*}
    \text{Insert Strange Inequality :c}
  \end{align*}
  By integration by parts we have that:
  \begin{align*}
    |\tilde{I}_{\beta}^{2}(x)|&\leq\left|\int_{-\infty}^{\infty}e^{i(x\eta+\eta^3)}|\eta|^{\beta}\phi_0(\eta)\phi_2(\eta)d\eta \right|\\
    &\leq\left| \int_{-\infty}^{\infty}\frac{d}{d\eta}\left(e^{i(x\eta+\eta^3)}\right)\frac{1}{i(x+3\eta^2)}|\eta|^{\beta}\phi_0(\eta)\phi_2(\eta)d\eta \right|\\
    &\leq \left| \frac{e^{i(x\eta+\eta^3)}}{i(x+3\eta^2)}|\eta|^{\beta}\phi_0(\eta)\phi_2(\eta)\Big|_{-\infty}^{\infty}-\int_{-\infty}^{\infty}e^{i(x\eta+\eta^3)}\frac{d}{d\eta}\left( \frac{|\eta|^{\beta}\phi_0(\eta)\phi_2(\eta)}{i(x+3\eta^2)} \right) d\eta \right|
  \end{align*}
\end{proof}
