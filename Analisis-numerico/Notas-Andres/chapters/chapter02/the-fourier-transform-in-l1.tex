\begin{definition}{}
  The \textit{Fourier Transform} of a function $f\in L^{1}(\mathbb{R}^n)$, denoted by $\hat{f}$, defined as:
	\begin{equation}
    \hat{f}(\xi)=\int_{\mathbb{R}^n}f(x)e^{-2\pi i (x \cdot \xi)}dx,\hspace{0.5cm}\text{for }\xi\in\mathbb{R}^{n},\label{definition:transform-of-f}
	\end{equation}
	where $(x\cdot \xi)=x^1\xi^1+\cdots+x^n\xi^n$.
\end{definition}
We list some basic properties of the Fourier transform in $L^1(\mathbb{R}^{n})$. 
\begin{theorem}{}\label{theorem:properties-of-fourier-transform-L1}
  Let $f\in L^1(\mathbb{R}^{n})$. Then:
  \begin{enumerate}
    \item $f\to \hat{f}$ defines a linear transform from $L^1(\mathbb{R}^{n})$ to $L^{\infty}(\mathbb{R}^{n})$ with
      \begin{equation}
        \|\hat{f}\|_{\infty}\leq \|f\|_{1}.\label{eq:2-2}
      \end{equation}
    \item $\hat{f}$ is continuos.\label{property:f-transform-is-continuos-l1}
    \item $\hat{f}(\xi)\to 0$ as $|\xi|\to \infty$ (\textit{Riemman-Lebesgue}).
    \item if $\tau_{h}f(x)=f(x-h)$ denotes the translation by $h\in\mathbb{R}^{n}$, then
      \begin{equation}
        \hat{(\tau_{h}f)}(\xi)=e^{-2\pi i(h\cdot \xi)}\hat{f}(\xi),\label{eq:2-3}
      \end{equation}
      and
      \begin{equation}
        (\hat{e^{-2\pi i(x\cdot h)}f})(\xi)=(\tau_{-h}\hat{f})(\xi).\label{eq:2-4}
      \end{equation}\label{property:traslation-f-transform-l1}
    \item if $\delta_{a}f(x)=f(ax)$ denotes a dilation by $a>0$, then
      \begin{equation}
        (\hat{\delta_{a}f})(\xi)=a^{-n}\hat{f}(a^{-1}\xi).\label{eq:2-5}
      \end{equation}
    \item Let $g\in L^1(\mathbb{R}^{n})$ and $f*g$ be the convolution of $f$ and $g$. Then,
      \begin{equation}
        (\hat{f*g})(\xi)=\hat{f}(\xi)\hat{g}(\xi).\label{eq:2-6}
      \end{equation}
    \item Let $g\in L^1(\mathbb{R}^{n})$. Then,
      \begin{equation}
        \int_{\mathbb{R}^{n}}\hat{f}g(y)dy=\int_{\mathbb{R}^{n}}f(y)\hat{g}(y)dy.\label{eq:2-7}
      \end{equation}\label{property:change-transform-l1}
  \end{enumerate}
\end{theorem}
Let's see the proof of this theorem
\begin{proof} 
	\begin{enumerate}
    \item Note that using the definition \cref{definition:transform-of-f}:
			\begin{align*}
				|\hat{f}(\xi)|\leq& |\int_{\mathbb{R}^n}f(x)e^{-2\pi i(x\cdot \xi)}dx|,\\
				\leq& \int_{\mathbb{R}^n}|f(x)||e^{-2\pi i(x\cdot \xi)}|dx,\\
				\leq& \int_{\mathbb{R}^n}|f(x)|dx,\\
				\leq& \|f\|_{1},
			\end{align*}
			then we can affirm that:
			\begin{align*}
				\sup_{\xi\in\mathbb{R}^n}|\hat{f}(\xi)|=\|\hat{f}\|_{\infty}\leq\|f\|_{1},
			\end{align*}
      which allow us to conclude \cref{eq:2-2}. On the other hand, note that equality is satisfied when $f>0$.
		\item Let us verify the conditions of the Lebesgue dominated convergence theorem.\\
			Suppose $\{f_{h}\}=\{f(x)e^{-2\pi i (x\cdot \xi + h)}\}$ that converges to $f(x)e^{-2\pi i (x\cdot\xi)}$ when $h$ tends to $0$, on the other hand note that from the previous numeral we know that $|f_{h}|\leq |f|$ for all $h$, also as $f\in L^{1}(\mathbb{R}^n)$, then we can assure that:
			\begin{align*}
				\lim_{h\rightarrow 0}\hat{f}(\xi+h)&=\lim_{h\rightarrow 0}\int_{\mathbb{R}^n}f(x)e^{-2\pi i (x\cdot \xi + h)}dx,\\
				&=\int_{\mathbb{R}^n}\lim_{h\rightarrow 0} f(x)e^{-2\pi i (x\cdot \xi + h)}dx,\\
				&=\int_{\mathbb{R}^n}f(x)e^{-2\pi i (x\cdot\xi)}dx,\\
				&=\hat{f}(\xi).
			\end{align*}
			Therefore it is demostrated that $\hat{f}$ is continuous.
		\item For this it is important remember that $C^{\infty}_{c}(\mathbb{R}^n)$ is dense in $L^{1}(\mathbb{R}^n)$, therefore we can determine the behavior of a functions in $L^{1}(\mathbb{R}^n)$ by functions in $C^{\infty}_{c}(\mathbb{R}^n)$.\\
			Suppose $f\in C^{\infty}_{c}(\mathbb{R}^n)$, then using integration by parts it is satisfied:
			\begin{align*}
				|\hat{f}(\xi)|&=\left| \int_{\mathbb{R}^n}f(x)e^{-2\pi i (x\cdot \xi)}dx \right| ,\\
        &=\left| \lim_{R \rightarrow \infty}\int_{B_R(0)}f(x)e^{-2\pi i (x\cdot \xi)}dx \right| ,\\
        &=\left|\lim_{R \rightarrow \infty}\int_{\partial B_R(0)}f(x)e^{-2\pi i (x\cdot \xi)}\eta_jdS(x)-\int_{B_R(0)}\frac{\frac{\partial f(x)}{\partial x_j}e^{-2\pi i (x\cdot \xi)}}{-2\pi i \xi_j}dx\right|,\\
        &=\left|\frac{1}{2\pi i \xi_j}\int_{\mathbb{R}^n} \frac{\partial f(x)}{\partial x_j}e^{2\pi i (x\cdot\xi)}dx\right|,\\
        &\leq \frac{1}{|2\pi\xi_j|}\int_{\mathbb{R}^n}\left|\frac{\partial f(x)}{\partial x_j}\right|dx,\\
				&\leq \frac{\|\frac{\partial f}{\partial x_j}\|_{1}}{|2\pi\xi|},
			\end{align*}
			then:
			\begin{align*}
				\lim_{|\xi|\rightarrow \infty}|\hat{f}(\xi)|\leq\lim_{|\xi|\rightarrow \infty}\frac{\|\frac{\partial f}{\partial x_j}\|_{1}}{|2\pi\xi|} = 0,
			\end{align*}
			which allow us to conclude that if $|\xi|\rightarrow \infty$, then $\hat{f}(\xi)\rightarrow 0$.
    \item Note that:
			\begin{align*}
				(\widehat{\tau_{h}f})(\xi)&=\int_{\mathbb{R}^n}(\tau_{h}f)(x)e^{-2\pi i (x\cdot\xi)}dx,\\
				&=\int_{\mathbb{R}^n}f(x-h)e^{-2\pi i (x\cdot\xi)}dx,\\
				&=\int_{\mathbb{R}^n}f(u)e^{-2\pi i (u+h\cdot\xi)}du,\\
				&=\int_{\mathbb{R}^n}f(u)e^{-2\pi i (u\cdot\xi)}e^{-2\pi i (h\cdot\xi)}du,\\
				&=e^{-2\pi i (h\cdot\xi)}\hat{f}(\xi).
			\end{align*}
			On the other hand:
			\begin{align*}
				(\widehat{e^{-2\pi i(x\cdot h)}f})(\xi)&=\int_{\mathbb{R}^n}e^{-2\pi i(x\cdot h)}f(x)e^{-2\pi i (x\cdot\xi)}dx,\\
				&=\int_{\mathbb{R}^n}f(x)e^{-2\pi i (x\cdot\xi+h)}dx,\\
				&=\hat{f}(\xi+h),\\
				&=(\tau_{-h}\hat{f})(\xi).
			\end{align*}
    \item By definition \cref{definition:transform-of-f}:
			\begin{align*}
				(\widehat{\delta_{a}f})(\xi)&=\int_{\mathbb{R}^n}\delta_{a}f(x)e^{-2\pi i (x\cdot\xi)}dx,\\
				&=\int_{\mathbb{R}^n}f(ax)e^{-2\pi i (x\cdot\xi)}dx,\\
				&=\int_{\mathbb{R}^n}f(u)e^{-2\pi i (\frac{u}{x}\cdot\xi)}a^{-n}du,\\
				&=a^{-n}\int_{\mathbb{R}^n}f(u)e^{-2\pi i (u\cdot\frac{\xi}{a})}du,\\
				&=a^{-n}\hat{f}\left(\frac{\xi}{a}\right)=a^{-n}\hat{f}(a^{-1}\xi).
			\end{align*}
		\item Firts let's see if $f,g\in L^{1}(\mathbb{R}^n)$, then $f*g\in L^{1}(\mathbb{R}^n)$, for this we see that:
			\begin{align*}
				\|f*g\|_{1}&=\int_{\mathbb{R}^n}|f*g(x)|dx,\\
				&=\int_{\mathbb{R}^n}\left|\int_{\mathbb{R}^n}f(y)g(x-y)dy\right|dx,\\
				&\leq\int_{\mathbb{R}^n}\int_{\mathbb{R}^n}|f(y)||g(x-y)|dydx,\\
				&\leq\int_{\mathbb{R}^n}\int_{\mathbb{R}^n}|f(y)||g(x-y)|dxdy,\\
				&\leq\int_{\mathbb{R}^n}|f(y)|\int_{\mathbb{R}^n}|g(x-y)|dxdy,\\
				&\leq\int_{\mathbb{R}^n}|f(y)|\|g\|_{1}dy,\\
				&\leq\|f\|_{1}\|g\|_{1},
			\end{align*}
			then $f*g\in L^{1}(\mathbb{R}^n)$.\\
			Now let's see that:
			\begin{align*}
				\widehat{f*g}(\xi)&=\int_{\mathbb{R}^n}f*g(x)e^{-2\pi i (x\cdot\xi)}dx,\\
				&=\int_{\mathbb{R}^n}\int_{\mathbb{R}^n}f(y)g(x-y)e^{-e\pi i (x\cdot\xi)}dydx,\\
				&=\int_{\mathbb{R}^n}\int_{\mathbb{R}^n}f(y)g(x-y)e^{-e\pi i (x\cdot\xi)}dxdy,\\
				&=\int_{\mathbb{R}^n}f(y)e^{-2\pi i (y\cdot \xi)}\hat{g}(\xi)dy,\\
				&=\left(\int_{\mathbb{R}^n}f(y)e^{-2\pi i (y\cdot \xi)}dy\right) \hat{g}(\xi),\\
				&=\hat{f}(\xi)\hat{g}(\xi).
			\end{align*}
		\item Note that:
			\begin{align*}
				\int_{\mathbb{R}^n}\hat{f}(y)g(y)dy&=\int_{\mathbb{R}^n}\int_{\mathbb{R}^n}f(x)e^{-2\pi i(x\cdot y)}dxg(y)dy,\\
				&=\int_{\mathbb{R}^n}\int_{\mathbb{R}^n}f(x)g(y)e^{-2\pi i(x\cdot y)}dxdy,\\
				&=\int_{\mathbb{R}^n}\int_{\mathbb{R}^n}f(x)g(y)e^{-2\pi i(y\cdot x)}dydx,\\
				&=\int_{\mathbb{R}^n}f(x)\hat{g}(x)dx,\\
				&=\int_{\mathbb{R}^n}f(y)\hat{g}(y)dy.
			\end{align*}
	\end{enumerate}
\end{proof}
Now, it will be interesting to see examples that allow us to contrast some of the properties demostrated in \cref{theorem:properties-of-fourier-transform-L1}.
\begin{example}{}\label{example:f-in-l1-but-f-transform-not}
  Let's see examples of functions $f\in L^{1}(\mathbb{R}^n)$ such that $\hat{f}\notin L^{1}(\mathbb{R}^n)$.
  \begin{itemize}
  	\item Consider $n=1$, $a\neq b$ and $f(x)=\chi_{(a,b)}(x)$, then:
  	\begin{align*}
  		\hat{f}(\xi)&=\int_{-\infty}^{\infty}f(x)e^{-2\pi x\xi}dx,\\
  		&=\int_{a}^{b}e^{-2\pi i x\xi}dx,\\
  		&=\left. \left(\frac{e^{-2\pi ix\xi}}{-2\pi i\xi}\right) \right|_{a}^{b},\\
  		&=-\frac{e^{-2\pi ib\xi}-e^{-2\pi ia\xi}}{2\pi i\xi},\\
  		&=\frac{\sin(\pi(a-b)\xi)}{\pi\xi}e^{-\pi i (a+b)\xi}.
  	\end{align*}
  \end{itemize}
  To exemplify that this function is not in $L^{1}(\mathbb{R})$ take the case $a=-1$ and $b=1$, then:
  $$\hat{f}(\xi)=\frac{sen(2\pi \xi)}{\pi \xi}.$$
  Now, note that:
  \begin{align*}
  	\frac{1}{2 \pi}\int_{1}^{\infty}\frac{1}{x}&=\int_{1}^{\infty}\frac{\frac{1}{2}}{\pi x},\\
  	&\leq \int_{-\infty}^{\infty}\left|\frac{sen(2\pi x)}{\pi x}\right|dx=\|\hat{f}\|_1.
  \end{align*}
  Then, if supose that $\hat{f}\in L^{1}(\mathbb{R})$, then $\frac{1}{2 \pi}\int_{1}^{\infty}\frac{1}{x}< \infty$ and by the integral test we hace that:
  \begin{align*}
  	\frac{1}{2 \pi}\sum_{k=1}^{\infty}\frac{1}{k}<\infty,
  \end{align*}
  What implies that the harmonic series converges, which is a contradiction, then we can conclude that $\hat{f}\notin L^{1}(\mathbb{R})$.  
\end{example}
\begin{proposition}{}
  Suppose $x_kf\in L^{1}(\mathbb{R}^n)$, where $x_k$ denotes the $k$-th coordinate of $x$, then $\hat{f}$ is differentiable with respect to $x_k$ and:
  \begin{equation}
    \frac{\partial \hat{f}}{\partial \xi_k}(\xi)=(\widehat{-2\pi i x_k f(x)})(\xi)\label{eq:2-8}.
  \end{equation}
  In the other words, the Fourier Transform of the product $x_k f(x)$ is equal to a multiple of the partial derivative of $\hat{f}(\xi)$ with respect to the $k$-th variable. 
\end{proposition}
\begin{proof} 
  Note that formally let's see that:
  \begin{align*}
    (\hat{-2\pi ix_kf(x)})(\xi)&=\int_{\mathbb{R}^{n}}-2\pi i x_kf(x)e^{-2\pi i(x\cdot \xi)}dx,\\
    &=\int_{\mathbb{R}^{n}}f(x)\frac{\partial e^{-2\pi i(x\cdot \xi)}}{\partial \xi_k}dx,\\
    &=\frac{\partial }{\partial \xi_k}\int_{\mathbb{R}^{n}}f(x)e^{-2\pi i(x\cdot \xi)}dx.\\
    &=\frac{\partial \hat{f}}{\partial \xi_k}(\xi).
  \end{align*}
  Therefore it will be sufficient prove that:
  \begin{equation}
    \frac{\partial}{\partial \xi_k}\int_{\mathbb{R}^{n}}f(x)e^{-2\pi i(x\cdot \xi)}dx=\int_{\mathbb{R}^{n}}\frac{\partial }{\partial \xi_k}f(x)e^{-2\pi i(x\cdot \xi)}dx.\label{eq:2-9}
  \end{equation}
  For prove this we will use the Lebesgue dominated convergence theorem.\\
  Note that using the mean value theorem and suppose $c\in [0,|h|]$:
  \begin{align*}
    g_h(x)&=\frac{1}{|h_k|}\left| f(x)e^{-2\pi i(x\cdot(\xi+h\epsilon_k))}-f(x)e^{-2\pi i(x\cdot\xi)} \right|,\\
    &=\frac{|f(x)|}{|h_k|}\left| e^{-2\pi i(x\cdot (\xi+h\epsilon_k))} - e^{-2\pi i(x\cdot\xi)} \right|,\\
    &\leq \left| \frac{f(x)}{h} \right|\left| -2\pi ix_ke^{-2\pi i(x\cdot (\xi+c\epsilon_k))}h \right|,\\
    &\leq \left|-2\pi x_kf(x)\right|,\\
    &\leq 2\pi \|x_k f(x)\|_{1}.
  \end{align*}
  That as $x_kf\in L^1(\mathbb{R}^{n})$ and $g_{h_k}\to \frac{\partial f(x)e^{-2\pi i(x\cdot \xi)}}{\partial \xi_k}$ when $h_k\to 0$, then we can ensure that the \cref{eq:2-9} is true, hence we can conclude the theorem.
\end{proof}
To consider the converse result, we need to introduce a definition.
\begin{definition}{}
	Let $1\leq p < \infty$. A function $f\in L^{p}(\mathbb{R}^n)$ is differentiable in $L^{p}(\mathbb{R}^n)$ with respect to the $k$-th variable, if there exist $g\in L^{p}(\mathbb{R}^n)$ such that:
	\begin{align*}
		\int_{\mathbb{R}^n}\left|\frac{f(x+h\epsilon_k)-f(x)}{h}-g(x)\right|^p dx\rightarrow0 
	\end{align*}
	when $h\to 0$.\\
	Where $\epsilon_k$ is $1$ in the $k$-th coordinate and in the other case $0$. Furthemore if such function $g$  exist (in this case unique) it's called the partial derivativeSuponer of $f$ with respect to $x_k$ in the norm $L^p$.
\end{definition}
\begin{theorem}{}
  Let $f\in L^1(\mathbb{R}^{n})$ and $g$ be its partial derivative with respect to the $k$-th variable in the $L^1$-norm. Then, $\hat{g}(\xi)=2\pi i\xi_k\hat{f}(\xi)$.
\end{theorem}
\begin{proof} 
  Using the properties \cref{property:f-transform-is-continuos-l1} and \cref{property:traslation-f-transform-l1} in Theorem \cref{theorem:properties-of-fourier-transform-L1} we can argue that:
  \begin{align*}
    \left| \hat{\frac{f(x+h\epsilon_k)-f(x)}{h}-g(x)} \right|&=\left| \frac{\hat{f(x+h\epsilon_k)}(\xi)-\hat{g}(\xi)}{h}-\hat{g}(\xi) \right|,\\
    &=\left| \frac{\hat{f}(k)e^{-2\pi ih(\epsilon_k\cdot \xi)}-\hat{f}(k)}{h}-\hat{g}(\xi) \right|,\\
    &=\left| \hat{f}(\xi)\frac{(e^{-2\pi ih(\epsilon_k\cdot \xi)}-1)}{h}-\hat{g}(\xi) \right|,\\
    &=\left| \hat{g}(\xi)-\hat{f}(\xi)\frac{(1-e^{-2\pi ih(\epsilon_k\cdot\xi)})}{h} \right|,
  \end{align*}
  now, take the limit $h\to 0$, then:
  \begin{align*}
    \lim_{h \rightarrow 0}\left| \hat{g}(\xi)-\hat{f}(\xi)\frac{(1-e^{-2\pi ih(\epsilon_k\cdot\xi)})}{h} \right|&=\left| \hat{g}(\xi)-(2\pi i\xi_k)\hat{f}(\xi) \right|=0,
  \end{align*}
  what it implies that $\hat{g}(\xi)=(2\pi i\xi_k)\hat{f}(\xi)$.
\end{proof}
Aquí sería bueno preguntarle al profesor Oscar como escribir esta propiedad.
Now we turn our attention to the following question: Give the Fourier transform $\hat{f}$ of function in $L^1(\mathbb{R}^{n})$, how we can recover $f$?\\
Perhaps it would be appropriate for the reader to suggest the following formula:
\begin{align*}
  f(x)=\int_{\mathbb{R}^{n}}\hat{f}(\xi)e^{2\pi i(x\cdot \xi)d\xi}.
\end{align*}
Unfortunately, $\hat{f}(\xi)$ may be nonintegrable (see \cref{example:f-in-l1-but-f-transform-not}). To avoid this problem, one needs to use the so called method of summability (Abel and Gauss) similar to those used in the study Fourier series. Combining the ideas behind the Gauss summation method and the identities \cref{property:traslation-f-transform-l1}, \cref{property:change-transform-l1} and:
\begin{equation}
  \hat{e^{-4\pi^2t|x|^2}}(\xi)=\frac{e^{-\frac{|\xi|^2}{4t}}}{(4\pi t)^{n/2}},
\end{equation}
we obtain the following equalities:
\begin{align*}
  f(x)&=\lim_{t \rightarrow 0}\frac{e^{-\frac{|\cdot|^2}{4t}}}{(4\pi t)^{n/2}}*f(x),\\
  &=\lim_{t \rightarrow 0}\int_{\mathbb{R}^{n}}\frac{e^{-\frac{|x-y|^2}{4t}}}{(4\pi t)^{n/2}}f(y)dy,\\
  &=\lim_{t \rightarrow 0}\int_{\mathbb{R}^{n}}\tau_{x}\frac{e^{-\frac{|y|^2}{4t}}}{(4\pi t)^{n/2}}f(y)dy,\\
  &=\lim_{t \rightarrow 0}\int_{\mathbb{R}^{n}}(\hat{e^{2\pi i(x\cdot \xi)}e^{-4\pi^2t|\xi|^2}})(y)f(y)dy,\\
  &=\lim_{t \rightarrow 0}\int_{\mathbb{R}^{n}}e^{2\pi i(x\cdot \xi)}e^{-4\pi^2t|\xi|^2}\hat{f}(\xi)d\xi,\\
  &=\lim_{t \rightarrow 0}\hat{f}(\xi)e^{2\pi i(x\cdot \xi)}e^{-4\pi^2t|\xi|^2}d\xi,
\end{align*}
where the limit is taken in the $L^1$-norm.\\
Thus, if $f$ and $\hat{f}$ are both integrable, the Lebesgue dominated convergence theorem guarantees the point-wise equality. Also, if $f \in L^{1}(\mathbb{R}^{n})$ is continuous at the point $x_0$, we get:
\begin{align*}
  f(x_0)=\lim_{t \rightarrow 0}\frac{e^{-|\cdot|^2/4t}}{(4\pi t)^{n/2}}*f(x_0)=\lim_{t \rightarrow 0}\int_{\mathbb{R}^{n}}\hat{f}(\xi)e^{2\pi i(x_0\cdot \xi)}e^{-4\pi^2t|\xi|^2}d\xi.  
\end{align*}
Collecting this information, we get the following result.
\begin{proposition}{}
  Let $f\in L^{1}(\mathbb{R}^{n})$. Then,
  \begin{align*}
    f(x)=\lim_{t \rightarrow 0}\int_{\mathbb{R}^{n}}\hat{f}(\xi)e^{2\pi i (x\cdot \xi)}e^{-4\pi^2 t |\xi|^2}d\xi,
  \end{align*}
  where the limit is taken in the $L^1$-norm. Moreover, if $f$ is continuous at the point $x_0$, then the following point-wise equality holds:
  \begin{align*}
    f(x_0)=\lim_{t \rightarrow 0}\int_{\mathbb{R}^{n}}\hat{f}(\xi)e^{2\pi i (x_0\cdot \xi)}e^{-4\pi^2 t |\xi|^2}d\xi.
  \end{align*}
\end{proposition}
let $f,\hat{f}\in L^1(\mathbb{R}^{n})$. Then,
\begin{align*}
  f(x)=\int_{\mathbb{R}^{n}}\hat{f}(\xi)e^{2\pi i(x\cdot \xi)}d\xi, \text{ almost everywhere }x\in\mathbb{R}^{n}.
\end{align*}
From this result and \cref{theorem:properties-of-fourier-transform-L1} we can conclude that
\begin{align*}
  \hat{}:L^1(\mathbb{R}^{n})\to C_{\infty}(\mathbb{R}^{n})
\end{align*}
is a linear, one-to-one, bounded map. However, it is not surjective.
