\lipsum[5]
\begin{definition}{}
  Let $\phi$ be an integrable function on $\mathbb{R}^n$ such that $\int_{\mathbb{R}^n}\phi=1$, and define for all $t>0$ 
  $$\phi_t(x)=t^{-n}\phi(t^{-1}x).$$ 
  If $t\rightarrow 0$, $\phi_t$ converges in the distributional sense of $\mathcal{S}'$ to $\delta$, the Dirac measure at the origin, and the pointwise limit holds:
  $$\lim_{t\rightarrow0}\phi_t*g(x)=g(x).$$
  Then, $\{\phi_t|t>0\}$ is an approximation of the identity.
\end{definition}
Let’s see that this makes sense in practice, because:\\
If $g\in\mathcal{S}(\mathbb{R}^{n})$, then:
\begin{align*}
  \phi_t(g)&=\int_{\mathbb{R}^n}t^{-n}\phi(t^{-1}x)g(x)dx && \text{Taking $y=t^{-1}x$, then $x=ty$ and $dy=t^{-n}dx$}\\
  &=\int_{\mathbb{R}^n}\phi(y)g(ty)dy
\end{align*}
Then, by the dominated convergence theorem, we can conclude that:
\begin{align*}
  \lim_{t\rightarrow 0} \phi_t(g)&=\int_{\mathbb{R}^n}\phi(y)\lim_{t\rightarrow0}g(ty)dy\\
  &=\int_{\mathbb{R}^n}\phi(y)g(0)dy\\
  &=g(0)\int_{\mathbb{R}^n}\phi(y)dy\\
  &=g(0)\\
  &=\delta(g).
\end{align*}
Since $\delta*g=g$, for any $g\in\mathcal{S}$, because
$$\delta * g(x)=\delta (g(x-\cdot))=\delta (\tau_x \tilde{g})=\tau_x(\tilde{g}(0))=g(x).$$
Recall that:
$$\tau_x g(y)=g(y-x)\qquad \text{and} \qquad \tilde{g}(x)=g(-x).$$
This indicates that pointwise we have to:\\
$$\lim_{t\rightarrow 0}\phi_t*g(x)=g(x).$$
Now, let’s see an important result:
\begin{theorem}{}
  Let $\{\phi_t: t>0\}$ be an approximation of the identity, then:
  $$\lim_{t\rightarrow 0}||\phi_t*f-f||_p=0,$$
  if $f\in\mathcal{L}^p$, with $1\leq p < \infty$, $f$ is continuous and tends to $0$ at infinity, that is, $f\in C_0(\mathbb{R}^n)$ (For the case $p=\infty$, the convergence will be uniform).
\end{theorem}
\begin{proof}{}
  Note that:
  \begin{align*}
    \phi_t*f(x)-f(x)&=\int_{\mathbb{R}^n}\phi_t(x-y)f(y)dy-\int_{\mathbb{R}^n}\phi(y)f(x)dy\\
    &=\int_{\mathbb{R}^n}t^{-n}\phi(t^{-1}(x-y))f(y)dy-\int_{\mathbb{R}^n}\phi(y)f(x)dy\\ 
    &\text{Taking $y=x-tz$, $z=t^{-1}(x-y)$ and $dz=t^{-n}dy$}\\
    &=\int_{\mathbb{R}^n}\phi(z)f(x-tz)dz-\int_{\mathbb{R}^n}\phi(z)f(x)dz\\
    &=\int_{\mathbb{R}^n}\phi(y)(f(x-ty)-f(x))dy
  \end{align*}
  Given $\epsilon>0$, there exists $\delta>0$ such that if $|h|<\delta$, then:
  $$||f(\cdot+h)-f(\cdot)||_p<\frac{\epsilon}{2||\phi||_1}.$$
  (Note that $\delta$ depends on $f$.) For some fixed $\delta$, note that if $t$ is small enough, then:
  $$\int_{|y|\geq \delta/t}|\phi(y)|dy\leq \frac{\epsilon}{4||f||_p}.$$
  Then, by the Minkowski inequality:
  \begin{align*}
    ||\phi_t*f-f||_p&\leq\int_{|y|<\delta/t}|\phi(y)| \,||f(\cdot+ty)-f( \cdot)||_pdy+2||f||_p\int_{|y|\geq \delta/t}|\phi(y)|dy\\
    &< \epsilon.
  \end{align*}
  This allows us to conclude that:
  $$||\phi_t*f-f||_p=0.$$  
\end{proof}
\begin{note}{}
  As a consequence of this theorem, we know that there exists a sequence $\{t_k\}$ that depends on $f$ such that $t_k\rightarrow 0$ and:
  $$\lim_{k\rightarrow \infty}\phi_k*f(x)=f(x) \quad \text{c.t.p.}$$
  Therefore, if the limit $\lim_{t\to 0} \phi_t*f(x)$ exists, it must equal $f(x)$ almost everywhere. (Our intention will be to study the existence of this limit and the independence of the sequence $\{t_k\}$ in future work).  
\end{note}
