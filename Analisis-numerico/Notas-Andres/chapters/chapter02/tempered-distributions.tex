 With the definitions of the Fourier transform that we have developed so far, it is natural to consider what happens with functions in $L^p(\mathbb{R}^n)$ where $1 \leq p \leq 2$. For this, it will be useful to observe the following result.
\begin{proposition}{}
  Let $f \in L^p(\mathbb{R}^n)$ with $1 \leq p \leq 2$. Then $f \in L^1(\mathbb{R}^n) + L^2(\mathbb{R}^n)$.
\end{proposition}
\begin{proof}{}
  Let $f \in L^p(\mathbb{R}^n)$. Note that $f$ can be decomposed as $f_1 + f_2$, defined as follows:
  \begin{align*}
    f_1(x) &= \begin{cases}
      f(x),\hspace{1cm} & \text{if } |f(x)| > 1. \\
      0,\hspace{1cm} & \text{otherwise.}
    \end{cases} \\
    f_2(x) &= \begin{cases}
      f(x),\hspace{1cm} & \text{if } |f(x)| \leq 1. \\
      0,\hspace{1cm} & \text{otherwise.}
    \end{cases}
  \end{align*}
  Now let's see that $f_1 \in L^1(\mathbb{R}^n)$:
  \begin{align*}
    \int_{\mathbb{R}^n}|f_1(x)|dx &\leq \int_{\{x \in \mathbb{R}^n: |f(x)| > 1\}} |f(x)| dx \\
    &\leq \int_{\{x \in \mathbb{R}^n: |f(x)| > 1\}} |f(x)|^p dx && \text{Since if } |f(x)| > 1, |f(x)| \leq |f(x)|^p \\
    &\leq \int_{\mathbb{R}^n} |f(x)|^p dx = \|f\|_p^p.
  \end{align*}
  Thus, $f_1 \in L^{1}(\mathbb{R}^n)$. \\
  Now let's see that $f_2 \in L^2(\mathbb{R}^n)$:
  \begin{align*}
    \int_{\mathbb{R}^n}|f_2(x)|^2dx &\leq \int_{\{x \in \mathbb{R}^n: |f(x)| \leq 1\}} |f(x)|^2 dx \\
    &\leq \int_{\{x \in \mathbb{R}^n: |f(x)| \leq 1\}} |f(x)|^p dx && \text{Since if } |f(x)| \leq 1, |f(x)|^2 \leq |f(x)|^p \\
    &\leq \int_{\mathbb{R}^n} |f(x)|^p dx = \|f\|_p^p.
  \end{align*}
  Thus, $f_2 \in L^2(\mathbb{R}^n)$, which allows us to conclude that $f \in L^1(\mathbb{R}^n) + L^2(\mathbb{R}^n)$.
\end{proof}
The above allows us to extend our definitions of the Fourier transform to any function $f \in L^p(\mathbb{R}^n)$, with $1 \leq p \leq 2$, since we can write $f = f_1 + f_2 \in L^1(\mathbb{R}^n) + L^2(\mathbb{R}^n)$, using the Fourier transform in the sense of $L^1(\mathbb{R}^n)$ for $f_1$ and the Fourier transform in the sense of $L^2(\mathbb{R}^n)$ for $f_2$. However, we still need to demonstrate the uniqueness of this transform and see to what extent it preserves the properties we developed earlier.\\
On the other hand, one would like to extend the Fourier transform to spaces $L^p(\mathbb{R}^n)$ with $p > 2$. However, as we have already seen, the Fourier operator is discontinuous in the spaces $L^p(\mathbb{R}^n)$ with $p > 2$, which indicates that we cannot extend our definition in the same way we did with $L^2(\mathbb{R}^n)$. Thus, from now on, we will focus on demonstrating that the Fourier transform for $L^p(\mathbb{R}^n)$ spaces exists in the distributional sense. However, the codomain of this will not be a functional space, but rather a space of distributions, which we will call \textbf{Tempered Distributions}. On the other hand, we will first focus on studying to what extent we can apply \cref{definition:transform-of-f}.
\begin{example}{}
  Take $n \geq 1$ and $f(x) = \delta_0(x)$, the delta function, i.e., the measure of a unit mass concentrated at the origin. Using \cref{definition:transform-of-f}, we have that:
  \begin{align*}
    \hat{\delta_0}(\xi) &= \int_{\mathbb{R}^n} \delta_0(x)e^{-2\pi i (x \cdot \xi)}dx \\
    &= e^{-2\pi i (0 \cdot \xi)} \\
    &= 1.
  \end{align*}
  \newline
  In fact, \cref{definition:transform-of-f} tells us that if $\mu$ is a bounded measure, then $\hat{\mu}$ represents a function in $L^\infty(\mathbb{R}^n)$. \\
  \begin{align*}
    \hat{\mu}(\xi) &= \int_{\mathbb{R}^n} e^{-2\pi i (x \cdot \xi)}d\mu(x).
  \end{align*}
  Then:
  \begin{align*}
    |\hat{\mu}(\xi)| &\leq \int_{\mathbb{R}^n} |e^{-2\pi i (x \cdot \xi)}|d\mu(x) \\
    &\leq \int_{\mathbb{R}^n} d\mu(x) \\
    &\leq \mu(\mathbb{R}^n) \\
    &< \infty.
  \end{align*}
  Thus:
  \begin{align*}
    \sup_{\xi \in \mathbb{R}^n} |\hat{\mu}| < \infty,
  \end{align*}
  meaning $\hat{\mu} \in L^{\infty}(\mathbb{R}^n)$.
\end{example}
Now, suppose that given $f \equiv 1$, we want to find $\hat{f}(\xi)$. In this case, \cref{definition:transform-of-f} cannot be used directly, so it will be necessary to introduce the notion of tempered distribution.\\
To begin with this, we will first study the following family of seminorms.
\begin{definition}{}
  For each $(\nu,\beta) \in (\mathbb{Z}^+ \cup \{0\})^{2n}$, we will denote the seminorm $\seminorm{\cdot}_{(\nu,\beta)}$ defined as:
  $$ \seminorm{f}_{(\nu,\beta)} = \|x^\nu \partial^\beta f\|_{\infty}. $$ 
\end{definition}
Now, using this, we will define the Schwartz space $\mathcal{S}(\mathbb{R}^n)$ as the space of functions $C^{\infty}(\mathbb{R}^n)$ with infinite decay, that is:
\begin{align*}
  \mathcal{S}(\mathbb{R}^n) := \{\varphi \in C^{\infty}(\mathbb{R}^n): \seminorm{\varphi}_{(\nu,\beta)} < \infty, \text{ for all } (\mathbb{Z}^+ \cup \{0\})^{2n}\}
\end{align*} 
In this way, we can verify the following proposition. 
\begin{proposition}{}
  $C^{\infty}_{0}(\mathbb{R}^n) \subsetneq \mathcal{S}(\mathbb{R}^n)$.
\end{proposition}
\begin{proof}{}
  First, let's see that $C^{\infty}_{0}(\mathbb{R}^n) \subset \mathcal{S}(\mathbb{R}^n)$.\\
  Suppose $f \in C^{\infty}_{0}(\mathbb{R}^n)$, now, let's see that $|\|f\||_{(\nu,\beta)} < \infty$ for all $(\nu,\beta) \in (\mathbb{Z}^+ \cup \{0\})$.\\
  Note that since $Supp(f)$ is compact, there exists an $R > 0$ such that $Supp(f) \subseteq B_{R}(0)$. Furthermore, since $f \in C^{\infty}(\mathbb{R}^n)$, the quantity $|\partial^{\beta}f|$ achieves its maximum (in particular its supremum) in $B_{R}(0)$, thus:
  \begin{align*}
    |x^{\nu} \partial^{\beta}f| &\leq |x_1^{\nu_1}x_2^{\nu_2}\cdots x_n^{\nu_n}\partial^{\beta}f| \\
    &\leq |x_1|^{\nu_1}|x_2|^{\nu_2}|\cdots||x_n|^{\nu_n}|\partial^{\beta}f| \\
    &\leq |x|^{\nu_1}|x|^{\nu_2}|\cdots||x_n|^{\nu_n}|\partial^{\beta}f| \\
    &\leq |x|^{\nu_1 + \nu_2 + \cdots + \nu_n}|\partial^{\beta}f| \\
    &\leq |x|^{|\nu|}|\partial^{\beta}f| \\
    &\leq R^{|\nu|}\sup_{x \in B_{R}(0)}|\partial^{\beta}f| \\
    &< \infty
  \end{align*}
  Then $\sup_{x \in \mathbb{R}^n} |x^{\nu} \partial^{\beta} f| \leq R^{|\nu|} \sup_{x \in B_{R}(0)} |\partial^{\beta} f| < \infty$, so we can conclude that $f \in \mathcal{S}(\mathbb{R}^n)$.\\
  Now, we only need to find a function $f \in \mathcal{S}(\mathbb{R}^n)$ such that $f \notin C^{\infty}_{0}(\mathbb{R}^n)$.\\
  For this, we will use the Gaussian function $f(x) = e^{-|x|^2}$. First, let's see that $f \in \mathcal{S}(\mathbb{R}^n)$:
  To begin, note that using an inductive argument, we can ensure that:
  \begin{align*}
    e^{-|x|^2} &= e^{-(x_1^2 + x_2^2 + \cdots + x_n^2)}\\
    \frac{\partial (e^{-|x|^2})}{\partial x_i} &= -2x_i e^{-|x|^2}\\
    \frac{\partial^3 (e^{-|x|^2})}{\partial x_i^2 x_j} &= (4x_i^2 - 2)(-2x_j)e^{-|x|^2}\\
    &\vdots\\
    \frac{\partial^{|\beta|} (e^{-|x|^2})}{\partial x^{\beta}} &= P(x)e^{-|x|^2}\\
  \end{align*}
  where $P(x)$ is a polynomial.\\
  Now, let's see that $|\|f\||_{(\nu,\beta)} < \infty$ for all $\nu$ and $\beta$.\\
  First, note that given a polynomial $q(x)$ of degree $m > 0$, there always exists $c \in \mathbb{R}^+$ such that:
  \begin{align*}
    |q(x)| &\leq \left| \sum_{n=0}^{m} a_n x^n \right|\\
    &\leq \sum_{n=0}^{m} |a_n| |x|^n\\
    &\leq \begin{cases}
      \sum_{n=0}^{m} |a_n|, & \text{if } |x| \leq 1\\
      a_0 + |x|^m \left( \sum_{n=1}^{m} |a_n| \right), & \text{if } |x| > 1.
    \end{cases}\\
    &\leq c(1 + |x|^m)\\
    &\leq c(1 + |x|^2)^{\frac{m}{2}}\\
    &\leq c \langle x \rangle^m 
  \end{align*}
  Now, suppose $c \in \mathbb{R}^+$ and $m \in \mathbb{Z}^+$ is even, such that $c \langle x \rangle^m \geq |x^{\nu} P(x)|$ for all $x \in \mathbb{R}^n$:
  \begin{align*}
    |x^{\nu} P(x) e^{-|x|^2}| &\leq c \frac{\langle x \rangle^m}{e^{|x|^2}}
  \end{align*}
  Then, since $\left| \frac{c \langle x \rangle^m}{e^{|x|^2}} \right|$ is continuous on all of $\mathbb{R}^n$, it attains its maximum on any compact set. Furthermore:
  \begin{align*}
    \lim_{|x| \rightarrow \infty} \left| \frac{c \langle x \rangle^m}{e^{|x|^2}} \right| = 0
  \end{align*}
  Thus, we can assume that:
  \begin{align*}
    \|x^{\nu} \partial^{\beta} f\|_{\infty} &\leq \sup_{x \in \mathbb{R}^n} \left| \frac{c \langle x \rangle^m}{e^{|x|^2}} \right|\\
    &< \infty
  \end{align*}
  Hence, it has been shown that $f(x) = e^{-|x|^2} \in \mathcal{S}(\mathbb{R}^n)$.\\
  Then, since $e^{-|x|^2} \neq 0$ for all $x \in \mathbb{R}^n$, we have $Supp(e^{-|x|^2}) = \mathbb{R}^n$, so we can conclude that $e^{-|x|^2} \in \mathcal{S}(\mathbb{R}^n) \setminus C^{\infty}_{0}(\mathbb{R}^n)$.
  Then $\sup_{x\in\mathbb{R}^n}|x^{\nu}\partial^{\beta}f|\leq R^{|\nu|}\sup_{x\in B_{R}(0)}|\partial^{\beta}f|<\infty$.
\end{proof}
Now we will see how the topology of $\mathcal{S}(\mathbb{R}^n)$ is determined by the family of seminorms $|\|\cdot\||_{(\nu,\beta)}$ with $(\nu,\beta)\in(\mathbb{Z}^+\cup\{0\})^{2n}$.
\begin{definition}{}
  Let $\{\varphi_j\}\subset \mathcal{S}(\mathbb{R}^n)$, we will say that $\varphi_j \rightarrow 0$ as $j \rightarrow \infty$ if for every $(\nu,\beta)\in(\mathbb{Z}^+\cup\{0\})^{2n}$ it holds that:
  $$|\|\varphi_j\||_{(\nu,\beta)}\rightarrow 0, \hspace{1cm} \text{as }\hspace{0.5cm}j\rightarrow \infty$$
\end{definition}
\begin{note}{}
  Similarly, we will say that $\varphi_j \rightarrow \varphi$ as $j\rightarrow \infty$ if and only if $|\|\varphi_j - \varphi \||_{(\nu,\beta)}\rightarrow 0$ as $j\rightarrow \infty$.
\end{note}
Now we will study the following proposition.
\begin{proposition}{}
  Let $d:(\mathcal{S}(\mathbb{R}^n))^2\rightarrow \mathbb{R}^+\cup \{0\}$ defined as follows:
  \begin{align*}
    d(\varphi,\psi)=\sum_{\nu,\beta}\frac{1}{2^{|\nu|+|\beta|}}\frac{|\|\varphi-\psi\||_{(\nu,\beta)}}{1+|\|\varphi-\psi\||_{(\nu,\beta)}}
  \end{align*}
  Then the following conditions hold:
  \begin{enumerate}
    \item $d$ defines a metric for $\mathcal{S}(\mathbb{R}^n)$.
    \item $d$ generates a topology on $\mathcal{S}(\mathbb{R}^n)$ equivalent to the topology generated by the seminorms of $\mathcal{S}(\mathbb{R}^n)$.
  \end{enumerate}
\end{proposition}
\begin{proof}{}
	\begin{enumerate}
		\item Let us see that $d$ indeed defines a metric for $\mathcal{S}(\mathbb{R}^n)$.\\
		  \begin{itemize}
			  \item First, we will show that $d$ is well-defined for any $(\varphi,\psi)\in (\mathcal{S}(\mathbb{R}^n))^2$, which means:
			    \begin{align*}
				    d(\varphi,\psi)&\leq\sum_{\nu,\beta}\frac{1}{2^{|\nu|+|\beta|}}\frac{|\|\varphi-\psi\||_{(\nu,\beta)}}{1+|\|\varphi-\psi\||_{(\nu,\beta)}}\\
				    &\leq \sum_{\nu,\beta}\frac{1}{2^{|\nu|+|\beta|}}\\
				    &\leq \sum_{\nu,\beta}\frac{1}{2^{|\nu|}}\frac{1}{2^{|\beta|}}\\
				    &\leq \sum_{\nu\in(\mathbb{Z}^+\cup\{0\})^n}\frac{1}{2^{|\nu|}} \sum_{\beta\in(\mathbb{Z}^+\cup\{0\})^n}\frac{1}{2^{|\beta|}}\\
				    &\leq \left(\sum_{\nu\in(\mathbb{Z}^+\cup\{0\})^n}\frac{1}{2^{|\nu|}}\right)^2\\
				    &\leq\left( \sum_{\nu_1=0}^{\infty}\sum_{\nu_2=0}^{\infty}\cdots\sum_{\nu_n=0}^{\infty}\frac{1}{2^{\nu_1+\nu_2+\cdots+\nu_n}}\right)^2\\
				    &\leq\left(\sum_{k=0}^{\infty}\frac{1}{2^k}\right)^{2n}\\
				    &<\infty
			    \end{align*}
			    Therefore, we can ensure that $d$ is well-defined for all $(\varphi,\psi)\in(\mathcal{S}(\mathbb{R}^n))^2$\\
			  \item Now let us see that $d(\varphi,\psi)=0$ if and only if $\varphi=\psi$.\\
			    Note that, in the definition of $d$, all summands are positive real numbers, so we can affirm that $d(\varphi,\psi)=0$ if and only if $|\|\varphi-\psi\||_{(\nu,\beta)}=0$ for all $(\nu,\beta)\in(\mathbb{Z}^+\cup \{0\})^{2n}$.\\
			    Now, assume $\nu=0$ and $\beta=0$, then:
			    \begin{align*}
				    |\|\varphi-\psi\||_{(0,0)}&=\|x^{0}\partial^{0}(\varphi-\psi)\|_{\infty}\\
				    &=\|\varphi-\psi\|_{\infty}\\
				    &=\sup_{x\in\mathbb{R}^n}|\varphi-\psi|=0
			    \end{align*}
			    Then $\varphi(x)=\psi(x)$ for all $x\in\mathbb{R}^n$.
			  \item Note that $d(\varphi,\psi)=d(\psi,\varphi)$ follows immediately, since $|\|\varphi-\psi\||_{(\nu,\beta)}=|\|\psi-\varphi\||_{(\nu,\beta)}$ for all $\nu$ and $\beta$.
			  \item Now let us see that $d$ satisfies the triangle inequality.\\
			    Using the triangle inequality of the seminorms $|\|\cdot\||_{(\nu,\beta)}$, we have:
		      \begin{align*}
				    |\|\varphi-\psi\||_{(\nu,\beta)}&\leq |\|\varphi-\phi\||_{(\nu,\beta)}+|\|\phi-\psi\||_{(\nu,\beta)}\\
				    1+|\|\varphi-\psi\||_{(\nu,\beta)}&\leq 1+ |\|\varphi-\phi\||_{(\nu,\beta)}+|\|\phi-\psi\||_{(\nu,\beta)}\\
				    \frac{1}{1+ |\|\varphi-\phi\||_{(\nu,\beta)}+|\|\phi-\psi\||_{(\nu,\beta)}}&\leq \frac{1}{1+|\|\varphi-\psi\||_{(\nu,\beta)}}\\
				    -\frac{1}{1+|\|\varphi-\psi\||_{(\nu,\beta)}}&\leq -\frac{1}{1+ |\|\varphi-\phi\||_{(\nu,\beta)}+|\|\phi-\psi\||_{(\nu,\beta)}}
			    \end{align*}
			    Then:
			    \begin{align*}
				    d(\varphi,\psi)&\leq \sum_{\nu,\beta}\frac{1}{2^{|\nu|+|\beta|}}\frac{|\|\varphi-\psi\||_{(\nu,\beta)}}{1+|\|\varphi-\psi\||_{(\nu,\beta)}}\\
				    &\leq \sum_{\nu,\beta}\frac{1}{2^{|\nu|+|\beta|}}\left(1-\frac{1}{1+|\|\varphi-\psi\||_{(\nu,\beta)}}\right)\\
				    &\leq \sum_{\nu,\beta}\frac{1}{2^{|\nu|+|\beta|}}\left(1-\frac{1}{1+|\|\varphi-\phi\||_{(\nu,\beta)}+|\|\phi-\psi\||_{(\nu,\beta)}}\right)\\
				    &\leq\sum_{\nu,\beta}\frac{1}{2^{|\nu|+|\beta|}}\frac{|\|\varphi-\phi\||_{(\nu,\beta)}+|\|\phi-\psi\||_{(\nu,\beta)}}{1+|\|\varphi-\phi\||_{(\nu,\beta)}+|\|\phi-\psi\||_{(\nu,\beta)}}\\
				    &\leq\sum_{\nu,\beta}\frac{1}{2^{|\nu|+|\beta|}}\frac{|\|\varphi-\phi\||_{(\nu,\beta)}}{1+|\|\varphi-\phi\||_{(\nu,\beta)}+|\|\phi-\psi\||_{(\nu,\beta)}}\\
            &\phantom{\leq}+\sum_{\nu,\beta}\frac{1}{2^{|\nu|+|\beta|}}\frac{|\|\phi-\psi\||_{(\nu,\beta)}}{1+|\|\varphi-\phi\||_{(\nu,\beta)}+|\|\phi-\psi\||_{(\nu,\beta)}}\\
				    &\leq\sum_{\nu,\beta}\frac{1}{2^{|\nu|+|\beta|}}\frac{|\|\varphi-\phi\||_{(\nu,\beta)}}{1+|\|\varphi-\phi\||_{(\nu,\beta)}}+\sum_{\nu,\beta}\frac{1}{2^{|\nu|+|\beta|}}\frac{|\|\phi-\psi\||_{(\nu,\beta)}}{1+|\|\phi-\psi\||_{(\nu,\beta)}}\\
				    &\leq d(\varphi,\phi)+d(\phi,\psi)
			    \end{align*}
			    Thus, we can conclude that $d$ is a metric for the space $\mathcal{S}(\mathbb{R}^n)$.\\
		  \end{itemize}
		\item Now let us see that $d$ generates the same topology for $\mathcal{S}(\mathbb{R}^n)$ as the seminorms of $\mathcal{S}(\mathbb{R}^n)$.
		  \begin{itemize}
			  \item First, we will see that if $\{\varphi_j\}\rightarrow \varphi$ as $j\rightarrow \infty$, then $d(\varphi_j,\varphi)\rightarrow 0$ as $j\rightarrow \infty$.\\
			    Note that since the series $\sum_{\nu,\beta}\frac{1}{2^{|\nu|+|\beta|}}$ converges, we can ensure that given $\epsilon >0$, there exists $N>0$ such that:
			    \begin{align*}
				    \sum_{|\nu|,|\beta|> N}\frac{1}{2^{|\nu|+|\beta|}}<\frac{\epsilon}{M}
			    \end{align*}
			    On the other hand, since $\varphi_j\rightarrow \varphi$, for $j\geq N$ we can obtain:
			    \begin{align*}
            d(\varphi_j,\varphi)&\leq \sum_{\nu,\beta}\frac{1}{2^{|\nu|+|\beta|}}\frac{|\|\varphi_j-\varphi\|_{(\nu,\beta)}}{1+|\|\varphi_j-\varphi\||_{(\nu,\beta)}}\\
            &\leq \sum_{|\nu|,|\beta|\leq N}\frac{1}{2^{|\nu|+|\beta|}}\frac{|\|\varphi_j-\varphi\|_{(\nu,\beta)}}{1+|\|\varphi_j-\varphi\||_{(\nu,\beta)}}\\
            &\phantom{\leq}+ \sum_{|\nu|,|\beta|>N}\frac{1}{2^{|\nu|+|\beta|}}\frac{|\|\varphi_j-\varphi\|_{(\nu,\beta)}}{1+|\|\varphi_j-\varphi\||_{(\nu,\beta)}}\\
            &\leq \sum_{|\nu|,|\beta|\leq N}\frac{1}{2^{|\nu|+|\beta|}}|\|\varphi_j-\varphi\|_{(\nu,\beta)} + \sum_{|\nu|,|\beta|>N}\frac{1}{2^{|\nu|+|\beta|}}\\
            &<\frac{\epsilon}{M}\left(\sum_{|\nu|,|\beta|\leq N}\frac{1}{2^{|\nu|+|\beta|}}+1\right)\\
            &\phantom{<}\text{Taking $M=\left(\sum_{|\nu|,|\beta|\leq N}\frac{1}{2^{|\nu|+|\beta|}}+1\right)$.}\\
            &<\epsilon
			    \end{align*}
			    Thus, we can conclude that $d(\varphi_j,\varphi)\rightarrow 0$ as $j\rightarrow \infty$.
			  \item Now let us see the converse, which means that if $d(\varphi_j,\varphi)\rightarrow 0$ as $j\rightarrow \infty$, then $\varphi_j\rightarrow\varphi$ as $j\rightarrow \infty$.\\
			    Suppose $d(\varphi_j,\varphi)\rightarrow 0$ when $j\rightarrow \infty$, then given multi-index $(\tilde{\nu},\tilde{\beta})$ exists $j_0> 0$ and $\epsilon\in(0,2^{-(|\nu|+|\beta|+1)})$ such that $d(\varphi_j,\varphi)<\epsilon$, then for everything $j>j_0$ it is true that:
			    \begin{align*}
            \frac{1}{2^{|\tilde{\nu}|+|\tilde{\beta}|}}\frac{|\|\varphi_j-\varphi\||_{(\tilde{\nu},\tilde{\beta})}}{1+|\|\varphi_j-\varphi\||_{(\tilde{\nu},\tilde{\beta})}}&\leq \sum_{\nu,\beta}\frac{1}{2^{|\nu|+|\beta|}}\frac{|\|\varphi_j-\varphi\||_{(\nu,\beta)}}{1+|\|\varphi_j-\varphi\||_{(\nu,\beta)}}\\
    				&< \epsilon
			    \end{align*}
			    From here it can be inferred that:
          \begin{align*}
            |\|\varphi_j-\varphi\||_{(\tilde{\nu},\tilde{\beta})}&< \epsilon (2^{|\tilde{\nu}|+|\tilde{\beta}|})(1+|\|\varphi_j-\varphi\||_{(\tilde{\nu},\tilde{\beta})})\\
            &< \epsilon (2^{|\tilde{\nu}|+|\tilde{\beta}|})+\epsilon(2^{|\tilde{\nu}|+|\tilde{\beta}|})(|\|\varphi_j-\varphi\||_{(\tilde{\nu},\tilde{\beta})})\\
            (|\|\varphi_j-\varphi\||_{(\tilde{\nu},\tilde{\beta})})(1-\epsilon(2^{|\tilde{\nu}|+|\tilde{\beta}|}))&< \epsilon(2^{|\tilde{\nu}|+|\tilde{\beta}|})
          \end{align*}
          Then as $\epsilon < 2^{-(|\nu|+|\beta|+1)}$, so $(1-\epsilon(2^{|\tilde{\nu}|+|\tilde{\beta}|}))\geq \frac{1}{2}$, hence:
          \begin{align*}
            \frac{1}{2}(|\|\varphi_j-\varphi\||_{(\tilde{\nu},\tilde{\beta})})&< \epsilon (2^{|\tilde{\nu}|+|\tilde{\beta}|})\\
            (|\|\varphi_j-\varphi\||_{(\tilde{\nu},\tilde{\beta})})&< \epsilon (2^{|\tilde{\nu}|+|\tilde{\beta}|+1})
          \end{align*}
          then for $j>j_0$ is true that $\lim_{j\rightarrow \infty}|\|\varphi_j-\varphi\||_{(\nu,\beta)}=0$, so we can assure $\varphi_j \rightarrow \varphi$ as $j\rightarrow \infty$.
		  \end{itemize}
	\end{enumerate}
\end{proof}
Before going to the main result, let's review the following proposition that will help us to prove the next theorem.
\begin{proposition}{}
	Let $\phi\in\mathcal{S}$, then there exists $c>0$, and $m>\frac{n}{2}$ such that:
	$$\|x^{\nu}\partial^{\beta}\phi\|_{1}\leq c(|\|\phi\||_{(\nu,\beta)}+\sum_{|\gamma|\leq 2m}|\|\phi\||_{(\nu+\gamma,\beta)})$$
\end{proposition}
\begin{proof}{}
	\begin{align*}
		\|x^{\nu}\partial^{\beta}\phi\|_{1}&\leq \|\frac{1+|x|^{2m}}{1+|x|^{2m}}x^{\nu}\partial^{\beta}\phi\|_{1} &&\text{Using Hölder's inequality.}\\
		&\leq \|\frac{1}{1+|x|^{2m}}\|_{1}\|(1+|x|^{2m})x^{\nu}\partial^{\beta}\phi\|_{\infty}\\
		&\leq c(\|x^{\nu}\partial^{\beta}\phi\|_{\infty}+\||x|^{2m}x^{\nu}\partial^{\beta}\phi\|_{\infty})\\
		&\leq c(|\|\phi\||_{(\nu,\beta)}+\sum_{|\gamma|<2m}\|x^{\nu+\gamma}\partial^{\beta}\phi\|_{\infty})\\
		&\leq c(|\|\phi\||_{(\nu,\beta)}+\sum_{|\gamma|<2m}|\|\phi\||_{(\nu+\gamma,\beta)})
	\end{align*}
\end{proof}
Now, the relationship between the space of $\mathcal{S}(\mathbb{R}^n)$ and the Fourier transform will be more natural after seeing the following result.
\begin{theorem}{}
	The Fourier transform $\phi\rightarrow \hat{\phi}$ is an isomorphism of $\mathcal{S}(\mathbb{R}^n)$ onto itself.
\end{theorem}
\begin{proof}{}
	\begin{itemize}
		\item First, let's see that if $\phi\in\mathcal{S}(\mathbb{R}^n)$, then $\hat{\phi}\in\mathcal{S}(\mathbb{R}^n)$.\\
	    Note that since $\mathcal{S}(\mathbb{R}^n)\subsetneq \mathcal{L}^p(\mathbb{R}^n)$ with $1\leq p \leq \infty$, then $\phi\in\mathcal{L}(\mathbb{R}^n)$, so we can use the Fourier transform properties in the sense of $\mathcal{L}(\mathbb{R}^n)$, as follows:
	    \begin{align*}
	      \partial^{\beta}\hat{\phi}(\xi)&=[\hat{(-2\pi i x)^{\beta}\phi(x)}](\xi)
	    \end{align*}
	    Then, for all $\nu\in(\mathbb{Z}^+\cup\{0\})^n$, we have that $x^\nu \phi \in \mathcal{L}(\mathbb{R}^n)$, so we can affirm that $\hat{\phi}\in C^{\infty}$.\\
	    Now let's see that $|\|\hat{\phi}\||_{(\nu,\beta)}<\infty$ for all $(\nu,\beta)\in(\mathbb{Z}^+\cup\{0\})^{2n}$.
	    \begin{align*}
	    	\sup_{\xi\in\mathbb{R}^n}|\xi^{\nu}\partial^{\beta}\hat{\phi}(\xi)|&\leq \sup_{\xi\in\mathbb{R}^n}|\xi^{\nu}(\hat{(-2\pi i x)^{\beta}\phi(x)})(\xi)|\\
	    	&\leq \sup_{\xi\in\mathbb{R}^n}\left|\frac{[\hat{\partial^{\nu}(-2\pi i x)^{\beta}\phi(x)}]}{(2\pi i)^{\nu}}(\xi)\right|
	    \end{align*}
	    Since $x^{\beta}\phi\in\mathcal{S}(\mathbb{R}^n)$, $\partial^{\nu}(-2\pi ix)^{\beta}\phi\in\mathcal{L}(\mathbb{R}^n)$, we can ensure that $\hat{\partial^{\nu}(-2\pi ix)^{\beta}\phi}\in\mathcal{L}^{\infty}(\mathbb{R}^n)$, meaning that for all $(\nu,\beta)\in(\mathbb{Z}^+\cup\{0\})^{2n}$ we have $|\|\hat{\phi}\||_{(\nu,\beta)}<\infty$.
		\item Now let's see that the operator is injective.\\
		  Suppose $\hat{\phi}=\hat{\psi}$, then, since $\phi,\psi\in\mathcal{L}(\mathbb{R}^n)$, using the inversion formula, we have:
		  \begin{align*}
			  \phi(x)&=\lim_{t\rightarrow 0}\int_{\mathbb{R}^n}\hat{\phi}(\xi)e^{2\pi i (x\cdot \xi)}e^{-4\pi^2t|\xi|^2}d\xi\\
			  &=\lim_{t\rightarrow 0}\int_{\mathbb{R}^n}\hat{\psi}(\xi)e^{2\pi i (x\cdot \xi)}e^{-4\pi^2t|\xi|^2}d\xi\\
			  &=\psi(x)
		  \end{align*}
		  Thus, we can conclude that $\phi=\psi$.
		\item Now, let's see that the operator is surjective.\\
		  Similarly to how we demonstrated that if $\phi\in\mathcal{S}(\mathbb{R}^n)$, then $\hat{\phi}\in\mathcal{S}(\mathbb{R}^n)$, it can be shown that if $\phi\in\mathcal{S}(\mathbb{R}^n)$, then $\check{\phi}\in\mathcal{S}(\mathbb{R}^n)$, so for every $\phi\in\mathcal{S}(\mathbb{R}^n)$, there exists $\check{\phi}\in\mathcal{S}(\mathbb{R}^n)$ such that $\hat{\check{\phi}}=\phi$, thus demonstrating that the Fourier transform operator is surjective.
		\item Now let's see that continuity is maintained.\\
		  This means that if we take $\{\phi_j\}\subset \mathcal{S}(\mathbb{R}^n)$, such that $\phi_j \rightarrow \phi$ when $j\rightarrow \infty$, then $\hat{\phi_j}\rightarrow \hat{\phi}$ when $j\rightarrow \infty$.\\
		  Thus, we want to see that for all $(\nu,\beta)\in(\mathbb{Z}\cup\{0\})^{2n}$, $|\|\hat{\phi_j}-\hat{\phi}\||_{(\nu,\beta)}\rightarrow 0$ as $j\rightarrow \infty$. Note that:
      \begin{align*}
			  \|\xi^{\nu}\partial^{\beta}(\hat{\phi_j}-\hat{\phi})\|_{\infty}&=\|\xi^{\nu}\partial^{\beta}(\hat{\phi_j-\phi})\|_{\infty}\\
			  &=\|\xi^{\nu}(\hat{(-2\pi ix)^{\beta}(\phi_j-\phi)})\|_{\infty}\\
			  &=\|\frac{(-2\pi i)^{\beta}\hat{\partial^{\nu}[(x^\beta)(\phi_j-\phi)]}}{(2\pi i)^{\nu}}\|_{\infty}\\
			  &\leq\|\frac{(-2\pi i)^{\beta}\partial^{\nu}[(x^\beta)(\phi_j-\phi)]}{(2\pi i)^{\nu}}\|_{1}\\
			  &\leq c\|\partial^{\nu}[(x^\beta)(\phi_j-\phi)]\|_{\infty}+c\sum_{|\gamma|\leq 2m}\|\partial{\nu+\gamma}[(x^\beta)(\phi_j-\phi)]\|_{\infty}\\
			  &\leq k \sum_{|\gamma_1|,|\gamma_2|\leq2m+|\nu|+|\beta|}|\|\phi_j-\phi\||_{(\gamma_1,\gamma_2)}
		  \end{align*}
		  Since for all $(\gamma_1,\gamma_2)\in(\mathbb{Z}\cup\{0\})^{2n}$, $|\|\phi_j-\phi\||_{(\gamma_1,\gamma_2)}\rightarrow 0$ as $j\rightarrow \infty$, then for all $(\nu,\beta)\in(\mathbb{Z}\cup\{0\})^{2n}$, we have that $|\|\hat{\phi_j}-\hat{\phi}\|_{(\nu,\beta)}\rightarrow 0$ as $j\rightarrow \infty$.\\
      Note that the converse case of continuity is analogous, so it is demonstrated that \phantom{  }$\hat{\empty}:\mathcal{S}(\mathbb{R}^n)\rightarrow\mathcal{S}(\mathbb{R}^n)$ is an isomorphism.
	\end{itemize}
\end{proof}
\begin{definition}{}
	We will say that $\psi:\mathcal{S}(\mathbb{R}^n)\rightarrow \mathbb{C}$ defines a tempered distribution, that is, $\psi\in\mathcal{S}'(\mathbb{R}^n)$ if:
	\begin{enumerate}
		\item $\psi$ is linear.
		\item $\psi$ is continuous, meaning that if we take $\{\varphi_l\}\subset\mathcal{S}(\mathbb{R}^n)$ such that $\varphi_j\rightarrow 0$ as $j\rightarrow \infty$, then the numerical sequence $\psi(\varphi_j)\rightarrow 0$ as $j\rightarrow \infty$.
	\end{enumerate}
\end{definition}
\begin{proposition}{}
	Let $\psi\in\mathcal{S}'(\mathbb{R}^n)$ be linear. Then, $\psi$ is continuous if and only if there exist $N\in(\mathbb{Z}^+\cup\{0\}),c>0$ such that for all $\varphi\in\mathcal{S}(\mathbb{R}^n)$, it holds that:
	$$|\psi(\varphi)|\leq c\sum_{|\nu|,|\beta|\leq N}|\|\varphi\||_{(\nu,\beta)}$$
\end{proposition}
\begin{proof}{}
	$\leftarrow$\\
	Let $\{\varphi_j\}\subset \mathcal{S}(\mathbb{R}^n)$ such that $\varphi_j\rightarrow \varphi$ as $j\rightarrow \infty$, then:
	\begin{align*}
		|\psi(\varphi_j)-\psi(\varphi)|&=|\psi(\varphi_j-\varphi)|\\
		&\leq c\sum_{|\nu|,|\beta|\leq N}|\|\varphi_j-\varphi\||_{(\nu,\beta)}
	\end{align*}
	Since $\varphi_j\rightarrow \varphi$ as $j\rightarrow \infty$, the finite sum converges to $0$, and thus $\psi(\varphi_j)\rightarrow 0$ as $j\rightarrow \infty$.\\
	$\rightarrow$\\
	By contradiction, suppose that if we take $\{\varphi_j\}\subset \mathcal{S}(\mathbb{R}^n)$ such that $\varphi_j\rightarrow \varphi$ as $j\rightarrow \infty$, then $\psi(\varphi_j)\rightarrow 0$ as $j\rightarrow \infty$, and that for all $N\in\mathbb{Z}^+\cup\{0\}$ and any constant $c>0$, we have:
	\begin{align*}
		|\psi(\varphi)|>c\sum_{|\nu|,|\beta|\leq N}|\|\varphi\||_{(\nu,\beta)}
	\end{align*}
	Let:
	\begin{align*}
		g_j=\frac{\varphi_j}{j(\sum_{|\nu|,|\beta|\leq J}|\|\varphi_j\||_{(\nu,\beta)})}
	\end{align*}
	Note that:
	\begin{itemize}
		\item $g_j\in\mathcal{S}(\mathbb{R}^n)$ since it is simply $\varphi_j$ scaled by a constant.
		\item $\psi(g_j)>1$ by the inequality of the hypothesis.
		\item $g_j\rightarrow 0$ as $j\rightarrow \infty$.\\
		  Let $\tilde{\nu}$ and $\tilde{\beta}$ be multi-indices such that $|\|g_j\||_{(\tilde{\nu},\tilde{\beta})}\rightarrow 0$ as $j\rightarrow \infty$, and take $j_0>|\tilde{\nu}|+|\tilde{\beta}|$. If we take $j>j_0$, then:
		\begin{align*}
			|\|g_j\||_{\tilde{\nu},\tilde{\beta}}&=\frac{|\|\varphi_j\||_{(\tilde{\nu},\tilde{\beta})}}{j\sum_{|\nu|,|\beta|\leq j}|\|\varphi_j\||_{(\nu,\beta)}}\\
			&\leq \frac{1}{j}
		\end{align*}
		Thus as $j\rightarrow \infty$, $g_j\rightarrow 0$
	\end{itemize}
	Therefore:
	$$|\psi(g_j)|\rightarrow 0 \hspace{1cm}\text{as $j\rightarrow \infty$}$$
	\textbf{Contradiction}, since $|\psi(g_j)|>1$ for all $j$.\\
	Hence, if $\psi$ is continuous, there exists $N\in\mathbb{Z}^+\cup\{0\}$ and $c>0$ such that for all $\varphi\in\mathcal{S}(\mathbb{R}^n)$ we have:
	$$|\psi(\varphi)|\leq c\sum_{|\nu|,|\beta|\leq N}|\|\varphi\||_{\nu,\beta}$$
\end{proof}
\begin{example}{}
	Let's see that every $f\in\mathcal{L}(\mathbb{R}^n)$ defines a tempered distribution.\\
	Define $\psi_{f}(\varphi)$ as:
	\begin{align*}
		\psi_{f}(\varphi)=\int_{\mathbb{R}^n}f(x)\varphi(x)dx
	\end{align*}
	Note that linearity is inherited from the linearity of the integral; furthermore, continuity holds because:
	\begin{align*}
		|\psi_{f}(\varphi)|&=|\int_{\mathbb{R}^n}f(x)\varphi(x)dx|\\
		&\leq\int_{\mathbb{R}^n}|f(x)||\varphi(x)|dx\\
		&\leq\|f\|_{1} \|\varphi\|_{\infty}\\
		&\leq c|\|\varphi\||_{(0,0)}
	\end{align*}
	Thus, it is well-defined, linear, and continuous, so $\psi_{f}(\varphi)\in \mathcal{S}'(\mathbb{R}^n)$.
\end{example}
\begin{example}{}
	In general, every $f\in\mathcal{L}^p(\mathbb{R}^n)$ defines a tempered distribution. If we define $\psi_{f}$ as in the previous example, we see that:
	\begin{align*}
		|\psi_{f}(\varphi)|&=|\int_{\mathbb{R}^n}f(x)\varphi(x)dx|\\
		&\leq\int_{\mathbb{R}^n}|f(x)||\varphi(x)|dx &&\text{Using Hölder's inequality.}\\
		&\leq\|f\|_{p} \|\varphi\|_{q}\\
		&\leq\|f\|_{p} \|\frac{1+|x|^{2m}}{1+|x|^{2m}}\varphi\|_{q}\\
		&\leq\|f\|_{p} \|\frac{1}{1+|x|^{2m}}\|_{q}\|(1+|x|^{2m})\varphi\|_{\infty}
	\end{align*}
	Thus, it is well-defined, linear, and continuous, so $\psi_{f}(\varphi)\in \mathcal{S}'(\mathbb{R}^n)$.
\end{example}
\begin{example}{}
	Note that from the previous example, every $\phi\in\mathcal{S}(\mathbb{R}^n)$ defines a tempered distribution.
\end{example}
\begin{example}{}
  Let’s see an example of a function that does not define a tempered distribution.\\
  Take $f(x)=e^{|x|^2}$ and note that $\psi_f\notin\mathcal{S}(\mathbb{R}^n)$, since otherwise, if we take $\varphi\in\mathcal{S}(\mathbb{R}^n)$ as $\varphi(x)=e^{-\frac{|x|^2}{2}}$, note that:
  \begin{align*}
    \psi_{f}(x)&=\int_{\mathbb{R}^n}e^{|x|^2}e^{-\frac{|x|^2}{2}}dx\\
    &=\int_{\mathbb{R}^n}e^{\frac{|x|^2}{2}}dx\\
    &=\infty
  \end{align*}
  Therefore, $\psi_{f}(\varphi)$ would not be defined and hence $\psi_{f}\notin\mathcal{S}(\mathbb{R}^n)$.
\end{example}
\begin{example}{}
  Let $x\in\mathbb{R}^n$ and define $\delta_x$ as follows:
  $$\delta_x(\varphi)=\varphi(x)$$
  Let’s see that $\delta_x\in\mathcal{S}'(\mathbb{R}^n)$.\\
  Note that $\delta_x$ is well defined since $\varphi$ is well defined for all $x\in\mathbb{R}^n$; on the other hand, the linearity of $\delta_x$ is trivial, and continuity can be easily verified as follows.
  \begin{align*}
    |\delta_x(\varphi)|&=|\varphi(x)|\\
    &\leq\|\varphi\|_{\infty}\\
    &\leq |\|\varphi\||_{(0,0)}
  \end{align*}
  Then, $\delta_x\in\mathcal{S}'(\mathbb{R}^n)$.
\end{example}
\begin{proposition}{}
  Let $a\in\mathbb{R}^n$, then there does not exist $f\in\mathcal{L}_{LOC}(\mathbb{R}^n)$ (measurable function) such that $\psi_{f}=\delta_a$.
\end{proposition}
\begin{proof}{}
  Let’s reason by contradiction.\\
  Suppose there exists $f\in\mathcal{L}_{LOC}(\mathbb{R}^n)$ such that:
  $$\int_{\mathbb{R}^n}f(x)\varphi(x)dx=\varphi(a)$$
  Suppose $r>0$ and $y\neq a$, such that the ball $a\notin B_{r}(y)$, and note that $\chi_{B_{r}(y)}\in\mathcal{L}(\mathbb{R}^n)$; moreover, there exists $\{\phi_j\}\subset C^{\infty}_{C}(\mathbb{R}^n)$ such that $\phi_j\rightarrow \chi_{B_{r}(y)}$ as $j\rightarrow \infty$. Then we will have that as $j\rightarrow \infty$:
  \begin{align*}
    \int_{\mathbb{R}^n}f(x)\phi_{j}(x)dx \rightarrow 0
  \end{align*}
  Additionally:
  \begin{align*}
    \int_{\mathbb{R}^n}f(x)\chi_{B_{r}(y)}(x)dx&=0\\
    \frac{1}{|B_{r}(y)|}\int_{B_{r}(y)}f(x)dx&=0
  \end{align*}
  Then by the Lebesgue differentiation theorem, we have that:
  \begin{align*}
    f(x)&=\frac{1}{|B_{r}(y)|}\int_{B_{r}(y)}f(x)dx\\
    &=0
  \end{align*}
  Almost everywhere. Then, since $y$ is arbitrary, we have $f(x)=0$ almost everywhere.
  Then if we take any $\phi\in\mathcal{S}(\mathbb{R}^n)$ such that $\phi(a)\neq 0$, we will have:
  $$0\neq \phi(a)=\delta_a(\phi)=\psi_f(\phi)=0$$
  Which is a \textbf{contradiction}, therefore $\delta_a$ cannot be defined by a function $f\in\mathcal{L}_{LOC}(\mathbb{R}^n)$.
\end{proof}
Let’s see another example of a distribution that is not defined from a function:
\begin{example}{}
  Let’s show that $v.p \left(\frac{1}{x}\right)\in\mathcal{S}'(\mathbb{R}^n)$.\\
  Note that $v.p\left(\frac{1}{x}\right)(\varphi)$ is defined as:
  \begin{align*}
    v.p\left(\frac{1}{x}\right)(\varphi)&=\lim_{\epsilon\rightarrow 0}\int_{\epsilon<|x|<\frac{1}{\epsilon}}\frac{\varphi(x)}{x}dx
  \end{align*}
  Now:
  \begin{align*}
    \left|v.p\left(\frac{1}{x}\right)(\varphi)\right|&=\left|\lim_{\epsilon\rightarrow 0}\int_{\epsilon<|x|<\frac{1}{\epsilon}}\frac{\varphi(x)}{x}dx \right|\\
    &\leq\left|\lim_{\epsilon\rightarrow 0}\int_{\epsilon<|x|<\frac{1}{\epsilon}}\frac{\varphi(x)}{x} dx\right|\\
    &\leq\left|\lim_{\epsilon\rightarrow 0} \int_{\epsilon<|x|<1}\frac{\varphi(x)}{x}dx+\int_{1<|x|<\frac{1}{\epsilon}}\frac{\varphi(x)}{x}dx\right|\\
    &\leq\left|\lim_{\epsilon\rightarrow 0} \int_{\epsilon<|x|<1}\frac{\varphi(x)-\varphi(0)}{x}dx+\int_{1<|x|<\frac{1}{\epsilon}}\frac{\varphi(x)}{x}dx\right|\\
    &\leq\lim_{\epsilon\rightarrow 0} \int_{\epsilon<|x|<1}\left|\frac{\varphi(x)-\varphi(0)}{x}\right|dx+\int_{1<|x|<\frac{1}{\epsilon}}\left|\frac{\varphi(x)}{x}\right|dx\\
    &\phantom{\leq}\text{Using the Mean Value inequality.}\\
    &\leq\lim_{\epsilon\rightarrow 0} \|\varphi'\|_{\infty}\int_{\epsilon<|x|<1}\frac{|x|}{|x|}dx+\int_{1<|x|<\frac{1}{\epsilon}}\left|\frac{\varphi(x)}{x}\right|dx\\
    &\leq\lim_{\epsilon\rightarrow 0} 2(1-\epsilon)|\|\varphi\||_{(0,1)}+\int_{1<|x|<\frac{1}{\epsilon}}\left|\frac{\varphi(x)}{x}\right|dx\\
    &\leq\lim_{\epsilon\rightarrow 0} 2(1-\epsilon)|\|\varphi\||_{(0,1)}+\int_{1<|x|<\frac{1}{\epsilon}}\left|\frac{x\varphi(x)}{x^2}\right|dx\\
    &\leq\lim_{\epsilon\rightarrow 0} 2(1-\epsilon)|\|\varphi\||_{(0,1)}+\|x\varphi\|_{\infty}\int_{1<|x|<\frac{1}{\epsilon}}\left|\frac{1}{x^2}\right|dx\\
    &\leq \lim_{\epsilon\rightarrow 0}2(1-\epsilon)|\|\varphi\||_{(0,1)}+c\|x\varphi\|_{\infty}\\
    &\leq \lim_{\epsilon\rightarrow 0}2(1-\epsilon)|\|\varphi\||_{(0,1)}+c|\|\varphi\||_{(1,0)}\\
    &\leq k(|\|\varphi\||_{(0,1)}+|\|\varphi\||_{(1,0)})
  \end{align*}
  Then $v.p\left(\frac{1}{x}\right)(\varphi)$ is well defined, is linear by the linearity of the integral, and is continuous, that is, $v.p\left(\frac{1}{x}\right)\in\mathcal{S}'(\mathbb{R}^n)$.
\end{example}
Note that, as we saw in the examples, the Schwartz space is a space that we can associate with many others. This is why the following definition will provide us with the tools that we initially hoped to have.
\begin{definition}{}
  Let $\psi\in\mathcal{S}'(\mathbb{R}^n)$, we will define the Fourier transform of $\psi$ as:
  $$\hat{\psi}(\varphi)=\psi(\hat{\varphi})$$
\end{definition}
\begin{note}{}
  Note that since $\hat{\empty}:\mathcal{S}(\mathbb{R}^n)\rightarrow \mathcal{S}(\mathbb{R}^n)$ is an isomorphism, then $\hat{\psi}\in\mathcal{S}'(\mathbb{R}^n)$.
\end{note}
\begin{note}{}
  Create a correspondence diagram between the Schwartz space, $L^2$, and the space of tempered distributions.
\end{note}
\begin{example}{}
  Let’s show that $\hat{1}=\delta_{0}$ in the sense of tempered distributions.
  \begin{align*}
    \hat{\psi_{1}}(\varphi)&=\psi_{1}(\hat{\varphi})\\
    &=\int_{\mathbb{R}^n}\hat{\varphi}(\xi)e^{2\pi i(0\cdot \xi)}d\xi &&\text{Using the inversion formula.}\\
    &=\varphi(0)\\
    &=\delta_{0}(\varphi)
  \end{align*}
\end{example}
\begin{example}{}
  Let’s calculate $\hat{\delta_a}$.
  \begin{align*}
    \hat{\delta_{a}}(\varphi)&=\delta_{a}(\hat{\varphi})\\
    &=\hat{\varphi}(a)\\
    &=\int_{\mathbb{R}^n}\varphi(x)e^{-2\pi i (x\cdot a)}dx\\
    &=\psi_{e^{-2\pi i (x\cdot a)}}(\varphi)
  \end{align*}
  Note that if $a=0$, then $\hat{\delta_0}=\psi_{e^{-2\pi i (x\cdot 0)}}(\varphi)=\psi_{1}=1$ as a distribution. 
\end{example}
\begin{example}{}
  Example to fill in, v.p of $1/x$.
\end{example}
\begin{definition}{}
  Let $\{\psi_j\}\subset \mathcal{S}'(\mathbb{R}^n)$. We will say that $\psi_j\rightarrow 0$ as $j \rightarrow \infty$ in $\mathcal{S}'(\mathbb{R}^n)$ if for all $\varphi\in\mathcal{S}(\mathbb{R}^n)$ it follows that $\psi_j(\varphi)\rightarrow 0$ as $j \rightarrow \infty$.
\end{definition}
\begin{theorem}{}
  The operator $\mathcal{F}:\psi\rightarrow \hat{\psi}$ is an isomorphism from $\mathcal{S}'(\mathbb{R}^n)$ onto itself.
\end{theorem}
\begin{proof}{}
  \begin{itemize}
    \item Let’s show that $\mathcal{F}$ is an injective operator.\\
      Assume $\psi_1,\psi_2\in\mathcal{S}'(\mathbb{R}^n)$ such that $\hat{\psi_1}(\varphi)=\hat{\psi_2}(\varphi)$ for all $\varphi\in\mathcal{S}(\mathbb{R}^n)$, from which we can deduce that:
      \begin{align*}
        \psi_1(\hat{\varphi})&=\hat{\psi_1}(\varphi)\\
        &=\hat{\psi_2}(\varphi)\\
        &=\psi_2(\hat{\varphi})				
      \end{align*}
      Then, since \hspace*{0.1cm}$\hat{} :\mathcal{S}(\mathbb{R}^n)\rightarrow\mathcal{S}(\mathbb{R}^n)$ is an isomorphism, we can conclude that if $\hat{\psi_1}(\varphi)=\hat{\psi_2}(\varphi)$ for all $\varphi\in\mathcal{S}(\mathbb{R}^n)$, then $\psi_1(\varphi)=\psi_2(\varphi)$ for all $\varphi\in\mathcal{S}(\mathbb{R}^n)$.
    \item Let’s show that $\mathcal{F}$ is surjective.\\
      Define $\mathcal{F}^{-1}:\mathcal{S}'(\mathbb{R}^n)\rightarrow\mathcal{S}'(\mathbb{R}^n)$ such that $\widecheck{\psi}(\varphi)=\psi(\widecheck{\varphi})$. Then, similarly to how we did with $\mathcal{F}$, we can see that $\mathcal{F}^{-1}$ is injective, so we can ensure that for every $\psi\in\mathcal{S}'(\mathbb{R}^n)$ there exists $\widecheck{\psi}\in\mathcal{S}'(\mathbb{R}^n)$ such that $\hat{\widecheck{\psi}}=\psi$. Thus, it is demonstrated that $\mathcal{F}$ is a bijective operator.
    \item Let’s show that $\mathcal{F}$ and $\mathcal{F}^{-1}$ are continuous operators.\\
      Let $\{\psi_j\}\subset\mathcal{S}'(\mathbb{R}^n)$ and $\psi\in\mathcal{S}'(\mathbb{R}^n)$ such that $\psi_j\rightarrow \psi$ in the sense of $\mathcal{S}'(\mathbb{R}^n)$ as $j\rightarrow \infty$.\\
      Let’s show that $\hat{\psi}_j\rightarrow \hat{\psi}$ in the sense of $\mathcal{S}'(\mathbb{R}^n)$ as $j\rightarrow \infty$, since for any $\varphi\in\mathcal{S}(\mathbb{R}^n)$ we have:
      \begin{align*}
        \lim_{j\rightarrow \infty}\hat{\psi}_j(\varphi)&=\lim_{j\rightarrow \infty}\psi_j(\hat{\varphi})\\
        &=\psi(\hat{\varphi})\\
        &=\hat{\psi}(\varphi)
      \end{align*}
      Then, by applying the same idea with $\mathcal{F}^{-1}$, we can arrive at an analogous result, allowing us to conclude that $\mathcal{F}$ is an isomorphism from $\mathcal{S}'(\mathbb{R}^n)$ onto itself.
  \end{itemize}
\end{proof}
\begin{example}{}
  Let’s see that $\hat{e^{-a|x|^2}}(\xi)=\left(\frac{\pi}{a}\right)^{\frac{n}{2}}e^{-\frac{\pi^2|\xi|^2}{a}}$ with $\text{Re}(a)\geq 0$ and $a\neq 0$.
\end{example}
\begin{proof}{}
  We will examine this by cases:
  \begin{itemize}
    \item Let $a\in\mathbb{R}^{+}$.\\
      From example 1.3 in the book by \cite{MR3308874}, it can be deduced that:
      \begin{align*}
        \hat{e^{-4\pi^2 t|x|^2}}(\xi)=\left(\frac{1}{4\pi t}\right)^{\frac{n}{2}}e^{-\frac{|\xi|^2}{4t}}
      \end{align*}
      for all $t>0$. Then, given $a\in\mathbb{R}^{+}$, making $4\pi^2 t=a$ in the previous identity, we have:
      \begin{align*}
        \hat{e^{-a|x|^2}}(\xi)=\left(\frac{1}{4\pi t}\right)^{\frac{n}{2}}e^{-\frac{\pi^2|\xi|^2}{a}}
      \end{align*}
    \item Let $a\in\mathbb{C}$ with $\text{Re}(a)>0$.\\
      Given $\varphi\in\mathcal{S}(\mathbb{R}^n)$ for each $a\in\mathcal{A}:=\{a\in\mathbb{C}:\text{Re}(a)>0\}$, we define the following functions:
      \begin{align*}
        F_1(a):=&\langle e^{-a|x|^2},\hat{\varphi}\rangle:=\int_{\mathbb{R}^n}e^{-a|x|^2}\hat{\varphi}(x)dx\\
        F_2(a):=&\left(\frac{\pi}{a}\right)^{\frac{n}{2}}\langle e^{-\frac{\pi^2|\xi|^2}{a}},\varphi\rangle:=\left(\frac{\pi}{a}\right)^{\frac{n}{2}}\int_{\mathbb{R}^n}e^{-\frac{\pi^2|\xi|^2}{a}}\varphi(\xi) d\xi\\
      \end{align*}
      Now, let’s see that these functions are analytic in $\mathcal{A}$.\\
      To do this, note that for any closed triangle $\Delta$:
      \begin{align*}
        \int_{\partial \Delta}F_1(a)da&=\int_{\partial \Delta}\int_{\mathbb{R}^n}e^{-a|x|^2}\hat{\varphi}(x)dxda\\
        &=\int_{\mathbb{R}^n}\int_{\partial \Delta}e^{-a|x|^2}\hat{\varphi}(x)dadx\\
        &=\int_{\mathbb{R}^n}\int_{\partial\Delta}e^{-a|x|^2}da\hat{\varphi}(x)dx
      \end{align*}
      Then, since $e^{-a|x|^2}$ is an analytic function, using the \textbf{Cauchy-Goursat} theorem, we have that for any closed triangle $\Delta$ contained in $\mathcal{A}$, it holds that:
      \begin{align*}
        \int_{\partial\Delta}e^{-a|x|^2}da=0
      \end{align*}
      Thus, we can continue with:
      \begin{align*}
        \int_{\partial \Delta}F_1(a)da&=\int_{\mathbb{R}^n}\int_{\partial\Delta}e^{-a|x|^2}da\hat{\varphi}(x)dx\\
        &=\int_{\mathbb{R}^n}(0)\hat{\varphi}(x)dx\\
        &=0
      \end{align*}
      Then, using \textbf{Morera's theorem}, we can conclude that $F_1$ is analytic.\\
      Similarly, note that:
      \begin{align*}
        \int_{\partial\Delta}F_2(a)da&=\left(\frac{\pi}{a}\right)^{\frac{n}{2}}\int_{\partial\Delta}\int_{\mathbb{R}^n}e^{-\frac{\pi^2|\xi|^2}{a}}\varphi(\xi) d\xi da\\
        &=\left(\frac{\pi}{a}\right)^{\frac{n}{2}}\int_{\mathbb{R}^n}\int_{\partial\Delta}e^{-\frac{\pi^2|\xi|^2}{a}}\varphi(\xi) da d\xi\\
        &=\left(\frac{\pi}{a}\right)^{\frac{n}{2}}\int_{\mathbb{R}^n}\int_{\partial\Delta}e^{-\frac{\pi^2|\xi|^2}{a}} da \varphi(\xi)d\xi
      \end{align*}
      Then, since $e^{-\frac{\pi^2|\xi|^2}{a}}$ is analytic in $\mathcal{A}$ (where $a\neq 0$), by the \textbf{Cauchy-Goursat} theorem, it holds that for any closed triangle $\Delta$ contained in $\mathcal{A}$:
      \begin{align*}
        \int_{\partial\Delta}e^{-\frac{\pi^2|\xi|^2}{a}}da=0
      \end{align*}
      Thus, we can continue with:
      \begin{align*}
        \int_{\partial\Delta}F_2(a)da&=\left(\frac{\pi}{a}\right)^{\frac{n}{2}}\int_{\mathbb{R}^n}\int_{\partial\Delta}e^{-\frac{\pi^2|\xi|^2}{a}} da\varphi(\xi) d\xi\\
        &=\int_{\mathbb{R}^n}(0)\varphi(\xi)d\xi\\
        &=0
      \end{align*}
      Then, using \textbf{Morera's theorem}, we can conclude that $F_2$ is analytic.\\
      On the other hand, from the first case, we have that $F_1(a)=F_2(a)$ for all $a\in\mathbb{R}^{+}\setminus\{0\}$, then since $F_1$ and $F_2$ are analytic in $\mathbb{R}^{+}\setminus\{0\}\subset \mathcal{A}$, by the uniqueness theorem of analytic functions, we have that $F_1=F_2$. In other words, we have:
      \begin{align*}
      \langle e^{-a|x|^2}, \hat{\varphi} \rangle&=\left(\frac{\pi}{a}\right)^{\frac{n}{2}}\langle e^{-\frac{\pi^2|\xi|^2}{a}}, \varphi \rangle
      \end{align*}
      for all $a\in\mathcal{A}=\{a\in\mathbb{C}:\text{Re}(a)>0\}$
    \item Let $a\in\mathbb{C}$ with $\text{Re}(a)=0$ and $a\neq 0$.\\
      Let $a\in\mathbb{C}$ with $\text{Re}(a)=0$ and $a\neq 0$, that is $a=yi$ for some $y\in\mathbb{R}\setminus\{0\}$.\\
      Given $\varphi\in\mathcal{S}(\mathbb{R}^n)$ from the previous case, we can conclude that:
      \begin{align*}
        \langle e^{-a|x|^2},\hat{\varphi} \rangle &= \lim_{\epsilon\rightarrow 0}\langle e^{-(a+\epsilon)|x|^2},\hat{\varphi} \rangle\\
        &=\lim_{\epsilon\rightarrow 0}\left(\frac{\pi}{a+\epsilon}\right)^{\frac{n}{2}}\langle e^{-\frac{\pi^2|\xi|^2}{a+\epsilon}}, \varphi \rangle\\
        &=\left(\frac{\pi}{a}\right)^{\frac{n}{2}}\langle e^{-\frac{\pi^2|\xi|^2}{a}}, \varphi \rangle
      \end{align*}
      Therefore, it is demonstrated that $\hat{e^{-a|x|^2}}(\xi)=\left(\frac{\pi}{a}\right)^{\frac{n}{2}}e^{-\frac{\pi^2|\xi|^2}{a}}$ with $\text{Re}(a)\geq 0$ and $a\neq 0$.
  \end{itemize}
\end{proof}
\begin{note}{}
  With the theorem and the previous example, we can justify the following calculation related to the fundamental solution of the time-dependent Schrödinger equation.
\end{note}
\begin{example}{}
  Let’s see that $\hat{e^{-4\pi^2it|x|^2}}(\xi)=-\left(\frac{1}{4\pi it}\right)^{\frac{n}{2}}e^{\frac{i|\xi|^2}{4t}}$.\\
  Note that this is immediate by setting $a=4\pi^2it$ and using the previous example.
\end{example}
\begin{definition}{}
  Let $\psi\in\mathcal{S}'(\mathbb{R}^n)$ and let $\beta$ be a multi-index.\\
  We define $\partial^{\beta}\psi:\mathcal{S}(\mathbb{R}^n)\rightarrow \mathbb{C}$ as:
  $$<\partial^{\beta}\psi,\phi>:=(-1)^{|\beta|}<\psi,\partial^{\beta}\phi>$$
  for all $\phi\in\mathcal{S}(\mathbb{R}^n)$.
\end{definition}
\begin{note}{}
  This definition is motivated by the following procedure:
  Suppose $\psi\in\mathcal{S}'(\mathbb{R}^n)$ is a function, then using integration by parts we have:
  \begin{align*}
    \int_{|x|\leq R}\partial_{x_i}\psi \phi dx &= \int_{|x|= R}(\psi\phi)\eta_i dS(x) - \int_{|x|\leq R} \psi \partial_{x_i}\phi dx
  \end{align*}
  where $\eta_i$ is the $i$-th component of the normal vector to $\{x:|x| = R\}$.\\
  Thus:
  \begin{align*}
    \langle \partial_{x_i}\psi,\phi \rangle &= \int_{\mathbb{R}^n}\partial_{x_i}\psi\phi dx \\
    &= \lim_{R\rightarrow \infty} \int_{|x|\leq R}\partial_{x_i}\psi \phi dx\\
    &= \lim_{R\rightarrow \infty} \int_{|x|= R}(\psi\phi)\eta_i dS(x) - \int_{|x|\leq R} \psi \partial_{x_i}\phi dx\\
    &= -\int_{\mathbb{R}^n}\psi\partial_{x_i}\phi dx
  \end{align*}
  Then, using an inductive argument, we can conclude the following:
  \begin{align*}
    \langle \partial^{\beta}\psi, \phi \rangle &= (-1)^{|\beta|}\int_{\mathbb{R}^n}\psi\partial_{\beta}\phi dx\\
    &=(-1)^{|\beta|}\langle \psi, \partial_{\beta}\phi \rangle
  \end{align*}
\end{note}
Now let’s see some examples related to this:
\begin{example}{}
  Let’s calculate the derivative of $\delta_0$:
  \begin{align*}
    \langle \partial^{\alpha}\delta_0, \phi \rangle &= (-1)^{|\alpha|}\langle \delta_0,\partial_{\alpha}\phi \rangle\\
    &= (-1)^{|\alpha|}\partial_{\alpha}\phi(0)
  \end{align*}
\end{example}
\begin{example}{}
  Let $H(x)=0\mathcal{X}_{x<0}+1\mathcal{X}_{x\geq 0}$.
  Note that $H\in L^\infty(\mathbb{R})$ and by previous results we have that $H\in\mathcal{S}'(\mathbb{R})$. Also note that since $H$ is a step function, it does not have a derivative in the weak sense.
  Now let’s find its derivative in the sense of tempered distributions:
  \begin{align*}
    \langle H',\phi \rangle&=-\langle H, \phi'\rangle \\
    &=-\int_{-\infty}^{\infty}H(x)\phi'(x)dx\\
    &=-\int_{0}^{\infty}\phi'(x)dx\\
    &=-\lim_{R\rightarrow\infty}\int_{0}^{R}\phi'(x)dx\\
    &=-\lim_{R\rightarrow\infty}(\phi(R)-\phi(0))\\
    &=-(-\phi(0))\\
    &=\phi(0)\\
    &=\langle \delta_0, \phi\rangle
  \end{align*}
  Therefore, we could conclude that $H^{k}=\delta_0^{k-1}$.
\end{example}
Now, in order to give an application to the previous ideas, we will introduce the following definition.
\begin{definition}{}
  For $\varphi\in\mathcal{S}(\mathbb{R}^n)$, we define \textbf{The Hilbert transform} $\mathcal{H}(\varphi)$ as:
  \begin{align*}
    \mathcal{H}(\varphi)(y)&=\frac{1}{\pi}v.p\frac{1}{x}(\varphi(y-\cdot))=\frac{1}{\pi}(v.p\frac{1}{x}*\varphi)(y)
  \end{align*}
\end{definition}
Note that the previous definition involves a kind of convolution between a tempered distribution and a function from the Schwartz space, for which it will be interesting to prove the following proposition.
\begin{proposition}{}
  Let $\psi\in\mathcal{S}'(\mathbb{R}^n)$ and $\varphi\in\mathcal{S}(\mathbb{R}^n)$. If we define:
  $$(\psi*\varphi)(x)=\psi(\varphi(x-\cdot))$$
  then, $\psi*\varphi\in C^{\infty}(\mathbb{R}^n)\cap\mathcal{S}'(\mathbb{R}^n)$ and it holds that:
  $$\hat{\psi*\varphi}=\hat{\psi}\hat{\varphi}$$
  where $\hat{\psi}\hat{\varphi}$ is defined as $\hat{\psi}\hat{\varphi}(\phi)=\hat{\psi}(\hat{\varphi}\phi)$ for all $\phi\in\mathcal{S}(\mathbb{R}^n)$.
\end{proposition}{}
\begin{proof}{}
  \begin{itemize}
    \item Let’s show that $\psi*\varphi\in C^{\infty}(\mathbb{R}^n)$.\\
      We compute:
      \begin{align*}
        \lim_{h\rightarrow 0}\frac{(\psi*\varphi)(x+h\epsilon_j)-(\psi*\varphi)(x)}{h}&=\lim_{h\rightarrow 0}\frac{\langle \psi, \varphi(x+h\epsilon_j-\cdot) \rangle-\langle \psi, \varphi(x-\cdot) \rangle}{h}\\
        &=\lim_{h\rightarrow 0}\langle \psi, \frac{\varphi(x+h\epsilon_j-\cdot)-\varphi(x-\cdot)}{h} \rangle\\
        &=\langle \psi, \frac{\partial \varphi}{\partial x_j} (x-\cdot) \rangle\\
        &=(\psi * \partial \varphi_{x_j})(x)
      \end{align*}
      This, of course, assumes that $\frac{\varphi(x+h\epsilon_j-\cdot)-\varphi(x-\cdot)}{h} \rightarrow \partial\varphi_{x_j}(x-\cdot)$ in the sense of the Schwartz space as $h\rightarrow 0$. We see that this holds (using Taylor's formula):
      \begin{align*}
        \hspace{-1.2cm}\frac{\varphi(x+h\epsilon_j-\cdot)-\varphi(x-\cdot)}{h} - \partial\varphi_{x_j}(x-\cdot)&=\frac{1}{h}(\varphi(x+h\epsilon_j-\cdot)-\varphi(x-\cdot)-h\partial\varphi_{x_j}(x-\cdot))\\
        &=\frac{1}{h}\left[\int_{0}^{1}(1-t)\partial^{2}\varphi_{x^2_{j}}(x-\cdot+th\epsilon_j)dt\right]h^2\\
        &=h\left[\int_{0}^{1}(1-t)\partial^{2}\varphi_{x^2_{j}}(x-\cdot+th\epsilon_j)dt\right]\\
      \end{align*}
      Now, to conclude, we note that this is equivalent to showing that:
      \begin{align*}
        |\|h\left[\int_{0}^{1}(1-t)\partial^{2}\varphi_{x^2_{j}}(x-\cdot+th\epsilon_j)dt\right]\||_{\nu,\beta}\rightarrow 0
      \end{align*}
      when $h\rightarrow 0$.
      \begin{align*}
        \left|hy^{\nu}\left[\int_{0}^{1}(1-t)\partial^{2}\varphi_{x^2_{j}}(x-\cdot+th\epsilon_j)dt\right]\right|&\leq |h|\int_{0}^{1}(1-t)|y^{\nu}\partial^{2}\varphi_{x^2_{j}}(x-\cdot+th\epsilon_j)|dt
      \end{align*}
      Then, since for $0\leq t \leq 1$, we have that $1-t$ is bounded and $|y^{\nu}\partial^2\varphi_{x^2_j}|\leq |\|\varphi\||_{\nu,\beta}$ for some multi-index $\beta$, we can ensure that there exists a constant $M>0$ such that:
      \begin{align*}
        \left|hy^{\nu}\left[\int_{0}^{1}(1-t)\partial^{2}\varphi_{x^2_{j}}(x-\cdot+th\epsilon_j)dt\right]\right|&\leq |h|\int_{0}^{1}(1-t)|y^{\nu}\partial^{2}\varphi_{x^2_{j}}(x-\cdot+th\epsilon_j)|dt\\
        &\leq|h|M
      \end{align*}
      This allows us to conclude that $\frac{\varphi(x+h\epsilon_j-\cdot)-\varphi(x-\cdot)}{h} \rightarrow \partial\varphi_{x_j}(x-\cdot)$ in the sense of the Schwartz space as $h\rightarrow 0$.
      Thus, using an inductive argument, we can reach that $\psi*\varphi\in C^{\infty}$.
    \item Now let’s show that $\psi*\varphi \in\mathcal{S}'(\mathbb{R}^n)$.\\
      The linearity is inherited immediately from the linearity of $\psi$, now let’s check the continuity.\\
      Note that since $\psi\in\mathcal{S}'(\mathbb{R}^n)$, there exists $N>0$ and $C>0$ such that for any $f\in\mathcal{S}(\mathbb{R}^n)$ it holds that:
      \begin{align*}
        |\psi(f)|\leq C\sum_{|\nu|,|\beta|\leq N}|\|f\||_{\nu,\beta}
      \end{align*}
      Then, since $(\psi*\varphi)(x)=\langle \psi, \varphi(x-\cdot) \rangle$ and $\varphi(x-\cdot)\in\mathcal{S}(\mathbb{R}^n)$, we know that there exists $N>0$ and $C>0$ such that:
      \begin{align*}
        |\psi(\varphi(x-\cdot))|\leq C\sum_{|\nu|,|\beta|\leq N}|\|\varphi(x-\cdot)\||_{\nu,\beta}
      \end{align*}
      Thus, it is proven that $\psi*\varphi \in\mathcal{S}'(\mathbb{R}^n)$.
    \item Now let’s show that $\hat{\psi*\varphi}=\hat{\psi}\hat{\varphi}$.\\
      For this, first let’s prove that $\int_{\mathbb{R}^n}\langle \psi,\varphi(x) \rangle dx=\langle \psi,\int_{\mathbb{R}^n}\varphi(x)dx  \rangle$.\\
      To do this, assume $\varphi\in C^{\infty}_{c}(\mathbb{R}^n)$. Thus, we know that there exists $R>0$ such that $Supp(\varphi)\subseteq [-R,R]^n$, so we have:
      \begin{align*}
        \int_{\mathbb{R}^n}\langle \psi,\varphi(x) \rangle dx&=\int_{[-R,R]^n}\langle \psi,\varphi(x) \rangle dx \hspace{1cm}\text{Taking $\mathcal{P}$ a partition of $[-R,R]^n$}\\
        &=\lim_{|p|\rightarrow 0}\sum_{p\in\mathcal{P}}\langle \psi,\varphi(x^k) \rangle \Delta x_1\Delta x_2\cdots\Delta x_n\\
        &=\lim_{|p|\rightarrow 0}\langle \psi,\sum_{p\in\mathcal{P}}\varphi(x^k)\Delta x_1\Delta x_2\cdots\Delta x_n \rangle \\
        &=\langle \psi, \int_{\mathbb{R}^n}\varphi(x)dx \rangle
      \end{align*}
      Now, if $\varphi\notin C^{\infty}_{c}(\mathbb{R}^n)$, we can take $\{\varphi_j\}\subseteq \mathcal{S}'(\mathbb{R}^n)$ such that $\varphi_j\rightarrow \varphi$ as $j\rightarrow \infty$, and the argument would be similar.
      Now, using this we can perform the following calculation:
      \begin{align*}
        \langle \hat{\psi*\varphi},\phi \rangle&=\langle \psi*\varphi,\hat{\phi} \rangle\\
        &=\int_{\mathbb{R}^n}\langle \psi,\varphi(x-y) \rangle\hat{\phi}(x)dx\\
        &=\langle \psi, \varphi(x-y)\hat{\phi}(x) \rangle dx\\
        &=\langle \psi, \int_{\mathbb{R}^n}\varphi(x-y)\hat{\phi}(x)dx \rangle\\
        &=\langle \psi, \int_{\mathbb{R}^n}\tilde{\varphi}(y-x))\hat{\phi}(x)dx \rangle &&\tilde{\varphi}(x)=\varphi(-x)\\
        &=\langle \psi, \hat{\phi}*\tilde{\varphi} \rangle\\
        &=\langle \psi, \hat{\phi\check{\tilde{\varphi}}} \rangle\\
        &=\langle \hat{\psi}, \phi\check{\tilde{\varphi}} \rangle\\
        &=\langle \hat{\psi}, \phi\hat{\varphi} \rangle\\
        &=\langle \hat{\psi}\hat{\varphi},\phi \rangle                
      \end{align*}
      Therefore, the proposition is proven.
  \end{itemize}
\end{proof}
With this, it will be interesting to see the following example.
\begin{example}{}
  Note that with the previous proposition, it is easy to see that:
  \begin{align*}
    \hat{\mathcal{H}(\varphi)}(\xi)&=\hat{\frac{1}{\pi}(v.p\frac{1}{x}*\varphi)(y)}\\
    &=\frac{1}{\pi}\hat{v.p\frac{1}{x}}\hat{\varphi}(\xi)\\
    &=-i sgn(\xi) \hat{\varphi}(\xi)
  \end{align*}
\end{example}
Now, based on what has been seen, it will be interesting to show that the Hilbert transform is an isometry in $\mathcal{L}^2(\mathbb{R}^n)$, since it can be verified that:
$$\|\mathcal{H}(\varphi)\|_2=\|\varphi\|_2 \hspace{1cm}\text{and}\hspace{1cm}\mathcal{H}(\mathcal{H}(\varphi))=-\varphi$$
It would be interesting to see some other properties of the Hilbert transform.
