\begin{definition}{}
    Let $(X,\mu)$ and $(Y, \nu)$ be two measure spaces, and let $T$ be an operator from $\mathcal{L}^{p}(X,\mu)$ to the space of measurable functions on $Y$ in $\mathbb{C}$.
    $$T:\mathcal{L}^p(X,\mu)\rightarrow\mathcal{M}(Y,\mathbb{C})$$
    \begin{itemize}
        \item[i.] We say that $T$ is $(p,q)$-weak (with $q<\infty$) if for every $\lambda>0$, there exists $C>0$ such that:
        $$\nu(\{y\in Y:|(Tf)(y)|>\lambda\})\leq \left(\frac{C||f||_p}{\lambda}\right)^q.$$
        \item[ii.] We say that $T$ is $(p,\infty)$-weak if the operator is bounded from $\mathcal{L}^p(X,\mu)$ to $\mathcal{L}^{\infty}(Y,\nu)$.
        \item[iii.] We say that $T$ is $(p,q)$-strong if the operator is bounded from $\mathcal{L}^p(X,\mu)$ to $\mathcal{L}^q(Y,\nu)$.
    \end{itemize}
\end{definition}
\begin{proposition}{}
  Note that if $T$ is $(p,q)$-strong, then it is $(p,q)$-weak.
\end{proposition}
\begin{proof}{}
  Let $$E_\lambda=\{ y\in Y: |(Tf)(y)|>\lambda \},$$
  \begin{align*}
    \nu(E_\lambda)=\int_{E_\lambda}d_{\nu}&\leq\int_{E_\lambda}\Bigl|\frac{(Tf)(y)}{\lambda}\Bigr|^qd\nu \\
    &\leq\frac{1}{\lambda^q}\int_{Y}|(Tf)(y)|^qd\nu=\frac{1}{\lambda^q}||Tf||_q^q\\
    &\leq \frac{1}{\lambda^q}(C||f||_p)^q=\left(\frac{C||f||_p}{\lambda}\right)^q
  \end{align*}  
\end{proof}
When $(X,\mu)=(Y,\nu)$ and $T$ is the identity operator, the $(p,p)$-weak inequality is the classical Chebyshev inequality.\\
On the other hand, the relationship between the inequalities, particularly the $(p,q)$-weak inequality and convergence almost everywhere, is contained in the following theorem. (We also assume that $(X,\mu)=(Y,\nu)$).\\
\begin{theorem}{}
  Let $\{T_t\}$ be a family of linear operators in $\mathcal{L}^p(X,\mu)$:
  \begin{align*}
    T_t:\mathcal{L}^p(X,\mu)&\rightarrow\mathcal{L}^q(X,\mu)\\
    f&\rightarrow T_tf
  \end{align*}
  and
  $$T^*f(x)=\sup_{t>0}|T_tf(x)|$$
  If $T^*$ is $(p,q)$-weak, the set $\{f\in\mathcal{L}^p|\lim_{t\rightarrow t_0} T_tf(x)=f(x)\, \text{a.e.}\}$ is closed in $\mathcal{L}^p$.\\
  $T^*$ is called the maximal operator associated with $\{T_t\}$.
\end{theorem}
\begin{proof}{}
  Let $\left\{f_n\right\}\subseteq \mathcal{L}^p$ be a sequence of functions for which $T_t f_n(x) \underset{t \rightarrow t_0}{\rightarrow} f_n(x)$ almost everywhere and let $f$ be its limit in the $L^p$ norm, that is: $$\left\|f_n-f\right\|_{L^p} \stackrel{n\to \infty}{\rightarrow} 0.$$\\
  We want to prove that $f \in\left\{\left.h \in \mathcal{L}^p\right|\lim_{t \rightarrow t_0} T_t h(x)=h(x)\,\text{a.e.}\right\}$ (this is the same as saying that the set contains all its accumulation points). Then, the idea will be to show that $$\mu\left(\left\{x \in X: \lim _{t \rightarrow t_0} \sup _{t>0}\left|T_t f(x)-f(x)\right|>0\right\}\right)=0.$$
  This is equivalent to saying that the set of $x\in X$ such that $\lim _{t \rightarrow t_0}\sup _{t>0}\left|T_t f(x)-f(x)\right|>0$ has measure zero. For this, we observe that:
  $$\hspace{-6cm}\mu\left(\left\{x \in X: \limsup _{t \rightarrow t_0}\left|T_t f(x)-f(x)\right|>\lambda\right\}\right) \leq$$
  $$\hspace{4cm}\mu\left(\left\{x \in X: \limsup _{t \rightarrow t_0}\left|T_t\left(f-f_n\right)(x)-\left(f-f_n\right)(x)\right|>\lambda\right\}\right)$$
  Since $ \lim_{t \rightarrow t_0} T_t f_n(x)=f_n(x)$ a.e., it follows that given $\epsilon>0$, there exists $\delta>0$ such that if $t\in \left(t_0-\delta, t_0+\delta\right)$, then $\left|T_t f_n(x)-f_n(x)\right|<\varepsilon$.\\
  Then, $$\left|T_t f(x)-f(x)\right| \leq\left|T_t\left(f-f_n\right)(x)-\left(f-f_n\right)(x)\right|+\left|T_t f_n(x)-f_n(x)\right|$$ 
  and thus, 
  $$\lambda<\limsup _{t \rightarrow t_0}\left|T_t f(x)-f(x)\right| \leqslant \limsup_{t \rightarrow t_0}\left|T_t\left(f-f_n\right)(x)-\left(f-f_n\right)(x)\right|+0$$
  \newpage
  Therefore:
  $$\hspace{-6cm}\left\{x \in X:\limsup_{t \rightarrow t_0}|T_t f(x)-f(x)|>\lambda\right\} \subseteq$$
  $$\hspace{4cm}\left\{x\in X:\limsup_{t \rightarrow t_0}|T_t(f-f_n)(x)-(f-f_n)(x)|>\lambda\right\}$$
  Now, 
  \begin{align*}
    \lambda &< \limsup_{t \rightarrow t_0}|T_t(f-f_n)(x)-(f-f_n)(x)| \\
    &\leq \limsup_{t \rightarrow t_0}|T_t(f-f_n)(x)|+|(f-f_n)(x)| \\
    &\leq \limsup_{t \rightarrow t_0} |T_t(f-f_n)(x)|+\limsup_{t \rightarrow t_0}|(f-f_n)(x)| \\
    &\leq 2\limsup_{t \rightarrow t_0}|T^*(f-f_n)(x)|\hspace{1cm} \text{when}\quad |(f-f_n)(x)|\leq \limsup_{t \rightarrow t_0}|T^*(f-f_n)(x)| \\
    &\leq 2\limsup_{t \rightarrow t_0}|(f-f_n)(x)|\hspace{1cm} \text{when}\quad \limsup_{t \rightarrow t_0}|T^*(f-f_n)(x)|\leq |(f-f_n)(x)|
  \end{align*}
  Thus:
  \begin{align*}
    \mu(\{&x\in X\,:\, \limsup_{t \rightarrow t_0}|T_t(f-f_n)(x)-(f-f_n)(x)|>\lambda\})\\
    &\subseteq \mu(\{x\in X\,:\, |T^*(f-f_n)(x)|>\lambda/2\}\cup \{x\in X\,:\,|(f-f_n)(x)|>\lambda/2\}) \\
    &\leq \Bigl(\frac{2C}{\lambda}||f-f_n||_p\Bigr)^q+\Bigl(\frac{2}{\lambda}||f-f_n||_p\Bigr)^p\\
    &\leq 0 
  \end{align*}
  Since $T^*$ is $(p,q)$-weak and using the Chebyshev $(p,p)$-weak inequality, moreover when $n\to \infty, \|f-f_n\|=0$.
  Thus:
  \begin{align*}
    \hspace{-0.4cm}\mu(\{x \in X:\limsup_{t \rightarrow t_0}|T_t f(x)-f(x)|>0\}) &\leq\sum_{k=1}^{\infty}\mu(\{x\in X:\limsup_{t\rightarrow t_0}|T_t f(x)-f(x)|>\frac{1}{k}\})\\
    &=0.
  \end{align*}
  If $T_tf(x)$ is complex, we apply the reasoning separately to its real and imaginary parts.\\
  As for the approximations of the identity, we know the pointwise convergence to $f$ for functions in $\mathcal{S}$, so it is enough to prove weak bounds on the maximal operator $\sup_{t>0}|\phi_t*f(x)|$ to deduce pointwise convergence for $f\in\mathcal{L}^p$, $1\leq p <\infty$, or for $f\in C_0$.
\end{proof}
\begin{definition}{}
  Let $(X,\mu)$ be a measure space and let $f:X\rightarrow \mathbb{C}$ be a measurable function.\\
  We will call the distribution function of $f$ associated with $\mu$ the function:
  \begin{align*}
    a_f:(0,\infty)&\rightarrow [0,\infty]\\
    \lambda&\rightarrow \mu(\{x\in X: |f(x)|>\lambda\})	
  \end{align*}
\end{definition}
\begin{proposition}{}
  Let $\varphi:[0,\infty)\rightarrow[0,\infty)$ be a differentiable and increasing function such that $\varphi(0)=0$. Then:
  $$\int_X \varphi(|f(x)|)d\mu=\int_0^\infty \varphi'(\lambda)a_f(\lambda)d\lambda$$
\end{proposition}
\begin{proof}{}
  Note that:
  $$a_f(\lambda)=\mu(\{x\in X: |f(x)|>\lambda\})=\int_{\{x\in X: |f(x)|>\lambda\}}\,d\mu$$
  Thus, we have:
  \begin{align*}
    \int_0^\infty \varphi'(\lambda)a_f(\lambda)\,d\lambda&=\int_0^\infty \varphi'(\lambda)\left(\int_{\{x\in X: |f(x)|>\lambda\}}\,d\mu\right)\,d\lambda\\
    &=\int_0^\infty\int_{\{x\in X: |f(x)|>\lambda\}}\varphi'(\lambda)\,d\mu d\lambda
  \end{align*}
  By changing the order of integration, we have:
  $$\int_0^\infty\int_{\{x\in X: |f(x)|>\lambda\}}\varphi'(\lambda)\,d\mu d\lambda=\int_X\int_0^{|f(x)|}\varphi'(\lambda)\,d\lambda d\mu$$
  This is clearer with the following graph where the region of integration is below the curve.
  \begin{center}
    \tikzset{every picture/.style={line width=0.75pt}} %set default line width to 0.75pt        
  
    \begin{tikzpicture}[x=0.75pt,y=0.75pt,yscale=-1,xscale=1]
      %uncomment if require: \path (0,310); %set diagram left start at 0, and has height of 310
          
      %Shape: Axis 2D [id:dp3395163420666467] 
      \draw  (83.69,252.24) -- (476.26,249.22)(109.59,25.34) -- (111.56,282.15) (469.22,244.28) -- (476.26,249.22) -- (469.3,254.28) (104.64,32.38) -- (109.59,25.34) -- (114.64,32.3)  ;
      %Curve Lines [id:da5277146836580144] 
      \draw    (111.33,252.02) .. controls (205.33,77.02) and (179.33,309.02) .. (269.33,148.02) ;
      %Curve Lines [id:da03980337538940737] 
      \draw    (269.33,148.02) .. controls (310.33,85.02) and (327.33,233.02) .. (369.33,148.02) ;
      %Curve Lines [id:da7123629736179811] 
      \draw    (369.33,148.02) .. controls (398.33,62.02) and (423.33,185.02) .. (461.33,92.02) ;
      %Straight Lines [id:da7957859031909573] 
      \draw [color={rgb, 255:red, 255; green, 0; blue, 0 }  ,draw opacity=1 ][line width=2.25]    (269.33,148.02) -- (271.5,251) ;
      %Shape: Free Drawing [id:dp04683283542392558] 
      \draw  [color={rgb, 255:red, 184; green, 233; blue, 134 }  ,draw opacity=1 ][line width=2.25] [line join = round][line cap = round] (239.5,51) .. controls (239.5,50.67) and (239.5,50.33) .. (239.5,50) ;
  
      % Text Node
      \draw (450,71.4) node [anchor=north west][inner sep=0.75pt]    {$|f( x) |$};
      % Text Node
      \draw (483,240.4) node [anchor=north west][inner sep=0.75pt]    {$\mu $};
      % Text Node
      \draw (114,9.4) node [anchor=north west][inner sep=0.75pt]    {$\lambda $};
      % Text Node
      \draw (275,169.4) node [anchor=north west][inner sep=0.75pt]    {$d\lambda $};
          
    \end{tikzpicture}  
  \end{center}
  Therefore:
  \begin{align*}
    \int_0^\infty \varphi'(\lambda)a_f(\lambda)\,d\lambda&=\int_X\int_0^{|f(x)|}\varphi'(\lambda)\,d\lambda d\mu\\
    &=\int_X\varphi(\lambda)|_0^{|f(x)|}\,d\mu\\
    &=\int_X(\varphi(|f(x)|)-\varphi(0))\,d\mu\\
    &=\int_X\varphi(|f(x)|)\,d\mu && \text{since } \varphi(0)=0
  \end{align*}
\end{proof}
\begin{note}{}
  If in particular, $\varphi(\lambda)=\lambda^p$, then we can conclude that:
  \begin{equation}
    ||f||_{L^p}^p=p\int_0^\infty \lambda^{p-1}a_f(\lambda)d\lambda 
  \end{equation}
\end{note}
Since weak inequalities measure the size of the distribution function, this representation of the norm in $L^p$ is suitable for proving the following interpolation theorem, which allows us to deduce bounds in $L^p$ from weak inequalities. It applies to slightly more general operators than linear ones (maximal operators are not linear) that we will define next.
\begin{definition}{}
  An operator $T$ from a vector space of measurable functions to measurable functions is said to be sublinear if:
  \begin{itemize}
    \item $|T(f_1+f_2)(x)|\leq |T(f_1)(x)|+|T(f_2)(x)|$.\\
    \item $|T(\lambda f)| = |\lambda||T(f)(x)|$ for all $\lambda\in\mathbb{C}$
  \end{itemize} 
\end{definition}
