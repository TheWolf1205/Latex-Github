\lipsum[10]
\begin{definition}{The Hardy-Littlewood Maximal Function}
  Given $f\in L^1_{loc}(\mathbb{R}^{n})$, we will define $\mathcal{M}f(x)$, the Hardy-Littlewood maximal function of a $f$ by:
    \begin{align*}
      \mathcal{M}f(x)&=\sup_{r>0}\frac{1}{|B_r(x)|}\int_{B_r(x)}|f(y)|dy,\\
      &=\sup_{r>0}\frac{1}{\omega_n}\int_{B_1(0)}|f(x-ry)|dy,\\
      &=\sup_{r>0}(|f|*\frac{1}{|B_r(0)|}\chi_{B_r(0)})(x).
    \end{align*}
\end{definition}
\begin{proposition}{}
  the Hardy-Littlewood maximal functions satisfies:
  \begin{enumerate}
    \item $\mathcal{M}$ define a sublineal operator:
      \begin{align*}
        |\mathcal{M}(f+g)(x)|\leq |\mathcal{M}f(x)|+|\mathcal{M}g(x)|, \hspace{0.5cm}x\in\mathbb{R}^{n}.
      \end{align*}
    \item If $f\in L^{\infty}(\mathbb{R}^{n})$, then $\mathcal{M}f(x)\in L^{\infty}(\mathbb{R}^{n})$, i.e, $\mathcal{M}$ is of type $(\infty,\infty)$.
    \item If $f\in L^{1}(\mathbb{R}^{n})$, then $\mathcal{M}f(x)\in L^{1}(\mathbb{R}^{n})$ if $f\equiv 0$, i.e, $\mathcal{M}$ is not of type $(1,1)$-strong. 
    \end{enumerate}
\end{proposition}

\begin{proof} 
  \begin{enumerate}
    \item $\mathcal{M}(f+g)(x)$ by definition is:
      \begin{align*}
        \mathcal{M}(f+g)(x)&\leq\sup_{r>0}\frac{1}{|B_r(x)|}\int_{B_r(x)}|f(y)+g(y)|dy\\
        &\leq \sup_{r>0}\frac{1}{|B_r(x)|}\int_{B_r(x)}|f(y)|+|g(y)|dy\\
        &\leq \sup_{r>0}\frac{1}{|B_r(x)|}\left( \int_{B_x(r)}|f(y)|dy+\int_{B_r(x)}g(y)dy\right)\\
        &\leq \sup_{r>0}\frac{1}{|B_r(x)|}\int_{B_r(x)}|f(y)|dy + \frac{1}{|B_r(x)|}\int_{B_r(x)}g(y)dy\\
        &\leq \mathcal{M}f(x)+\mathcal{M}g(x)
      \end{align*}
    \item If $f\in L^{\infty}(\mathbb{R}^{n})$, then:
      \begin{align*}
        \mathcal{M}f(x)&\leq \sup_{r>0}\frac{1}{|B_r(x)|}\int_{B_r(x)}|f(y)|dy\\
        &\leq \sup_{r>0} \frac{1}{|B_r(x)|}\int_{B_r(x)}\|f\|_{\infty}dy\\
        &\leq \sup_{r>0} \|f\|_{\infty}\frac{1}{|B_r(x)|}\int_{B_r(x)}dy\\
        &\leq \|f\|_{\infty}
      \end{align*}
    \item reasoning by contradiction suppose $f\in L^1(\mathbb{R}^{n})$ such that $f\neq 0$, i.e, exist $r>0$, such that $\int_{B_r(0)}|f(x)|dx=a$ with $a>0$ and $\mathcal{M}f(x)\in L^1(\mathbb{R}^{n})$.\\
      If suposse $|x|>r$, then:
      \begin{align*}
        \mathcal{M}f(x)&\geq \sup_{R>0}\frac{1}{|B_R(x)|}\int_{B_{R}(x)}|f(y)|dy\\
        &\geq \frac{1}{|B_{3|x|}(x)|}\int_{B_{3|x|}(x)}|f(y)|dy\\
        &\geq \frac{1}{|B_{3|x|}(x)|}\int_{B_r(0)}|f(y)|dy\\
        &\geq \frac{c_{n}}{3^{n}|x|^{n}}\int_{B_r(0)}|f(y)|dy\\
        &\geq \frac{\tilde{c}_{n}}{3^{n}|x|^{n}}
      \end{align*}
      Now, with $\mathcal{M}f(x)\in L^1(\mathbb{R}^{n}\setminus\{x:|x|\leq r\})$, then:
      \begin{align*}
        \|\mathcal{M}f\|_{1}&\geq\norm{\frac{\tilde{c}_{n}}{3^{n}|x|^{n}}}_{1}\\
        &\geq \frac{\tilde{c}_{n}}{3^n}\norm{\frac{1}{|x|^n}}_{1}\\
        &\geq \frac{\tilde{c}_{n}}{3^{n}}\int_{|x|>r}\left| \frac{1}{|x|^n}dx \right|\\
        &\geq \frac{\tilde{c}_{n}}{3^{n}}\lim_{R \to \infty}\int_{r}^{R}\left| \frac{1}{r^n}r^{n-1}dx \right|\\
        &\geq \frac{\tilde{c}_{n}}{3^{n}}\lim_{R \to \infty}(\ln(R)-\ln(r))\\
        &\geq \infty
      \end{align*}
      \textbf{Contradiction}, then we can conclude that if $f\in L^1(\mathbb{R}^{n})$ and $\mathcal{M}f\in L^1(\mathbb{R}^{n})$, then $f\equiv 0$, i.e, $\mathcal{M}$ is not of type $(1,1)$-strong. 
  \end{enumerate}
\end{proof}

\begin{definition}{The Hardy-Littlewood Maximal Function not centered}
  Given $f\in L^{1}_{loc}(\mathbb{R}^{n})$, we will define $\overline{\mathcal{M}}f(x)$, the Hardy-Littlewood maximal function not centered of a $f$ by:
  \begin{align*}
    \overline{\mathcal{M}}f(x)=\sup_{\{B_r(z): x\in B_{r}(z)\}}\frac{1}{|B_{r}(z)|}\int_{B_{r}(z)}|f(y)|dy
  \end{align*}
\end{definition}

\begin{proposition}{}
  The operators $\overline{\mathcal{M}}$ and $\mathcal{M}$ are equivalent operators, i.e, there exist $c_n, C_n\in \mathbb{R}^{n}$ such that:
  \begin{align*}
    c_n\mathcal{M}f(x)\leq \overline{\mathcal{M}}f(x)\leq C_n\mathcal{M}f(x)
  \end{align*}
\end{proposition}

\begin{proof} 
  Suppose $B_r(z)$ such that $x\in B_{r}(z)$, then:
  \begin{align*}
    \frac{1}{|B_{r}(z)|}\int_{B_{r}(z)}|f(y)|dy&\leq \frac{1}{|B_{2r}(x)|}\int_{B_{2r}}^{}|f(y)|dy\\
    &\leq \frac{1}{|B_{r}(z)|}\frac{|B_{2r}(x)|}{|B_{2r}(x)|}\int_{B_{2r}(x)}|f(y)|dy\\
    &\leq \frac{|B_{2r}(x)|}{|B_{r}(z)|}\mathcal{M}f(x)\\
    &\leq \frac{(2r)^n}{r^n}\mathcal{M}f(x)\\
    &\leq 2^n\mathcal{M}f(x)
  \end{align*}
  then:
  \begin{align*}
    \overline{\mathcal{M}}f(x)&\leq C_n\mathcal{M}f(x) &&\text{with $C_n=2^n$.}
  \end{align*}
  Now, suppose $B_r(x)$ such that $z\in B_r(x)$, then:
  \begin{align*}
    \frac{1}{|B_r(x)|}\int_{B_{r}(x)}|f(y)|dy&\leq \frac{1}{|B_{r}(x)|}\int_{B_{2r}(z)}|f(y)|dy\\
    &\leq \frac{1}{|B_{r}(x)|}\frac{|B_{2r}(z)|}{|B_{2r}(z)|}\int_{B_{2r}(z)}|f(y)|dy\\
    &\leq \frac{|B_{2r}(z)|}{|B_{r}(x)|}\overline{\mathcal{M}}f(x)\\
    &\leq \frac{(2r)^n}{r^n}\overline{\mathcal{M}}f(x)\\
    &\leq 2^n\overline{\mathcal{M}}f(x)
  \end{align*}
  later:
  \begin{align*}
    \mathcal{M}f(x)\leq 2^n\overline{\mathcal{M}}f(x)
  \end{align*}
  then:
  \begin{align*}
    c_n\mathcal{M}f(x)&\leq \overline{\mathcal{M}}f(x) &&\text{with $c_n=\frac{1}{2^n}$.}
  \end{align*}
  Hence we can conclude that:
  \begin{align*}
    c_n\mathcal{M}f(x)\leq \overline{\mathcal{M}}f(x)\leq C_n\mathcal{M}f(x)
  \end{align*}
  with $c_n=2^{-n}$ and $C_n=2^n$. 
\end{proof}

\begin{proposition}{}
  Given $f\in L^{1}_{loc}(\mathbb{R}^{n})$, then $\overline{\mathcal{M}}f$ is measurable. 
\end{proposition}

\begin{proof} 
  Given $\lambda>0$, we will show that the set $E_{\lambda}=\{x:|\overline{\mathcal{M}}f(x)|>\lambda\}$ is a open set.\\
  Consider $x\in E_{\lambda}$. then it's satisfaces that exist $r>0$, $x_0\in \mathbb{R}^{n}$ such that $x\in B_{r}(x_0)$, suppose $z\in B_{r}(x_0)$, then:
  \begin{align*}
    \lambda &<  \frac{1}{|B_{r}(x_0)|}\int_{B_{r}(x_0)}|f(y)|dy\\
    &< \sup_{\{B_{r}(w):z\in B_{r}(w)\}}\frac{1}{|B_{r}(w)|}\int_{B_{r}(w)}|f(y)|dy\\
    &< \overline{\mathcal{M}}f(z) 
  \end{align*}
  hence so we can say that exist an open ball $B_{r}(x_0)\subseteq E_{\lambda}$ and it's conclude that $E_\lambda$ is a measurable set and hence $\overline{\mathcal{M}}f$ is measurable. 
\end{proof}

\begin{lemma}{Vitali's covering lemma}\label{lemma:vitalis_covering}
  Let $E\subseteq \mathbb{R}^{n}$ measurable, such that $E\subseteq \bigcup_{\alpha}B_{r_{\alpha}}(x_\alpha)$ with the family of open balls  $\{B_{r_{\alpha}}(x_{\alpha})\}_{\alpha}$ that satisfies $\sup_{\alpha}r_{\alpha}=c_0<\infty$.\\
  then there exist a subfamily $\{B_{r_j}(x_j)\}_{j\in\mathbb{Z}^{+}}$ disjoint and numerable, such that:
  \begin{align*}
    |E|\leq 5^{n}\sum_{j=1}^{\infty}|B_{r_j}(x_j)|
  \end{align*}
\end{lemma}

\begin{proof} 
  Take $r_1\geq \frac{c_0}{2}$, define the set:
  \begin{align*}
    A_1:=\{r_{\alpha}\in\mathbb{R}^{n}:B_{r_1}(x_1)\cap B_{r_{\alpha}}(x_{\alpha})=\emptyset\}
  \end{align*}
  Now, see by cases:
  \begin{enumerate}
    \item $A=\emptyset$.\\
      Suppose $z\in B_{r_1}(x_1)\cap B_{r_{\alpha}}(x_{\alpha})$ and $y\in B_{r_{\alpha}}(x_{\alpha})$, then: 
      \begin{align*}
        d(x_1,y)&\leq d(x_1,z) + d(z,x_{\alpha}) + d(x_{\alpha},y),\\
        &\leq r_1+2r_1+2r_1,\\
        &\leq 5r_1.
      \end{align*}
      Then, $\bigcup_{\alpha}B_{r_{\alpha}}(x_{\alpha})\subseteq B_{5r_1}(x_1)$, which concludes the proof.
    \item $A\neq \emptyset$.\\
      Take $r_2> \frac{\sup_{A_1}}{2}$, such that $B_{r_1}(x_1)\cap B_{r_2}(x_2)=\emptyset$, consider:
      \begin{align*}
        A_3:=\left\{r_\alpha\in\mathbb{R}^{n}:\left(\bigcup_{i=1}^{2}B_{r_i}(x_i)\right)\cap B_{r_{\alpha}}(x_{\alpha})=\emptyset\right\}
      \end{align*}
      We again think of the 2 cases discussed.\\
      Suppose:
      $$A_j=\{r_{\alpha}\in\mathbb{R}^{n}:\left( \bigcup_{i=1}^{j-1}B_{r_{i}}(x_i) \right)\cap B_{r_{\alpha}}(x_{\alpha})=\emptyset\}$$
      And take $r_{j}>\frac{\sup A_j}{2}$, then:
      \begin{enumerate}
        \item If $|\bigcup_{i=1}^{\infty}B_{r_i}(x_i)|=\infty$.
          \begin{align*}
            |E|\leq 5^n\sum_{i=1}^{\infty}|B_{r_i}(x_i)|=\infty
          \end{align*}
        \item If $|\bigcup_{i=1}^{\infty}B_{r_i}(x_i)|<\infty$.\\
          In this case as $\left| \bigcup_{i=1}^{\infty}B_{r_i}(x_i) \right|=\sum_{i=1}^{\infty}|B_{r_{i}}(x_i)|<\infty$, we know that $r_i\rightarrow 0$ when $i\rightarrow\infty$, then see that for all $B_{r_{\gamma}}(x_{\gamma})\in\{B_{r_{\alpha}}(x_{\alpha})\}$ it is satisface that:
          \begin{align*}
            B_{r_{\gamma}}(x_{\gamma})\subseteq \bigcup_{i=1}^{\infty}B_{5r_i}(x_i)  
          \end{align*}
          we can define $i_{\alpha}$ as the $i$ smaller such that $r_i<\frac{r_\alpha}{2}$.\\
          Note that there exist $i\in\{1,\cdots,i_{\alpha}-1\}$ such that $B_{r_{\alpha}}(x_\alpha)\cap B_{r_i}(x_i)\neq \emptyset$.\\
          Otherwise, if:
          \begin{align*}
            B_{r_{\alpha}(x_{\alpha})}\cap B_{r_i}(x_i)=\emptyset,
          \end{align*}
          then:
          \begin{align*}
            r_{\alpha}\in\{r_{\beta}\in\mathbb{R}: B_{r_{\alpha}}\cap B_{r_i}(x_i)=\emptyset\}=A,
          \end{align*}
          then, $r_{i}\geq \frac{r_{\alpha}}{2}$, but $r_{i}<\frac{r_{\alpha}}{2}$, \textbf{Contradiction}.\\
          Then exist $i\in\{1,\cdots,i_{\alpha}-1\}$ such that $B_{r_{\alpha}}(x_\alpha)\cap B_{r_i}(x_i)\neq \emptyset$.\\
          We denote $i^{*}$ like that index, hence:
          Suppose $z\in B_{r_i}(x_i)\cap B_{r_{\alpha}}(x_{\alpha})$ and $y\in B_{r_{\alpha}}(x_{\alpha})$, then: 
          \begin{align*}
            d(x_i,y)&\leq d(x_i,z) + d(z,x_{\alpha}) + d(x_{\alpha},y),\\
            &\leq r_i+2r_i+2r_i,\\
            &\leq 5r_i.
          \end{align*}
          Then
          \begin{align*}
            B_{r_{\alpha}}(x_{\alpha})\subseteq B_{5r_{i^{*}}}
          \end{align*}
      \end{enumerate}
  \end{enumerate}
  which concludes the proof.
\end{proof}

\begin{lemma}{}
  Let $f\in L^1_{loc}(\mathbb{R}^{n})$. Then the function $\mathcal{M}f$ is measurable. 
\end{lemma}

\begin{proof} 
  Given $\lambda>0$, we will show that  the set $E_{\lambda}=\{x:|\mathcal{M}f(x)|>\lambda\}$ is a open set.\\
  Consider $x\in E_{\lambda}$. then it's satisfaces that exist $r>0$ such that:
  \begin{align*}
    \frac{1}{|B_{r}(x)|}\int_{B_{r}(x)}|f(y)|dy>\lambda
  \end{align*}
  Note that exist $\lambda_1>0$ such that:
  \begin{align*}
    \frac{1}{|B_{r}(x)|}\int_{B_{r}(x)}|f(y)|dy\geq \lambda_1 > \lambda 
  \end{align*}
  Suppose $z\in B_{\epsilon}(x)$, fore some $\epsilon$. Observe that $B_{r}(x)\subset B_{r+\epsilon}(z)$. From this, and previous inequality, we deduce:
  \begin{align*}
    \lambda_1&\leq \frac{1}{|B_{r}(x)|}\int_{B_{r}(x)}|f(y)|dy\\
    &\leq \frac{1}{|B_{r}(x)|}\int_{B_{r+\epsilon}(z)}|f(y)|dy\\
    &\leq \frac{1}{|B_{r}(x)|}\frac{|B_{r+\epsilon}(z)|}{|B_{r+\epsilon}(z)|}\int_{B_{r+\epsilon}(x)}|f(y)|dy\\
    &\leq \frac{|B_{r+\epsilon}(z)|}{|B_{r}(x)|}\mathcal{M}f(z)\\
    &\leq \frac{(r+\epsilon)^{n}}{r^{n}}\mathcal{M}f(z)
  \end{align*}
  Then:
  \begin{align*}
    \frac{\lambda_1 r^{n}}{(r+\epsilon)^n}&\leq \mathcal{M}f(z), &&\text{for all $z\in B_{\epsilon}(x)$}
  \end{align*}
  This lead us to look for $\epsilon>0$ such that $\frac{\lambda_1 r^{n}}{(r+\epsilon)^{n}}>\lambda$.\\
  This is that:
  \begin{align*}
    \lambda_1r^n>\lambda(r+\epsilon)^{n}
  \end{align*}
  then:
  \begin{align*}
    \sqrt{\lambda_1}r > \sqrt{\lambda}(r+\epsilon)
  \end{align*}
  finally:
  \begin{align*}
    \frac{(\sqrt{\lambda_1}-\sqrt{\lambda})r}{\sqrt{\lambda}}>\epsilon>0
  \end{align*}
  Hence, we deduce that for $\epsilon>0$ as above, one has that $\mathcal{M}f(z)>\lambda$ for all $z\in B_{\epsilon}(x)$, which implies $B_{\epsilon}(x)\subset E_{\lambda}$. Since $x\in E_{\lambda}$ is arbitrary, we can conclude that $E_{\lambda}$ is open and finally measurable. 
\end{proof}

\begin{theorem}{Theorem of Hardy-Littlewood}
  Take $1<p\leq \infty$, then $\mathcal{M}$ is an sublinear operator of type $(p,p)-strong$, i.e, there exist $c_{p}$ such that:
  \begin{align*}
    \|\mathcal{M}f\|_{p}\leq c_p\|f\|_{p}, \hspace{0.5cm}\text{for all $f\in L^{p}(\mathbb{R}^{n}).$}
  \end{align*}
\end{theorem}

\begin{proof} 
  We can see that $\mathcal{M}$ is of type $(1,1)$-week, i.e, there exist a constant $c_1$ such that for all $f\in L(\mathbb{R}^{n})$ is satisfacied that:
  \begin{align*}
    m(\lambda,\mathcal{M}f(x))\leq \frac{c_1\|f\|_{1}}{\lambda}
  \end{align*}
  We define $E_{\lambda}=\{x:|\mathcal{M}f(x)|>\lambda\}$, take $x\in E_{\lambda}$, then $\mathcal{M}f(x)>\lambda$.\\
  There exist $r_x$ such that:
  \begin{align*}
    \frac{1}{|B_{r_x}(x)|}\int_{B_{r_x}(x)}|f(y)|dy\geq \lambda
  \end{align*}
  Which implies that:
  \begin{align*}
    \lambda|B_{r_x}(x)|\leq \int_{B_{r_x}(x)}|f(y)|dy
  \end{align*}
  Note that if $E_\lambda\subseteq \bigcup_{x\in E_{\lambda}}B_{r_x}(x)$, by the Vitali's convering lemma \cref{lemma:vitalis_covering}, exist $\{B_{r_{x_i}}(x)\}\subseteq \{B_{r_x}(x)\}$ disjoint and numerable such that:
  \begin{align*}
    |E_\lambda|=m(\lambda,\mathcal{M}f(x))&\leq 5^{n}\sum_{i=1}^{\infty}|B_{r_{x}}(x_i)|\\
    &\leq \frac{5^n}{\lambda}\sum_{i=1}^{\infty}\int_{B_{r_x}(x_i)}|f(y)|dy\\
    &\leq \frac{5^n}{\lambda}\|f\|_{1}
  \end{align*}
  Hence we can conclude that $\mathcal{M}$ is an sublinear operator of type $(1,1)$-week.\\
  Then using \cref{theorem:marcienkiewickz-theorem} we can conclude that $\mathcal{M}$ is an sublinear operator of type $(p,p)-strong$ for $1<p\leq \infty$. 
\end{proof}
