\lipsum[5]
\subsection{Weak derivaties}
\begin{notation}{}
  Let $C_c^{\infty}(U)$ denote the space of infinitely differentiable functions $\phi:U\to \mathbb{R}$, with compact support in $U$. We will sometimes call a function $\phi$ belonging to $C_{c}^{\infty}(U)$ a test function. 
\end{notation}
\textbf{Motivation for definition of weak derivative.} Assume we are given a function $u \in C^{1}(U)$. Then if $\phi\in C_{c}^{\infty}(U)$, we see from the integration by parts formula that:
\begin{equation}
  \int_{U}u \phi_{x_{i}} dx=-\int_{U}u_{x_{i}}\phi(x) dx \hspace{0.3cm}(i=1,2,\cdots).
\end{equation}
There are no boundary terms, since $\phi$ has compact support in $U$ and thus vanishes near $\partial U$. More generally now, if $k$ is a positive integer, $u\in C^{k}(U)$, and $\alpha=(\alpha_1,\cdots,\alpha_{n})$ is a multiindex of order $|\alpha|=\alpha_1+\cdots+\alpha_n=k$, then:
\begin{equation}\label{eq:weak-derivative-motivation}
  \int_{U}u\partial^{\alpha}\phi dx=(-1)^{|\alpha|}\int_{U}\partial^{\alpha}u \phi dx.
\end{equation}
We next examine formula \cref{eq:weak-derivative-motivation}, valid for $u\in C^{k}(U)$, and ask whether some variant of it might still be true even if $u$ is not $k$ times continuously differentiable. Now the left-hand side of \cref{eq:weak-derivative-motivation} makes sense if $u$ is only locally summable: the problem is rather that if $u$ is not $C^{k}(U)$, then the expression ``$\partial^{\alpha}u$'' on the right-hand side of \cref{eq:weak-derivative-motivation} has no obvious meaning. We resolve this difficulty by asking if there exists a locally summable function $v$ for which formula \cref{eq:weak-derivative-motivation} is valid, with $v$ replacing $\partial^{\alpha}u$:
\begin{definition}{}
  Suppose $u,v\in L^1_{loc}(U)$ and $\alpha$ is a multiindex. We say that $v$ is the $\alpha$-weak partial derivative of $u$, written
  \begin{align*}
    D^{\alpha}u=v,
  \end{align*}
  provided
  \begin{equation}{}\label{eq:int-weak-derivative}
    \int_{U}uD^{\alpha}\phi dx=(-1)^{|\alpha|}\int_{U}v\phi dx
  \end{equation}
  for all functions $\phi\in C^{\infty}_{c}(U)$.
\end{definition}
In other words, if we are given $u$ and if there happens to exist a function $v$ which verifies \ref{eq:int-weak-derivative} for all $\phi$, we say that $D^{\alpha}u=v$ in the weak sense. If there does not exist such a function $v$, then $u$ does not possess a weak $\alpha^{th}$-partial derivative.
\begin{lemma}{Uniqueness of weak derivaties}
  A weak $\alpha^{th}$-partial derivative of $u$, if it exist, is uniquely up to a set of measure zero.
\end{lemma}
\begin{proof} 
  Assume that $v,\tilde{v}\in L^{1}_{loc}(U)$ satisfy:
  \begin{align*}
    \int_{U}uD^{\alpha}\phi dx=(-1)^{|\alpha|}\int_{U}v\phi dx=(-1)^{|\alpha|}\int_{U}\tilde{v}\phi dx
  \end{align*}
  for all $\phi\in C^{\infty}_{c}(U)$, then:
  \begin{align*}
    \int_{U}v\phi dx-\int_{U}\tilde{v}\phi dx&=\int_{U}(v-\tilde{v})\phi dx\\
    &=0
  \end{align*}
  For all $\phi\in C^{\infty}_{c}(U)$, hence we can conclude that $v-\tilde{v}=0$ in a set of measure zero, i.e $v=\tilde{v}$ in a set of measure zero. 
\end{proof}
\begin{example}{}
  Let $n=1$, $U=(0,2)$, and:
  \begin{align*}
    u(x)= 
    \begin{cases}
      x, &\text{ if } 0<x\leq 1 \text{,} \\
      1, &\text{ if } 1<x\leq 2 .
    \end{cases}
  \end{align*}
  Define:
  \begin{align*}
    v(x)= 
    \begin{cases}
      1, &\text{ if } 0<x\leq 1 \text{,} \\
      0, &\text{ if } 1<x\leq 2 .
    \end{cases}
  \end{align*}
  Let us show $u'=v$ in a weak sense. To see this, choose any $\phi\in C^{\infty}_{c}(U)$.\\
  We must demostrate:
  \begin{align*}
    \int_{0}^{2}u\phi' dx&=-\int_{0}^{2}v\phi dx.
  \end{align*}
  But we easily calculate
  \begin{align*}
    \int_{0}^{2}u(x)\phi'(x)dx&=\int_{0}^{1}x\phi'(x)dx+\int_{1}^{2}(1)\phi'(x) dx\\
    &=-\int_{0}^{1}(1)\phi(x)dx - \int_{1}^{2}(0)\phi'(x)dx\\
    &=-\int_{0}^{2}v(x)\phi(x)dx
  \end{align*}
  as required.
\end{example}
\begin{example}{}
  Let $n=1$, $U=(0,2)$, and
  \begin{align*}
    u(x)= 
    \begin{cases}
      x, &\text{ if } 0<x\leq 1 \text{,} \\
      2, &\text{ if } 1<x\leq 2 .
    \end{cases}
  \end{align*}
  We assert $u'$ does not exist in the weak sense. To check this, we must show there does not exist any function $v\in L^{1}_{loc}(U)$ satisfying:
  \begin{equation}\label{eq:weak-example-2}
    \int_{0}^{2}u(x)\phi(x)dx=-\int_{0}^{2}v(x)\phi(x)dx,
  \end{equation}
  for all $\phi\in C^{\infty}_{c}(U)$. Suppose, to the contrary, \ref{eq:weak-example-2} were valid for some $v$ and all $\phi$. Then:
  \begin{align*}
    -\int_{0}^{2}v(x)\phi(x)dx&=\int_{0}^{2}u(x)\phi'(x)dx,\\
    &=\int_{0}^{1}x\phi'(x)dx+\int_{1}^{2}2\phi'(x)dx,\\
    &=-\int_{0}^{1}\phi(x)dx+2\phi(x)\bigg|_{1}^{2},\\
    &=-\int_{0}^{1}\phi(x)dx-2\phi(1).
  \end{align*}
  then:
  \begin{equation}\label{eq:weak-example-2.1}
    2\phi(1)=\int_{0}^{2}v(x)\phi(x)dx-\int_{0}^{1}\phi(x)dx,
  \end{equation}
  Choose a secuence $\{\phi_{m}\}_{m=1}^{\infty}$ of smooth functions satisfying
  \begin{align*}
    0\leq \phi_{m}\leq 1, \phi_m(1)=1,\phi_m(x)\to 0\text{ for all }x\neq 1.
  \end{align*}
  Replacing $\phi$ by $\phi_m$ in \ref{eq:weak-example-2.1} and sending $m\to \infty$, we discover.
  \begin{align*}
    2&=\lim_{m \to \infty}2\phi_m(1),\\
    &=\lim_{m \to \infty}\int_{0}^{2}v(x)\phi_m(x)dx-\int_{0}^{1}\phi_m(x)dx,\\
    &=0,
  \end{align*}
  a contradiction.
\end{example}
\begin{example}{Absolute value}
  let $n=1$, $U=\mathbb{R}$ and $u(x)=|x|$, then define:
  \begin{align*}
    v(x)= 
    \begin{cases}
      1, &\text{ if } x > 0 \text{,} \\
      0, &\text{ if } x = 0\text{,}\\
      -1, &\text{ if } x <  0.
    \end{cases}
  \end{align*}
  Let us show $u'=v$ in a weak sence. To see this, choose any $\phi\in C^{\infty}_{c}(\mathbb{R})$ and:
  \begin{align*}
    \int_{\mathbb{R}}u\phi'(x)dx&=\int_{\mathbb{R}} |x|\phi'(x)dx\\
    &=\int_{0}^{\infty}x\phi'(x)dx-\int_{-\infty}^{0}x\phi'(x)dx\\
    &=-\int_{0}^{\infty}(1)\phi(x)dx - \int_{-\infty}^{0}(-1)\phi(x)dx\\
    &=-\int_{\mathbb{R}}v(x)\phi(x)dx
  \end{align*}
\end{example}
\begin{example}{Characteristic function of $\mathbb{Q}$}
  
\end{example}
More sophisticated examples appear soon.
\subsection{Definition of Sobolev spaces.}
Fix $1\leq p\leq \infty$ and let $k$ be a nonnegative integer. We define now certain function spaces, whose members have weak derivaties of various orders lying in various $L^p$ spaces. 
\begin{definition}{The Sobolev space}
  \begin{align*}
    W^{k,p}(U)
  \end{align*}
  consist of all locally summable functions $u:U\to \mathbb{R}$ such that for each multiindex $\alpha$ with $|\alpha|\leq k$, $D^{\alpha}u$ exist in the weak sense and belongs to $L^{p}(U)$. 
\end{definition}
\begin{note}{}
  \begin{enumerate}[i)]
    \item If $p=2$, we usually write:
      \begin{align*}
        H^{k}(U)&=W^{k,2}(U) &&(k=1,2,\cdots)
      \end{align*}
      The letter $H$ is used , since as we will see $H^{k}(U)$ is a Hilbert space. Note that $H^{0}(U)=L^2(U)$.
    \item We henceforth identify functions in $W^{k,p}(U)$ wich agree a.e. 
  \end{enumerate}
\end{note}
\begin{definition}{}
  If $u\in W^{k,p}(U)$, we define a norm to be:
  \begin{align*}
    \|u\|_{W^{k,p}(U)}:= 
    \begin{cases}
      \left( \sum_{|\alpha|\leq k}\int_{U}|D^{\alpha}u|^{p}dx \right)^{\frac{1}{p}}, &\text{ if } 1\leq p<\infty \text{,} \\
      \sum_{|\alpha|\leq k}\sup_{U}|D^{\alpha}u|, &\text{ if } p=\infty .
    \end{cases}
  \end{align*}
\end{definition}
\begin{definition}{}
  \begin{enumerate}[i)]
    \item Let $\{u_m\}_{m=1}^{\infty}$, $u\in W^{k,p}(U)$. We say $u_m$ converges to $u$ in $W^{k,p}(U)$, written:
      \begin{align*}
        u_{m}\to u\text{ in }W^{k,p}(U),
      \end{align*}  
      provided
      \begin{align*}
        \lim_{m \to \infty}\|u_{m}-u\|_{W^{k,p}(U)}=0.
      \end{align*}
    \item We writte:
      \begin{align*}
        u_m\to u\text{ in }W^{k,p}_{loc}(U),
      \end{align*}
      to mean:
      \begin{align*}
        u_m\to u\text{ in }W^{k,p}(V),
      \end{align*}
      for each $V\subset \subset U$.
  \end{enumerate}
\end{definition}
\begin{definition}{}
  We denote by:
  \begin{align*}
    W^{k,p}_0(U)
  \end{align*}
  The closure of $C^{\infty}_{c}(U)$ in $W^{k,p}(U)$. 
\end{definition}
Thus $u\in W^{k,p}_0(U)$ if and only if there exist functions $u_m\in C^{\infty}_{c}(U)$ such that $u_m\to u$ in $W^{k,p}(U)$. We interpret $W^{k,p}_{0}(U)$ as comprising those functions $u\in W^{k,p}(U)$ such that:
\begin{align*}
  "D^{\alpha}u=0\text{ on $\partial U$" for all $|\alpha|\leq k-1$.}
\end{align*}
\begin{note}{}
  It is customary to write:
  \begin{align*}
    H^{k}_{0}(U)=W^{k,2}_{0}(U).
  \end{align*}
\end{note}
We will see in the exercises that if $n=1$ and $U$ is an open interval in $\mathbb{R}^{1}$, then $u\in W^{k,p}(U)$ if and only if $u$ equals a.e. an absolutely continuous function whose ordinary derivative (which exists a.e.) belongs to $L^{p}(U)$.\\
Such a simple characterization is however only available for $n=1$. In general a function can belong to a Sobolev space and yet be discontinuous and/or unbounded.
\begin{example}{}
  Take $U=B_1(0)$, the open unit ball in $\mathbb{R}^{n}$, and
  \begin{align*}
    u(x)&=|x|^{-\alpha} && x\in U, x\neq 0.
  \end{align*}
  For which values of $\alpha>0$, $n$, $p$ does $u$ belong to $W^{1,p}(U)$? To answer, note first that $u$ is smooth away from $0$, with
  \begin{align*}
    u_{x_{i}}(x)&=\frac{-\alpha x_{i}}{|x|^{\alpha+2}} && x\neq 0,
  \end{align*}
  and so:
  \begin{align*}
    |Du(x)|&=\frac{|\alpha|}{|x|^{\alpha+1}} &&x\neq 0.
  \end{align*}
  Let $\phi\in C^{\infty}_{c}(U)$ and fix $\epsilon>0$. Then
  \begin{align*}
    \int_{U\setminus B_{\epsilon}(0)}u\phi_{x_{i}}dx&=-\int_{U-B_{\epsilon}(0)}u_{x_{i}}\phi dx+\int_{\partial B_{\epsilon}(0)}u\phi\eta_{i}dS(x),
  \end{align*}
  $\eta=(\eta_1,\cdots,\eta_n)$ denoting the inward pointing normal on $\partial B_{\epsilon}(0)$. Now if $\alpha+1<n$ $|Du(x)|\in L^{1}(U)$:
  \begin{align*}
    \int_{U}|Du(x)|dx&=\int_{U}\frac{|\alpha|}{|x|^{\alpha+1}}dx\\
    &=|\alpha|\int_{0}^{1}\frac{r^{n-1}}{r^{\alpha+1}}dr\\
    &=|\alpha|\int_{0}^{1}r^{n-1-(\alpha+1)}dr < \infty 
  \end{align*}
  In this case:
  \begin{align*}
    \left| \int_{\partial B_{\epsilon}(0)}u\phi\eta_i dS(x)\right|&\leq \|\phi\|_{\infty}\int_{\partial B_{\epsilon}(0)}\left||x|^{-\alpha}\right||\eta_i|dS(x)\\
    &\leq \|\phi\|_{\infty}\int_{\partial B_{\epsilon}(0)}\epsilon^{-\alpha}dS(x)\\
    &\leq \|\phi\|_{\infty}\epsilon^{-\alpha}\int_{\partial B_{\epsilon}(0)}dS(x)\\
    &\leq C\epsilon^{n-1-\alpha}\to 0.
  \end{align*}
  Thus:
  \begin{align*}
    \int_{U}u\phi_{x_{i}}dx=-\int_{U}u_{x_{i}}\phi dx
  \end{align*}
  for all $\phi\in C^{\infty}_{c}(U)$, provided $0\leq \alpha < n-1$. Furthermore $|Du(x)|=\frac{|\alpha|}{|x|^{\alpha+1}}\in L^{p}(U)$ if and only if $(\alpha+1)p<n$. Consequenly $u\in W^{1,p}(U)$ if and only if $\alpha <\frac{n-p}{p}$. In particular $u \notin W^{1,p}(U)$ for each $p\leq n$. 
\end{example}
\begin{example}{}
  Por revisar
\end{example}
This last example illustrates a fundamental fact of life, that although a function $u$ belonging to a Sobolev space possesses certain smoothness properties, it can still be rather badly behaved in other ways.
\subsection{Elementary properties.}
Next we verify certain properties of weak derivatives. Note very carefully that whereas these various rules are obviously true for smooth functions, functions in Sobolev space are not necessarily smooth: we must always rely solely upon the definition of weak derivatives.
\begin{theorem}{Properties of weak derivaties}
  Assume $u,v\in W^{k,p}(U)$, and $|\alpha|\leq k$. Then:
  \begin{enumerate}
    \item $D^{\alpha}u\in W^{k-|\alpha|,p}(U)$ and $D^{\beta}(D^{\alpha}u)=D^{\alpha}(D^{\beta}u)=D^{\alpha+\beta}u$ for all multiindex $\alpha,\beta$ with $|\alpha|+|\beta|\leq k$.
    \item For each $\lambda,\mu\in\mathbb{R},\lambda u+\mu v\in W^{k,p}(U)$ and $D^{\alpha}(\lambda u+\mu v)=\lambda D^{\alpha}u+\mu D^{\alpha}v$, $|\alpha|\leq k$.
    \item If $V$ is an open subset of $U$, then $u\in W^{k,p}(V)$.
    \item If $\zeta\in C^{\infty}_{c}(U)$, then $\zeta u\in W^{k,p}(U)$ and:
    \begin{equation}
      D^{\alpha}(\zeta u)=\sum_{\beta\leq \alpha}\binom{\alpha}{\beta}D^{\beta}\zeta D^{\alpha-\beta}u \text{ Leibniz's formula},
    \end{equation}
    where $\binom{\alpha}{\beta}=\frac{\alpha!}{\beta!(\alpha-\beta)!}$.
  \end{enumerate}
\end{theorem}
