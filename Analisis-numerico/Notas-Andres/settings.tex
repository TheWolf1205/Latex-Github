\usepackage[utf8]{inputenc}
\usepackage{graphicx}
\usepackage{epigraph}
\usepackage{enumerate}
\usepackage[spanish]{babel}

\usepackage{amsmath,amssymb}  % Para matemáticas avanzadas
\usepackage{hyperref}         % Para enlaces dentro del documento
\usepackage{cleveref}
\usepackage{tikz}
\usepackage{xcolor}
\usepackage{csquotes}
\usepackage[most]{tcolorbox}

\graphicspath{ {images/} }
\usepackage{caption}
\usepackage{subcaption}
\usepackage{float}
\usepackage[width=150mm,top=35mm,bottom=25mm,bindingoffset=6mm]{geometry}
\usepackage{fancyhdr}
\usepackage{setspace}
\usepackage{lipsum} % Libreria con el texto de prueba

% widehat y widecheck
\DeclareFontFamily{U}{mathx}{}
\DeclareFontShape{U}{mathx}{m}{n}{<-> mathx10}{}
\DeclareSymbolFont{mathx}{U}{mathx}{m}{n}
\DeclareMathAccent{\widecheck}{0}{mathx}{"71}
\renewcommand{\check}{\widecheck}
\renewcommand{\hat}{\widehat}
\renewcommand{\check}{\widecheck}

% Integral de media
\def\Xint#1{\mathchoice
{\XXint\displaystyle\textstyle{#1}}%
{\XXint\textstyle\scriptstyle{#1}}%
{\XXint\scriptstyle\scriptscriptstyle{#1}}%
{\XXint\scriptscriptstyle\scriptscriptstyle{#1}}%
\!\int}
\def\XXint#1#2#3{{\setbox0=\hbox{$#1{#2#3}{\int}$ }
\vcenter{\hbox{$#2#3$ }}\kern-.6\wd0}}
\def\ddashint{\Xint=}
\def\dashint{\Xint-}

% norma 3 lineas
\newcommand{\seminorm}[1]{{\left\vert\kern-0.25ex\left\vert\kern-0.25ex\left\vert #1 
    \right\vert\kern-0.25ex\right\vert\kern-0.25ex\right\vert}}

% norma alargada
\newcommand{\norm}[1]{\left\lVert#1\right\rVert}

% Comportamiento de cref
\usepackage{hyperref}
\usepackage{cleveref}

% Definir colores personalizados para las cajas
\definecolor{mygrayback}{RGB}{245, 245, 245}
\definecolor{mygrayframe}{RGB}{80, 80, 80}
\definecolor{mygrayframeproof}{RGB}{140, 140, 140}

% Redefinir el ambiente de Teorema
\newtcolorbox[auto counter, number within=section]{theorem}[2][]{%
  colback=mygrayback, colframe=mygrayframe, fonttitle=\bfseries,
  title=Teorema~\thetcbcounter: #2,#1, breakable}

% Redefinir el ambiente de Lema
\newtcolorbox[auto counter, number within=section]{lemma}[2][]{%
  colback=mygrayback, colframe=mygrayframe, fonttitle=\bfseries,
  title=Lema~\thetcbcounter: #2,#1, breakable}

% Redefinir el ambiente de Proposición
\newtcolorbox[auto counter, number within=section]{proposition}[2][]{%
  colback=mygrayback, colframe=mygrayframe, fonttitle=\bfseries,
  title=Proposición~\thetcbcounter: #2,#1, breakable}

% Redefinir el ambiente de Nota
\newtcolorbox[auto counter, number within=section]{note}[2][]{%
  colback=mygrayback, colframe=mygrayframe, fonttitle=\bfseries,
  title=Nota~\thetcbcounter: #2,#1, breakable}

% Redefinir el ambiente de Corolario
\newtcolorbox[auto counter, number within=section]{corollary}[2][]{%
  colback=mygrayback, colframe=mygrayframe, fonttitle=\bfseries,
  title=Corolario~\thetcbcounter: #2,#1, breakable}

% Redefinir el ambiente de Ejemplo
\newtcolorbox[auto counter, number within=section]{example}[2][]{%
  colback=mygrayback, colframe=mygrayframe, fonttitle=\bfseries,
  title=Ejemplo~\thetcbcounter: #2,#1, breakable}

% Redefinir el ambiente de Definición
\newtcolorbox[auto counter, number within=section]{definition}[2][]{%
  colback=mygrayback, colframe=mygrayframe, fonttitle=\bfseries,
  title=Definición~\thetcbcounter: #2,#1, breakable}

% Redefinir el ambiente de Notación
\newtcolorbox[auto counter, number within=section]{notation}[2][]{%
  colback=mygrayback, colframe=mygrayframe, fonttitle=\bfseries,
  title=Notación~\thetcbcounter: #2,#1, breakable}


% Redefinir el ambiente de Demostración
\newtcolorbox{proof}[1][]{%
  colback=mygrayback, colframe=mygrayframeproof, fonttitle=\bfseries, 
  title=Demostración, #1, breakable,
  after upper={\hfill\(\square\)} % Cuadro al final
}

% Redefinir el ambiente de Solución
\newtcolorbox{solution}[1][]{%
  colback=mygrayback, colframe=mygrayframeproof, fonttitle=\bfseries, title=Solución,#1, breakable}

