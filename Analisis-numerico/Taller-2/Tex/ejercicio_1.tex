\begin{homeworkProblem}
  Sea $A\in \mathbb{R}^{m\times n}$. Entonces se satisface:
  \begin{enumerate}
    \item $\|A\|_2=\|A^{T}\|_{2}\leq \|A\|_{F}=\|A^{T}\|_{F}$,
    \item $\|A\|_{\infty}\leq \sqrt{n}\|A\|_{2}$,
    \item $\|A\|_{2}\leq \sqrt{m}\|A\|_{\infty}$,
    \item $\|A\|_{2}\leq \sqrt{\|A\|_{1}\|A\|_{\infty}}$.
  \end{enumerate}
  \begin{solucion}
    \begin{enumerate}
      \item $\|A\|_2=\|A^{T}\|_{2}\leq \|A\|_{F}=\|A^{T}\|_{F}$.\\
        Veamos primero que $\|A\|_{2}=\|A^{T}\|_{2}$, para esto recordemos que:
        \begin{align*}
          \|A\|_{2}=\sqrt{\lambda_{m\acute{a}x}(A^TA)}
        \end{align*}
        Por lo que sabemos que $\|A\|_{2}$ depende totalmente de sus valores propios, por lo que será suficiente para concluir el resultado con ver que los valores propios de $A^{T}A$ son los mismos valores propios de $(A^{T})^{T}A^{T}=AA^{T}$, para esto será suficiente con verificar que si tomamos $\lambda\neq 0$ valor propio de $A^TA$, entonces:
        \begin{align*}
          A^TAx&=\lambda x\\
          AA^TAx&=\lambda Ax\\
          AA^Ty&=\lambda y
        \end{align*}
        Por lo que podríamos decir que $\lambda$ también es un valor propio de $AA^T$, note que de forma análoga se puede repetir el mismo procedimiento para ver que los valores propios de $AA^T$ son valores propios de $A^TA$ y por ende estas 2 matrices comparten valores propios distintos de $0$, luego se puede concluir que $\sqrt{\lambda_{m\acute{a}x}(A^TA)}=\sqrt{\lambda_{m\acute{a}x}(AA^T)}$, es decir, $\|A\|_2=\|A^T\|_{2}$.
        Ahora, veamos que $\|A\|_{F}=\|A^T\|_{F}$, para esto recordemos que:
        \begin{align*}
          \|A\|_{F}=\left( \sum_{i=1}^n \sum_{j=1}^{m}|a_{ij}|^2 \right)^{\frac{1}{2}}
        \end{align*}
        por lo que si tomamos $a_{ij}\in A$ y $\tilde{a}_{ji}\in A^T$, debería ser fácil seguir el siguiente cálculo:
        \begin{align*}
          \|A\|_{F}&=\left( \sum_{i=1}^n \sum_{j=1}^{m}|a_{ij}|^2 \right)^{\frac{1}{2}}\\
          &=\left( \sum_{i=1}^n \sum_{j=1}^{m}|\tilde{a}_{ji}|^2 \right)^{\frac{1}{2}}\\
          &=\left( \sum_{j=1}^m \sum_{i=1}^{n}|\tilde{a}_{ji}|^2 \right)^{\frac{1}{2}}\\
          &=\|A^T\|_{F}
        \end{align*}
        Ahora veamos que $\|A\|_{2}\leq \|A\|_{F}$.\\
        Primero note que si tomamos la matriz $A^TA\in R^{n\times n}$ se cumple que:
        \begin{align*}
          [A^TA]_{ij}=\sum_{k=1}^{m}a_{ki}a_{kj}
        \end{align*}
        Luego:
        \begin{align*}
          tr(A^TA)&=\sum_{i=1}^n [A^TA]_{ii}\\
          &=\sum_{i=1}^{n}\sum_{j=1}^{m}a_{ji}a_{ji}\\
          &=\|A\|_{F}
        \end{align*}
        además sabemos que $tr(A^{T}A)=\sigma_1^2+\sigma_2^2+\cdots+\sigma_r^2$ y que $\|A\|_2^2=\sigma_{1}^2$, por lo que podemos asegurar que:
        \begin{align*}
          \|A\|_2&\leq \sigma_1^2\\
          &\leq \sigma_1^2+\sigma_2^2+\cdots+\sigma_r^2\\
          &\leq \|A\|_{F}^2
        \end{align*}
        lo que implica que $\|A\|_2\leq \|A\|_F$, lo que concluye el ejercicio.
        \demostrado
      \item $\|A\|_{\infty}\leq \sqrt{n}\|A\|_2$.\\
        Note que:
        \begin{align*}
          \|A\|_{\infty}&=\max_{i=1,\cdots,m}\sum_{j=1}^{n}|a_{ij}|\\
        \end{align*}
        Ahora, suponga $A_i=(a_{i1},a_{i2},\cdots,a_{in})\in \mathbb{R}^{n}$, note que:
        \begin{align*}
          \sum_{j=1}^{n}|a_{ij}|&= (|a_{i1}|,|a_{i2}|,\cdots,|a_{in}|)\cdot (1,1,\cdots,1) &&\text{Luego por Cauchy-Schwarz.}\\
          &\leq \|(|a_{i1}|,|a_{i2}|,\cdots,|a_{in}|)\|_2\|(1,1,\cdots,1)\|_{2}\\
          &\leq \sqrt{\sum_{j=1}^n|a_{ij}|^{2}}\times \sqrt{\sum_{j=1}^{n}|1|^2}\\
          &\leq \sqrt{n}\|A_i\|_{2}
        \end{align*}
        Ahora, como $\|A_i\|_{2}\leq \|A\|_{2}$ para todo $i=1,\cdots,m$ al ser consistente, al tomar máximo en ambos lados de la desigualdad se cumple que:
        \begin{align*}
          \|A\|_{1}&\leq \max_{i=1,\cdots,m}\sum_{j=1}^{n}|a_{ij}|\\
          &\leq \sqrt{n}\|A\|_{2}
        \end{align*}
        lo que concluye el resultado esperado.
        \demostrado
      \item $\|A\|_{2}\leq\sqrt{m}\|A\|_{\infty}$.\\
        Sea $x\in \mathbb{R}^{n}$ tal que $\|x\|_2=1$, y denotemos:
        \begin{align*}
          (Ax)_{i}=\sum_{j=1}^{n}a_{ij}x_{j}
        \end{align*}
        entonces:
        \begin{align*}
          |(Ax)_i|&=\left|\sum_{j=1}^na_{ij}x_{j}\right|\\
          &\leq \sum_{j=1}^{n}|a_{ij}||x_j|\\
          &\leq \sqrt{\sum_{j=1}^{n}|a_{ij}|^2}\|x\|_{2} &&\text{Por Cauchy-Schawrz.}\\
          &\leq \sqrt{\sum_{j=1}^{n}|a_{ij}|^2}
        \end{align*}
        luego como:
        \begin{align*}
          \|Ax\|_2^2&=\sum_{i=1}^{m}|(Ax)_i|^2\\
          &=\sum_{i=1}^{m}\sum_{j=1}^n|a_{ij}^2|\\
          &\leq \sum_{i=1}^{m}\|A\|^2_{\infty}\\
          &\leq m\|A\|^2_{\infty}
        \end{align*}
        luego tomando raiz cuadrada en ambos lados y el máximo de los $\|Ax\|_2$ con $x$ de norma 1, se puede concluir que $\|A\|_2\leq \sqrt{m}\|A\|_{\infty}$, lo que concluye el resultado esperado.
      \item $\|A\|_2\sqrt{\|A\|_1\|A\|_\infty}$.\\
        Sea $x\in\mathbb{R}^{n}$ con $\|x\|=1$ arbitrario, entonces:
        \begin{align*}
          (Ax)_i=\sum_{j=1}^{n}a_ijx_j
        \end{align*}
        Y utilizando la definición de $\|x\|_1$ tenemos que:
        \begin{align*}
          \|Ax\|_1&=\sum_{i=1}^m\left| (Ax)_i \right|\\
          &=\sum_{i=1}^{m}\left| \sum_{j=1}^{n}a_{ij}x_j \right|
        \end{align*}
        usando el procedimiento que usamos en el primer ejercicio sabemos que:
        \begin{align*}
          \|Ax\|_{2}^2&\leq\sum_{i=1}^{m}\left| \sum_{j=1}^{n}a_{ij}x_j \right|^2\\
          &\leq\sum_{i=1}^{m}\left|\sum_{j=1}^{n}a_{ij}x_j\right|\max_{i=1,\cdots,m}\left|\sum_{j=1}^{n}a_{ij}x_j\right|\\
          &\leq \|Ax\|_1\|Ax\|_{\infty}\\
          &\leq \|A\|_{1}\|A\|_{\infty}
        \end{align*}
        Luego tomando raíz y luego tomando el máximo variando los $\|x\|=1$ se concluye que:
        \begin{align*}
          \|A\|_2\leq \sqrt{\|A\|_{\infty}\|A\|_{1}}
        \end{align*}
        Lo que finaliza el ejercicio.
        \demostrado
    \end{enumerate}  
  \end{solucion}
\end{homeworkProblem}
