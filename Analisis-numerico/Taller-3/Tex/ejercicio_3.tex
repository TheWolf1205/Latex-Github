\begin{homeworkProblem}
  \begin{align*}
    A =  
\begin{pmatrix}  
a & 0 & 0 \\  
a\delta & a & 0 \\  
0 & a\delta & a  
\end{pmatrix}, \quad a < 0, \quad \delta > 0
\quad\text{y}\quad  b =  
\begin{pmatrix}  
-1 \\  
-1.1 \\  
0  
\end{pmatrix}.
\end{align*}


\begin{enumerate}
    \item[(a)] Obtenga el número de condición de \( A \).
    \begin{solucion}
      Hallemos el número de condición $\kappa_\infty(A) = \|A^{-1}\|_\infty\|A\|_\infty $.Para ello,  encontremos la inversa de $A$:
        


\[
\left(
\begin{array}{ccc|ccc}
a & 0 & 0 & 1 & 0 & 0 \\
a\delta & a & 0 & 0 & 1 & 0 \\
0 & a\delta & a & 0 & 0 & 1
\end{array}
\right)
R_2 \rightarrow R_2 - \delta R_1,\\
\quad R_3 \rightarrow R_3 - \delta R_2
\]

\[
\left(
\begin{array}{ccc|ccc}
a & 0 & 0 & 1 & 0 & 0 \\
0 & a & 0 & -\delta & 1 & 0 \\
0 & 0 & a & \delta^2 & -\delta & 1
\end{array}
\right)
R_1 \rightarrow \frac{R_1}{a}, \quad R_2 \rightarrow \frac{R_2}{a}, \quad R_3 \rightarrow \frac{R_3}{a}
\]

\[
\left(
\begin{array}{ccc|ccc}
1 & 0 & 0 & \frac{1}{a} & 0 & 0 \\
0 & 1 & 0 & -\frac{\delta}{a} & \frac{1}{a} & 0 \\
0 & 0 & 1 & \frac{\delta^2}{a} & -\frac{\delta}{a} & \frac{1}{a}
\end{array}
\right)\hspace{0.5cm}\Longrightarrow \hspace{0.5cm}
A^{-1} =
\begin{pmatrix}
\frac{1}{a} & 0 & 0 \\
-\frac{\delta}{a} & \frac{1}{a} & 0 \\
\frac{\delta^2}{a} & -\frac{\delta}{a} & \frac{1}{a}
\end{pmatrix}
\]
Ahora calculemos las normas correspondientes,
\begin{align*}
    \kappa_\infty(A) = \|A^{-1}\|_\infty\|A\|_\infty &=\left(\left|\frac{\delta^2}{a}\right|+\left|\frac{-\delta}{a}\right|+\left|\frac{1}{a}\right|\right)\left(|a\delta|+|a|\right)\\
    &=(\delta+1)(\delta^2+\delta+1) \hspace{1cm},\delta>0\\
    &=\delta^3+2\delta^2+2\delta+1
\end{align*}
Por consiguiente, $\kappa_\infty(A)=\delta^3+2\delta^2+2\delta+1$, con $\delta>0$.  
    \end{solucion}
    
    \item[(b)] Estudiar el condicionamiento del sistema \( Ax = b \) en función de los valores de \( \delta \). Interprete su resultado.
    \begin{solucion}
      Como $\kappa_\infty(A) = \delta^3 + 2\delta^2 + 2\delta + 1$ con $\delta > 0$, notemos que:  
\begin{itemize}
    \item Si $\delta \rightarrow 0$, entonces $\kappa_\infty(A) \rightarrow 1$.  
    \item Si $\delta \rightarrow \infty$, entonces $\kappa_\infty(A) \rightarrow \infty$. 
\end{itemize}

Por consiguiente, si $\delta \ll 1$, la matriz estará bien condicionada, lo que implica que los cambios relativos en $b$ controlan los cambios relativos en $x$. En cambio, si $\delta \gg 1$, el sistema será mal condicionado, es decir, pequeños cambios en los datos pueden generar grandes errores en la solución.    
    \end{solucion}
    \item[(c)] Si \( a = -1 \), \( \delta = 0.1 \) y se considera $x^* =  \left(1, \frac{9}{10},  1 \right)^T$ como solución aproximada del sistema \( Ax = b \) (sin obtener la solución exacta), determine un intervalo en el que esté comprendido el error relativo. ¿Es coherente con la respuesta dada en el apartado anterior?
    \begin{solucion}
      Reemplazando $a=-1$ y $\delta=-0.1$ en la matriz y tomando $x^* =  \left(1, \frac{9}{10},  1 \right)^T$, obtenemos:

\begin{align*}
    \underbrace{
        \begin{pmatrix}
            -1 & 0 & 0\\
            -0.1 & -1 & 0\\
            0 & -0.1 & -1
        \end{pmatrix}
        }_{A}
        \underbrace{
        \begin{pmatrix}
            1  \\
            \frac{9}{10}  \\
            1 
        \end{pmatrix}
        }_{x^*}=
        \underbrace{
        \begin{pmatrix}
            -1 \\
            -1 \\
            -\frac{109}{100}
        \end{pmatrix}
        }_{b^*}\approx \underbrace{
        \begin{pmatrix}
            -1 \\
            -1.1 \\
            0
        \end{pmatrix}
        }_{b}
\end{align*}

Usemos la siguiente expresión para encontrar una cota para el error relativo:
\begin{align*}
    \frac{1}{\kappa_\infty(A)}\frac{\|b-b^*\|}{\|b\|} \leq \frac{\|x-x^*\|}{\|x\|} \leq \kappa_\infty(A)\frac{\|b-b^*\|}{\|b\|}
\end{align*}

Antes de reemplazar, calculemos algunos valores:
\begin{align*}
    \kappa_\infty(A) &= \delta^3 + 2\delta^2 + 2\delta + 1\\
    &= \left(\frac{1}{10}\right)^3 + 2\left(\frac{1}{10}\right)^2 + 2\left(\frac{1}{10}\right) + 1 = \frac{1221}{1000}\\
    \|b\|_\infty &= \left|-1.1\right| = 1.1\\
    \|b-b^*\|_\infty &= \left|\frac{-109}{100}\right| = \frac{109}{100}\\
    \frac{\|b-b^{*}\|_{\infty}}{\|b\|_{\infty}}&=\frac{109}{110}
\end{align*}

Reemplazando, la cota del error relativo queda de la siguiente forma:
\begin{align*}
    \frac{1}{1.221} \frac{109}{110} &\leq \frac{\|x-x^*\|}{\|x\|} \leq (1.221) \frac{109}{110}\\
    0.81156 &\leq \frac{\|x-x^*\|}{\|x\|} \leq 1.2099
\end{align*}

Es decir, el error relativo está entre $81.16\%$ y $120.99\%$, lo cual considero que no hace perder coherencia con los resultados anteriores, ya que si bien el error relativo parece ser grande, aunque se está tomando un $\delta \ll 1$, que deja la matriz bien condicionada, la matriz no puede controlar lo malas que seas las soluciones aproximadas propuestas, como lo es en este caso.  
    \end{solucion}
    \item[(d)] Si \( a = -1 \) y \( \delta = 0.1 \), ¿es convergente el método de Jacobi aplicado a la resolución del sistema \( Ax = b \)?  
    Realice tres iteraciones a partir de $x^* =  \left(0, 0,  0\right)^T$.
    \begin{solucion}
      Sea
\[
A = \begin{pmatrix}
            -1 & 0 & 0\\
            -0.1 & -1 & 0\\
            0 & -0.1 & -1
        \end{pmatrix}
\]
Reescribamos la matriz como $ A = D + L + U $, donde $ D $ es la matriz diagonal de $ A $, $ L $ es la matriz con los elementos debajo de la diagonal, y $ U $ es la matriz con los elementos por encima de la diagonal:

\[
A = \begin{pmatrix}
            -1 & 0 & 0\\
            0 & -1 & 0\\
            0 & 0 & -1
        \end{pmatrix}+
        \begin{pmatrix}
            0 & 0 & 0\\
            -0.1 & 0 & 0\\
            0 & -0.1 & 0
        \end{pmatrix}+
        \begin{pmatrix}
            0 & 0 & 0\\
            0 & 0 & 0\\
            0 & 0 & 0
        \end{pmatrix}
\]

Ahora, calculemos la matriz de iteración $ T_J $ para el método iterativo de Jacobi, dada por:
\[
T_J = -D^{-1}(L+U).
\]
Dado que en este caso $ U = 0 $, la expresión se reduce a $ T_J = -D^{-1} L $:
\[
T_J = -D^{-1} L =
\begin{pmatrix}
        1 & 0 & 0\\
        0 & 1 & 0\\
        0 & 0 & 1
\end{pmatrix}
\begin{pmatrix}
            0 & 0 & 0\\
            -0.1 & 0 & 0\\
            0 & -0.1 & 0
\end{pmatrix}
=
\begin{pmatrix}
            0 & 0 & 0\\
            0.1 & 0 & 0\\
            0 & 0.1 & 0
\end{pmatrix}
\]
Como $ \|T_J\|_\infty = 0.1 < 1 $, el método de Jacobi converge según el criterio de la norma.

A continuación, escribamos explícitamente la forma matricial de las iteraciones, que se define como:
\[
x^{(k+1)} = -D^{-1} L x^{(k)} + D^{-1} b.
\]
Sustituyendo los valores:
\[
\begin{pmatrix}
            x_1^{(k+1)} \\
            x_2^{(k+1)} \\
            x_3^{(k+1)}
\end{pmatrix}
=
\begin{pmatrix}
            0 & 0 & 0\\
            -0.1 & 0 & 0\\
            0 & -0.1 & 0
\end{pmatrix}
\begin{pmatrix}
            x_1^{(k)} \\
            x_2^{(k)} \\
            x_3^{(k)}
\end{pmatrix}
+
\begin{pmatrix}
            1 \\
            1.1 \\
            0
\end{pmatrix}.
\]

Ahora, realicemos tres iteraciones del método de Jacobi:

\[
\begin{pmatrix}
            x_1^{(1)} \\
            x_2^{(1)} \\
            x_3^{(1)}
\end{pmatrix}
=
\begin{pmatrix}
            1 \\
            1.1 \\
            0
\end{pmatrix}.
\]

\[
\begin{pmatrix}
            x_1^{(2)} \\
            x_2^{(2)} \\
            x_3^{(2)}
\end{pmatrix}
=
\begin{pmatrix}
            1 \\
            1 \\
            -0.11
\end{pmatrix}.
\]

\[
\begin{pmatrix}
            x_1^{(3)} \\
            x_2^{(3)} \\
            x_3^{(3)}
\end{pmatrix}
=
\begin{pmatrix}
            1 \\
            1 \\
            -0.1
\end{pmatrix}.
\]

Notemos que:

\[
Ax^{(3)} = \begin{pmatrix}
            -1 & 0 & 0\\
            -0.1 & -1 & 0\\
            0 & -0.1 & -1
        \end{pmatrix}
        \begin{pmatrix}
            1 \\
            1 \\
            -0.1
        \end{pmatrix}
        =
        \begin{pmatrix}
            -1 \\
            -1.1 \\
            0
        \end{pmatrix}.
\]

Por lo tanto, el método de Jacobi no solo converge, sino que lo hace rápidamente y en solo tres iteraciones alcanza la solución exacta.  
    \end{solucion}
\end{enumerate}
\end{homeworkProblem}
