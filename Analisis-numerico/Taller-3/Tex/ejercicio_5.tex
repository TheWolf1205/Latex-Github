\begin{homeworkProblem}
  Considere el sistema $Ax=b$ donde
  \begin{align*}
    A:=\begin{pmatrix}
      3 & -1 & -1 & 0 & 0 \\
      -1 & 4 & 0 & -2 & 0 \\
      -1 & 0 & 3 & -1 & 0 \\
      0 & -2 & -1 & 5 & -1 \\
      0 & 0 & 0 & -1 & 2
    \end{pmatrix}
    \quad b:=\begin{pmatrix}
      2 \\
      -26 \\
      3 \\
      47 \\
      -10 
    \end{pmatrix}
  \end{align*}
  \begin{enumerate}
    \item Investigue la convergencia de los métodos de Jacobi, Gauss-Seidel y sobrerelajación.
      \begin{solucion}
        \begin{enumerate}
          \item Convergencia en Jacobi.\\
            Note que $A$ es una matriz diagonal dominante, por lo que se puede asegurar que el método converge desde cualquier solución inicial $\textbf{x}^{0}$.
          \item Convergencia en Gauss-Seidel.\\
            Note que $T_{GS}$ es:
            \begin{align*}
              T_{GS}=(D-L)^{-1}U=\begin{pmatrix}
                0 & \frac{1}{3}  & \frac{1}{3}  & 0            & 0\\
                0 & \frac{1}{12} & \frac{1}{12} & \frac{1}{2}  & 0\\
                0 & \frac{1}{9}  & \frac{1}{9}  & \frac{1}{3}  & 0\\
                0 & \frac{1}{18} & \frac{1}{18} & \frac{4}{15} & \frac{1}{5}\\
                0 & \frac{1}{36} & \frac{1}{36} & \frac{2}{15} & \frac{1}{10}
              \end{pmatrix}
            \end{align*}
            Luego el radio espectral de $T_{GS}$ es $1061/2071\approx 0.51 < 1$, por lo que sabemos que la sucesión converge con cualquier $\textbf{c}$ y $\textbf{x}^0$.\\
            Estos cálculos se ayudaron de matlab
            \newpage
            \begin{lstlisting}[language = matlab]
% Definir la matriz
A = [ 3  -1  -1   0   0;
     -1   4   0  -2   0;
     -1   0   3  -1   0;
      0  -2  -1   5  -1;
      0   0   0  -1   2];
D = [ 3   0   0   0   0;
      0   4   0   0   0;
      0   0   3   0   0;
      0   0   0   5   0;
      0   0   0   0   2];
L = [ 0   0   0   0   0;
      1   0   0   0   0;
      1   0   0   0   0;
      0   2   1   0   0;
      0   0   0   1   0];
U = [ 0   1   1   0   0;
      0   0   0   2   0;
      0   0   0   1   0;
      0   0   0   0   1;
      0   0   0   0   0];

TGS = inv(D-L)*U;
latex_codigo = latex(sym(TGS));

% Obtener valores propios
valores_propiosGS = eig(TGS);

% Calcular el radio espectral
radio_espectralGS = max(abs(valores_propiosGS));

% Mostrar resultado
disp('Radio espectral en Gauss-Seidel:')
disp(radio_espectral)
            \end{lstlisting}
          \item Convergencia en SOR.\\
            Note que $A$ es una matriz simétrica definida positiva, ya que el polinomio carácteristico es $P_A(x)=167 - 373 x + 295 x^2 - 105 x^3 + 17 x^4 - x^5$, el cuál tiene raíces reales, positivas (esto se puede comprobar utilizando graficadoras o un programa de cómputo de factorización, omitiremos ese paso ya que lo único importante es saber que son reales), por lo que sabemos que si tomamos $0<w<2$ entonces el método SOR converge a la única solución del sistema $Ax=b$ desde cualquier solución inicial $\textbf{x}^{0}$. 
        \end{enumerate}
      \end{solucion}
      \newpage
    \item ¿Cuál es el radio espectral de la matriz $J$ y de la matriz $S$?
      \begin{solucion}
        \begin{enumerate}
          \item Jacobi.\\
          La matriz $T_J$ es:
          \begin{align*}
            \begin{pmatrix}
              0           & \frac{1}{3} & \frac{1}{3} & 0           & 0\\
              \frac{1}{4} & 0           & 0           & \frac{1}{2} & 0\\
              \frac{1}{3} & 0           & 0           & \frac{1}{3} & 0\\
              0           & \frac{2}{5} & \frac{1}{5} & 0           & \frac{1}{5}\\
              0           & 0           & 0           & \frac{1}{2} & 0 
            \end{pmatrix}
          \end{align*}
          Luego el radio espectral de $T_J$ es $\frac{1040}{1453}\approx 0.72 < 1$.
          \item SOR.\\
          La matriz $T_{SOR}$ que depende de $w$ es:
          \begin{align*}
            \begin{pmatrix}
              1-w & \frac{w}{3} & \frac{w}{3} & 0 & 0\\
              -\frac{w\,\left(3\,w-3\right)}{12} & \frac{w^2}{12}-w+1 & \frac{w^2}{12} & \frac{w}{2} & 0\\
              -\frac{w\,\left(3\,w-3\right)}{9} & \frac{w^2}{9} & \frac{w^2}{9}-w+1 & \frac{w}{3} & 0\\
              -\frac{w^2\,\left(3\,w-3\right)}{18} & \frac{w^3}{18}-\frac{w\,\left(4\,w-4\right)}{10} & \frac{w^3}{18}-\frac{w\,\left(3\,w-3\right)}{15} & \frac{4\,w^2}{15}-w+1 & \frac{w}{5}\\
              -\frac{w^3\,\left(3\,w-3\right)}{36} & \frac{w^4}{36}-\frac{w^2\,\left(4\,w-4\right)}{20} & \frac{w^4}{36}-\frac{w^2\,\left(3\,w-3\right)}{30} & \frac{2\,w^3}{15}-\frac{w\,\left(5\,w-5\right)}{10} & \frac{w^2}{10}-w+1 
            \end{pmatrix}
          \end{align*}
          así, usemos el $w^*$ encontrado en el siguiente item para cálcular el radio espectral, por lo que $T_{SOR}(w^*)$ sería:
          \begin{align*}
            \begin{pmatrix}
              -\frac{17}{100} & \frac{39}{100} & \frac{39}{100} & 0 & 0\\
              -\frac{1989}{40000} & -\frac{2237}{40000} & \frac{4563}{40000} & \frac{117}{200} & 0\\
              -\frac{663}{10000} & \frac{1521}{10000} & -\frac{179}{10000} & \frac{39}{100} & 0\\
              -\frac{698697453389513}{18014398509481984} & \frac{678674449446225}{72057594037927936} & \frac{7090251080549995}{144115188075855872} & \frac{1219}{6250} & \frac{117}{500}\\
              -\frac{3269904081862921}{144115188075855872} & \frac{794049105852083}{144115188075855872} & \frac{8295593764243493}{288230376151711744} & \frac{4110828093788559}{36028797018963968} & -\frac{3311}{100000} 
            \end{pmatrix}
          \end{align*}
          Luego el radio espectral de $T_{SOR}(w^*)$ es $\frac{918}{3797}\approx 0.24 < 1$. 
        \end{enumerate}
      \end{solucion}
    \item Aproxime con dos cifras decimales el parámetro de sobrerelajación $w^*$.
      \begin{solucion}
        Utilizemos la fórmula:
        \begin{align*}
          w^*&=\frac{2}{1+\sqrt{1-\rho(J)^2}}\\
            &\approx 1.17
        \end{align*}
      \end{solucion}
    \item ¿Qué reducción en el costo operacional ofrece el método de sobrerelajación con el parámetro $w^*$, en comparación con el método de Gauss-Seidel?
      \begin{solucion}
        Note que como $w^*\approx 1,17$, o sea $1<w^*<2$, es decir el método sería de sobrerelajación que debería de acelerar la convergencia del método original (Gauss-Seidel).  
      \end{solucion}
    \item ¿Cuántas iteraciones más requiere el método de Gauss-Seidel para lograr una precisión mejorada en una cifra decimal? ¿Cuántas necesita el método de sobrerelajación con $w^*$?
      \begin{solucion}
        \begin{enumerate}
          \item Gauss-Seidel.\\
            Aquí para llegar al $x$ esperado realizó $9$ iteraciones, en donde
            \begin{align*}
              x=\begin{pmatrix}
                \frac{4306}{1367}\\  
                \frac{615}{617}\\
                \frac{4142}{635}\\
                \frac{2205}{164}\\
                \frac{3845}{328}     
              \end{pmatrix}
            \end{align*}
          \item SOR.\\
            Aquí para llegar al $x$ esperado realizó $7$ iteraciones, en donde
            \begin{align*}
              \begin{pmatrix}
                \frac{1867}{585}\\
                \frac{509}{493}\\
                \frac{1219}{186}\\
                \frac{5173}{384}\\
                \frac{2711}{231}     
              \end{pmatrix}
            \end{align*}
        \end{enumerate}
        Con lo que podemos concluir que Gauss-Seidel realizó $2$ iteraciones más que $SOR$ y el método de sobrerelajación con $w^*$ necesitó $7$ iteraciones. 
      \end{solucion}
      Para realizar el ejercicio se usó el siguiente script de matlab
      \begin{lstlisting}
% Definir la matriz
A = [ 3  -1  -1   0   0;
     -1   4   0  -2   0;
     -1   0   3  -1   0;
      0  -2  -1   5  -1;
      0   0   0  -1   2];
b = [ 2;
    -26;
      3;
     47;
     10;
    ];
D = [ 3   0   0   0   0;
      0   4   0   0   0;
      0   0   3   0   0;
      0   0   0   5   0;
      0   0   0   0   2];
L = [ 0   0   0   0   0;
      1   0   0   0   0;
      1   0   0   0   0;
      0   2   1   0   0;
      0   0   0   1   0];
U = [ 0   1   1   0   0;
      0   0   0   2   0;
      0   0   0   1   0;
      0   0   0   0   1;
      0   0   0   0   0];
% Jacobi y Gauss-Seidel

TGS = inv(D-L)*U;
TJ = inv(D)*(U+L);

latex_codigo = latex(sym(TGS));
latex_codigo = latex(sym(TJ));

% Obtener valores propios
valores_propiosGS = eig(TGS);
valores_propiosJ = eig(TJ);

% Calcular el radio espectral
radio_espectralGS = max(abs(valores_propiosGS));
radio_espectralJ = max(abs(valores_propiosJ));

% Mostrar resultado
disp('Radio espectral en Gauss-Seidel:')
disp(radio_espectralGS)
disp('Radio espectral en Jacobi:')
disp(radio_espectralJ)

% SOR
w = 2 / (1+sqrt(1-(radio_espectralJ^2)));
woptimo = 1.17;
TSOR = inv(D-woptimo*L)*((1-woptimo)*D+woptimo*U);
latex_codigo = latex(sym(TSOR));
valores_propiosSOR = eig(TSOR);
radio_espectralSOR = max(abs(valores_propiosSOR));
disp('Radio espectral en SOR:')
disp(radio_espectralSOR)

% Encontremos el valor de x en ambos métodos, Gauss-Seidel y SOR con woptimo

cGS = inv(D-L)*b;
cSOR = woptimo*inv(D-woptimo*L)*b

% Gauss-Seidel

x = zeros(5,1); % x0 inicial
x_anterior = x+1; % Para comparar iteraciones
tolerancia = 0.05; % Para una cifra decimal
iteraciones = 0;

% Iterar hasta que cada componente de x tenga precisión de una cifra decimal
while any(abs(x - x_anterior) > tolerancia)
    x_anterior = x; % Guardar el x anterior
    x = TGS * x + cGS; % Nueva iteración
    iteraciones = iteraciones + 1;
end

iteraciones
x

% SOR

x = zeros(5,1); % x0 inicial
x_anterior = x+1; % Para comparar iteraciones
tolerancia = 0.05; % Para una cifra decimal
iteraciones = 0;

% Iterar hasta que cada componente de x tenga precisión de una cifra decimal
while any(abs(x - x_anterior) > tolerancia)
    x_anterior = x; % Guardar el x anterior
    x = TSOR * x + cSOR; % Nueva iteración
    iteraciones = iteraciones + 1;
end

iteraciones
x  
      \end{lstlisting}
  \end{enumerate}
\end{homeworkProblem}
