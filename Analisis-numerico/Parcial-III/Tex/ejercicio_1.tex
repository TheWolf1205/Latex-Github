\begin{homeworkProblem}
  Dada la función
  \begin{align*}
    f(x)=x-\frac{1}{x}-2,\quad f:\mathbb{R}^{+}\to \mathbb{R},
  \end{align*}
  se construye el siguiente algoritmo para aproximar la raíz $r=1$:
  \begin{align*}
    x_{n+1}=2-\frac{1}{x_n}.
  \end{align*}
  \begin{enumerate}
    \item Verificar que si $x_0>1$ entonces la sucesión $\{x_{n}\}$ es monótona decreciente y acotada inferiormente por $1$. Concluir que $x_n\to 1$ aunque esta iteración no esta en las hipótesis del teorema del punto fijo. ¿Qué hipótesis no se cumple?
      \begin{solucion}
        Dada la sucesión:
        \begin{align*}
          x_{n+1} = 2 - \frac{1}{x_n},
        \end{align*}
        queremos demostrar que si $x_0 > 1$, entonces $\{x_n\}$ es monótona decreciente y acotada inferiormente por $1$.\\
        Demostramos por inducción que $x_n \geq 1$ para todo $n$:
        \begin{itemize}
          \item El caso base se tiene por hipótesis, ya que $x_0>1$.
          \item Ahora suponga que se tiene para $n$, es decir.\\
            Supongamos que $x_n \geq 1$. Veamos si $x_{n+1} \geq 1$:
            \begin{align*}
              x_{n+1} = 2 - \frac{1}{x_n}.
            \end{align*}
            Como $x_n \geq 1$, se tiene $\frac{1}{x_n} \leq 1$, lo que implica que:
            \begin{align*}
              x_{n+1} = 2 - \frac{1}{x_n} \geq 2 - 1 = 1.
            \end{align*}
            Por lo tanto, la sucesión es acotada inferiormente por $1$.
        \end{itemize}     
        Ahora afirmamos que $x_{n+1} \leq x_n$, ya que:
        \begin{align*}
            2 \leq x_n + \frac{1}{x_n}.
        \end{align*}
        Para todo $x_n>0$, en particular para todo $x_n>1$ (los nuestros) ahora si reordenamos la desigualdad:
        \begin{align*}
          x_{n+1}=2-\frac{1}{x_n}\leq x_n
        \end{align*}
        Por lo tanto, la sucesión es monótona decreciente.\\
        Ahora, para ver la convergencia note que $\{x_n\}$ es monótona decreciente y acotada inferiormente por $1$, por el teorema de monotonía, converge a un límite $L$. Tomando el límite en la ecuación de recurrencia:
        \begin{align*}
          L = 2 - \frac{1}{L}.
        \end{align*}
        Multiplicamos por $L$:
        \begin{align*}
          L^2 - 2L + 1 &= 0,\\
          (L - 1)^2 &= 0 \quad\Rightarrow L = 1.
        \end{align*}
        Por lo tanto, $x_n \to 1$.\\
        Ahora el teorema del punto fijo requiere que la función de iteración sea contractiva en un intervalo alrededor del punto fijo. La derivada de la función de iteración:
        \begin{align*}
          \varphi(x) = 2 - \frac{1}{x}
        \end{align*}
        es
        \begin{align*}
            \varphi'(x) = \frac{1}{x^2}.
        \end{align*}
        Evaluando en $x = 1$:
        \begin{align*}
          \varphi'(1) = 1.
        \end{align*}
        Como $|\varphi'(1)| = 1$, la condición de contractividad ($|\varphi'(x)| < 1 $ en un entorno del punto fijo) no se cumple, por lo que no se puede garantizar la convergencia por el teorema del punto fijo.
      \end{solucion}
    \item Dar un algoritmo para aproximar la raíz de $f$ que converja cuadráticamente.
      \begin{solucion}
        Ahora veamos un algoritmo de convergencia cuadrática que nos permita aproximar la raíz.\\
        Para obtener un algoritmo que converja cuadráticamente a la raíz $x = 1$, podemos usar el \textbf{método de Newton}:
        \begin{align*}
            x_{n+1} = x_n - \frac{f(x_n)}{f'(x_n)}.
        \end{align*}
        Calculamos $f(x) = x + \frac{1}{x} - 2$ y su derivada:
        \begin{align*}
          f'(x) = 1 - \frac{1}{x^2}.
        \end{align*}
        Aplicamos la fórmula de Newton:
        \begin{align*}
          x_{n+1} = x_n - \frac{x_n + \frac{1}{x_n} - 2}{1 - \frac{1}{x_n^2}}.
        \end{align*}
        Este método tiene \textbf{convergencia cuadrática}, lo que significa que el error disminuye aproximadamente como el cuadrado del error anterior en cada iteración. 
      \end{solucion}
  \end{enumerate}
\end{homeworkProblem}
