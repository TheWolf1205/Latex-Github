%!TEX program = lualatex
\PassOptionsToPackage{usenames, dvipsnames}{color}
\documentclass{aleph-revista}
\usepackage{fontspec}
\setmainfont{Times New Roman}
\usepackage{unicode-math}
\setmathfont{TeX Gyre Termes Math}
\usepackage{tikz}
\usepackage{graphicx}
\usepackage{aleph-comandos} 
\usepackage{multicol}    
\usepackage[usenames]{color}
\usepackage{parskip}
\usepackage{hyperref}
\usepackage[spanish]{babel}
\usepackage{amsmath, amssymb, amsthm, mathrsfs}
\usepackage{enumitem}
\newtheorem{proposition}{Proposición}
\newtheorem{lemma}{Lema}

\hypersetup{
    colorlinks=true,
    linkcolor=blue,
    filecolor=blue,      
    urlcolor=cyan,
    pdftitle={Sharelatex Example},
    pdfpagemode=FullScreen,
    }
\addbibresource{bibliografia.bib}

% Fourier
% hat y widecheck
\DeclareFontFamily{U}{mathx}{}
\DeclareFontShape{U}{mathx}{m}{n}{<-> mathx10}{}
\DeclareSymbolFont{mathx}{U}{mathx}{m}{n}
\providecommand{\check}{\widecheck}
\renewcommand{\hat}{\widehat}
\providecommand{\norm}[1]{\left\|#1\right\|}


\newcommand{\ident}{\hspace{0.5cm}}
\renewcommand\qedsymbol{$\blacksquare$}
\fechapubli{2025}
\periodouno{Julio}

\titulo{Estimación tipo conmutador para transformadas de Hilbert y derivadas fraccionarias.}

\tituloingles{Models of the universe from differential geometry.}

\autor{Andrés David Cadena Simons}
\institucion{Universidad Nacional de Colombia, Facultad de ciencias, Sede Bogotá}
\correo{acadenas@unal.edu.co}

\fecha{\today}
\resumen{
XXXXX
}
\palabrasc{XXX}
\begin{document}
\membrete
%%%%%%%%%%%%%%%%%%%%%%%%%%%%%%%%%%%%%%%%%%%%%%%%%%%
\section{Introducción}
El objetivo de este trabajo es demostrar en detalle la siguiente estimación tipo conmutador:
\begin{proposition}[Estimación tipo conmutador no local]
Sea $1 < p < \infty$, $0 < \alpha, \beta \leq 1$, tal que $\alpha + \beta = 1$. Entonces
\begin{align*}
  \norm{D_x^{\alpha} [H_x, g] D_x^{\beta} f}_{L^p(\mathbb{R})} \lesssim_{p,\alpha,\beta} \norm{\partial_x g}_{L^\infty(\mathbb{R})}\norm{f}_{L^p(\mathbb{R})}, 
\end{align*}
para toda función $g$ suave con derivada acotada y toda $f \in \mathcal{S}(\mathbb{R})$.
\end{proposition}
%%%%%%%%%%%%%%%%%%%%%%%%%%%%%%%%%%%%%%%%%%%%%%%%%%%
\section{Preliminares}
Recordamos que la transformada de Hilbert $H_x$ está definida en la transformada de Fourier como
\begin{align*}
  \hat{H_x f}(\xi) = -isign(\xi) \hat{f}(\xi). 
\end{align*}
Además, las derivadas fraccionarias están dadas por el operador
\begin{align*}
  \hat{D_x^s f}(\xi) = |\xi|^s \hat{f}(\xi), \qquad s \in \mathbb{R}. 
\end{align*}
También usaremos las proyecciones de Littlewood–Paley $P^x_N$ definidas por
\begin{align*}
  \hat{P^x_N f}(\xi) = \psi_N(\xi) \hat{f}(\xi), 
\end{align*}
donde $\psi_N(\xi)$ es un multiplicador suave con soporte en frecuencias de orden $|\xi| \sim N$, con $N$ número diádico.\\
Utilizaremos además las siguientes herramientas fundamentales
\begin{lemma}[Fefferman–Stein]
  Sea $f = (f_j)_{j=1}^{\infty}$ una secuencia de funciones localmente integrables en $\mathbb{R}$. Si $1 < p < \infty$, entonces
  \begin{align*}
    \norm{(Mf_j)_{l^2}}_{L^p} \leq C_p \norm{(f_j)_{l^2}}_{L^p}, 
  \end{align*}
  donde $M$ denota la función maximal de Hardy–Littlewood.
\end{lemma}
\begin{lemma}[Estimación tipo Calderón]
  Para $l + m \geq 1$, se tiene
  \begin{align*}
    \norm{\partial_x^l [H_x, g] \partial_x^m f}_{L^p} \lesssim \norm{\partial_x^{l + m} g}_{L^\infty} \norm{f}_{L^p}. 
  \end{align*}
\end{lemma}
%%%%%%%%%%%%%%%%%%%%%%%%%%%%%%%%%%%%%%%%%%%%%%%%%%
\section{Demostración de la Proposición}

Comenzamos suponiendo el caso no trivial donde $0 < \alpha, \beta < 1$ y $\alpha + \beta = 1$. El caso $\beta = 1$, $\alpha = 0$ se sigue directamente de la estimación clásica de Calderón.

Queremos estimar
\[
D_x^\alpha [H_x, g] D_x^\beta f.
\]
En el dominio de Fourier, tenemos
\[
\mathcal{F} \left( D_x^\alpha [H_x, g] D_x^\beta f \right)(\xi) = -i \int_{\mathbb{R}} |\xi|^\alpha \left[ \mathrm{sign}(\xi) - \mathrm{sign}(\xi - \xi') \right] \hat{g}(\xi - \xi') |\xi'|^\beta \hat{f}(\xi') \, d\xi'.
\]
Observamos que el integrando solo es distinto de cero si los signos de $\xi$ y $\xi'$ son distintos, lo que equivale a que $|\xi'| < |\xi - \xi'|$.
Aplicando la descomposici\'on tipo paraproducto y la propiedad del soporte en frecuencia, podemos escribir:
\[
D_x^\alpha [H_x, g] D_x^\beta f = A_1 + A_2 + A_3 + A_4,
\]
donde
\begin{align*}
A_1 &= H_x \left( \sum_{N>0} D_x^\alpha (P_N^x g \cdot P_{\ll N}^x D_x^\beta f) \right), \\
A_2 &= - \sum_{N>0} D_x^\alpha (P_N^x g \cdot P_{\ll N}^x H_x D_x^\beta f), \\
A_3 &= H_x \left( \sum_{N>0} D_x^\alpha (P_N^x g \cdot \widetilde{P}_N^x D_x^\beta f) \right), \\
A_4 &= - \sum_{N>0} D_x^\alpha (P_N^x g \cdot \widetilde{P}_N^x H_x D_x^\beta f).
\end{align*}

\subsection*{Estimaci\'on del t\'ermino $A_1$}

Dado que $\alpha + \beta = 1$ y que $H_x$ es acotado en $L^p$, aplicamos la desigualdad de Littlewood--Paley:
\[
\|A_1\|_{L^p} \lesssim \left\| \left( \sum_{M} |P_M^x \sum_{N > 0} D_x^\alpha (P_N^x g \cdot P_{\ll N}^x D_x^\beta f)|^2 \right)^{1/2} \right\|_{L^p}.
\]

El soporte en frecuencia nos permite restringir la suma a los \( N \sim M \), as\'i que:
\[
\|A_1\|_{L^p} \lesssim \left\| \left( \sum_{N} |D_x^\alpha (P_N^x g \cdot P_{\ll N}^x D_x^\beta f)|^2 \right)^{1/2} \right\|_{L^p}.
\]

Usamos que
\[
D_x^\alpha (P_N^x g \cdot P_{\ll N}^x D_x^\beta f) \sim P_N^x (\partial_x g_N^{-\beta} \cdot P_{\ll N}^x D_x^\beta f),
\]
donde \( g_N^{-\beta} \) representa una reescalada del tipo \( N^{-\beta} \partial_x g \), y por el Lema 7.1 (estimaci\'on tipo Calder\'on-Coifman-Meyer):
\[
|P_N^x (\partial_x g_N^{-\beta} \cdot P_{\ll N}^x D_x^\beta f)(x)| \lesssim M(\partial_x g)(x) \cdot N^{-\beta} M(P_{\ll N}^x D_x^\beta f)(x).
\]

Aplicando la desigualdad de Fefferman--Stein sobre la suma en \( N \):
\[
\|A_1\|_{L^p} \lesssim \| \partial_x g \|_{L^\infty} \cdot \left\| \left( \sum_{N} |M(P_{\ll N}^x D_x^\beta f)|^2 \right)^{1/2} \right\|_{L^p}.
\]

Finalmente, aplicamos la desigualdad de Littlewood--Paley y acotamos el maximal por:
\[
\|A_1\|_{L^p} \lesssim \| \partial_x g \|_{L^\infty} \cdot \|f\|_{L^p}.
\]

Esto completa la estimaci\'on del t\'ermino $A_1$.
\subsection*{Estimaci\'on del t\'ermino $A_2$}

La estimaci\'on de $A_2$ sigue exactamente los mismos pasos que la de $A_1$, ya que $H_x$ es un operador lineal acotado en $L^p$, y aparece aplicado sobre $f$ antes del producto. Observamos que:
\[
A_2 = - \sum_{N > 0} D_x^\alpha (P_N^x g \cdot P_{\ll N}^x H_x D_x^\beta f).
\]

Como $H_x$ conmuta con las proyecciones y es acotado, podemos reemplazar $f$ por $H_x f$ en la estimaci\'on anterior. Por tanto:
\[
\|A_2\|_{L^p} \lesssim \| \partial_x g \|_{L^\infty} \cdot \| H_x f \|_{L^p} \lesssim \| \partial_x g \|_{L^\infty} \cdot \|f\|_{L^p}.
\]

Esto concluye la estimaci\'on del t\'ermino $A_2$.
\subsection*{Estimaci\'on del t\'ermino $A_3$}

Ahora consideramos el t\'ermino
\[
A_3 = H_x \left( \sum_{N>0} D_x^\alpha (P_N^x g \cdot \widetilde{P}_N^x D_x^\beta f) \right).
\]

Recordamos que \( \widetilde{P}_N^x \) es una proyecci\'on que selecciona las frecuencias del mismo orden que $N$, por lo que $P_N^x g$ y $\widetilde{P}_N^x f$ est\'an oscilando en frecuencias comparables. En este caso se trata del llamado "interacci\'on de alta con alta frecuencia".

Aplicamos nuevamente la desigualdad de Littlewood--Paley para estimar
\[
\|A_3\|_{L^p} \lesssim \left\| \left( \sum_{M} |P_M^x \sum_{N > 0} D_x^\alpha (P_N^x g \cdot \widetilde{P}_N^x D_x^\beta f)|^2 \right)^{1/2} \right\|_{L^p}.
\]

Dado que el soporte en frecuencia de $P_M^x$ solo interact\'ua con frecuencias $N \sim M$, reducimos a
\[
\|A_3\|_{L^p} \lesssim \left\| \left( \sum_{N} |D_x^\alpha (P_N^x g \cdot \widetilde{P}_N^x D_x^\beta f)|^2 \right)^{1/2} \right\|_{L^p}.
\]

Ahora usamos la regla del producto para derivadas fraccionarias (o simb\'olicamente en Fourier), teniendo en cuenta que $D_x^\alpha (f g)$ se comporta como la suma de productos $D_x^\alpha f \cdot g$ y $f \cdot D_x^\alpha g$, pero dado que ambos factores est\'an en la misma escala de frecuencia, podemos estimar como:
\[
|D_x^\alpha (P_N^x g \cdot \widetilde{P}_N^x D_x^\beta f)(x)| \lesssim N^\alpha |P_N^x g(x)| \cdot |\widetilde{P}_N^x D_x^\beta f(x)|.
\]

Recordamos que $\beta = 1 - \alpha$, por lo que $D_x^\beta f \sim N^\beta P_N^x f$, y por tanto
\[
|\widetilde{P}_N^x D_x^\beta f(x)| \lesssim N^\beta |P_N^x f(x)|.
\]

Entonces
\[
|D_x^\alpha (P_N^x g \cdot \widetilde{P}_N^x D_x^\beta f)(x)| \lesssim N^\alpha |P_N^x g(x)| \cdot N^\beta |P_N^x f(x)| = N |P_N^x g(x)| \cdot |P_N^x f(x)|.
\]

Pero como $\partial_x g \in L^\infty$, tenemos $|P_N^x g(x)| \lesssim N^{-1} M(\partial_x g)(x)$. Entonces
\[
|D_x^\alpha (P_N^x g \cdot \widetilde{P}_N^x D_x^\beta f)(x)| \lesssim M(\partial_x g)(x) \cdot |P_N^x f(x)|.
\]

Por tanto,
\[
\left( \sum_{N} |D_x^\alpha (P_N^x g \cdot \widetilde{P}_N^x D_x^\beta f)|^2 \right)^{1/2} \lesssim M(\partial_x g)(x) \cdot \left( \sum_N |P_N^x f(x)|^2 \right)^{1/2}.
\]

Finalmente, aplicamos Fefferman--Stein y Littlewood--Paley:
\[
\|A_3\|_{L^p} \lesssim \|M(\partial_x g)\|_{L^\infty} \cdot \|f\|_{L^p} \lesssim \|\partial_x g\|_{L^\infty} \cdot \|f\|_{L^p}.
\]

Esto concluye la estimaci\'on del t\'ermino $A_3$.
\subsection*{Estimaci\'on del t\'ermino $A_4$}

Finalmente, consideramos el t\'ermino
\[
A_4 = - \sum_{N>0} D_x^\alpha (P_N^x g \cdot \widetilde{P}_N^x H_x D_x^\beta f).
\]

Como en el caso de $A_3$, se trata de un producto de funciones en frecuencias comparables. Usamos que $H_x$ es un operador acotado en $L^p$ y conmuta con derivadas fraccionarias, por lo que:
\[
|\widetilde{P}_N^x H_x D_x^\beta f(x)| \lesssim N^\beta |P_N^x f(x)|.
\]

Por tanto, repitiendo los mismos argumentos que en la estimaci\'on de $A_3$, tenemos:
\begin{align*}
|D_x^\alpha (P_N^x g \cdot \widetilde{P}_N^x H_x D_x^\beta f)(x)| &\lesssim N^\alpha |P_N^x g(x)| \cdot |\widetilde{P}_N^x H_x D_x^\beta f(x)| \\
&\lesssim N^\alpha |P_N^x g(x)| \cdot N^\beta |P_N^x f(x)| \\
&= N |P_N^x g(x)| \cdot |P_N^x f(x)|.
\end{align*}

Usando que $|P_N^x g(x)| \lesssim N^{-1} M(\partial_x g)(x)$, obtenemos:
\[
|D_x^\alpha (P_N^x g \cdot \widetilde{P}_N^x H_x D_x^\beta f)(x)| \lesssim M(\partial_x g)(x) \cdot |P_N^x f(x)|.
\]

Por tanto,
\[
\left( \sum_{N} |D_x^\alpha (P_N^x g \cdot \widetilde{P}_N^x H_x D_x^\beta f)|^2 \right)^{1/2} \lesssim M(\partial_x g)(x) \cdot \left( \sum_N |P_N^x f(x)|^2 \right)^{1/2}.
\]

Aplicando nuevamente Fefferman--Stein y Littlewood--Paley, concluimos:
\[
\|A_4\|_{L^p} \lesssim \|M(\partial_x g)\|_{L^\infty} \cdot \|f\|_{L^p} \lesssim \|\partial_x g\|_{L^\infty} \cdot \|f\|_{L^p}.
\]

Esto concluye la estimaci\'on del t\'ermino $A_4$ y por tanto, la demostraci\'on completa de la Proposici\'on 1.1.
%%%%%%%%%%%%%%%%%%%%%%%%%%%%%%%%%%%%%%%%%%%%%%%%%%
\section{Implicaciones}
%%%%%%%%%%%%%%%%%%%%%%%%%%%%%%%%%%%%%%%%%%%%%%%
\section{Conclusiones}
%%%%%%%%%%%%%%%%%%%%%%%%%%%%%%%%%%%%%%%%%%%%%%%
\newpage
\nocite{*}
\printbibliography

\end{document}
