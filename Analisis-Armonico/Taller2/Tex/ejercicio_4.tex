\begin{homeworkProblem}
  \textbf{Topología sobre $\mathcal{S}(\mathbb{R}^{n})$} Definimos la aplicación
  \begin{align*}
    d : \mathcal{S}(\mathbb{R}^{n}) \times \mathcal{S}(\mathbb{R}^{n})) &\longrightarrow \mathbb{R}+\\
    (\phi, \psi) &\longmapsto \sum_{\alpha, \beta \in \mathbb{N}^{n}} 2^{-(|\alpha| + |\beta|)} \frac{\|\phi - \psi\|_{\alpha, \beta}}{1 + \|\phi - \psi\|_{\alpha, \beta}}
  \end{align*}
  \begin{enumerate}[(i)]
    \item Pruebe que $(\mathcal{S}(\mathbb{R}^{n}); d)$ es un espacio métrico completo.
    \item Pruebe que para cualquier sucesión $(\phi_k)_k \subset \mathcal{S}(\mathbb{R}^{n})$ y $\phi \in \mathcal{S}(\mathbb{R}^{n})$, vale
      \begin{align*}
        \phi_k \overset{d}{\to} \phi \text{ si y solo si } \|\phi_k - \phi\|_{\alpha, \beta} \to 0, \forall \alpha, \beta \in \mathbb{N}^{n}
      \end{align*}
    \item Sea $f \in C^\infty(\mathbb{R}^{n})$. Pruebe que
      \begin{align*}
        f \in \mathcal{S}(\mathbb{R}^{n}) \text{ si y solo si } x^{\alpha} \partial^{\beta}f \in L^2(\mathbb{R}^{n}), \forall \alpha, \beta \in \mathbb{N}^{n}.
      \end{align*}
    \item Muestre que
      \begin{align*}
        \mathcal{F} : \mathcal{S}(\mathbb{R}^{n}) &\longrightarrow \mathcal{S}(\mathbb{R}^{n})\\
        \phi &\longmapsto \hat{\phi}
      \end{align*}
      es un isomorfismo topológico.
  \end{enumerate}
  \begin{solution}
    (I)\\
    Primero veamos que $d$ está bien definida, ya que
    \begin{align*}
			d(\phi,\psi)&\leq\sum_{\alpha,\beta\in \mathbb{N}^{n}}\frac{1}{2^{|\alpha|+|\beta|}}\frac{|\|\phi-\psi\||_{(\alpha,\beta)}}{1+|\|\phi-\psi\||_{(\alpha,\beta)}},\\
			&\leq \sum_{\alpha,\beta\in\mathbb{N}^{n}}\frac{1}{2^{|\alpha|+|\beta|}},\\
			&\leq \sum_{\alpha,\beta\in\mathbb{N}^{n}}\frac{1}{2^{|\alpha|}}\frac{1}{2^{|\beta|}},\\
			&\leq \sum_{\alpha\in\mathbb{N}^{n}}\frac{1}{2^{|\alpha|}} \sum_{\beta\in\mathbb{N}^{n}}\frac{1}{2^{|\beta|}},\\
			&\leq \left(\sum_{\alpha\in\mathbb{N}^{n}}\frac{1}{2^{|\alpha|}}\right)^2,\\
			&\leq\left( \sum_{\alpha_1=0}^{\infty}\sum_{\alpha_2=0}^{\infty}\cdots\sum_{\alpha_n=0}^{\infty}\frac{1}{2^{\alpha_1+\alpha_2+\cdots+\alpha_n}}\right)^2,\\
			&\leq\left(\sum_{k=0}^{\infty}\frac{1}{2^k}\right)^{2n},\\
			&<\infty.
    \end{align*}
    Ahora veamos que $d(\phi,\psi)=0$ si y sólo si $\phi=\psi$.\\
		Note que, en la definición de $d$, todos los sumandos son reales positivos, nosotros afirmamos que $d(\phi,\psi)=0$ si y sólo si $|\|\phi-\psi\||_{(\alpha,\beta)}=0$ para todo $(\alpha,\beta)\in\mathbb{N}^{2n}$.\\
		Ahora, suponga que $\alpha=0$ y $\beta=0$, entonces:
		\begin{align*}
			|\|\phi-\psi\||_{(0,0)}&=\|x^{0}\partial^{0}(\phi-\psi)\|_{\infty},\\
			&=\|\phi-\psi\|_{\infty},\\
			&=\sup_{x\in\mathbb{R}^n}|\phi-\psi|=0.
		\end{align*}
		Entonces, $\phi(x)=\psi(x)$ para todo $x\in\mathbb{R}^n$.\\
    Por otro lado, note que $d(\phi,\psi)=d(\psi,\phi)$ es inmediato, ya que $|\|\phi-\psi||_{(\alpha,\beta)}=|\|\psi-\phi\||_{(\alpha,\beta)}$ para todo $\alpha$ y $\beta$.\\
    Ahora veamos que $d$ satisface la desigualdad triangular.\\
		Usando las desigualdades triangulares de las seminormas $|\|\cdot\||_{(\alpha,\beta)}$, nosotros tenemos que
		\begin{align*}
			|\|\phi-\psi\||_{(\alpha,\beta)}&\leq |\|\phi-\varphi\||_{(\alpha,\beta)}+|\|\varphi-\psi\||_{(\alpha,\beta)},\\
			1+|\|\phi-\psi\||_{(\alpha,\beta)}&\leq 1+ |\|\phi-\varphi\||_{(\alpha,\beta)}+|\|\varphi-\psi\||_{(\alpha,\beta)},\\
			\frac{1}{1+ |\|\phi-\varphi\||_{(\alpha,\beta)}+|\|\varphi-\psi\||_{(\alpha,\beta)}}&\leq \frac{1}{1+|\|\phi-\psi\||_{(\alpha,\beta)}},\\
			-\frac{1}{1+|\|\phi-\psi\||_{(\alpha,\beta)}}&\leq -\frac{1}{1+ |\|\phi-\varphi\||_{(\alpha,\beta)}+|\|\varphi-\psi\||_{(\alpha,\beta)}}.
		\end{align*}
		Entonces:
		\begin{align*}
		  d(\phi,\psi)&\leq \sum_{\alpha,\beta}\frac{1}{2^{|\alpha|+|\beta|}}\frac{|\|\phi-\psi\||_{(\alpha,\beta)}}{1+|\|\phi-\psi\||_{(\alpha,\beta)}},\\
			&\leq \sum_{\alpha,\beta}\frac{1}{2^{|\alpha|+|\beta|}}\left(1-\frac{1}{1+|\|\phi-\psi\||_{(\alpha,\beta)}}\right),\\
			&\leq \sum_{\alpha,\beta}\frac{1}{2^{|\alpha|+|\beta|}}\left(1-\frac{1}{1+|\|\phi-\varphi\||_{(\alpha,\beta)}+|\|\varphi-\psi\||_{(\alpha,\beta)}}\right),\\
			&\leq\sum_{\alpha,\beta}\frac{1}{2^{|\alpha|+|\beta|}}\frac{|\|\phi-\varphi\||_{(\alpha,\beta)}+|\|\varphi-\psi\||_{(\alpha,\beta)}}{1+|\|\phi-\varphi\||_{(\alpha,\beta)}+|\|\varphi-\psi\||_{(\alpha,\beta)}},\\
			&\leq\sum_{\alpha,\beta}\frac{1}{2^{|\alpha|+|\beta|}}\frac{|\|\phi-\varphi\||_{(\alpha,\beta)}}{1+|\|\phi-\varphi\||_{(\alpha,\beta)}+|\|\varphi-\psi\||_{(\alpha,\beta)}}\\
      &\phantom{\leq}+\sum_{\alpha,\beta}\frac{1}{2^{|\alpha|+|\beta|}}\frac{|\|\varphi-\psi\||_{(\alpha,\beta)}}{1+|\|\phi-\varphi\||_{(\alpha,\beta)}+|\|\varphi-\psi\||_{(\alpha,\beta)}},\\
			&\leq\sum_{\alpha,\beta}\frac{1}{2^{|\alpha|+|\beta|}}\frac{|\|\phi-\varphi\||_{(\alpha,\beta)}}{1+|\|\phi-\varphi\||_{(\alpha,\beta)}}+\sum_{\alpha,\beta}\frac{1}{2^{|\alpha|+|\beta|}}\frac{|\|\varphi-\psi\||_{(\alpha,\beta)}}{1+|\|\varphi-\psi\||_{(\alpha,\beta)}},\\
			&\leq d(\phi,\varphi)+d(\varphi,\psi).
		\end{align*}
		Luego, podemos concluir que $d$ es una métrica para el espacio $\mathcal{S}(\mathbb{R}^n)$.\\
    Ahora veamos que $(\mathcal{S}(\mathbb{R}^{n}),d)$ es un espacio completo.\\
    Suponga $\{f_{k}\}\subset\mathcal{S}(\mathbb{R}^{n})$ una sucesión de Cauchy, es decir que dado $\epsilon>0$ existe $N>0$ tal que si $k,l>N$, entonces
    \begin{align*}
      d(f_{n},f_{m})<\epsilon.
    \end{align*}

    (II)\\
    Note que si $\norm{\phi_{k}-\phi}\to 0$, para todo $\alpha,\beta\in\mathbb{N}^{n}$, entonces como $d$ está bien definida es válido afirmar que
    \begin{align*}
      \lim_{k \to \infty}d(\phi_{k},\phi)&=\lim_{k \to \infty}\sum_{\alpha,\beta\in\mathbb{N}^{n}}\frac{1}{2^{|\alpha|+|\beta|}}\frac{\norm{\phi-\psi}_{\alpha,\beta}}{1+\norm{\phi-\psi}_{\alpha,\beta}},\\
      &=\sum_{\alpha,\beta\in\mathbb{N}^{n}}\frac{1}{2^{|\alpha|+|\beta|}}\lim_{k \to \infty}\frac{\norm{\phi_{k}-\phi}_{\alpha,\beta}}{1+\norm{\phi_{k}-\phi}_{\alpha,\beta}},\\
      &=\sum_{\alpha,\beta\in\mathbb{N}^{n}}\frac{1}{2^{|\alpha|+|\beta|}}\left( 0 \right),\\
      &=0.
    \end{align*}
    Luego podemos afirmar que $\phi_{k}\overset{d}{\to}\phi$ cuando $k\to\infty$.\\
    Ahora veamos que si $\phi_{k}\overset{d}{\to}\phi$, entonces $\norm{\phi_{k}-\phi}_{\alpha,\beta}\to 0$, para todo $\alpha,\beta\in\mathbb{N}^{n}$.\\
    Razonemos por contradicción, suponga que existe $\alpha_{0},\beta_{0}\in\mathbb{N}^{n}$ tal que $\norm{\phi_{k}-\phi}_{\alpha_{0},\beta_{0}}\cancel{\to} 0$. 
  \end{solution}
\end{homeworkProblem}
