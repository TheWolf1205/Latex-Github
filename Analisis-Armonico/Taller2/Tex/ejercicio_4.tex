\begin{homeworkProblem}
  \textbf{Topología sobre $\mathcal{S}(\mathbb{R}^{n})$} Definimos la aplicación
  \begin{align*}
    d : \mathcal{S}(\mathbb{R}^{n}) \times \mathcal{S}(\mathbb{R}^{n})) &\longrightarrow \mathbb{R}+\\
    (\phi, \psi) &\longmapsto \sum_{\alpha, \beta \in \mathbb{N}^{n}} 2^{-(|\alpha| + |\beta|)} \frac{\|\phi - \psi\|_{\alpha, \beta}}{1 + \|\phi - \psi\|_{\alpha, \beta}}
  \end{align*}
  \begin{enumerate}[(i)]
    \item Pruebe que $(\mathcal{S}(\mathbb{R}^{n}); d)$ es un espacio métrico completo.
    \item Pruebe que para cualquier sucesión $(\phi_k)_k \subset \mathcal{S}(\mathbb{R}^{n})$ y $\phi \in \mathcal{S}(\mathbb{R}^{n})$, vale
      \begin{align*}
        \phi_k \overset{d}{\to} \phi \text{ si y solo si } \|\phi_k - \phi\|_{\alpha, \beta} \to 0, \forall \alpha, \beta \in \mathbb{N}^{n}
      \end{align*}
    \item Sea $f \in C^\infty(\mathbb{R}^{n})$. Pruebe que
      \begin{align*}
        f \in \mathcal{S}(\mathbb{R}^{n}) \text{ si y solo si } x^{\alpha} \partial^{\beta}f \in L^2(\mathbb{R}^{n}), \forall \alpha, \beta \in \mathbb{N}^{n}.
      \end{align*}
    \item Muestre que
      \begin{align*}
        \mathcal{F} : \mathcal{S}(\mathbb{R}^{n}) &\longrightarrow \mathcal{S}(\mathbb{R}^{n})\\
        \phi &\longmapsto \hat{\phi}
      \end{align*}
      es un isomorfismo topológico.
  \end{enumerate}
  \begin{solution}
    (I)\\
    Primero veamos que $d$ está bien definida, ya que
    \begin{align*}
			d(\phi,\psi)&\leq\sum_{\alpha,\beta\in \mathbb{N}^{n}}\frac{1}{2^{|\alpha|+|\beta|}}\frac{|\|\phi-\psi\||_{(\alpha,\beta)}}{1+|\|\phi-\psi\||_{(\alpha,\beta)}},\\
			&\leq \sum_{\alpha,\beta\in\mathbb{N}^{n}}\frac{1}{2^{|\alpha|+|\beta|}},\\
			&\leq \sum_{\alpha,\beta\in\mathbb{N}^{n}}\frac{1}{2^{|\alpha|}}\frac{1}{2^{|\beta|}},\\
			&\leq \sum_{\alpha\in\mathbb{N}^{n}}\frac{1}{2^{|\alpha|}} \sum_{\beta\in\mathbb{N}^{n}}\frac{1}{2^{|\beta|}},\\
			&\leq \left(\sum_{\alpha\in\mathbb{N}^{n}}\frac{1}{2^{|\alpha|}}\right)^2,\\
			&\leq\left( \sum_{\alpha_1=0}^{\infty}\sum_{\alpha_2=0}^{\infty}\cdots\sum_{\alpha_n=0}^{\infty}\frac{1}{2^{\alpha_1+\alpha_2+\cdots+\alpha_n}}\right)^2,\\
			&\leq\left(\sum_{k=0}^{\infty}\frac{1}{2^k}\right)^{2n},\\
			&<\infty.
    \end{align*}
    Ahora veamos que $d(\phi,\psi)=0$ si y sólo si $\phi=\psi$.\\
		Note que, en la definición de $d$, todos los sumandos son reales positivos, nosotros afirmamos que $d(\phi,\psi)=0$ si y sólo si $|\|\phi-\psi\||_{(\alpha,\beta)}=0$ para todo $(\alpha,\beta)\in\mathbb{N}^{2n}$.\\
		Ahora, suponga que $\alpha=0$ y $\beta=0$, entonces:
		\begin{align*}
			|\|\phi-\psi\||_{(0,0)}&=\|x^{0}\partial^{0}(\phi-\psi)\|_{\infty},\\
			&=\|\phi-\psi\|_{\infty},\\
			&=\sup_{x\in\mathbb{R}^n}|\phi-\psi|=0.
		\end{align*}
		Entonces, $\phi(x)=\psi(x)$ para todo $x\in\mathbb{R}^n$.\\
    Por otro lado, note que $d(\phi,\psi)=d(\psi,\phi)$ es inmediato, ya que $|\|\phi-\psi||_{(\alpha,\beta)}=|\|\psi-\phi\||_{(\alpha,\beta)}$ para todo $\alpha$ y $\beta$.\\
    Ahora veamos que $d$ satisface la desigualdad triangular.\\
		Usando las desigualdades triangulares de las seminormas $|\|\cdot\||_{(\alpha,\beta)}$, nosotros tenemos que
		\begin{align*}
			|\|\phi-\psi\||_{(\alpha,\beta)}&\leq |\|\phi-\varphi\||_{(\alpha,\beta)}+|\|\varphi-\psi\||_{(\alpha,\beta)},\\
			1+|\|\phi-\psi\||_{(\alpha,\beta)}&\leq 1+ |\|\phi-\varphi\||_{(\alpha,\beta)}+|\|\varphi-\psi\||_{(\alpha,\beta)},\\
			\frac{1}{1+ |\|\phi-\varphi\||_{(\alpha,\beta)}+|\|\varphi-\psi\||_{(\alpha,\beta)}}&\leq \frac{1}{1+|\|\phi-\psi\||_{(\alpha,\beta)}},\\
			-\frac{1}{1+|\|\phi-\psi\||_{(\alpha,\beta)}}&\leq -\frac{1}{1+ |\|\phi-\varphi\||_{(\alpha,\beta)}+|\|\varphi-\psi\||_{(\alpha,\beta)}}.
		\end{align*}
		Entonces:
		\begin{align*}
		  d(\phi,\psi)&\leq \sum_{\alpha,\beta}\frac{1}{2^{|\alpha|+|\beta|}}\frac{|\|\phi-\psi\||_{(\alpha,\beta)}}{1+|\|\phi-\psi\||_{(\alpha,\beta)}},\\
			&\leq \sum_{\alpha,\beta}\frac{1}{2^{|\alpha|+|\beta|}}\left(1-\frac{1}{1+|\|\phi-\psi\||_{(\alpha,\beta)}}\right),\\
			&\leq \sum_{\alpha,\beta}\frac{1}{2^{|\alpha|+|\beta|}}\left(1-\frac{1}{1+|\|\phi-\varphi\||_{(\alpha,\beta)}+|\|\varphi-\psi\||_{(\alpha,\beta)}}\right),\\
			&\leq\sum_{\alpha,\beta}\frac{1}{2^{|\alpha|+|\beta|}}\frac{|\|\phi-\varphi\||_{(\alpha,\beta)}+|\|\varphi-\psi\||_{(\alpha,\beta)}}{1+|\|\phi-\varphi\||_{(\alpha,\beta)}+|\|\varphi-\psi\||_{(\alpha,\beta)}},\\
			&\leq\sum_{\alpha,\beta}\frac{1}{2^{|\alpha|+|\beta|}}\frac{|\|\phi-\varphi\||_{(\alpha,\beta)}}{1+|\|\phi-\varphi\||_{(\alpha,\beta)}+|\|\varphi-\psi\||_{(\alpha,\beta)}}\\
      &\phantom{\leq}+\sum_{\alpha,\beta}\frac{1}{2^{|\alpha|+|\beta|}}\frac{|\|\varphi-\psi\||_{(\alpha,\beta)}}{1+|\|\phi-\varphi\||_{(\alpha,\beta)}+|\|\varphi-\psi\||_{(\alpha,\beta)}},\\
			&\leq\sum_{\alpha,\beta}\frac{1}{2^{|\alpha|+|\beta|}}\frac{|\|\phi-\varphi\||_{(\alpha,\beta)}}{1+|\|\phi-\varphi\||_{(\alpha,\beta)}}+\sum_{\alpha,\beta}\frac{1}{2^{|\alpha|+|\beta|}}\frac{|\|\varphi-\psi\||_{(\alpha,\beta)}}{1+|\|\varphi-\psi\||_{(\alpha,\beta)}},\\
			&\leq d(\phi,\varphi)+d(\varphi,\psi).
		\end{align*}
		Luego, podemos concluir que $d$ es una métrica para el espacio $\mathcal{S}(\mathbb{R}^n)$.\\
    Ahora veamos que $(\mathcal{S}(\mathbb{R}^{n}),d)$ es un espacio completo.\\
    Suponga $\{f_{k}\}\subset\mathcal{S}(\mathbb{R}^{n})$ una sucesión de Cauchy, es decir que dado $\epsilon>0$ existe $N>0$ tal que si $k,l>N$, entonces
    \begin{align*}
      d(f_{n},f_{m})<\epsilon.
    \end{align*}

    (II)\\
    Note que si $\norm{\phi_{k}-\phi}\to 0$, para todo $\alpha,\beta\in\mathbb{N}^{n}$, entonces como $d$ está bien definida es válido afirmar que
    \begin{align*}
      \lim_{k \to \infty}d(\phi_{k},\phi)&=\lim_{k \to \infty}\sum_{\alpha,\beta\in\mathbb{N}^{n}}\frac{1}{2^{|\alpha|+|\beta|}}\frac{\norm{\phi-\psi}_{\alpha,\beta}}{1+\norm{\phi-\psi}_{\alpha,\beta}},\\
      &=\sum_{\alpha,\beta\in\mathbb{N}^{n}}\frac{1}{2^{|\alpha|+|\beta|}}\lim_{k \to \infty}\frac{\norm{\phi_{k}-\phi}_{\alpha,\beta}}{1+\norm{\phi_{k}-\phi}_{\alpha,\beta}},\\
      &=\sum_{\alpha,\beta\in\mathbb{N}^{n}}\frac{1}{2^{|\alpha|+|\beta|}}\left( 0 \right),\\
      &=0.
    \end{align*}
    Luego podemos afirmar que $\phi_{k}\overset{d}{\to}\phi$ cuando $k\to\infty$.\\
    Ahora veamos que si $\phi_{k}\overset{d}{\to}\phi$, entonces $\norm{\phi_{k}-\phi}_{\alpha,\beta}\to 0$, para todo $\alpha,\beta\in\mathbb{N}^{n}$.\\
    Razonemos por contradicción, suponga que $\phi_k\overset{d}{\to}\phi$ y que existe $\alpha_{0},\beta_{0}\in\mathbb{N}^{n}$ tal que $\norm{\phi_{k}-\phi}_{\alpha_{0},\beta_{0}}\cancel{\to}$ $0$, luego
    \begin{align*}
      0 &= \lim_{k \to \infty}d(\phi_{k},\phi),\\
      &=\lim_{k \to \infty}\sum_{\alpha,\beta\in\mathbb{N}^{n}}\frac{1}{2^{|\alpha|+|\beta|}}\frac{\norm{\phi_{k}-\phi}_{\alpha,\beta}}{1+\norm{\phi_{k}-\phi}_{\alpha,\beta}},\\
      &=\sum_{\alpha,\beta\in\mathbb{N}^{n}}\frac{1}{2^{|\alpha|+|\beta|}}\lim_{k \to \infty}\frac{\norm{\phi_{k}-\phi}_{\alpha,\beta}}{1+\norm{\phi_k-\phi}_{\alpha,\beta}},\\
      &=\frac{1}{2^{|\alpha_{0}|+|\beta_{0}|}}\frac{\norm{\phi_{k}-\phi}_{\alpha_0,\beta_0}}{1+\norm{\phi_k-\phi}_{\alpha_0,\beta_0}}\neq 0.
    \end{align*}
    Lo cuál es una contradicción, por lo que podemos afirmar que si $\phi_{k}\overset{d}{\to}\phi$, entonces $\norm{\phi_k-\phi}_{\alpha,\beta}\to 0$ para todo $\alpha,\beta\in\mathbb{N}^{n}$, lo que concluye el resultado.
    (III)\\
    (IV)\\
    Primero veamos que si $\phi\in\mathcal{S}(\mathbb{R}^n)$, entonces $\hat{\phi}\in\mathcal{S}(\mathbb{R}^n)$.\\
	    Note que como $\mathcal{S}(\mathbb{R}^n)\subsetneq \mathcal{L}^p(\mathbb{R}^n)$ con $1\leq p \leq \infty$, entonces $\phi\in L(\mathbb{R}^n)$, por lo que nosotros podremos usar las propiedades de la transformada de Fourier en el sentido de $L(\mathbb{R}^n)$, así se sigue que
	    \begin{align*}
	      \partial^{\beta}\hat{\phi}(\xi)&=[\hat{(-2\pi i x)^{\beta}\phi(x)}](\xi).
	    \end{align*}
      Entonces, como para todo $\alpha\in\mathbb{N}^{n}$, nosotros sabemos que $x^\alpha \phi \in L(\mathbb{R}^n)$, luego podemos afirmar que $\hat{\phi}\in C^{\infty}(\mathbb{R}^{n})$.\\
	    Ahora, note que $\|\hat{\phi}\|_{(\alpha,\beta)}<\infty$ para todo $(\alpha,\beta)\in\mathbb{N}^{2n}$, así
	    \begin{align*}
        \sup_{\xi\in\mathbb{R}^n}|\xi^{\alpha}\partial^{\beta}\hat{\phi}(\xi)|&= \sup_{\xi\in\mathbb{R}^n}|\xi^{\alpha}(\hat{(-2\pi i x)^{\beta}\phi(x)})(\xi)|,\\
        &= \sup_{\xi\in\mathbb{R}^n}\left|\frac{[\hat{\partial^{\alpha}(-2\pi i x)^{\beta}\phi(x)}]}{(2\pi i)^{\alpha}}(\xi)\right|,
	    \end{align*}
      luego $x^{\beta}\phi\in\mathcal{S}(\mathbb{R}^n)$, $\partial^{\alpha}(-2\pi ix)^{\beta}\phi\in L(\mathbb{R}^n)$, por lo que podemos asegurar que $\hat{\partial^{\alpha}(-2\pi ix)^{\beta}\phi}\in L^{\infty}(\mathbb{R}^n)$, lo que implica que para todo $(\alpha,\beta)\in\mathbb{N}^{2n}$ nosotros tenemos que $\|\hat{\phi}\|_{\alpha,\beta}<\infty$, es decir, $\hat{\phi}\in\mathcal{S}(\mathbb{R}^{n})$.\\
      Ahora veamos que la transformada de Fourier es un operador inyectivo.
      Suponga $\hat{\phi}=\hat{\psi}$, entonces como $\phi,\psi\in L(\mathbb{R}^n)$, usando la fórmula de inversión se tiene que
		  \begin{align*}
			  \phi(x)&=\lim_{t\rightarrow 0}\int_{\mathbb{R}^n}\hat{\phi}(\xi)e^{2\pi i (x\cdot \xi)}e^{-4\pi^2t|\xi|^2}d\xi\\
			  &=\lim_{t\rightarrow 0}\int_{\mathbb{R}^n}\hat{\psi}(\xi)e^{2\pi i (x\cdot \xi)}e^{-4\pi^2t|\xi|^2}d\xi\\
			  &=\psi(x)
		  \end{align*}
		  Luego se puede concluir que $\phi=\psi$.
		  Ahora, veamos que es un operador sobreyectivo.\\
      De manera similar, como en $\mathcal{S}(\mathbb{R}^{n})$ es válida la fórmula de inversión de fourier en el sentido de $L(\mathbb{R}^{n})$, sabemos que si $\phi\in\mathcal{S}(\mathbb{R}^n)$, entonces $\hat{\phi}\in\mathcal{S}(\mathbb{R}^n)$, de lo que se puede ver que si $\phi\in\mathcal{S}(\mathbb{R}^n)$, entonces $\check{\phi}\in\mathcal{S}(\mathbb{R}^n)$, así que para cada $\phi\in\mathcal{S}(\mathbb{R}^n)$, sabemos que existe $\check{\phi}\in\mathcal{S}(\mathbb{R}^n)$ tal que $\hat{\check{\phi}}=\phi$, lo que demuestra que la transformada de Fourier es un operador sobreyectivo.
		Ahora, veamos que la transformada de Fourier es un operador continuo, es decir que si tomamos una sucesión $\{\phi_j\}\subset \mathcal{S}(\mathbb{R}^n)$, tal que $\phi_j \rightarrow \phi$ cuando $j\rightarrow \infty$, entonces $\hat{\phi_j}\rightarrow \hat{\phi}$ cuando $j\rightarrow \infty$.\\
		  De este modo, nosotros queremos ver que para todo $(\alpha,\beta)\in\mathbb{N}^{2n}$, $\|\hat{\phi_j}-\hat{\phi}\|_{\alpha,\beta}\rightarrow 0$ cuando $j\rightarrow \infty$. Note que
      \begin{align*}
			  \|\xi^{\alpha}\partial^{\beta}(\hat{\phi_j}-\hat{\phi})\|_{\infty}&=\|\xi^{\alpha}\partial^{\beta}(\hat{\phi_j-\phi})\|_{\infty},\\
			  &=\|\xi^{\alpha}(\hat{(-2\pi ix)^{\beta}(\phi_j-\phi)})\|_{\infty},\\
			  &=\|\frac{(-2\pi i)^{\beta}\hat{\partial^{\alpha}[(x^\beta)(\phi_j-\phi)]}}{(2\pi i)^{\alpha}}\|_{\infty},\\
			  &\leq\|\frac{(-2\pi i)^{\beta}\partial^{\alpha}[(x^\beta)(\phi_j-\phi)]}{(2\pi i)^{\alpha}}\|_{1},\\
			  &\leq c\|\partial^{\alpha}[(x^\beta)(\phi_j-\phi)]\|_{\infty}+c\sum_{|\gamma|\leq 2m}\|\partial{\alpha+\gamma}[(x^\beta)(\phi_j-\phi)]\|_{\infty},\\
			  &\leq k \sum_{|\gamma_1|,|\gamma_2|\leq2m+|\alpha|+|\beta|}|\|\phi_j-\phi\||_{(\gamma_1,\gamma_2)}.
		  \end{align*}
		  Luego para todo $(\gamma_1,\gamma_2)\in\mathbb{N}^{2n}$, $\|\phi_j-\phi\|_{(\gamma_1,\gamma_2)}\rightarrow 0$ cuando $j\rightarrow \infty$, entonces para todo $(\alpha,\beta)\in\mathbb{N}^{n}$, nosotros tenemos que $\|\hat{\phi_j}-\hat{\phi}\|_{(\alpha,\beta)}\rightarrow 0$ cuando $j\rightarrow \infty$.\\
      Note que ver que el operador inverso es continuo es un caso análogo, solo que usando la transformada inversa de Fourier por lo que queda demostrado que \phantom{  }$\hat{\empty}:\mathcal{S}(\mathbb{R}^n)\rightarrow\mathcal{S}(\mathbb{R}^n)$ es un isomorfismo.
  \end{solution}
\end{homeworkProblem}
