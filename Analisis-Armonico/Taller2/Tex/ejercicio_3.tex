\begin{homeworkProblem}
  \textbf{Producto de convolución $\mathcal{S}'*\mathcal{S}$}.
  \begin{enumerate}[(i)]
    \item Sean $f, \phi$ y $\psi \in \mathcal{S}(\mathbb{R}^{n})$, pruebe que
      \begin{align*}
        \int_{\mathbb{R}^{n}} f * \phi(x) \psi(x) \, dx = \int_{\mathbb{R}^{n}} f(x) \tilde{\phi} * \psi(x) \, dx,
      \end{align*}
      donde $\tilde{\phi}(x) = \phi(-x)$. Esto motiva la siguiente definición: Sean $T \in \mathcal{S}'(\mathbb{R}^{n})$ y $\phi \in \mathcal{S}(\mathbb{R}^{n})$,
      \begin{align*}
        T * \phi : \mathcal{S}(\mathbb{R}^{n}) \longrightarrow \mathbb{C}, \quad \psi \longmapsto T * \phi(\psi) := T(\tilde{\phi} * \psi).
      \end{align*}
      Pruebe que $T * \phi \in \mathcal{S}'(\mathbb{R}^{n})$ y que
      \begin{align*}
        \hat{(T * \phi)} = \hat{T} \hat{\phi} \quad \text{en} \quad \mathcal{S}'(\mathbb{R}^{n}).
      \end{align*}
    \item Por otro lado, si $T \in \mathcal{S}'(\mathbb{R}^{n})$ y $\phi \in \mathcal{S}(\mathbb{R}^{n})$, se define:
      \begin{align*}
        T *_1 \phi : \mathbb{R}^{n} \longrightarrow \mathbb{C}, \quad x \longmapsto T *_1 \phi(x) := T(\tau_x \tilde{\phi}),
      \end{align*}
      donde $\tau_x \phi(y) = \phi(y - x)$. Pruebe entonces que
      \begin{align*}
        T *_1 \phi \in C^\infty(\mathbb{R}^{n}) \cap \mathcal{S}'(\mathbb{R}^{n}) \quad \text{y} \quad T *_1 \phi = T * \phi.
      \end{align*}
  \end{enumerate}
  \begin{solution}
    (I)\\
    Veamos que $T*\phi \in \mathcal{S}'(\mathbb{R}^{n})$.\\
    Por claridad defina el operador $T_{T}(\phi)=T*\phi$, note que como $T\in\mathcal{S}'(\mathbb{R}^{n})$ y la convolución para funciones en $\mathcal{S}(\mathbb{R}^{n})$ es lineal, entonces $T_{T}$ es lineal, ya que si tomamos $\phi,\psi\in\mathcal{S}(\mathbb{R}^{n})$ y $\lambda$ escalar, entonces 
    \begin{align*}
      T_{T}(\phi+\lambda\psi)&=\left(T*\left( \phi+\lambda\psi \right)\right)(f),\\
      &=T\left( \tilde{\left( \phi+\lambda\psi \right)}*f \right),\\
      &=T\left( \tilde{\phi}*f+\lambda\tilde{\psi}*f \right),\\
      &=\left( T*\phi \right)(f)+\lambda\left( T*\psi \right)(f),\\
      &=T_{T}(\phi)+\lambda T_{T}(\psi).
    \end{align*}
    \newpage
    Ahora veamos que $T_T$ es un operador acotado, para esto recuerde que como $T\in\mathcal{S}(\mathbb{R}^{n})$ entonces se satisface que existe una constante $C>0$ y enteros $m,l$ tales que
    \begin{align*}
      |T(\phi)|\leq C\sum_{\substack{|\alpha|\leq l\\|\beta|\leq m}}\rho_{\alpha,\beta}(\phi)
    \end{align*}
    para toda $\phi\in\mathcal{S}(\mathbb{R}^{n})$.\\
    Asumiendo esto podemos ver que como $\tilde{\phi}*f\in \mathcal{S}(\mathbb{R}^{n})$, entonces
    \begin{align*}
      |T_{T}(\phi)|&=\left| \left( T*\phi \right)(f) \right|,\\
      &=\left| T\left( \tilde{\phi}*f \right) \right|,\\
      &\leq C\sum_{\substack{|\alpha|\leq l\\|\beta|\leq m}}\rho_{\alpha,\beta}(\tilde{\phi}*f)
    \end{align*}
    Lo que concluye que $T_{T}$ es un operador lineal continuo, es decir, $T*\phi\in\mathcal{S}'(\mathbb{R}^{n})$.\\
    Ahora, veamos que $(\hat{T*\phi})=\hat{T}\hat{\phi}$ en $\mathcal{S}'(\mathbb{R}^{n})$.\\
    Note que
    \begin{align*}
      (\hat{T*\phi})(f)&=(T*\phi)(\hat{f}),\\
      &=T\left( \tilde{\phi}*\hat{f} \right),\\
      &=T\left( \hat{\hat{\phi}}*\hat{f} \right),\\
      &=T\left( \hat{\left(\hat{\phi}f\right)} \right),\\
      &=\hat{T}\left( \hat{\phi}f \right),\\
      &=\hat{T}\hat{\phi}(f).
    \end{align*}
    Lo que concluye el ejercicio.\\
    (II)\\
    Veamos que $T*_{1}\phi\in C^{\infty}(\mathbb{R}^{n})\cup \mathcal{S}'(\mathbb{R}^{n})$.\\
    Primero veamos que $T*_{1}\phi\in C^{\infty}(\mathbb{R}^{n})$, para esto note que como $T$ es continuo, entonces 
      \begin{align*}
        \partial_{x_j}\left(T*_{1}\phi\right)(x)&=\lim_{h\rightarrow 0}\frac{(T*_{1}\phi)(x+h\epsilon_j)-(T*\phi)(x)}{h},\\
        &=\lim_{h \to 0}\frac{T\left( \tau_{x+h\epsilon_{j}}\tilde{\phi}(y)\right)-T\left( \tau_{x}\tilde{\phi}(y) \right)}{h},\\
        &=\lim_{h\rightarrow 0}\frac{T\left( \phi\left( x+h\epsilon_{j}-y \right) \right)-T\left( \phi\left( x-y \right) \right)}{h},\\
        &=\lim_{h \to 0}T\left( \frac{\phi(x+h\epsilon-y)-\phi(x-y)}{h} \right),\\
        &=T\left( \lim_{h \to 0}\frac{\phi(x+h\epsilon_{j}-y)-\phi(x-y)}{h}\right),\\
        &=T\left( \partial_{x_{j}}\phi(x-y) \right),\\
        &=T\left( \tau_{x}\tilde{\partial_{x_j}\phi}(y) \right),\\
        &= \left(T*_{1}\partial_{x_j}\phi\right)(x),
      \end{align*}
      luego usando un argumento inductivo, como $\phi\in\mathcal{S}(\mathbb{R}^{n})$ podemos concluir que $T*_{1}\phi\in C^{\infty}(\mathbb{R}^{n})$.\\
      Ahora veamos que $T*_{1}\phi\in\mathcal{S}'(\mathbb{R}^{n})$, para esto con el fin de ser más claros definiremos el operador $T_{T}(\phi)=T*_{1}\phi$, note que como las traslaciones y reflexiones son lineales, entonces dadas $\phi,\varphi\in\mathcal{S}(\mathbb{R}^{n})$ con $\lambda$ escalar se cumple que
      \begin{align*}
        T_{T}(\phi+\lambda\varphi)&=T*_{1}\left( \phi+\lambda\varphi \right),\\
        &=T\left( \tau_{x}\tilde{\phi+\lambda\varphi} \right),\\
        &=T\left( \tau_{x}\tilde{\phi}+\lambda\tau_{x}\tilde{\varphi} \right),\\
        &=T\left( \tau_{x}\tilde{\phi} \right)+\lambda T\left( \tau_{x}\tilde{\varphi} \right),\\
        &=T*_{1}\phi+\lambda T*_{1}\varphi.
      \end{align*}
      Ahora veamos la continuidad, note que como $T\in\mathcal{S}'(\mathbb{R}^{n})$, entonces existe una constante $C>0$ y enteros $m$ y $l$ tales que para toda $\phi\in \mathcal{S}(\mathbb{R}^{n})$ se cumple que
      \begin{align*}
        |T(\phi)|\leq C\sum_{\substack{|\alpha|\leq l\\ |\beta|\leq m}}\rho_{\alpha,\beta}(\phi),
      \end{align*}
      usando esto se puede ver que
      \begin{align*}
        |T_{T}(\phi)|&=\left| T*_{1}\phi(x) \right|,\\
        &=\left| T\left( \tau_{x}\tilde{\phi}(y) \right) \right|,\\
        &\leq C\sum_{\substack{|\alpha|\leq l\\|\beta|\leq m}}\rho_{\alpha,\beta}(\tau_{x}\tilde{\phi}),
      \end{align*}
      lo que nos permite concluir que $T_{T}\in\mathcal{S}'(\mathbb{R}^{n})$, es decir que $T*_{1}\phi\in\mathcal{S}'(\mathbb{R}^{n})$.\\
      Ahora, con el fin de ver que $T*_{1}\phi=T*\phi$, note que como $T*_{1}\phi\in C^{\infty}(\mathbb{R}^{n})$ es localmente integrable, entonces si tomamos $f\in C^{\infty}_{c}(\mathbb{R}^{n})$
      \begin{align*}
        \left( T*_{1}\phi \right)(f)&=\left( T(\tau_{x}\tilde{\phi}) \right)(f),\\
        &=\int_{\mathbb{R}^{n}}T(\tau_{x}\tilde{\phi})(y)f(x)\, dx,\\
        &=T\left( \int_{\mathbb{R}^{n}}\tau_{x}\tilde{\phi}(y)f(x)\, dx \right),\\
        &=T\left( \int_{\mathbb{R}^{n}}\phi(x-y)f(x)\, dx \right),\\
        &=T\left( \int_{\mathbb{R}^{n}}\tilde{\phi}(y-x)f(x)\, dx \right),\\
        &=T\left( \tilde{\phi}*f \right),\\
        &=(T*\phi)(f).
      \end{align*}
      Lo que concluye el resultado.
  \end{solution}
\end{homeworkProblem}
