\begin{homeworkProblem}
  Pruebe que el operador $\mathcal{F}$ de la transformada de Fourier es un isomorfismo de $\mathcal{S}'(\mathbb{R}^{n})$ en si mismo. Dada $\Psi \in \mathcal{S}'(\mathbb{R}^{n})$ y $\alpha\in\mathbb{N}^{n}$ multi-índice, pruebe que:
  \begin{enumerate}[(i)]
    \item $\hat{\partial^{\alpha}\Psi}=(2\pi i)^{|\alpha|}\xi^{\alpha}\hat{\Psi}$;
    \item $(-2\pi i)^{|\alpha|}\hat{x^{\alpha}\Psi}=\partial^{\alpha}\hat{\Psi}$;
    \item $\check{\hat{\Psi}}=\Psi=\hat{\check{\Psi}}$;
    \item $\mathcal{F}^{4}=Id$.
  \end{enumerate}
  \begin{solution}
    Veamos que el operador $\mathcal{F}$ de la transformada de Fourier es un isomorfismo de $\mathcal{S}'(\mathbb{R}^{n})$ en si mismo.\\
    Para esto primero mostremos que es un operador inyectivo, esto ya que si suponemos que dados $\Psi_{1},\Psi_{2}\in\mathcal{S}'(\mathbb{R}^{n})$ tales que $\hat{\Psi_{1}}=\hat{\Psi_{2}}$ se puede concluir que para toda $\phi\in\mathcal{S}(\mathbb{R}^{n})$ se cumple que
    \begin{align*}
      \Psi_{1}(\hat{\phi})&=\hat{\Psi_{1}}(\phi),\\
      &=\hat{\Psi_{2}}(\phi),\\
      &=\Psi_{2}(\hat{\phi}).
    \end{align*}
    De lo que se puede afirmar que $\Psi_{1}=\Psi_{2}$, lo que demuestra que $\mathcal{F}$ es un operador inyectivo.\\
    Ahora veamos que $\mathcal{F}$ es un operador sobreyectivo, ya que si definimos $\mathcal{F}^{-1}:\mathcal{S}'(\mathbb{R}^{n})\to\mathcal{S}'(\mathbb{R}^{n})$ tal que si tomamos $\Psi\in\mathcal{S}'(\mathbb{R}^{n})$, entonces $\check{\Psi}(\phi)=\Psi(\check{\phi})$, entonces de forma análoga a $\mathcal{F}$ se puede demostrar que $\mathcal{F}^{-1}$ es un operador inyectivo, luego dado $\Psi\in\mathcal{S}'(\mathbb{R}^{n})$ se sabe que existe $\check{\Psi}\in\mathcal{S}'(\mathbb{R}^{n})$ tal que $\hat{\check{\Psi}}=\Psi$, lo que demuestra que $\mathcal{F}$ es un operador sobreyectivo y por ende biyectivo.\\
    Ahora, veamos que es un operador continuo, sea $\{\psi_{j}\}\subset\mathcal{S}'(\mathbb{R}^{n})$ tal que $\psi_j\to\psi$ cuando $j\to\infty$ en el sentido de $\mathcal{S}'(\mathbb{R}^{n})$, veamos que $\hat{\psi_{j}}\to\hat{\psi}$ cuando $j\to\infty$ en el sentido de $\mathcal{S}'(\mathbb{R}^{n})$ ya que
    \begin{align*}
      \lim_{j \to \infty}\hat{\psi_{j}}(\phi)&=\lim_{j \to \infty}\psi_{j}(\hat{\phi}),\\
      &=\phi_{\hat{\phi}},\\
      &=\hat{\psi}(\phi).
    \end{align*}
    Luego aplicando el mismo razonamiento con $\mathcal{F}^{-1}$ nosotros podemos concluir que el operador $\mathcal{F}$ de la transformada de Fourier es un isomorfismo de $\mathcal{S}'(\mathbb{R}^{n})$ en si mismo.
    \begin{enumerate}
      \item Veamos que $\hat{\partial^{\alpha}\Psi}=(2\pi i)^{|\alpha|}\xi^{\alpha}\hat{\Psi}$, ya que
        \begin{align*}
          \hat{\partial^{\alpha}\Psi}(\phi)&=\partial^{\alpha}\Psi(\hat{\phi}),\\
          &=(-1)^{|\alpha|}\Psi(\partial^{\alpha}\hat{\phi}),\\
          &=(-1)^{|\alpha|}\Psi((-2\pi i)^{|\alpha|}\hat{x^{\alpha}\phi}),\\
          &=(2\pi i)^{|\alpha|}\Psi(\hat{x^{\alpha}\phi}),\\
          &=(2\pi i)^{|\alpha|}\hat{\Psi}(x^{\alpha}\phi),\\
          &=(2\pi i)^{|\alpha|}\xi^{\alpha}\hat{\Psi}(\phi),
        \end{align*}
        Lo que concluye el resultado.
      \item Veamos que $(-2\pi i)^{|\alpha|}\hat{x^{\alpha}\Psi}=\partial^{\alpha}\hat{\Psi}$, ya que
        \begin{align*}
          (-2\pi i)\hat{x^{\alpha}\Psi}(\phi)&=\hat{x^{\alpha}\Psi}((-2\pi i)^{|\alpha|}\phi),\\
          &=x^{\alpha}\Psi((-2\pi i)^{|\alpha|}\hat{\phi}),\\
          &=\Psi((-2\pi i)^{|\alpha|}\xi^{\alpha}\hat{\phi}),\\
          &=\Psi((-1)^{|\alpha|}\hat{\partial^{\alpha}\phi}),\\
          &=\hat{\Psi}((-1)^{\alpha}\partial^{\alpha}\phi),\\
          &=\partial^{\alpha}\Psi(\phi),
        \end{align*}
        lo que concluye el resultado.
      \item Veamos que $\check{\hat{\Psi}}=\Psi=\hat{\check{\Psi}}$, ya que
        \begin{align*}
          \check{\hat{\Psi}}(\phi)&=\hat{\Psi}(\check{\phi}),\\
          &=\Psi(hat{\check{\phi}}),\\
          &=\Psi(\phi),\\
          &=\Psi(\check{\hat{\phi}}),\\
          &=\check{\Psi}(\hat{\phi}),\\
          &=\hat{\check{\Psi}}(\phi),
        \end{align*}
        lo que concluye el resultado.
      \item Veamos que $\mathcal{F}^{4}=Id$ ya que
        \begin{align*}
          \mathcal{F}^{4}\Psi(\phi)&=\Psi(\mathcal{F}^{4}\phi),\\
          &=\Psi(\phi),
        \end{align*}
        lo que concluye que $\mathcal{F}^{4}=Id$.
    \end{enumerate}
  \end{solution}
\end{homeworkProblem}
