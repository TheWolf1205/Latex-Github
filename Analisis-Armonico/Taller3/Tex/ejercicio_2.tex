\begin{homeworkProblem}
  Pruebe la siguiente extensión en $\mathcal{S}'(\mathbb{R}^{n})$ de la fórmula $\hat{(e^{-\lambda\pi|x|^2})}=\lambda^{-\frac{n}{2}}e^{\pi\frac{|\xi|^2}{\lambda}}$ probada en la primera lista de ejercicios:
  \begin{align*}
    \hat{e^{-\lambda|x|^2}}(\xi)&=\left( \frac{\pi}{\lambda} \right)^{\frac{n}{2}}e^{-\pi^{2}\frac{|\xi|^2}{\lambda}}
  \end{align*}
  donde $\sqrt{\lambda}$ es definida como la rama con $Re \lambda>0$. Use un argumento de continuación analítica. Pruebe que la fórmula también vale en $\mathcal{S}'(\mathbb{R}^{n})$ cuando $Re \lambda=0$ y $\lambda\neq 0$. 
  \begin{solution}
    Usando la fórmula de la primera lista de ejercicios tenemos que
    \begin{align*}
      \hat{e^{-\lambda |x|^2}}(\xi)&=\hat{e^{-\pi \frac{\lambda}{\pi}|x|^2}}(\xi),\\
      &=\left( \frac{\pi}{\lambda} \right)^{\frac{n}{2}}e^{-\pi\frac{|\xi|^2}{\frac{\lambda}{\pi}}},\\
      &=\left( \frac{\pi}{\lambda} \right)^{\frac{n}{2}}e^{-\pi^2\frac{|\xi|^2}{\lambda}}, 
    \end{align*}
    para todo $\lambda>0$.\\
    Siendo así, sea $\phi\in\mathcal{S}(\mathbb{R}^{n})$ definamos
    \begin{align*}
      F(\lambda)&=\hat{e^{-\lambda|x|^2}}(\phi),\\
      &=\int_{\mathbb{R}^{n}}e^{-\lambda|x|^2}\hat{\phi}(x)\, dx,
    \end{align*}
    Note que si $Re \lambda>0$, esta integral converge absolutamente y define una función holomorfa en $\lambda$, además
    \begin{align*}
      F(\lambda)&=\left( \frac{\pi}{\lambda} \right)^{\frac{n}{2}}\int_{\mathbb{R}^{n}}e^{-\pi^2\frac{|\xi|^2}{\lambda}}\phi(\xi)\, d\xi,
    \end{align*}
    muestra que $F(\lambda)$ tiene una continuación analítica meromorfa a $\lambda\in \mathbb{C}\setminus\{0\}$, por lo que podemos definir una familia de distribuciones temperadas que dependen de $\lambda\in\mathbb{C}\setminus\{0\}$, por lo que vía continuación analítica podemos definir
    \begin{align*}
      \Psi_{\lambda}=\hat{e^{-\lambda|x|^2}}\in\mathcal{S}'(\mathbb{R}^{n}),
    \end{align*}
    para todo $\lambda\in\mathbb{C}\setminus\{0\}$.\\
    Veamos el caso en el que $Re \lambda>0$ y $\lambda\neq 0$, en un principio, si bien la función $e^{-\lambda|x|^2}\notin L^1(\mathbb{R}^{n})$, como $\Re \lambda=0$ esto se puede ver como $e^{-i\lambda_{2}|x|^2}$ con $\lambda_{2}\in\mathbb{R}\setminus\{0\}$, luego note que $e^{-i\lambda_{2}|x|^2}\in\mathcal{S}'(\mathbb{R}^{n})$ ya que este es un funcional continuo, ya que si tomamos $m$ tal que $2m>n$ entonces tenemos que 
    \begin{align*}
      |e^{-i\lambda_{2}|x|^2}(\phi)|&=\left| \int_{\mathbb{R}^{n}}e^{-i\lambda_{2}|x|^2}\phi(x)\, dx \right|,\\
      &\leq \int_{\mathbb{R}^{n}}|e^{-i\lambda_{2}|x|^2}\phi(x)|\, dx\\
      &\leq \int_{\mathbb{R}^{n}}|\phi(x)|\, dx,\\
      &\leq C_{1}\norm{\phi}_{0,0}+\int_{|x|>1}|\phi(x)|\, dx,\\
      &\leq C_{1}\norm{\phi}_{0,0}+\int_{|x|>1}\frac{x^{2m}|\phi(x)|}{x^{2m}}\, dx,\\
      &\leq C_{1}\norm{\phi}_{0,0}+\norm{\phi}_{2m,0}\int_{|x|>1}\frac{1}{x^2m} \, dx,\\
      &\leq C(\norm{\phi_{0,0}+\norm{\phi}_{2m,0}}).
    \end{align*}
    Lo que nos permite asegurar que $\Psi_{\lambda}$ define una distribución temperada y por ende la extensión es válida cuando $Re \lambda>0$ y cuando $Re \lambda=0$ con $\lambda\neq 0$. 
  \end{solution}
\end{homeworkProblem}
