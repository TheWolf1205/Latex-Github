\begin{homeworkProblem}
  Sean $B_R := \{x \in \mathbb{R}^n : |x| < R\}$ y $\chi_{B_R}$ la función característica del conjunto $B_R$. Se define
  \begin{align*}
    \mathcal{S}_{R}:L^2(\mathbb{R}^n) &\longrightarrow L^2(\mathbb{R}^n)\\
    f &\longmapsto \check{\left(\chi_{B_R} \hat{f} \right)}.   
  \end{align*}
  \begin{enumerate}[(i)]
    \item Probar que $\mathcal{S}_{R} \in \mathcal{B}(L^2(\mathbb{R}^n))$ y que $\|\mathcal{S}_{R}\|_{\mathcal{B}(L^2(\mathbb{R}^n))} \leq 1$.
    \item Mostrar que para todo $f \in L^2(\mathbb{R}^n)$, $\mathcal{S}_R f \to f$ en $L^2(\mathbb{R}^n)$ cuando $R \to +\infty$.
    \item Deducir que para cualquier $f \in L^2(\mathbb{R}^n)$, existe una sucesión $R_n \to +\infty$ cuando $n \to +\infty$ tal que
      \begin{align*}
        \int_{B_{R_n}} \widehat{f}(\xi) e^{2\pi i x \cdot \xi} \, d\xi \longrightarrow f(x), \quad \text{cuando } n \to +\infty, \quad \text{c.t.p. } x \in \mathbb{R}^n.
      \end{align*}
    \item Probar que para cualquier $f \in L^2(\mathbb{R}^n)$,
      \begin{align*}
        \xi \longmapsto \int_{B_R} f(x) e^{-2\pi i x \cdot \xi} \, dx \longrightarrow \hat{f}, \quad \text{cuando } R \to +\infty, \text{ en } L^2(\mathbb{R}^n).
      \end{align*}
  \end{enumerate}
  \begin{solution}
    \begin{note}{}
      $\mathcal{B}(L^2(\mathbb{R}^{n}))$ es el espacio de los operadores lineales acotados de $L^2(\mathbb{R}^{n})$ en $L^2(\mathbb{R}^{n})$.  
    \end{note}
    \begin{note}{}
      Veamos en un principio que el operador tiene sentido.\\
      Para esto recordemos que la transformada de $f$ es en $L^2(\mathbb{R}^{n})$, pero que nuestra definición de la inversa sigue estando en $L^1(\mathbb{R}^{n})$, verifiquemos que $\chi_{B_R}\hat{f}\in L^1(\mathbb{R}^{n})$, para esto note que usando la desigualdad de Cauchy-Schwarz, la identidad de Parseval y que $f\in L^2(\mathbb{R}^{n})$ se cumple que
      \begin{align*}
        \norm{\chi_{B_{R}}\hat{f}}_{1}&=\int_{\mathbb{R}^{n}}\left| \chi_{B_{R}}(x)\hat{f}(x) \right|\, dx,\\
        &=\int_{\mathbb{R}^{n}}\overline{\left| \chi_{B_{R}}(x) \right|}\left| \hat{f} \right|\, dx,\\
        &\leq \left( \int_{\mathbb{R}^{n}}|\chi_{B_{R}}(x)|^2\, dx \right)^{\frac{1}{2}}\left( \int_{\mathbb{R}^{n}}|\hat{f}(x)|^{2}\, dx \right)^{\frac{1}{2}},\\
        &\leq \norm{\chi_{B_{R}}}_{2}\norm{\hat{f}}_{2},\\
        &\leq \norm{\chi_{B_{R}}}_{2}\norm{f}_{2}.
      \end{align*}
      Por lo que podemos asegurar que $\norm{\chi_{B_{R}}\hat{f}}\in L^1(\mathbb{R}^{n})$ y por ende que $\mathcal{S}_{R}(f)$ existe y está bien definido. 
    \end{note}
    \begin{enumerate}[(i)]
      \item Sea $f,g\in L^2(\mathbb{R}^{n})$ y $\alpha$ escalar, entonces
        \begin{align*}
          \mathcal{S}_{R}\left( \alpha f+g \right)&=\check{\left( \chi_{B_{R}}(\hat{\alpha f+g}) \right)},\\
          &= \left( \check{\alpha\chi_{B_{R}}\hat{f}+\chi_{B_{R}}\hat{g}} \right),\\
          &=\alpha \left( \check{\chi_{B_{R}}\hat{f}} \right)+\left( \check{\chi_{B_{R}}\hat{g}} \right),\\
          &=\alpha \mathcal{S}_{R}(f)+\mathcal{S}_{R}(g).
        \end{align*}
        Ahora veamos que este es un operador acotado
        \begin{align*}
          \norm{\mathcal{S}_{R}(f)}_{2}&=\norm{\left( \check{\chi_{B_{R}}\hat{f}} \right)}_{2},\\
          &=\norm{\chi_{B_{R}}\hat{f}}_{2},\\
          &\leq\norm{\hat{f}}_{2},\\
          &\leq \norm{f}_{2}.
        \end{align*}
        Por lo que podemos concluir que $\mathcal{S}_{R}\in \mathcal{B}(L^2(\mathbb{R}^{n}))$ y que $\norm{S_{R}}\leq 1$ .
      \item Note que lo que buscamos es equivalente a
        \begin{align*}
          \lim_{R \to \infty}\norm{\mathcal{S}_{R}f-f}_{2}&=\lim_{R \to \infty}\norm{\left( \left( \check{\chi_{B_{R}}\hat{f}} \right)-\check{\hat{f}} \right)}_{2},\\
          &=\lim_{R \to \infty}\norm{\chi_{B_{R}}\hat{f}-\hat{f}}_{2},\\
          &=\lim_{R \to \infty}\norm{\left( \chi_{B_{R}-1} \right)\hat{f}}_{2},\\
          &=\lim_{R \to \infty}\norm{\chi_{B_{R}^{c}}\hat{f}}_{2},\\
          &=0.
        \end{align*}
        Siendo así, note que como $f\in L^2(\mathbb{R}^{n})$, entonces $\hat{f}\in L^2(\mathbb{R}^{n})$. Además, esto implica que dado $\epsilon>0$ existe $R>0$ tal que si $r\geq R$, entonces
        \begin{align*}
          \left(\int_{|x|\geq r}|\hat{f}(x)|^2\, dx\right)^{\frac{1}{2}} <\epsilon.
        \end{align*}
        Note que esto implica que
        \begin{align*}
          \norm{\chi_{B_{r}^c}\hat{f}}_{2}&=\left(\int_{\mathbb{R}^{n}}\left|\chi_{B_{r}^c}(x)\hat{f}(x)\right|^2\, dx\right)^{\frac{1}{2}},\\
          &=\left(\int_{|x|\geq r}|\hat{f}(x)|^2\, dx\right)^{\frac{1}{2}},\\
          &< \epsilon.
        \end{align*}
        Lo que implica que $\chi_{B_{R}^{c}}\hat{f}\to 0$ y por ende $\mathcal{S}_{R}f\to f$ en $L^2(\mathbb{R}^{n})$ cuando $R\to\infty$ para toda $f\in L^2(\mathbb{R}^{n})$.
      \item Note que como $\mathcal{S}_{R}f\to f$ en $L^2(\mathbb{R}^{n})$ cuando $R\to \infty$, entonces existe una subsucesión $\{R_n\}\subset \mathbb{R}$ tal que $\mathcal{S}_{R_{n}}f\to f$ en $L^2(\mathbb{R}^{n})$.\\
      Suponga $\{f_{m}\}\subset L^{1}(\mathbb{R}^{n})\cap L^2(\mathbb{R}^{n})$ tal que $\lim_{m \to \infty}f_m=f$ en $L^2(\mathbb{R}^{n})$, entonces note que por definición de la transformada de Fourier en $L^2(\mathbb{R}^{n})$.
      \begin{align*}
        \lim_{m \to \infty}\hat{f_{m}}&=\hat{f}.
      \end{align*}
      luego es válido afirmar que
      \begin{align*}
        \mathcal{S}_{R_n}(f)&=\check{\left( \chi_{B_{R_{n}}}\hat{f} \right)},\\
        &=\check{\left( \chi_{B_{R_n}}\lim_{m \to \infty}\hat{f_{m}} \right)},\\
        &=\lim_{m \to \infty}\check{\left( \chi_{B_{R_n}}\hat{f_{m}} \right)},\\
        &=\lim_{m \to \infty}\mathcal{S}_{R_n}(f_m).
      \end{align*}
      Luego usando que $\chi_{B_{R}}\hat{f}\in L^1(\mathbb{R}^{n})$ se tiene que
      \begin{align*}
        \lim_{m \to \infty}\mathcal{S}_{R_{n}}(f_m)(x)&=\lim_{m \to \infty}\left( \check{\chi_{B_{R_{n}}}\hat{f_{m}}} \right)(x),\\
        &=\lim_{m \to \infty}\int_{B_{R_{n}}}\hat{f_m}(\xi)e^{2\pi ix\cdot\xi}\, d\xi,\\
        &=\int_{B_{R_{n}}}\lim_{m \to \infty}\hat{f_m}(\xi)e^{2\pi ix\cdot\xi}\, d\xi,\\
        &=\int_{B_{R_{n}}}\hat{f}(\xi)e^{2\pi ix\cdot\xi}\, d\xi.
      \end{align*}
      Es decir, $\mathcal{S}_{R_n}(f)(x)=\int_{B_{R_{n}}}\hat{f}(\xi)e^{2\pi ix\cdot\xi}\, d\xi$ en casi todo $x\in\mathbb{R}^{n}$, pero recordemos que $\mathcal{S}_{R_n}(f)\to f$ cuando $R_{n}\to\infty$, entonces podemos asegurar que
      \begin{align*}
        \int_{B_{R_n}}\hat{f}(\xi)e^{2\pi ix\cdot\xi}\, d\xi \to f(x),
      \end{align*}
      para casi todo $x\in \mathbb{R}^{n}$.
      \item Note que
        \begin{align*}
          \int_{B_{R}}f(x)e^{-2\pi ix\cdot \xi}\, dx&=\int_{\mathbb{R}^{n}}\chi_{B_{R}}(x)f(x)e^{-2\pi ix\cdot\xi}\, dx,\\
          &=\hat{\chi_{B_{R}}f}(\xi).
        \end{align*}
        Luego lo que buscamos es equivalente a
        \begin{align*}
          \lim_{R \to \infty}\norm{\hat{\chi_{B_{R}}f}-\hat{f}}_{2}&=\lim_{R \to \infty}\norm{\chi_{B_{R}}f-f}_2,\\
          &=\lim_{R \to \infty}\norm{\left( \chi_{B_{R}}-1 \right)f}_{2},\\
          &=\lim_{R \to \infty}\norm{\chi_{B_{R}^{c}}f}_{2},\\
          &=0.
        \end{align*}
        posteriormente podemos ver que si razonamos similarmente a (II) podemos ver que dado $\epsilon>0$ existe $R>0$ tal que si $r\geq R$, entonces
        \begin{align*}
          \norm{\chi_{B_{r}^{c}}f}<\epsilon.
        \end{align*}
        Lo que nos permite concluir el ejercicio.
    \end{enumerate}    
  \end{solution}
\end{homeworkProblem}
