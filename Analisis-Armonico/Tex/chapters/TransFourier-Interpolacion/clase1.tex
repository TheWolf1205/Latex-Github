\section{Transformada de Fourier.}
  La transformada de Fourier tiene sus orígenes a principios del siglo XIX, cuando el matemático y físico francés Joseph Fourier (1768–1830) introdujo una revolucionaria idea al estudiar la propagación del calor. En 1807, presentó ante la Academia de Ciencias de París un manuscrito titulado \emph{Théorie de la propagation de la chaleur dans les solides}, en el cual proponía que cualquier función, bajo ciertas condiciones, podía expresarse como una serie infinita de senos y cosenos. Aunque sus ideas no fueron inmediatamente aceptadas —en parte por las dudas de sus contemporáneos sobre la legitimidad de representar funciones discontinuas mediante series trigonométricas—, con el tiempo se reconoció la profundidad de su enfoque y su enorme potencial analítico.\\
  Joseph Fourier nació en Auxerre, Francia, en 1768. Huérfano desde pequeño, ingresó a una escuela militar donde recibió una sólida formación matemática. Participó activamente en los cambios políticos de su época, y más tarde acompañó a Napoleón en la expedición a Egipto, donde mostró gran interés por la ciencia y la cultura de aquel país. A su regreso a Francia, ocupó diversos cargos administrativos y académicos, y fue nombrado prefecto del departamento de Isère. Durante su estancia en Grenoble desarrolló gran parte de sus investigaciones sobre la ecuación del calor, trabajo que culminó en su obra maestra \emph{Théorie analytique de la chaleur}, publicada en 1822.\\
  El impacto de las ideas de Fourier ha sido inmenso, extendiéndose mucho más allá de la física del calor. La noción de representar funciones como sumas de funciones armónicas abrió las puertas al análisis de señales, la mecánica cuántica, la teoría de la información, y múltiples ramas de la matemática pura y aplicada. La transformada de Fourier, como se conoce hoy en día, constituye una herramienta esencial en análisis armónico, proporcionando un puente entre el dominio temporal o espacial y el dominio frecuencial.\\
  Todo esto inspirado en la ecuación del calor
  \begin{align*}
    \begin{cases}
      \partial_{t}U=\partial^{2}_{x}u,&\text{con $(x,t)\in(0,\pi)\times (0,\infty)$,}\\
      u(0,t)=u(\pi,t)=0, &\text{para todo $t\geq 0$ },\\
      u(x,0)=f(x)&f(0)=f(\pi)=0.
    \end{cases}
  \end{align*}
  Fourier se inspira en buscar una solución de la forma separación de variables
  \begin{align*}
    u(x,t)&=X(x)T(t).
  \end{align*}
  Esto motiva a pensar en qué condiciones tiene que cumplir una función $f$ periódica de periodo $2\pi$ para poderla escribir como combinación lineal de funciones trigonométricas, es decir
  \begin{align*}
    f(x)=\sum_{k=0}^{\infty}a_{k}\cos(kx)+b_k\sen(kx).
  \end{align*}
  Por cuestiones de facilidad en la escritura vamos a pensar el problema en funciones de periodo $1$, para esto recordemos que
  \begin{align*}
    e^{ixk}=\cos(kx)+i\sen(kx),
  \end{align*}
  luego podemos reescribir la expresión anterior de la forma
  \begin{align*}
    f(x)=\sum_{k=-\infty}^{\infty}c_{k}e^{2\pi ixk}
  \end{align*}
  donde
  \begin{align*}
    c_{k}=\hat{f}(k)=\int_{0}^{1}f(x)e^{-2\pi i xk}\,dx,\quad\text{para todo $k\in\mathbb{Z}$.}
  \end{align*}
  El restante de teoría de Fourier se deja como lectura o motivante para entrar al curso de Series de Fourier.
\section{La transformada de Fourier en $L^{p}(\mathbb{R}^{n})$.}
  \begin{definition}{Transformada de Fourier en $L^{1}(\mathbb{R}^{n})$}
    Sea $f\in L^{1}(\mathbb{R}^{n})$, llamaremos la transformada de Fourier de $f$ a $\hat{f}$ definida por
    \begin{equation}
      \hat{f}(\xi)=\int_{\mathbb{R}^{n}}f(x)e^{-2\pi i x\cdot \xi}\,dx,
    \end{equation}
    donde $\xi\in\mathbb{R}^{n}$ y $x\cdot\xi$ es el producto punto entre $x$ y $\xi$. 
  \end{definition}
  \begin{note}{}\label{note:f-tranformada-sentido}
    Veamos que $\hat{f}$ tiene sentido como integral.\\
    Para ver esto note que
    \begin{align*}
      \left| \hat{f}(\xi) \right|&\leq\left| \int_{\mathbb{R}^{n}}f(x)e^{-2\pi ix\cdot\xi}\,dx \right|,\\
      &\leq \int_{\mathbb{R}^{n}}\left| f(x)e^{2\pi ix\cdot\xi} \right|\,dx,\\
      &\leq \int_{\mathbb{R}^{n}}\left| f(x) \right|\,dx,\\
      &\leq \norm{f}_{1}.
    \end{align*}
    luego se puede ver que
    \begin{align*}
      \norm{\hat{f}}_{\infty}\leq \norm{f}_{1}<\infty.
    \end{align*}
    Lo que nos permite definir la transformada de Fourier como un operador $\hat{\phantom{f}}:L^{1}(\mathbb{R}^{n})\to L^{\infty}(\mathbb{R}^{n})$.
  \end{note}
  Para ver el comportamiento de la transformada de Fourier anteriormente definida en $L^{p}(\mathbb{R}^{n})$, veamos la siguiente definición.
  \begin{definition}{}
    Una función $f\in L^{p}(\mathbb{R}^{n})$ es diferenciable en $L^{p}(\mathbb{R}^{n})$ con respecto a la $k$-ésima variable si existe $g\in L^{p}(\mathbb{R}^{n})$ tal que
    \begin{align*}
      \lim_{|h| \to 0}\norm{\frac{f(x+he_{k}-f(x))}{h}-g(x)}_{p}=0,
    \end{align*}
    en donde $e_k$ es el vector $k$-ésimo de la base canónica de $\mathbb{R}^{n}$.
    Si tal función $g$ existe es llamada la derivada parcial respecto a la $k$-ésima variable de $f$ en $L^{p}(\mathbb{R}^{n})$, lo denotaremos por $\frac{\partial f}{\partial x_{k}}$. 
  \end{definition}
  Ahora sí veamos el siguiente resultado.
  \begin{theorem}{}
    Sean $f,g\in L^{1}(\mathbb{R}^{n})$, entonces se satisface que
    \begin{enumerate}
      \item $\norm{\hat{f}}_{\infty}\leq\norm{f}_{1}$.
      \item Si $\tau_{h}f(x)=f(x-h)$ con $h\in\mathbb{R}^{n}$, entonces
        \begin{align*}
          \hat{\tau_{h}f}(\xi)&=e^{-2\pi ih\cdot \xi}\hat{f}(\xi),\\
          \tau_{h}\hat{f}(\xi)&=\hat{e^{2\pi ix\cdot h}f}(\xi).
        \end{align*}
      \item $T:\mathbb{R}^{n}\to\mathbb{R}^{n}$ es transformación lineal invertible y $S=\left( T^{-1} \right)^{t}=\left( T^{t} \right)^{-1}$, entonces
        \begin{align*}
          \hat{f\circ T}=|det(T)|^{-1}\left( \hat{f}\circ S \right).
        \end{align*}
        En particular, si $T$ es una rotación
        \begin{align*}
          \hat{f\circ T}=\hat{f}\circ T.
        \end{align*}
        Si $T$ es dilatación (o contracción) $Tx=rx$, entonces
        \begin{align*}
          \hat{f\circ T}(\xi)=\frac{1}{r^{n}}\hat{f}\left( \frac{\xi}{r} \right).
        \end{align*}
      \item Si $x^{\alpha}f\in L^{1}(\mathbb{R}^{n})$, para $|\alpha|\leq k$ entonces $\hat{f}\in C^{k}(\mathbb{R}^{n})$ y $\partial^{\alpha}\hat{f}=\hat{(-2\pi ix)^{\alpha}f}$.
      \begin{itemize}
        \item Si $f\in L^{1}(\mathbb{R}^{n})$ y $x_{k}f\in L^{1}(\mathbb{R}^{n})$, entonces 
          \begin{align*}
            \frac{\partial \hat{f}}{\partial \xi_k}=\hat{-2\pi ix_{k}f}(\xi)
          \end{align*}
          en la norma de $L^1(\mathbb{R}^{n})$.
      \end{itemize}
      \item Si $f\in C^{k}(\mathbb{R}^{n})$, $\partial^{\alpha}f\in L^{1}(\mathbb{R}^{n})$ para todo $|\alpha|\leq k$ y $\partial^{\alpha}f\in C^{0}_{\infty}(\mathbb{R}^{n})$ para todo $|\alpha|\leq k-1$, entonces
        \begin{align*}
          \hat{\partial^{\alpha}f}(\xi)=(2\pi i\xi)^{\alpha}\hat{f}(\xi).
        \end{align*}
        \begin{itemize}
          \item Si $f\in L^{1}(\mathbb{R}^{n})$ y $g=\frac{\partial f}{\partial x_k}$ en la norma de $L^1(\mathbb{R}^{n})$, entonces
            \begin{align*}
              \hat{g}(\xi)=\hat{\frac{\partial f}{\partial x_k}}(\xi)=2\pi i\xi_{k}\hat{f}(\xi)
            \end{align*}
        \end{itemize}
      \item $\hat{f*g}=\hat{f}\hat{g}$.
      \item $\int_{\mathbb{R}^{n}}\hat{\xi}g(\xi)\,d\xi=\int_{\mathbb{R}^{n}}f(x)\hat{g}(x)\,ds$
    \end{enumerate}
  \end{theorem}
  \begin{proof} 
    \begin{enumerate}
      \item Demostrado en la nota (\ref{note:f-tranformada-sentido}).
      \item Note que
        \begin{align*}
          \hat{\tau_{h}f}(\xi)&=\int_{\mathbb{R}^{n}}f(x-h)e^{-2\pi ix\cdot\xi}\,dx &&\text{Haciendo $u=x-h$},\\
          &=\int_{\mathbb{R}^{n}}f(u)e^{-2\pi i (u+h)\cdot\xi}\,du,\\
          &=e^{-2\pi ih\cdot\xi}\int_{\mathbb{R}^{n}}f(u)e^{-2\pi i u\cdot\xi}\,du,\\
          &=e^{-2\pi ih\cdot\xi} \hat{f}(\xi).
        \end{align*}
        Por otro lado
        \begin{align*}
          \tau_{h}\hat{f}(\xi)&=\hat{f}(\xi-h),\\
          &=\int_{\mathbb{R}^{n}}f(x)e^{-2\pi ix\cdot (\xi-h)}\,dx,\\
          &=\int_{\mathbb{R}^{n}}e^{2\pi ix\cdot h}f(x)e^{-2\pi ix\cdot \xi}\,dx,\\
          &=\hat{e^{2\pi ix\cdot h}f}(\xi).
        \end{align*}
      \item Note que
        \begin{align*}
          \hat{f\circ T}(\xi)&=\int_{\mathbb{R}^{n}}f(Tx)e^{-2\pi ix\cdot\xi}\,dx &&\text{Haciendo $u=Tx$,}\\
          &=\int_{\mathbb{R}^{n}}f(u)e^{-2\pi i u\cdot S\xi}\,\frac{du}{|det(T)|},\\
          &=|det(T)|^{-1}\hat{f}(S\xi),\\
          &=|det(T)|^{-1}(\hat{f}\circ S)(\xi).
        \end{align*}
        Ahora, note que si $T$ es una rotación, entonces $|det(T)|=1$ 
      \item Note que esto se puede reducir a verlo en una primera derivada y de manera inductiva se concluye el resultado, por lo que será suficiente ver el siguiente ítem.
      \begin{itemize}
        \item X
      \end{itemize}
      \item Si $f\in C^{k}(\mathbb{R}^{n})$, $\partial^{\alpha}f\in L^{1}(\mathbb{R}^{n})$ para todo $|\alpha|\leq k$ y $\partial^{\alpha}f\in C^{0}_{\infty}(\mathbb{R}^{n})$ para todo $|\alpha|\leq k-1$, entonces
        \begin{align*}
          \hat{\partial^{\alpha}f}(\xi)=(2\pi i\xi)^{\alpha}\hat{f}(\xi).
        \end{align*}
        \begin{itemize}
          \item Si $f\in L^{1}(\mathbb{R}^{n})$ y $g=\frac{\partial f}{\partial x_k}$ en la norma de $L^1(\mathbb{R}^{n})$, entonces
            \begin{align*}
              \hat{g}(\xi)=\hat{\frac{\partial f}{\partial x_k}}(\xi)=2\pi i\xi_{k}\hat{f}(\xi)
            \end{align*}
        \end{itemize}       
      \item $\hat{f*g}=\hat{f}\hat{g}$.
    \end{enumerate} 
  \end{proof}
Continuará...
